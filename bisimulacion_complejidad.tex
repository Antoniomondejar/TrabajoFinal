
\begin{lema}
    Sean $\model_1=\tup{\W,\R,\cset{\S_i}_{i \in \AGT},\V,\ACT}$ y $\model_2=\tup{\W',\R',\cset{\S_i'}_{i \in \AGT},\V',\ACT'}$ dos modelos tales que existe $Z \subseteq \W \times \W'$ \KHilogic-bisimulación entre ellos, entonces

    \begin{center}
        $Z' = \{(w,w') \in \W \times \W' \mid \V(w) = \V'(w')\}$
    \end{center}
    es una \KHilogic-bisimulación entre $\modults_1$ y $\modults_2$.
\end{lema}

Notemos que este teorema nos dice que para decidir si existe una \KHilogic-bisimulación entre dos modelos $\modults_1$ y $\modults_2$ basta con verificar que $Z'$ es una \KHilogic-bisimulación.

\begin{demostracion}
    Queremos ver que $Z'$ es una \KHilogic-bisimulación entre $\modults_1$ y $\modults_2$, dado que existe $Z \subseteq \W \times \W'$, \KHilogic-bisimulación entre ellos.

    Notemos que como $Z$ es una bisimulación, entonces cumple (atom). Luego, notemos que por la definición de $Z'$, $Z \subseteq Z'$, pues $Z'$ contiene todos los pares de $\W \times \W'$ que satisfacen (atom). Esto nos dice que $Z'$ satisface (A-zig) y (A-zag). A su vez, cómo mencionamos, $Z'$ contiene todos los pares que satisfacen (atom).

    Demostremos entonces que $Z'$ cumple (\KHilogic-zig) y (\KHilogic-zag).

    \begin{itemize}
        \item (\KHilogic-zig) Sean $U,T \subseteq \W$ tales que $U$ es proposicionalmente definible y $U \ultsExecAgi T$ queremos ver que existe $T' \subseteq \W'$ tal que:
    
        \begin{multicols}{2}
            \begin{itemize}
                \item $Z'(U) \ultsExecAgi T'$, 
                \item $T' \subseteq Z'(T)$.
            \end{itemize}
        \end{multicols}
    
        Notemos que como $Z$ es una bisimulación, entonces existe $T'' \subseteq \W'$ tal que:
    
        \begin{multicols}{2}
            \begin{itemize}
                \item $Z(U) \ultsExecAgi T''$, 
                \item $T'' \subseteq Z(T)$.
            \end{itemize}
        \end{multicols}
    
        Demostremos que $T' := T''$ cumple con lo mencionado. Es claro que como $Z \subseteq Z'$ y $T'' \subseteq Z(T)$ entonces $T'' \subseteq Z'(T)$. Entonces, nos queda demostrar que $Z'(U) \ultsExecAgi T''$.
    
        Esto lo demostraremos analizando que $Z(U) = Z'(U)$. Nuevamente, como $Z \subseteq Z'$ entonces $Z(U) \subseteq Z'(U)$. Luego, solo queda demostrar que $Z'(U) \subseteq Z(U)$.
    
        Sea $w' \in Z'(U)$, entonces existe $w \in U$ tal que $(w,w') \in Z'$, por lo que $\V(w) = \V'(w')$. Ahora bien, como $Z$ cumple (A-zag), existe $v \in \W$ tal que $(v,w') \in Z$, y cómo $Z$ cumple (atom) entonces $\V(v) = \V'(w')$. Luego $\V(w) = \V(v)$.
    
        Notemos que como $U$ es proposicionalmente definible entonces existe una fórmula $\varphi$ proposicional que lo define. Como $w \in U$, esto nos dice que $\modults, w \models \varphi$. Luego por Lema 1, $\modults, v \models \varphi$, por lo que $v \in U$. Como $v \in U$ y $(v,w') \in Z$ entonces $w' \in Z(U)$.
    
        Lo cuál demuestra que $Z(U) = Z'(U)$. Por lo que $Z'(U) \ultsExecAgi T''$. Finalmente, concluimos que $Z'$ cumple (\KHilogic-zig).
    
    
        \item (\KHilogic-zag) Análogo a (\KHilogic-zig).
    \end{itemize}
    
    Queda demostrado que $Z'$ es una \KHilogic-bisimulación.
\end{demostracion}


\begin{teorema}
    El problema de decidir si existe una \KHilogic-bisimulación entre dos modelos $\modults_1$ y $\modults_2$ ($\KHilogic-bisimilitud$)  es $co-\NP$ completo.
\end{teorema}

\begin{demostracion}
    Debemos demostrar que $\KHilogic-bisimilitud$ $\in co-\NP$ y que es $co-\NP$ hard.

    Para demostrar que está en $co-\NP$ debemos dar un algoritmo $A$ y un polinomio $p$ tal que dos modelos $\modults_1$ y $\modults_2$ son \KHilogic-bisimilares si y sólo si para todo $x \in \{0,1\}^*$ con $|x| \le p(|\modults_1|+|\modults_2|)$,  $A(\modults_1,\modults_2,x) = 1$. Esencialmente, a los $x$ los llamaremos ``contraejemplos'' y sólo nos interesarán los que representen un subconjunto del dominio de $\modults_1$ o del dominio de $\modults_2$.

\medskip\medskip

    Dados que los contraejemplos que nos interesan son subconjuntos de los dominios de ambos modelos, un polinomio que nos sirve para acotar los largos de los contraejemplos es $p(x) = x$. 

\medskip\medskip
    El algoritmo $A$ que propondremos hará lo siguiente:
    
    Dados $\modults_1, \modults_2$ y $x$, verificar que el conjunto $Z'$ mencionado en el lema 4 sea una \KHilogic-bisimulación entre los dos modelos recibidos. Notar que podemos verificar polinomialmente que $Z'$ cumple (A-zig) y (A-zag).

    Luego el algoritmo verificará que $Z'$ cumpla con (\KHilogic-zig) o (\KHilogic-zag) con respecto al subconjunto $x$,  dependiendo de si $x$ es un subconjunto del dominio de $\modults_1$ o del dominio de $\modults_2$. Notar que sólo nos interesarán los $x$ que sean proposicionalmente definibles, así que el algoritmo deberá encargarse de chequear que $x$ efectivamente sea un conjunto proposicionalmente definible. 

    En caso de serlo, simplemente analizará a qué conjunto se puede llegar desde $x$ utilizando cada plan del modelo y para cada uno de ellos, verificar que $Z'(x)$ tenga algún plan con el que cumpla la condición deseada. 
    
    (Queda escribir este algoritmo y demostrar más en profundidad el si y sólo si mencionado). 

\medskip\medskip

    Luego $\KHilogic-bisimilitud \in co-\NP$.
    
\medskip\medskip

    Ahora demostremos que $\KHilogic-bisimilitud$ es $co-\NP$ hard.

    Para demostrarlo, reduciremos el problema $DNF-TAUT$ a $\KHilogic-bisimilitud$. El problema $DNF-TAUT$ está compuesto por las fórmulas $\varphi$ proposicionales en forma disyuntiva normal que son tautologías. $DNF-TAUT$ es $co-\NP$ completo, por lo que si encontramos una reducción en tiempo polinomial de $DNF-TAUT$ a $\KHilogic-bisimilitud$ habremos demostrado que $\KHilogic-bisimilitud$ es $co-\NP$ hard.

    Es decir, queremos encontrar una reducción $f$ que transforme una $\varphi$ proposicional en forma disyuntiva normal a un par de modelos $(\modults_1,\modults_2)$ en tiempo polinomial en el tamaño de $\varphi$ tal que $\varphi \in DNF-TAUT$ si y sólo si $(\modults_1,\modults_2) \in \KHilogic-bisimilitud$.

    Describamos entonces a $f$.

    Sea $\varphi$ una fórmula proposicional en forma disyuntiva normal, entonces es de la forma $\varphi = \varphi_1 \vee ... \vee \varphi_m$. Sean a su vez $p_1,...,p_n$ las variables proposicionales que aparecen en $\varphi$, definimos los siguientes conjuntos:

    \begin{itemize}
        \item $\W := \{p_i\mid i\in\{1,...,n\}\} \cup \{\varphi_i \mid i \in \{1,...,m\}\} \cup \{e\}$
        \item $\ACT := \{in_{\varphi_i} \mid i \in \{1,...,m\}\} \cup \{out_{\varphi_i} \mid i \in \{1,...,m\}\}$
        \item $\S := \{\{in_{\varphi_i}out_{\varphi_i}\} \mid i \in \{1,...,m\}\}$
        \item $\R := \{(p_i,\varphi_j,in_{\varphi_j}) \mid p_i$ no aparece en forma negativa en $\varphi_j\} \cup \{(\varphi_j,p_i,out_{\varphi_j})\mid p_i$ aparece en forma positiva en $\varphi_j\} \cup \{(\varphi_j,e,out_{\varphi_j}) \mid \varphi_j$ no tiene variables positivas$\}$
    \end{itemize}

    Notar que decimos que $p_i$ aparece en forma `positiva' en $\varphi_j$ si $\varphi_j = ... \wedge p_i \wedge...$ . A su vez, decimos que $p_i$ aparece en forma `negativa' en $\varphi_j$ si $\varphi_j = ...\wedge \neg p_i \wedge...$ .

    Ahora, $f$ mapeará a $\varphi$ a $(\modults_1,\modults_2)$, los cuáles son definidos de la siguiente manera:

    \begin{itemize}
        \item En el caso de que $\varphi$ evalúe a $false$ en la asignación que asigna a $p_1,...,p_n$ el valor $false$:

        \begin{itemize}
            \item $\model_1=\tup{\{1\},\{\},\{\},\{(1,p)\},\{a\}}$
            \item $\model_2=\tup{\{1\},\{\},\{\},\{(1,q)\},\{a\}}$
        \end{itemize}

        Notar que estos dos modelos son claramente no \KHilogic-bisimilares, dado que no existe relación binaria no vacía que satisfaga (atom). Pero es correcto porque sólo utilizaremos esta definición en uno de los casos donde $\varphi \not \in DNF-TAUT$. 
        
        \item En el caso de que $\varphi$ evalúe a $true$ en la asignación que asigna a $p_1,...,p_n$ el valor $false$:
        
        \begin{itemize}
            \item $\model_1=\tup{\W,\R \cup \{(p_i,p_i,test) \mid i \in \{1,...n\}\} \cup \{(p_i,e,test)\mid i \in \{1,...,n\}\},\{\S \cup \{\{test\}\} \},\V,\ACT \cup \{test\}}$
            \item $\model_2=\tup{\W,\R,\cset{\S},\V,\ACT}$
        \end{itemize}
    \end{itemize}

    

    $\V$ no sé cómo definirlo pero básicamente me gustaría que le asignara a cada nodo una variable proposicional distinta ($\V$ inyectiva), así la única bisimulación posible sería $\{(w,w) \mid w \in \W\}$.

    Un detalle importante es que los modelos tienen un único conjunto de clases de planes (se considera un solo agente), en el que cada clase de equivalencia está formada por un único plan (singletons). 


\medskip\medskip

    Un poco la intuición de esto es que quiero usar los conjuntos proposicionalmente definibles de este modelo para considerar cada posible asignación de valores a las variables $p_1,...,p_n$. Dado un conjunto $U$ proposicionalmente definible, es como que considero que a las variables dentro de $U$ se les asignó $true$ y a las de fuera de $U$ se les asignó $false$. 

    El plan $test$ que sólo aparece en $\modults_1$ es la clave de la reducción. Lo que provoca es que dada una asignación (un conjunto proposicionalmente definible $U$), a $\modults_2$ le hago buscar algún término $\varphi_i$ al cuál pueda moverse (ninguna variable de $U$ aparezca negada en $\varphi_i$) y a su vez todo variable que aparecía positiva en $\varphi_i$ estaba en $U$. (Porque el plan $test$ es como que simplemente va de $U$ a $U \cup \{e\}$), el $e$ es como que considera los términos que no tienen ninguna variable positiva ($e$ de empty set).

    Si $\modults_2$ no encuentra ningún término al cuál moverse es que dicho $U$ sirve como contraejemplo para afirmar que $\varphi$ no es una tautología.

\medskip\medskip
        
    Notemos que ambos modelos son polinomiales en el tamaño de $\varphi$.

    Ahora queremos demostrar que $\varphi \in DNF-TAUT$ si y sólo si $(\modults_1,\modults_2) \in \KHilogic-bisimilitud$.
    \begin{itemize}

    \item $(\rightarrow)$ Sea $\varphi \in DNF-TAUT$, donde $\varphi = \varphi_1 \vee ... \vee \varphi_m$ y $p_1,...,p_n$ son las variables proposicionales que aparecen en $\varphi$, veamos que $(\modults_1,\modults_2) \in \KHilogic-bisimilitud$.

    Algo que debemos notar es que hay una única posible bisimulación $Z$ entre $\modults_1$ y $\modults_2$, dado que toda bisimulación debe satisfacer (A-zig), (A-zag) y (atom), y cada nodo tiene un único nodo en el modelo contrario con la misma imagen de $\V$.

    Esta bisimulación es $Z := \{(w,w) \mid w \in \W \}$. Es claro que $Z$ cumple (atom), (A-zig) y (A-zag).

    Veamos que que toda arista y todo plan que está en $\modults_2$ también está en $\modults_1$, motivo por el cuál podemos afirmar que $Z$ satisface (\KHilogic-zag). 

    Luego solo queda ver que $Z$ cumple (\KHilogic-zig).

    Sea $U, T \subseteq \W$ tal que $U$ proposicionalmente definible y $U \ultsExecAgi T$, queremos ver que existe $T' \subseteq \W$ tal que:

    \begin{multicols}{2}
        \begin{itemize}
            \item $Z(U) \ultsExecAgi T'$, 
            \item $T' \subseteq Z(T)$.
        \end{itemize}
    \end{multicols}

    Un detalle a tener en cuenta es que al ser $Z = \{(w,w)\}$, entonces siempre $Z(X) = X$ para todo $X \subseteq \W$.

    Notemos primero que si el plan que utiliza el agente es de la forma $in_{\varphi_i}out_{\varphi_i}$, entonces $T$ sirve como $T'$ dado que ese mismo plan y las mismas aristas con dichas etiquetas están en $\modults_2$.

    Entonces consideremos el caso en el que el plan utilizado fue $test$. En dicho caso, si analizamos la definición de $\R$, ese plan puede ser fuertemente ejecutable sólo si $U \subseteq \{p_1,...,p_n\}$ dado que son los únicos nodos que tienen aristas con etiqueta $test$. Así que solo necesitamos considerar los casos donde $U \subseteq \{p_1,...,p_n\}$.

    Notemos que $\R_{test}(U) = U \cup \{e\}$, entonces necesariamente $U \cup \{e\} \subseteq T$.  Luego, basta encontrar algún plan en $\modults_2$ que lleve de $Z(U) = U$ a algún $T' \subseteq Z(T)$.  

    Consideremos ahora la asignación $\overrightarrow{a}$ que asigna $true$ a las variables en $U$ y $false$ a las variables en $\{p_1,...,p_n\} / U$. Como $\varphi$ es una tautología, entonces existe $\varphi_j$ que se vuelve verdadero a partir de $\overrightarrow{a}$, es decir, en $\varphi_j$ las variables que aparecen en forma positiva son algún subconjunto de $U$ y las variables que aparecen en forma negativa son algún subconjunto de las variables de $\{p_1,...,p_n\}/U$.

    Como las variables que aparecen de forma negativa no están en $U$, entonces podemos ver, analizando la definición de $\R$, que el plan $in_{\varphi_j}out{\varphi_j}$ es fuertemente ejecutable en $U$. 
    
    En particular, la aristas con etiqueta $in_{\varphi_j}$ llevarían de cada nodo de $U$ al nodo $\varphi_j$.
    
    Ahora bien, notemos que la arista $out_{\varphi_j}$ lleva del nodo $\varphi_j$ a las variables que aparecen en forma positiva o al nodo $e$ en caso de no tener variables positivas. Pero, como dijimos, las variables positivas que aparecen en $\varphi_j$ son un subconjunto de $U$. Entonces necesariamente de $\varphi_j$ iremos a un conjunto $X \subseteq U \cup \{e\}$, es decir, $\R_{in_{\varphi_j}out_{\varphi_j}}(U) = X \subseteq U \cup \{e\} \subseteq T$. Finalmente, $T' := X$ nos sirve para demostrar que $Z(U) \ultsExecAgi T'$ y, a su vez, $T' \subseteq Z(T)$.

    Lo cual demuestra que $Z$ satisface (\KHilogic-zig). 

    Juntando los puntos mencionados, demostramos que $Z$ es una \KHilogic-bisimulación entre $\modults_1$ y $\modults_2$, por lo que $(\modults_1,\modults_2) \in \KHilogic-bisimilitud$. 

    \item ($\leftarrow$) Para demostrar este caso, veremos que siendo $\varphi \notin DNF-TAUT$, donde $\varphi = \varphi_1 \vee ... \vee \varphi_m$ y $p_1,...,p_n$ son las variables proposicionales que aparecen en $\varphi$, entonces $(\modults_1,\modults_2) \notin \KHilogic-bisimilitud$.

    Como $\varphi \notin DNF-TAUT$, existe una asignación $\overrightarrow{a}$ sobre las variables $p_1,...,p_n$ que hace fallar cada $\varphi_j$. Sea $U$ el conjunto de variables a las que se le asigna $true$ en $\overrightarrow{a}$.

    Si $U = \emptyset$, es claro que ($\modults_1,\modults_2) \notin \KHilogic-bisimilitud$, dado que en este caso se construyen dos modelos concretos que no son \KHilogic-bisimilares.

    Así que consideremos el caso donde $U \neq \emptyset$. Notemos que como mencionamos anteriormente, la única bisimulación posible entre $\modults_1$ y $\modults_2$ es $Z := \{(w,w) \mid w \in \W\}$. Veamos que $Z$ no es una \KHilogic-bisimulación.

    Notemos que el conjunto $U$ es proposicionalmente definible, dado que la función $\V$ está definida de forma inyectiva, podemos ver que $U$ cumple que es unión de clases de equivalencias de $\rho_\modults$, luego por Lema es proposicionalmente definible.

    A su vez, como $U \subseteq \{p_1,...,p_n\}$ entonces el plan $test$ es fuertemente ejecutable en $U$ y $\R_{test}(U) = U \cup \{e\}$, por lo que $U \ultsExecAgi U \cup \{e\}$.
    
    Veamos que no existe $T'$ tal que:

    \begin{multicols}{2}
        \begin{itemize}
            \item $Z(U) \ultsExecAgi T'$, 
            \item $T' \subseteq Z(U \cup \{e\})$.
        \end{itemize}
    \end{multicols}

    Notemos que los planes que tiene a su disposición $Z(U)$ son de la forma $in_{\varphi_j}out_{\varphi_j}$.

    Sea entonces $\varphi_j$, analicemos qué sucede en relación al plan $in_{\varphi_j}out_{\varphi_j}$. Un primer detalle a tener en cuenta es que como $\overrightarrow{a}$ hace fallar a $\varphi_j$, entonces ocurre que existe una variable en $U$ que aparece en forma negativa en $\varphi_j$, o existe una variable en $\{p_1,...,p_n\}/U$ que aparece en forma positiva en $\varphi_j$.

    Si ocurre que una variable de $U$ aparece en forma negativa en $\varphi_j$ entonces el plan $in_{\varphi_j}out_{\varphi_j}$ no será fuertemente ejecutable en dicha variable y, por lo tanto, no será fuertemente ejecutable en $U$. Así que este plan no permitiría encontrar un $T'$ adecuado.

    Por otro lado, si ninguna variable de $U$ aparece en forma negativa en $\varphi_j$ entonces $in_{\varphi_j}out_{\varphi_j}$ es fuertemente ejecutable en $U$. Sin embargo, notemos que entonces existe una variable $p_k$ en $\{p_1,...,p_n\}/U$ que aparece en forma positiva en $\varphi_j$. Ahora bien, como $p_k$ aparece en forma positiva en $\varphi_j$ entonces $p_k \in \R_{in_{\varphi_j}out_{\varphi_j}}(U)$. Luego $\R_{in_{\varphi_j}out_{\varphi_j}}(U) \not \subseteq U \cup \{e\}$, por lo que dicho plan no nos permite encontrar un $T'$ adecuado.
    
    Hemos analizado entonces todos los planes posibles y ninguno encuentra un $T'$ que cumpla con lo requerido, luego $Z$ no satisface (\KHilogic-zig).

    Hemos demostrado entonces que $Z$ no es una \KHilogic-bisimulación.

    Lo cual demuestra que $(\modults_1,\modults_2) \not \in \KHilogic-bisimilitud$.
    \end{itemize}

     Hemos demostrado que $f$ es una reducción de $DNF-TAUT$ a $\KHilogic-bisimilitud$, la cuál puede ser computada en tiempo polinomial. Lo cuál demuestra que $\KHilogic-bisimilitud$ es $co-\NP$ hard.

    Finalmente, como $\KHilogic-bisimilitud \in co-\NP$ y $\KHilogic-bisimilitud$ es $co-\NP$ hard, podemos concluir que $\KHilogic-bisimilitud$ es $co-\NP$ completo.
    
\end{demostracion}
