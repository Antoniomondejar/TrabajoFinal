
\begin{lema}
    Sean $\model_1=\tup{\W,\R,\cset{\S_i}_{i \in \AGT},\V,\ACT}$ y $\model_2=\tup{\W',\R',\cset{\S_i'}_{i \in \AGT},\V',\ACT'}$ dos modelos tales que existe $Z \subseteq \W \times \W'$ \KHilogic-bisimulación entre ellos, entonces

    \begin{center}
        $Z' = \{(w,w') \in \W \times \W' \mid \V(w) = \V'(w')\}$
    \end{center}
    es una \KHilogic-bisimulación entre $\modults_1$ y $\modults_2$.
\end{lema}

Notemos que este teorema nos dice que para decidir si existe una \KHilogic-bisimulación entre dos modelos $\modults_1$ y $\modults_2$ basta con verificar que $Z'$ es una \KHilogic-bisimulación.

\begin{demostracion}
    Queremos ver que $Z'$ es una \KHilogic-bisimulación entre $\modults_1$ y $\modults_2$, dado que existe $Z \subseteq \W \times \W'$, \KHilogic-bisimulación entre ellos.

    Notemos que como $Z$ es una bisimulación, entonces cumple (atom). Luego, notemos que por la definición de $Z'$, $Z \subseteq Z'$, pues $Z'$ contiene todos los pares de $\W \times \W'$ que satisfacen (atom). Esto nos dice que $Z'$ satisface (A-zig) y (A-zag). A su vez, cómo mencionamos, $Z'$ contiene todos los pares que satisfacen (atom).

    Demostremos entonces que $Z'$ cumple (\KHilogic-zig) y (\KHilogic-zag).

    \begin{itemize}
        \item (\KHilogic-zig) Sean $U,T \subseteq \W$ tales que $U$ es proposicionalmente definible y $U \ultsExecAgi T$ queremos ver que existe $T' \subseteq \W'$ tal que:
    
        \begin{multicols}{2}
            \begin{itemize}
                \item $Z'(U) \ultsExecAgi T'$, 
                \item $T' \subseteq Z'(T)$.
            \end{itemize}
        \end{multicols}
    
        Notemos que como $Z$ es una bisimulación, entonces existe $T'' \subseteq \W'$ tal que:
    
        \begin{multicols}{2}
            \begin{itemize}
                \item $Z(U) \ultsExecAgi T''$, 
                \item $T'' \subseteq Z(T)$.
            \end{itemize}
        \end{multicols}
    
        Demostremos que $T' := T''$ cumple con lo mencionado. Es claro que como $Z \subseteq Z'$ y $T'' \subseteq Z(T)$ entonces $T'' \subseteq Z'(T)$. Entonces, nos queda demostrar que $Z'(U) \ultsExecAgi T''$.
    
        Esto lo demostraremos analizando que $Z(U) = Z'(U)$. Nuevamente, como $Z \subseteq Z'$ entonces $Z(U) \subseteq Z'(U)$. Luego, solo queda demostrar que $Z'(U) \subseteq Z(U)$.
    
        Sea $w' \in Z'(U)$, entonces existe $w \in U$ tal que $(w,w') \in Z'$, por lo que $\V(w) = \V'(w')$. Ahora bien, como $Z$ cumple (A-zag), existe $v \in \W$ tal que $(v,w') \in Z$, y cómo $Z$ cumple (atom) entonces $\V(v) = \V'(w')$. Luego $\V(w) = \V(v)$.
    
        Notemos que como $U$ es proposicionalmente definible entonces existe una fórmula $\varphi$ proposicional que lo define. Como $w \in U$, esto nos dice que $\modults, w \models \varphi$. Luego por Lema 1, $\modults, v \models \varphi$, por lo que $v \in U$. Como $v \in U$ y $(v,w') \in Z$ entonces $w' \in Z(U)$.
    
        Lo cuál demuestra que $Z(U) = Z'(U)$. Por lo que $Z'(U) \ultsExecAgi T''$. Finalmente, concluimos que $Z'$ cumple (\KHilogic-zig).
    
    
        \item (\KHilogic-zag) Análogo a (\KHilogic-zig).
    \end{itemize}
    
    Queda demostrado que $Z'$ es una \KHilogic-bisimulación.
\end{demostracion}


\begin{lema}
    $\KHiBisim \in \coNP$ 
\end{lema}

\begin{demostracion}
    Para demostrar que $\KHiBisim \in \coNP$ debemos dar un algoritmo $A$ y un polinomio $p$ tal que dos modelos $\modults_1$ y $\modults_2$ son \KHilogic-bisimilares si y sólo si para todo $x \in \{0,1\}^*$ con $|x| \le p(|\modults_1|+|\modults_2|)$,  $A(\modults_1,\modults_2,x) = 1$. Esencialmente, a los $x$ los llamaremos ``contraejemplos'' y sólo nos interesarán los que representen un subconjunto del dominio de $\modults_1$ o del dominio de $\modults_2$.

\medskip\medskip

    Dados que los contraejemplos que nos interesan son subconjuntos de los dominios de ambos modelos, un polinomio que nos sirve para acotar los largos de los contraejemplos es $p(x) = x$. 

\medskip\medskip
    El algoritmo $A$ que propondremos hará lo siguiente:
    
    Dados $\modults_1, \modults_2$ y $x$, verificar que el conjunto $Z'$ mencionado en el lema 4 sea una \KHilogic-bisimulación entre los dos modelos recibidos. Notar que podemos verificar polinomialmente que $Z'$ cumple (A-zig) y (A-zag).

    Luego el algoritmo verificará que $Z'$ cumpla con (\KHilogic-zig) o (\KHilogic-zag) con respecto al subconjunto $x$,  dependiendo de si $x$ es un subconjunto del dominio de $\modults_1$ o del dominio de $\modults_2$. Notar que sólo nos interesarán los $x$ que sean proposicionalmente definibles, así que el algoritmo deberá encargarse de chequear que $x$ efectivamente sea un conjunto proposicionalmente definible. 

    En caso de serlo, simplemente analizará a qué conjunto se puede llegar desde $x$ utilizando cada plan del modelo y para cada uno de ellos, verificar que $Z'(x)$ tenga algún plan con el que cumpla la condición deseada. 
    
    (Queda escribir este algoritmo y demostrar más en profundidad el si y sólo si mencionado). 

\medskip\medskip

    Finalmente $\KHiBisim \in \coNP$.

\end{demostracion}

\begin{lema}
    $\KHiBisim$ es $\coNP$-hard.
\end{lema}

\begin{demostracion}

    Demostremos que $\KHiBisim$ es $\coNP$-hard.

    Para demostrarlo, reduciremos el problema $\DNFTAUT$ a $\KHiBisim$. El problema $\DNFTAUT$ está compuesto por las fórmulas $\varphi$ proposicionales en forma disyuntiva normal que son tautologías. $\DNFTAUT$ es $\coNP$ completo, por lo que si encontramos una reducción computable en tiempo polinomial de $\DNFTAUT$ a $\KHiBisim$ habremos demostrado que $\KHiBisim$ es $\coNP$-hard.

    Entonces queremos encontrar una reducción $f$ computable en tiempo polinomial que transforme una $\varphi$ proposicional en forma disyuntiva normal a un par de modelos $\tup{\modults_1,\modults_2}$ tal que $\varphi \in \DNFTAUT$ si y sólo si $\tup{\modults_1,\modults_2} \in \KHiBisim$.

    Describamos a $f$.

    Sea $\varphi$ una fórmula proposicional en forma disyuntiva normal, entonces es de la forma $\varphi = \varphi_1 \vee ... \vee \varphi_m$. Sean a su vez $p_1,...,p_n$ las variables proposicionales que aparecen en $\varphi$, definimos los siguientes conjuntos:

    \begin{itemize}
        \item $\W := \{p_i\mid i\in\{1,...,n\}\} \cup \{\varphi_i \mid i \in \{1,...,m\}\} \cup \{e\}$
        \item $\ACT := \{in_{\varphi_i} \mid i \in \{1,...,m\}\} \cup \{out_{\varphi_i} \mid i \in \{1,...,m\}\}$
        \item $\S := \{\{in_{\varphi_i}out_{\varphi_i}\} \mid i \in \{1,...,m\}\}$
        \item $\R := \{(p_i,\varphi_j,in_{\varphi_j}) \mid p_i$ no aparece en forma negativa en $\varphi_j\} \cup \{(\varphi_j,p_i,out_{\varphi_j})\mid p_i$ aparece en forma positiva en $\varphi_j\} \cup \{(\varphi_j,e,out_{\varphi_j}) \mid \varphi_j$ no tiene variables positivas$\}$
        \item $\V := \{(p_i,\{q_i\}) \mid i \in \{1,...,n\}\} \cup \{(\varphi_i,\{q_{i+n}\}) \mid i \in \{1,...,m\}\} \cup \{(e,\{q_{n+m+1}\})\}$ donde $q_1,...,q_{n+m+1} \in \PROP$ son $n+m+1$ variables proposicionales distintas.
    \end{itemize}

    Notar que decimos que $p_i$ aparece en forma `positiva' en $\varphi_j$ si $\varphi_j = ... \wedge p_i \wedge...$ . A su vez, decimos que $p_i$ aparece en forma `negativa' en $\varphi_j$ si $\varphi_j = ...\wedge \neg p_i \wedge...$ .

    Ahora, $f$ mapeará $\varphi$ a $\tup{\modults_1,\modults_2}$, con $\tup{\modults_1,\modults_2}$ definidos de la siguiente manera:

    \begin{itemize}
        \item En el caso de que $\varphi$ evalúe a $false$ en la asignación que asigna a $p_1,...,p_n$ el valor $false$:

        \begin{itemize}
            \item $\model_1=\tup{\{1\},\{\},\{\},\{(1,p)\},\{a\}}$
            \item $\model_2=\tup{\{1\},\{\},\{\},\{(1,q)\},\{a\}}$
        \end{itemize}

        Notar que estos dos modelos son claramente no \KHilogic-bisimilares, dado que no existe relación binaria no vacía que satisfaga (atom). Pero es correcto porque sólo utilizaremos esta definición en uno de los casos donde $\varphi \not \in \DNFTAUT$. 
        
        \item En el caso de que $\varphi$ evalúe a $true$ en la asignación que asigna a $p_1,...,p_n$ el valor $false$:
        
        \begin{itemize}
            \item $\model_1=\tup{\W,\R \cup \{(p_i,p_i,test) \mid i \in \{1,...n\}\} \cup \{(p_i,e,test)\mid i \in \{1,...,n\}\},\{\S \cup \{\{test\}\} \},\V,\ACT \cup \{test\}}$
            \item $\model_2=\tup{\W,\R,\cset{\S},\V,\ACT}$
        \end{itemize}
    \end{itemize}

    
    Notar que $\V$ está definida de forma tal que cada $w\in\W$ satisface una única variable proposicional distinta, es decir, es inyectiva.


\medskip\medskip
    Estos dos párrafos que siguen eran explicativos, pensaba sacarlos pero quizás está bueno dar una explicación más humana antes de la demo más técnica.

    La intuición de esta reducción es que queremos usar los conjuntos proposicionalmente definibles de este modelo para considerar cada posible asignación de valores a las variables $p_1,...,p_n$. Dado un conjunto $U \subseteq \{p_1,...,p_n\}$ , estamos considerando la asignación que asigna $true$ a las variables dentro de $U$ y asigna $false$ a las de $\{p_1,...,p_n\}/U$. Además, notemos que como la función $\V$ es inyectiva, entonces $U$ es proposicionalmente definible.
 
    El plan $test$ que sólo aparece en $\modults_1$ es la clave de la reducción. Lo que provoca es que dada una asignación (un conjunto proposicionalmente definible $U$), a $\modults_2$ le hago buscar algún término $\varphi_i$ al cuál pueda moverse (ninguna variable de $U$ aparezca negada en $\varphi_i$) y a su vez todo variable que aparecía positiva en $\varphi_i$ estaba en $U$. (Porque el plan $test$ es como que simplemente va de $U$ a $U \cup \{e\}$), el $e$ es como que considera los términos que no tienen ninguna variable positiva ($e$ de empty set). Si $\modults_2$ no encuentra ningún término al cuál moverse es que dicho $U$ sirve como contraejemplo para afirmar que $\varphi$ no es una tautología.

\medskip\medskip
        
    Notemos que ambos modelos son polinomiales en el tamaño de $\varphi$.

    A su vez, la construcción de los modelos, solo tiene el costo extra computacional de analizar si $\varphi$ evalúa a $true$ en la asignación que asigna $false$ a $p_1,...,p_n$, lo cuál es claramente computable en tiempo polinomial.

    Por lo que es fácil notar que $f$ es computable en tiempo polinomial en el tamaño de $\varphi$.

    Ahora queremos demostrar que $\varphi \in \DNFTAUT$ si y sólo si $\tup{\modults_1,\modults_2} \in \KHiBisim$.
    \begin{itemize}

    \item $(\rightarrow)$ Sea $\varphi \in \DNFTAUT$, donde $\varphi = \varphi_1 \vee ... \vee \varphi_m$ y $p_1,...,p_n$ son las variables proposicionales que aparecen en $\varphi$, veamos que $\tup{\modults_1,\modults_2} \in \KHiBisim$.

    Notar que como $\varphi$ es una tautología, entonces $\varphi$ evalúa a $true$ en la asignación que asigna $false$ a $p_1,...,p_n$, por lo que consideramos el segundo caso del mapeo $f$. 

    Veamos que $Z := \{(w,w) \mid w \in \W \}$ es una bisimulación entre $\modults_1$ y $\modults_2$. Es fácil ver que $Z$ satisface (atom), pues ambos modelos utilizan la misma función de valuación $\V$.

    A su vez, tanto (A-zig) como (A-zag) son claramente satisfechos por $Z$.

    Analicemos entonces (\KHilogic-zig) y (\KHilogic-zag).

    Primero, cabe resaltar que, por la definición de $Z$, para cada $X \subseteq \W$ se cumple que $X = Z(X)$.

    Ahora bien, notemos que cada plan de $\modults_2$ está también en $\modults_1$, y más aún, las aristas con las etiquetas que aparecen en dichos planes son exactamente las mismas en ambos modelos. Juntando lo recién mencionado y el hecho de que $U = Z(U)$, es fácil notar que Z satisface (\KHilogic-zag).

    Luego solo queda ver que $Z$ satisface (\KHilogic-zig).

    Sea $U, T \subseteq \W$ tal que $U$ es proposicionalmente definible y $U \ultsExecAgi T$, queremos ver que existe $T' \subseteq \W$ tal que:

    \begin{multicols}{2}
        \begin{itemize}
            \item $Z(U) \ultsExecAgi T'$, 
            \item $T' \subseteq Z(T)$.
        \end{itemize}
    \end{multicols}

    Notemos primero que si el plan que atestigua $U \ultsExecAgi T$ es de la forma $in_{\varphi_j}out_{\varphi_j}$, entonces $T$ sirve como $T'$ dado que ese mismo plan y las mismas aristas con dichas etiquetas están en $\modults_2$, similar al análisis realizado para el caso (\KHilogic-zag).

    Entonces consideremos el caso en el que el plan que atestigua $U \ultsExecAgi T$ es $test$. En dicho caso, si analizamos la definición de $\R$, ese plan puede ser fuertemente ejecutable sólo si $U \subseteq \{p_1,...,p_n\}$ dado que son los únicos nodos que tienen aristas con etiqueta $test$. Así que solo necesitamos considerar los casos donde $U \subseteq \{p_1,...,p_n\}$.

    Notemos que $\R_{test}(U) = U \cup \{e\}$, entonces necesariamente $U \cup \{e\} \subseteq T$.

    Consideremos ahora la asignación $\overrightarrow{a}$ que asigna $true$ a las variables en $U$ y $false$ a las variables en $\{p_1,...,p_n\} / U$. Como $\varphi$ es una tautología, entonces existe $\varphi_j$ que se vuelve verdadero a partir de $\overrightarrow{a}$, es decir, en $\varphi_j$ las variables que aparecen en forma positiva son algún subconjunto de $U$ y las variables que aparecen en forma negativa son algún subconjunto de $\{p_1,...,p_n\}/U$.

    Como las variables que aparecen en $\varphi_j$ en forma negativa no están en $U$, entonces, analizando la definición de $\R$, podemos ver que el plan $in_{\varphi_j}out_{\varphi_j}$ es fuertemente ejecutable en $Z(U) = U$. 
    
    En particular, las aristas con etiqueta $in_{\varphi_j}$ llevarían de cada nodo de $U$ al nodo $\varphi_j$.
    
    Ahora bien, notemos que las aristas con etiqueta $out_{\varphi_j}$ llevan del nodo $\varphi_j$ a las variables que aparecen en forma positiva en $\varphi_j$ o al nodo $e$ en caso de no tener variables positivas. Pero, como dijimos, las variables positivas que aparecen en $\varphi_j$ son un subconjunto, posiblemente vacío, de $U$. Entonces necesariamente de $\varphi_j$ las aristas con etiqueta $out_{\varphi_j}$ van a un conjunto $X \subseteq U \cup \{e\}$, es decir, $\R_{in_{\varphi_j}out_{\varphi_j}}(U) = X \subseteq U \cup \{e\} \subseteq T$. Finalmente, $T' := X$ nos sirve para demostrar que $Z(U) \ultsExecAgi T'$ y, a su vez, $T' \subseteq X \subseteq U \cup \{e\} \subseteq T = Z(T)$.

    Lo cual demuestra que $Z$ satisface (\KHilogic-zig). 

    Juntando los puntos mencionados, demostramos que $Z$ es una bisimulación entre $\modults_1$ y $\modults_2$, por lo que $\tup{\modults_1,\modults_2} \in \KHiBisim$. 

    \item ($\leftarrow$) Para demostrar este caso, veremos que siendo $\varphi \notin \DNFTAUT$, donde $\varphi = \varphi_1 \vee ... \vee \varphi_m$ y $p_1,...,p_n$ son las variables proposicionales que aparecen en $\varphi$, entonces $\tup{\modults_1,\modults_2} \notin \KHiBisim$.

    Como $\varphi \notin \DNFTAUT$, existe una asignación $\overrightarrow{a}$ sobre las variables $p_1,...,p_n$ que hace fallar cada $\varphi_j$. Denotemos a $U$ como el conjunto de variables a las que se le asigna $true$ en $\overrightarrow{a}$.

    Si $U = \emptyset$, es claro que $\tup{\modults_1,\modults_2} \notin \KHiBisim$, dado que en este caso se construyen dos modelos concretos que no son \KHilogic-bisimilares.

    Así que consideremos el caso donde $U \neq \emptyset$.
    
    Primero notemos que la fórmula $\psi := \bigvee\limits_{p_i \in U} q_i$ define a $U$. Pues, recordemos que por la definición de $\V$, $\V(p_i) = \{q_i\}$ para cada $i \in \{1,...,n\}$. A su vez, como $\V$ es inyectiva, solo los elementos de $U$ satisfacerán $\psi$. Luego, como $\psi$ es proposicional, $U$ es proposicionalmente definible.

    Ahora bien, como $U \subseteq \{p_1,...,p_n\}$ entonces el plan $test$ es fuertemente ejecutable en $U$ y $\R_{test}(U) = U \cup \{e\}$, por lo que $U \ultsExecAgi U \cup \{e\}$.
    
    Veamos que no existe $T'$ tal que:

    \begin{multicols}{2}
        \begin{itemize}
            \item $Z(U) \ultsExecAgi T'$, 
            \item $T' \subseteq Z(U \cup \{e\})$.
        \end{itemize}
    \end{multicols}

    Notemos que los planes que tiene a su disposición $Z(U) = U$ son de la forma $in_{\varphi_j}out_{\varphi_j}$.

    Sea entonces $\varphi_j$, analicemos qué sucede en relación al plan $in_{\varphi_j}out_{\varphi_j}$. Un primer detalle a tener en cuenta es que como $\overrightarrow{a}$ hace fallar a $\varphi_j$, entonces, o bien ocurre que existe una variable en $U$ que aparece en forma negativa en $\varphi_j$, o existe una variable en $\{p_1,...,p_n\}/U$ que aparece en forma positiva en $\varphi_j$.

    Si ocurre que una variable de $U$ aparece en forma negativa en $\varphi_j$ entonces el plan $in_{\varphi_j}out_{\varphi_j}$ no será fuertemente ejecutable en dicha variable y, por lo tanto, no será fuertemente ejecutable en $U$. Así que este plan no permitiría encontrar un $T'$ adecuado.

    Por otro lado, si ninguna variable de $U$ aparece en forma negativa en $\varphi_j$ entonces $in_{\varphi_j}out_{\varphi_j}$ es fuertemente ejecutable en $U$. Sin embargo, notemos que entonces existe una variable $p_k$ en $\{p_1,...,p_n\}/U$ que aparece en forma positiva en $\varphi_j$. Ahora bien, como $p_k$ aparece en forma positiva en $\varphi_j$ entonces $p_k \in \R_{in_{\varphi_j}out_{\varphi_j}}(U)$. Luego $\R_{in_{\varphi_j}out_{\varphi_j}}(U) \not \subseteq U \cup \{e\} = Z(U \cup \{e\})$, por lo que dicho plan no nos permite encontrar un $T'$ adecuado.
    
    Hemos analizado entonces todos los planes y ninguno permite encontrar un $T'$ que cumpla con lo requerido, luego $Z$ no satisface (\KHilogic-zig).

    Hemos demostrado entonces que $Z$ no es una \KHilogic-bisimulación.

    Lo cual demuestra que $\tup{\modults_1,\modults_2} \not \in \KHiBisim$.
    \end{itemize}

     Finalmente, demostramos que $f$ es una reducción de $\DNFTAUT$ a $\KHiBisim$, la cuál puede ser computada en tiempo polinomial. Lo cuál demuestra que $\KHiBisim$ es $\coNP$-hard.

\end{demostracion}


\begin{teorema}
    $\KHiBisim$ es $\coNP$-completo.
\end{teorema}


\begin{demostracion}

    A partir de los lemas 5 y 6, podemos afirmar que $\KHiBisim$ es $\coNP$-completo.
    
\end{demostracion}
