\chapter{Introducción}

La lógica y las ciencias de la computación han estado en contacto y en constante desarrollo a lo largo de los años. Desde la búsqueda de 
un procedimiento para determinar mecánicamente la validez de una sentencia matemática en los años 1930 con el programa de Hilbert hasta 
hoy en día donde abuda el uso de métodos formales en áreas de verificación de software e inteligencia artificial, estas dos disciplinas 
científicas no han dejado de nutrirse mutuamente.

En este trabajo contribuiremos novedosos aportes al estudio de la noción de bisimulación para 
\KHilogic. En primera instancia, propondremos una nueva definición para este concepto con el objetivo de otorgarle una 
naturaleza más algorítmica y mejor alíneada con las nociones clásicas de bisimulación, las cuales únicamente consideran 
características estructurales de los objetos matemáticos involucrados.
Luego, estudiaremos problemas computacionales relacionados con esta noción. Por un lado, caracterizaremos la complejidad computacional de 
determinar si una relación binaria es una \KHilogic-bisimulación y, haremos lo mismo para el problema de decidir la existencia de una 
\KHilogic-bisimulación entre dos modelos. Más aún, estudiaremos el problema de la minimización de modelos para \KHilogic a partir de 
la definición de contracciones por \KHilogic-bisimulación.
Este análisis computacional no solo es realizado con el objetivo de determinar la factibilidad de este lenguaje formal en aplicaciones 
prácticas sino que también para expandir el entendimiento a nivel conceptual de los objetos matemáticos 
considerados y, de esta forma, fortalecer el argumento a favor del estudio de aspectos computacionales en ámbitos propios de la lógica 
y la teoría de modelos.
