\chapter{Introducción}

La lógica y las ciencias de la computación han estado en contacto y en constante desarrollo a lo largo de los años. Desde de los años 1930
con la búsqueda de un procedimiento para determinar mecánicamente la validez de una sentencia matemática hasta 
hoy en día donde abunda el uso de métodos formales en áreas de verificación de software e inteligencia artificial, estas dos disciplinas 
científicas no han dejado de nutrirse mutuamente.

Como se menciona en \cite{HalpernHIKVV01}, la lógica se ha sumergido exitosamente en numerosas áreas de las ciencias de la computación. 
Un clásico ejemplo es la rama de la complejidad descriptiva, donde se buscan establecer relaciones entre la complejidad intrínseca 
de los problemas computacionales y la dificultad de expresar una propiedad en una determinada lógica. Un resultado que evidencia el enorme
potencial de esta propuesta es el Teorema de Fagin \cite{Fagin}, el cual establece que los problemas computacionales para los que se pueden 
verificar sus soluciones eficientemente (la clase \NP) son exactamente aquellos que pueden expresarse con una fórmula existencial de segundo 
orden, lo que permite establecer cotas superiores de dificultad computacional sin la necesidad de encontrar un algoritmo. Otros ejemplos 
relacionados con áreas prácticas de la computación incluyen el uso de variaciones sintácticas de lógica de primer orden como 
lenguajes de consulta de bases de datos y de herramientas de model-checking en verificación formal de sistemas.  

También se ha explorado el uso de formalismos lógicos para modelar nociones más abstractas vinculadas al razonamiento y la información.  
El ejemplo que constituye las bases teóricas en las que surge este trabajo es el de las lógicas epistémicas. Estas permiten representar 
y razonar formalmente sobre el conocimiento de agentes. De esta forma podemos expresar propiedades tales como ``Juan sabe que está 
lloviendo en Río Cuarto'', en la que el valor de verdad de la propiedad no depende de si llueve o no llueve en Río Cuarto sino que 
del conocimiento de Juan acerca del hecho. Así, las lógicas epistémicas proveen una base para el analisis de preguntas como: ¿Qué necesita 
saber un robot para cumplir un objetivo? ¿Qué necesita saber un agente acerca de los demás agentes para convencerlos de un argumento?

La semántica para este tipo de lógicas está basada en la noción de \textit{mundos posibles}. La idea es que dada la información actual 
que posee un agente existe una cierta cantidad de mundos posibles que podrían representar el estado actual de los hechos. 
Luego, un agente \textit{sabe} una propiedad si la misma es cierta en todos los mundos que considera que son posibles. 

\bigskip


En este trabajo contribuimos novedosos aportes al estudio de la noción de bisimulación para 
\KHilogic. En primera instancia, proponemos una nueva definición para este concepto con el objetivo de otorgarle una 
naturaleza más algorítmica y mejor alíneada con las nociones clásicas de bisimulación, las cuales únicamente consideran 
características estructurales de los objetos matemáticos involucrados.
Luego, estudiamos problemas computacionales relacionados con esta noción. Por un lado, caracterizamos la complejidad computacional de 
determinar si una relación binaria es una \KHilogic-bisimulación y, a su vez, la complejidad computacional de determinar la existencia 
de una \KHilogic-bisimulación entre dos modelos. Por último, encaramos el problema de la minimización de modelos para \KHilogic a partir 
de la definición de contracciones por \KHilogic-bisimulación.
Este análisis computacional no solo es realizado con el objetivo de determinar la factibilidad de este lenguaje formal en aplicaciones 
prácticas sino que también para expandir el entendimiento a nivel conceptual de los objetos matemáticos 
considerados y, de esta forma, fortalecer el argumento a favor del estudio de aspectos computacionales en ámbitos propios de la lógica.
