\documentclass{beamer}
\usepackage{xspace, amssymb, amsmath, tikz, hyperref,mathtools}
\usepgflibrary{arrows}
\tikzstyle{point}=[circle,draw]
\usetikzlibrary{arrows,automata,shapes,decorations.markings,decorations.pathmorphing,backgrounds,fit,snakes}
\usepackage[c]{esvect}
\usepackage{graphicx}
\usepackage{stmaryrd}
\usepackage[all]{xy}
%\usepackage{newmacros}
%\newtheorem{proposition}{Proposition}
%\usepackage{chronology}
\usepackage{chronosys}
\usepackage{soul}
\usetikzlibrary{calc}
%
\usetheme[progressbar=frametitle]{metropolis}
\setbeamertemplate{blocks}[rounded][shadow=true]
\setbeamertemplate{items}[ball]
%\setbeamertemplate{navigation symbols}{}
\setbeamercolor{title}{bg=red!65!black, fg=black}
\usepackage{ragged2e}
\usepackage[dvipsnames]{xcolor}
\justifying
%!TEX root = main.tex
%------------------------------------------------------------------------------------------------
% adjusting vertical space in the multicols environment
% default: \multicolsep = 12pt plus 4pt minus 3pt

\setlength{\multicolsep}{6pt plus 2pt minus 1.5pt}

%------------------------------------------------------------------------------------------------


% defining the format for theorems, propositions, corollaries and lemmas. The main point is the symbol at the end of them
\usepackage[hyperref,thmmarks]{ntheorem}
\usepackage{amsmath}
\usepackage{multicol}


% \theoremstyle{plain}
\theoremsymbol{\ensuremath{\triangleleft}}

\newtheorem{teorema}{Teorema}
\newtheorem{proposicion}{Proposición}
\newtheorem{corolario}{Corolario}
\newtheorem{lema}{Lema}
\newtheorem{hecho}{Hecho}

%------------------------------------------------

% defining the format for definitions, examples, fact and others. Same symbol at the end, but header and font different
\theoremheaderfont{\normalfont\bfseries}
\theorembodyfont{\normalfont}
\theoremsymbol{\ensuremath{\dashv}}

\newtheorem{definicion}{Definición}[section]
\newtheorem{ejemplo}{Ejemplo}
\newtheorem{nota}{Nota}

%------------------------------------------------

% defining the format for proofs
% \theoremstyle{nonumberplain}
\theoremheaderfont{\itshape}
\theorembodyfont{\normalfont}
\theoremsymbol{\ensuremath{\blacksquare}}

\newtheorem{demostracion}{Demostración.}

%-----------------------------------------------------------------------------

%------------------------------------------------
\usepackage{amssymb}


%------------------------------------------------




\newcommand{\modlts}{\mathcal{S}}
\newcommand{\modults}{\mathcal{M}}
\newcommand{\model}{\modults}


\newcommand{\tup}[1]{\langle #1 \rangle}

\newcommand{\W}{\mathrm{W}}
\newcommand{\R}{\mathrm{R}}
\renewcommand{\S}{\mathrm{S}}
\newcommand{\V}{\mathrm{V}}

\newcommand{\PROP}{{\rm \sf Prop}}
\newcommand{\ACT}{{\rm \sf Act}}
\newcommand{\AGT}{{\rm \sf Agt}}
\newcommand{\ACTP}{{\rm \sf A}}

\newcommand{\cset}[1]{\{ #1 \}}
% Sets with condition \csetsc{y}{xRy} = {y | xRy}
\newcommand{\csetsc}[2]{\{ #1  \mid #2 \}}
% Sets with domain and condition \csetc{y}{A}{xRy} = {y \in A | xRy}
\newcommand{\csetc}[3]{\{ #1 \in #2 \mid #3 \}}

\newcommand{\formatmodality}[1]{\ensuremath{\mathsf{#1}}}
\newcommand{\kh}{\formatmodality{Kh}}
\newcommand{\khi}{\formatmodality{Kh}_i}
% languages
\newcommand{\KHlogic}{\ensuremath{\mathsf{L}_{\kh}}}
\newcommand{\KHilogic}{\ensuremath{\mathsf{L}_{\khi}}}

\newcommand{\mycomment}[1]{}

\newcommand{\bigO}[0]{\mathcal{O}}

% complejidad 
\newcommand{\NP}{\rm\textsf{NP}}
\newcommand{\Poly}{\rm\textsf{P}}
\newcommand{\coNP}{\rm\textsf{coNP}}

\newcommand{\DNFTAUT}{\rm\textsf{DNF-TAUT}}

\newcommand{\KHBisim}{\rm\textsf{KhBisim}}

% estrategias
\newcommand{\mystackrel}[2]{%
  \mathrel{\vbox{\offinterlineskip\ialign{%
  \hfil##\hfil\cr
  \ensuremath{\scriptstyle#1}\cr
  \noalign{\kern.25ex}
  \ensuremath{#2}\cr
  }}}
}
\newcommand{\ultsExec}{\Rightarrow}

%\newcommand{\ultsExecStrat}[1]{\stackrel{#1}{\ultsExec}}
%\newcommand{\ultsExecStrat}[1]{\mystackrel{#1}{\ultsExec}}
\newcommand{\ultsExecStrat}[1]{\mathrel{\raisebox{-.2ex}{\ensuremath{\mystackrel{#1}{\ultsExec}}}}}

%\newcommand{\ultsExecAgi}{\stackrel{i}{\ultsExec}}
%\newcommand{\ultsExecAgi}{\mystackrel{i}{\ultsExec}}
\newcommand{\ultsExecAgi}{\mathrel{\raisebox{-.2ex}{\ensuremath{\mystackrel{i}{\ultsExec}}}}}

% no se que es esto
\usepackage{enumitem}
\newlist{cond-bisim}{enumerate}{1}
\setlist[cond-bisim]{label=\itmbisimformat{\arabic*}, leftmargin=3em, itemindent=1em}


\newcommand{\reff}[1]{\dgray{[#1]}}

\begin{document}
%\includeonlyframes{current}


%\addtobeamertemplate{footline}{\insertframenumber/\inserttotalframenumber}
%\addtobeamertemplate{footline}{\hfill \blue{\insertframenumber/\inserttotalframenumber}\hspace{2em}\null}
 
\title[Computational Aspects of Bisimulations for a Logic of Knowing How]{Computational Aspects of Bisimulations for a Logic of Knowing How}%{\large Logic in AI: A Modern View}
\author[Antonio Mondejar]
		{{\small Antonio Mondejar}  \\ {\tiny Logics, Interaction and Intelligent Systems Group (LIIS), \\  FAMAF-UNC }
        \\
        {\tiny \emph{(Joint work with Carlos Areces and Raul Fervari)}} \\
        \bigskip
}

\date[DL(R) 2025 Workshop]
{ \tiny{DL(R) 2025 Workshop - Instituto de Ingeniería Matemática y Computacional, 
        Universidad Católica de Chile}}
%\titlegraphic{\hfill \includegraphics[height=2.5cm]{accion.jpg}\\}
% %
%\centering\includegraphics[scale=.1]{logocjcc.png}


\begin{frame}
\titlepage
\end{frame}

\tikzset{
    >=stealth',
    ar/.style={
           shorten <=2pt,
           shorten >=2pt,}
}

%=============================================================

\begin{frame}
    \frametitle{The concept(s) of knowledge}
    
    \begin{small}
    \begin{itemize}
        \item Usually, \blue{epistemic} logic is about ``knowing that'' \gray{(Hintikka 1962)}: 
            \begin{itemize}
                \item \emph{John knows that it is raining in Paris}, 
                \item \emph{the robot knows that it is standing next to a wall}... \pause
                \item Typically, modal formulas $\mathsf{K_i}\varphi$ expressing \emph{``agent $i$ \orange{knows that}~$\varphi$''}.
            \end{itemize} \pause
       \item Study other patterns of reasoning: 
        \begin{itemize}
            \item knowing why, 
            \item knowing whether, 
            \item knowing who, 
            \item \orange{knowing how}.
        \end{itemize}
    \end{itemize}
    
    \end{small}
        
    \end{frame}

%=============================================================

\begin{frame}{Knowing How}
    \begin{small}
    \begin{itemize}
        \item Autonomous \orange{agent}: intelligent entities operating in a given en\-vir\-on\-ment (perception, decision making, etc).
        \item Related to the abilities of the agents to achieve a certain goal.
        \item Inspired by \orange{AI planning}.
        \item Interpreted as: \emph{there exists a \blue{proper} course of action (sequence of actions) that the agent can take to achieve the goal.} \pause
            \begin{itemize}
                \item How is the agent's knowledge represented?
                \item What does ``proper'' means?
            %    \item Different courses of actions?
            \end{itemize}
    \end{itemize}
\end{small}
\end{frame}


%=============================================================

% \begin{frame}
%     {History of Knowing How Approaches}

%     \begin{small}
%         \begin{itemize}
%             \item Knowing that + Abilities \gray{(Lespérance et al. 2000), (Herzig \& Troquard 2006), etc}.
%             \item A single binary modality for knowing how $\kh(\psi,\varphi)$ and variants: \gray{(Wang 2015)}.
%             \item Knowing how + knowing that \gray{(Fervari et al. 2017)}.
%             \item Semantics based on indistinguishability/uncertainty between plans \gray{(Areces et al. 2021)}.
%         \end{itemize} \pause

%         \blue{We argue the latter enables us to develop a full theory of epistemic logic of knowing how.}
%     \end{small}
% \end{frame}

%=============================================================

\begin{frame}{Models: Labelled Transition Systems (\ltss)}
    \begin{small}
    \begin{block}{}
    An LTS is a tuple $\modlts = \tup{\W, \cset{\R_a}_{a \in \ACT}, \V}$ where:
        \begin{itemize} 
            \item $\W$ is a non empty countable set of states \quad\quad $\bullet$ $\V:\W \to 2^{\PROP}$
            \item $\R_a \subseteq \W \times \W$, for each $a \in \ACT$.
        \end{itemize}
    \end{block}
    \pause 
    \begin{ctabular}{c}
    \begin{tikzpicture}[frame rectangle]
    \node [state, label = {[label-state]left:$w_1$}] (w1) {$p$};
    \node [state, label = {[label-state]above:$w_2$}, above right = 0.5em and 3em of w1] (w2) {};
    \node [state, label = {[label-state]above:$w_3$}, right = 3em of w2] (w3) {$q$};
    \node [state, label = {[label-state]below:$w_4$}, below right = 0.5em and 3em of w1] (w4) {};
    
    \path (w1) edge node [label-edge, pos = 0.35, above] {$a$} (w2)
                edge node [label-edge, pos = 0.35, below] {$a$} (w4)
            (w2) edge node [label-edge, above] {$b$} (w3);
    \end{tikzpicture}\pause
    \end{ctabular}
    
    A transition $a$ from $w_1$ to $w_2$ is read as \blue{``after executing action $a$ at state $w_1$, the agent reaches state $w_2$''.}
    \pause 
    
    \medskip 
    For a set of actions $\ACT$, a \orange{plan} $\sigma$ is an element from $\ACT^*$ (finite sequences of symbols from $\ACT$, such as $a$, $ab$ and the empty plan $\epsilon$).
    
    
    \end{small}
    \end{frame}

%=============================================================

    \begin{frame}{Strong executability}
    
    A plan must be \orange{fail-proof}: each partial execution must be completed. \pause
    
    \renewcommand{\arraystretch}{1.6}
    \begin{ctabular}{cc}
    \begin{tabular}{@{\ \ \ \ \ \ \ }c}
        \begin{tikzpicture}[frame rectangle]
    \node [state, label = {[label-state]left:$w_1$}] (w1) {$p$};
    \node [state, label = {[label-state]above:$w_2$}, above right = 0.5em and 3em of w1] (w2) {};
    \node [state, label = {[label-state]above:$w_3$}, right = 3em of w2] (w3) {$q$};
    \node [state, label = {[label-state]below:$w_4$}, below right = 0.5em and 3em of w1] (w4) {};
    
    \path (w1) edge node [label-edge, pos = 0.35, above] {$a$} (w2)
                edge node [label-edge, pos = 0.35, below] {$a$} (w4)
            (w2) edge node [label-edge, above] {$b$} (w3);
        \end{tikzpicture}
    \end{tabular}
    &
    \begin{tabular}{l}
      \footnotesize  $ab$ is not SE at $w_1$
    \end{tabular}
    \end{ctabular}
\end{frame}

\begin{frame}{Strong executability}

    \begin{block}{\textbf{Definition:}}
        A plan $\sigma$ is \orange{strongly executable} (SE) at $u \in \W$ iff every partial execution of $\sigma$ from $u$ can be completed. \pause
        \\ We will use $\stexec(\sigma)$ to denote the set of nodes of an \lts where $\sigma$ is SE. 
        \\ Also, let $\pi$ be a set of plans, we will use $\stexec(\pi)$ to denote the set of nodes of an \lts where every 
        $\sigma\in\pi$ is SE.  
    \end{block}
    
\end{frame}

%=============================================================
%     \begin{frame}{\KHlogic over \lts}
%     \begin{small}
%     \begin{block}{Definition (Syntax of \KHlogic)}
%     \begin{nscenter}
%         $\varphi ::= p \mid \neg\varphi \mid \varphi\vee\varphi \mid
%         \kh(\varphi,\varphi)$
%     \end{nscenter}
%     $\kh(\psi,\varphi)$: \emph{\blue{``whenever $\psi$ holds, the agent knows how to achieve $\varphi$''}}.
%     \end{block}
    
%     \pause
    
%     \begin{block}{\textbf{Definition (\KHlogic over \lts)}}
%         $\modlts,w \models p$  \  iff  \ $w\in\V(p)$ \\ \pause 
%         \medskip
%         $\modlts,w \models \orange{\kh(\psi,\varphi)}$  \  iff  \ 
%         \orange{there exists} a plan $\sigma \in \ACT^*$ such that:
        
%         \begin{quote}
%             \begin{enumerate}
%             \item $\sigma$ is SE at \orange{every state} satisfying $\psi$; and, 
%             \item from every $\psi$-state, executing $\sigma$ \orange{always ends} at $\varphi$-states.
%             \end{enumerate}
%         \end{quote}
%     \end{block}
%     \end{small}
%     \end{frame}
    
%     %=============================================================
%     \begin{frame}{Example}
    
    
%     %\renewcommand{\arraystretch}{1.6}
%     \begin{ctabular}{cc}
%     \begin{tabular}{c}
%         \begin{tikzpicture}[frame rectangle,scale=.9]
%     \node [state, label = {[label-state]left:$w_1$}] (w1) {$p$};
%     \node [state, label = {[label-state]above:$w_2$}, above right = 0.5em and 3em of w1] (w2) {$r$};
%     \node [state, label = {[label-state]above:$w_3$}, right = 3em of w2] (w3) {$q$};
%     \node [state, label = {[label-state]below:$w_4$}, below right = 0.5em and 3em of w1] (w4) {$r$};
    
%     \path (w1) edge node [label-edge, pos = 0.35, above] {$a$} (w2)
%                 edge node [label-edge, pos = 0.35, below] {$a$} (w4)
%             (w2) edge node [label-edge, above] {$b$} (w3);
%         \end{tikzpicture}
%     \end{tabular}
%     &
%     \begin{scriptsize}
%     \begin{tabular}{l}
%       $\modlts,w_1\models\kh(p,r)$\\ \pause
%       \quad the plan \orange{$a$} is SE $w_1$ (the only $p$-state), and  \\  
%       \quad takes the agent from $\orange{p}$ only to $\orange{r}$-states. \\ \pause 
%       \\
%       $\modlts,w_1\not\models\kh(p,q)$\\ \pause  
%       \quad - \orange{$\epsilon$} and \orange{$a$}: are SE at $w_1$ ($p$-state),  \\
%       \quad but do not lead to $q$; \\ \pause 
%       \quad - \orange{$ab$} is {\bf not} SE at $w_1$. 
%       \end{tabular}
%     \end{scriptsize}
%     \end{ctabular}
    
%     \end{frame}
% %=============================================================
%     \begin{frame}{Conceptual analysis}
%         \begin{small}
%             \begin{itemize}
%                 \item The logic reached some consensus in the community.
%                 \item Simple language and semantics, and features nice properties (e.g. decidability, axiomatizability). 
%                 \item But look at these properties:
%                 \begin{itemize}
%                     \item \orange{$\kh(\psi,\chi) \land \kh(\chi,\varphi)$} implies \orange{$\kh(\psi,\varphi)$} \\ 
%                         \quad \emph{\scriptsize \dgray{Should the agent know how to compose plans?}} \pause
%                     \item If \orange{$\psi \ra \varphi$} holds \textbf{everywhere} then \orange{$\kh(\psi,\varphi)$} holds. \\
%                         \quad \emph{\scriptsize \dgray{Universal validities should imply knowing how?}} \pause
%                 \end{itemize}
%                 \item Moreover, in this setting, abilities = knowledge. \pause
%             \end{itemize}
%             \begin{block}{}
%             Arguably, this is a logic of knowing how, but not an \textbf{epistemic} logic of knowing how.
%             \end{block}
%         \end{small}
%     \end{frame}

% %=============================================================

\begin{frame}{Towards an epistemic logic of knowing how}
    \begin{small}
    There are many reasons to {\bf not} knowing how. What if...
    \begin{itemize}
    
    \item The agent is not aware of the existence of certain plans? \pause
    
    
     \item The agent is not able to \orange{distinguish} certain plan from another?\pause
    
     \item The agent \orange{does not care} about the difference among certain plans? \pause
    \end{itemize}
    
    A notion of \orange{epistemic indistinguishability} at the level of plans is introduced in order to address these issues \gray{(Areces et al. 2021)}.
    \end{small}  %\pause 
    
    % \smallskip
    
    % \begin{footnotesize}
    % \begin{block}{ }
    % {C. Areces, R. Fervari, A. Saravia, F. Vel\'azquez-Quesada. 
    
    % \emph{\dgray{Uncertainty-Based Semantics for Multi-Agent Knowing How Logics. (TARK 2021).}}}
    % \end{block}
    % \end{footnotesize}
    
    \end{frame}

% %=============================================================

%     \begin{frame}{Knowing How + Uncertainty}
%         \begin{columns}
        
%         \column{.4\textwidth}
        
        
%         \includegraphics[scale=.2]{all.jpg}
        
        
%         \column{.7\textwidth}
        
%         \begin{block} {} 
%         \scriptsize{{C. Areces, R. Fervari, A. Saravia, F. Vel\'azquez-Quesada. 
        
%         \emph{\dgray{Uncertainty-Based Semantics for Multi-Agent Knowing How Logics. (TARK 2021).}}}}
        
%         \end{block}
        
%         \end{columns}
        
%         \bigskip
        
%         \footnotesize
%         \blue{\url{https://cs.famaf.unc.edu.ar/~rfervari/files/papers/2021-tark.pdf}}      
%     \end{frame}
        
    %=============================================================
    
    \begin{frame}{Uncertainty-based \lts (\ults)}
    \begin{small}
    \begin{block}{}
    An \orange{$\ults$} is a tuple $\modults = \tup{\W, \ACT, \cset{\R_a}_{a \in \ACT}, \orange{\cset{\sim_i}_{i \in \AGT}}, \V}$ where:
    \begin{itemize}
        \item $\tup{\W, \cset{\R_a}_{a \in \ACT}, \V}$ is an \lts, \quad $\bullet$ $\ACT$ is a non empty set of actions,  
        \item \orange{$\sim_i$} is an \orange{equivalence relation} over a non-empty set of plans, for each $i\in\AGT$ (a set of \orange{agent} symbols).
    \end{itemize}
    \end{block}
    
    \bigskip
    
    \begin{nscenter}
    % \begin{tikzpicture}[frame rectangle]
    %     \node [state, label = {[label-state]left:$w$}] (w1) {$p$};
    %     \node [state, right = of w1] (w2) {$q$};
    %     \node [state, right = of w2] (w3) {$r$};
    %     \node [state, above = of w3] (w4) {};
    %     \path (w1) edge node [label-edge, below] {$a$} (w2)
    %                 edge node [label-edge, above] {$c$} (w4)
    %         (w2) edge node [label-edge, below] {$b$} (w3);
    %     \node [above right = -10pt and 3em of w3] {$ab \sim_ic$};
    
    %         \end{tikzpicture}
    \begin{tikzpicture}[frame rectangle]
        \node [state, label = {[label-state]left:$w_1$}] (w1) {$p$};
        \node [state, label = {[label-state]above:$w_2$}, above right = 0.5em and 3em of w1] (w2) {$r$};
        \node [state, label = {[label-state]above:$w_3$}, right = 3em of w2] (w3) {$q$};
        \node [state, label = {[label-state]below:$w_4$}, below right = 0.5em and 3em of w1] (w4) {$r$};
     \node [above right = -30pt and 2em of w3] {$a \sim_i ab$};

        % \node [below = 0.01em of s] (t) { \scriptsize{$a\sim_j a, \ ab \sim_j ab$  (\orange{$\Unc(j)=\{\{a\},\{ab\}\}$})}};
        
        \path (w1) edge node [label-edge, pos = 0.35, above] {$a$} (w2)
                    edge node [label-edge, pos = 0.35, below] {$a$} (w4)
                (w2) edge node [label-edge, above] {$b$} (w3);
            \end{tikzpicture}
    \end{nscenter}
    \pause

    We call \blue{$\Unc(i)$} the set of equivalence classes (over plans) by $\sim_i$. 

    \end{small}
    \end{frame}

%=============================================================

\begin{frame}{Semantics over \ults}

    \begin{small}
        \begin{block}{Definition (\KHilogic over \ults).}
            Formulas of the language $\KHilogic$ are given by
            \[
                \varphi ::= p \mid \neg\varphi \mid \varphi\vee\varphi \mid \khi(\varphi,\varphi),
            \]
            with $p \in \PROP$ and $i \in \AGT$. \pause $\khi(\psi,\varphi)$ is read as \defstyle{``agent $i$ knows how to achieve $\varphi$ given $\psi$''.} \pause
            \\ 
        \medskip
        $\modults,w \models \orange{\khi(\psi,\varphi)}$ iff there exists \orange{a set of plans} $\strategy \in \Unc(i)$ such that:

        \begin{quote}
            \begin{enumerate}
            \item \orange{each plan in $\strategy$} is SE at every $\psi$-state; and
            \item from $\psi$-states, \orange{each plan in $\strategy$} always ends at $\varphi$-states.
            \end{enumerate}
        \end{quote}
        \end{block}

        % \pause

        % \begin{block}{Property:}
        %     Define $\orange{\A\varphi} := \bigvee_{i\in\AGT}\khi(\neg\varphi,\bot)$, we have:
        %     \begin{nscenter}
        %         $\model,w\models\A\varphi \mbox{ iff for all $v$, } \model,v\models\varphi;$
        %     \end{nscenter}
        %     i.e., $\A$ is the standard \orange{universal} modality (and its dual: $\E\varphi:=\neg\A\neg\varphi$).
        % \end{block}

    \end{small}


\end{frame}

%=============================================================

\begin{frame}{Example}

\begin{ctabular}{cc}
\begin{tabular}{c}
    \begin{tikzpicture}[frame rectangle]
\node [state, label = {[label-state]left:$w_1$}] (w1) {$p$};
\node [state, label = {[label-state]above:$w_2$}, above right = 0.5em and 3em of w1] (w2) {$r$};
\node [state, label = {[label-state]above:$w_3$}, right = 3em of w2] (w3) {$q$};
\node [state, label = {[label-state]below:$w_4$}, below right = 0.5em and 3em of w1] (w4) {$r$};

\node [below = 2em of w4] (s) { \scriptsize{$a\sim_i ab$ (\orange{$\Unc(i)=\{\{a,ab\}\}$})}};
\node [below = 0.01em of s] (t) { \scriptsize{$a\sim_j a, \ ab \sim_j ab$  (\orange{$\Unc(j)=\{\{a\},\{ab\}\}$})}};

\path (w1) edge node [label-edge, pos = 0.35, above] {$a$} (w2)
            edge node [label-edge, pos = 0.35, below] {$a$} (w4)
        (w2) edge node [label-edge, above] {$b$} (w3);
    \end{tikzpicture}
\end{tabular}
&
\begin{scriptsize}
\begin{tabular}{l}
  \blue{$\model,w_1\models\neg\khi(p,r)$}\\ \pause
  \ \ take $\strategy = \{a,ab\}$: \\ \pause
  \quad - \orange{$a$} is SE at $w_1$ ($p$-state), and  \\  
  \quad takes from $\orange{p}$-states to $\orange{r}$-states. \\ \pause 
 % \\
  %$\modlts,w_1\not\models\kh(p,q)$\\ \pause  
  \quad  - \orange{$ab$} is not SE at $w_1$ ($p$-state). \\ \pause
  \quad  - thus, $\strategy=\{a,ab\}$ is not SE at $w_1$. \\ \pause 
  \\ 
  \blue{$\model,w_1\models\kh_j(p,r)$}\\ \pause
  \ \ take $\strategy' = \{a\}$: \\ % \pause
  \quad - \orange{$a$} is SE at $w_1$ ($p$-state), and  \\  
  \quad takes from $\orange{p}$-states to $\orange{r}$-states. \\ %\pause 
  \quad  - thus, $\strategy'=\{a\}$ works as a witness.
  \end{tabular}
\end{scriptsize}
\end{ctabular}


\end{frame}

%=============================================================
% \begin{frame}{Ontic vs. Epistemic Information}
%     \begin{small}
%         Summing up, \ultss comprise two kinds of information:\pause
%         \begin{itemize}
%         \item \orange{Ontic information}: provided by the graph part.  \\
%             \begin{itemize}
%             \item[$\blacktriangleright$] the available states, the accessibility relations, etc.\pause
%             \end{itemize}
%         \item \orange{Epistemic information}: given by the indistinguishability sets $\Unc(i)$.   \\ 
%             \begin{itemize} 
%             \item[$\blacktriangleright$] the perception of each agent about her own abilities.
%             \end{itemize}
%         \end{itemize}
        
%         % So, different kinds of updates have different meanings.
%     \end{small}
%     \end{frame}

%=============================================================

% \begin{frame}{Ontic Updates vs. Epistemic Updates}
%     \begin{small}
%         \begin{itemize}
%             \item Updating the LTS = update what an agent \orange{can} do.
%             % \item \orange{Public annoucenments}-like operators: modify the set of states or the actions in the model.
%             % \item Not expressible in the underlying language. \pause
%             \item Updating the relation $\sim_i$ (or the set $\Unc(i)$) = \orange{epistemic updates} (affecting the ``knowing how'').\pause
%             \item {\bf Proposal:} refining the \highlight{indistinguishability  between plans}, i.e., making plans distinguishable for the agent.
%         \end{itemize}
%     \end{small}
% \end{frame}

% %=============================================================

%     \begin{frame}
%     \frametitle{Epistemic updates: Refinement ($\Reflogic$)}
%     \begin{small}
%         \begin{definition}[$\Reflogic$ formulas]
%         \begin{nscenter}
%         $\varphi ::= p \mid \neg\varphi \mid \varphi\vee\varphi \mid
%         \khi(\varphi,\varphi)\mid \highlight{\refdiam{\sigma_1}{\sigma_2}\varphi}$
%         \end{nscenter}
%         \end{definition}\pause
        
%         $\refdiam{\sigma}{\sigma_2}\varphi$: \emph{``After it is stated that plans $\sigma_1$ and $\sigma_2$ are distinguishable, $\varphi$ holds.''}
%     \end{small}
%     \end{frame}

% %=============================================================

%     \begin{frame}
%         \frametitle{Epistemic updates: Refinement ($\Reflogic$) (cont.)}
%     \begin{block}{}
%         \centering
%     \scalebox{.8}{
%         \begin{tikzpicture}[frame rectangle]
%             \node [state, label = {[label-state]left:$w_1$}] (w1) {$p$};
%             \node [state, label = {[label-state]above:$w_2$}, above right = 0.5em and 3em of w1] (w2) {$r$};
%             \node [state, label = {[label-state]above:$w_3$}, right = 3em of w2] (w3) {$q$};
%             \node [state, label = {[label-state]below:$w_4$}, below right = 0.5em and 3em of w1] (w4) {$r$};
            
%             \path (w1) edge node [label-edge, pos = 0.35, above] {$a$} (w2)
%                         edge node [label-edge, pos = 0.35, below] {$a$} (w4)
%                     (w2) edge node [label-edge, above] {$b$} (w3);
%         \end{tikzpicture}
%     }
%     \medskip
        
%     \begin{nscenter}
%     $\Unc(i) = \cset{\cset{\mathit{a}},\cset{\mathit{b}}}$, \quad\quad 
%     \only<1-3>{
%     $\Unc(j) = \cset{\cset{\mathit{a},\mathit{b}}}$}
%     \only<4->{
%     $\Unc(j) = \defstyle{\cset{\cset{\mathit{a}},\cset{\mathit{b}}}}$}
%     \end{nscenter}
%     \end{block}
    
%     \begin{small}
%     \begin{itemize}
%     \item $\model,w \models \khi(p,r)$ and $\model,w\models \refdiam{a}{b}\khi(p,r)$.\pause 
%         \begin{itemize}
%             \item {\bf  preserves knowledge} \pause
%         \end{itemize}
%     \item  $\model,w \not\models \khj(p,r)$\pause\ but $\model,w\models \refdiam{a}{b}\khj(p,r)$;\pause 
%         \begin{itemize}
%             \item {\bf  generates new knowledge} 
%         \end{itemize}
%     \end{itemize}
%     \end{small}
    
%     \end{frame}

%     %=============================================================

%     \begin{frame}{Expressivity}
%         \begin{small}
%         \begin{block}{Property: }
%             $\Reflogic$ is \highlight{more expressive} than $\KHilogic$.
%         \end{block}
%         \pause 
        
%         {\bf Proof:} Let $\model$ and $\model'$ be the LTSs below, with $\Unc(i):=\cset{\cset{a}}$ and $\S'_i:=\cset{\cset{a,b}}$:
%             \begin{center}
%                 \begin{tikzpicture}[->]
%             \node[state] at (0,1) (p) [label=left:$w$]{$p$};
%             \node[left = of p] (m) {$\model$};
%             \node[state] at (1.75,0.6) (nq) {\phantom{$p$}};
%             \node[state] at (1.75,1.4) (q) {$q$};
%             \path (p) edge [above] node {$a$} (q);
%             \path (p) edge [below] node {$a$} (nq);
%             \end{tikzpicture}
%             \hspace{2cm}
%             \begin{tikzpicture}[->]
%                 \node[state] at (0,1) (p) [label=left:$w'$]{$p$};
%                 \node[state] at (1.75,0.6) (nq) {\phantom{$p$}};
%             \node[state] at (1.75,1.4) (q) {$q$};
%             \path (p) edge [above] node {$a$} (q);
%             \path (p) edge [below] node {$b$} (nq);
%             \node[right = 7em of p] (m) {$\model'$};
%             \end{tikzpicture}

%             \end{center}

%             \medskip

%             $\model,w \models \neg\refdiam{a}{b}\khi(p,q)$ while $\model',w \models \refdiam{a}{b}\khi(p,q)$.

%         % \pause

%         % \begin{block}{Corollary.}
%         %     There are $\Reflogic$-formulas that cannot be expressed in $\KHilogic$.
%         % \end{block}
%     \end{small}

%     \end{frame}

%     %=============================================================

%     \begin{frame}
%         {Uniform substitution}
%         \begin{small}
%         \orange{Uniform substitution} is an standard property in axiomatizing the logic.

%         If $\varphi$ is valid, then $\varphi[p/\psi]$ is also valid.

%         \pause
%         \begin{block}{Property: }
%             Uniform substitution fails in $\Reflogic$.
%         \end{block}

%         \pause

%         \begin{block}{Proof: } Take the formula $\varphi = p \rightarrow \refdiam{a}{b}p$, and the \ults $\model$ below, with $\Unc(i)=\cset{\cset{a,b}}$:

%             \begin{center}
%             \begin{tikzpicture}[->]
%                 \node[state] at (0,1) (p) [label=left:$w$]{$p$};
%                 \node[left = of p] (m) {$\model$};
%                 \node[state] at (1.75,0.6) (nq) {\phantom{$p$}};
%                 \node[state] at (1.75,1.4) (q) {$q$};
%                 \path (p) edge [above] node {$a$} (q);
%                 \path (p) edge [below] node {$b$} (nq);
%                 \end{tikzpicture} 
%             \end{center}
%         \pause 
%         Replace $\orange{p}$ by $\neg\khi(p,q)$: $\neg\khi(p,q) \rightarrow \refdiam{a}{b}\neg\khi(p,q)$ is not valid.
%         \end{block}
%     \end{small}
%     \end{frame}

%     %=============================================================

%     \begin{frame}
%         {Challenges}

%         \begin{small}
%             \begin{itemize}
%                 \item This kind of updates increase the expressive power:
%                 \begin{itemize}
%                     \item Failure of uniform substitution.
%                     \item No reduction axioms.
%                 \end{itemize}
%                 \item Quite challenging to obtain axiomatizations. \pause 
%                 \item (A) solution: extend the expressivity of the underlying \orange{static} language. 
%             \end{itemize}
%         \end{small}
%     \end{frame}

%     %=============================================================

%     \begin{frame}
%         {Proposal: A Knowing How Logic with Explicit Actions}
%         \begin{small}
%         \begin{definition}[$\KHiMLlogic$]
%             Formulas of the language $\KHiMLlogic$ are given by
%             \[
%                 \varphi ::= p \mid \neg\varphi \mid \varphi\vee\varphi \mid \khi(\varphi,\varphi) \mid [a]\varphi,
%             \]
%             with $p \in \PROP$, $i \in \AGT$ and $a \in \ACT$. Define: $\tup{a}\varphi := \neg[a]\neg\varphi$.
%             % As usual, we define the dual modality: $\tup{a}\varphi := \neg[a]\neg\varphi$. Formulas of the form $[a]\varphi$ are read as: \emph{``every execution of action $a$ leads always to situations in which $\varphi$ holds''};
%             % whereas  $\tup{a}\varphi$ stands for 
%             % \emph{``there exists a situation after executing action $a$ in which $\varphi$ holds''}.
%             \end{definition}

%             \begin{definition}
%                 $\model,w\models[a]\varphi$ iff for all $v$ s.t. $(w,v)\in \R_a$, $\model,v\models\varphi$.
%             \end{definition}
%         \end{small}
%     \end{frame}

%     %=============================================================

%     \begin{frame}
%         {Action Refinement}

%         \begin{small}
%         \begin{definition}[$\KHiMLReflogic$]
%             $\model,w\models\orange{[!a]}\varphi$ iff $\orange{\model^a},w\models\varphi$,

%             \medskip

%             \noindent where $a\in\ACT$ and $\model^a$ is as $\model$, except that for all $i\in\AGT$ we have:

%             \begin{itemize}
%                 \item $\S^a_i = (\Unc(i)\setminus\pi) \ \cup \ \{\{a\}\}$, if $a\in\pi$;
%                 \item $\S^a_i = \Unc(i)\ \cup \ \{\{a\}\}$, otherwise.
%             \end{itemize}
%         \end{definition}
%         \pause
%         Let $\model$ be such that $\Unc(i):=\cset{\cset{a,b}}$. 
%         \begin{center}
%         \begin{tikzpicture}[->]
%         \node[state] at (0,1) (p) [label=left:$w$]{$p$};
%         \node[left = of p] (m) {$\model$};
%         \node[state] at (1.75,0.6) (nq) {{$q$}};
%         \node[state] at (1.75,1.4) (q) {\phantom{$p$}};
%         \path (p) edge [above] node {$a$} (q);
%         \path (p) edge [below] node {$b$} (nq);
%         \end{tikzpicture} 
%         \end{center} 
%        $\model,w\not\models\khi(p,q)$ and $\model,w\models[!b]\khi(p,q)$.
%     \end{small}
%     \end{frame}

% %=============================================================

%     \begin{frame}
%         {Reduction Axioms}

%         \begin{small}
%         \begin{enumerate}
%             \item $\srefbox{a}p \lra p$ 
%             \item  $\srefbox{a} \neg \varphi_1 \lra \neg \srefbox{a} \varphi_1$
%             \item $\srefbox{a} (\varphi_1 \vee \varphi_2) \lra (\srefbox{a} \varphi_1 \vee \srefbox{a} \varphi_2)$ 
%             \item $\srefbox{a}[a]\varphi_1 \lra [a]\srefbox{a}\varphi_1$ \pause 
%             \item $\orange{\srefbox{a}} \khi(\varphi_1,\varphi_2) \lra (\khi(\orange{\srefbox{a}} \varphi_1,\orange{\srefbox{a}} \varphi_2) \vee$ \\ 
%             \quad\quad\quad\quad\quad\quad\quad\quad \  $\blue{\A}(\orange{\srefbox{a}} \varphi_1 \ra (\blue{\tup{a}\top} \wedge [a] \orange{\srefbox{a}}\varphi_2)))$
%         \end{enumerate} \pause

%         \medskip 

%         Via reduction axioms, we can \orange{eliminate all the occurrences of a [!a] modality} (i.e., embed $\KHiMLReflogic$ into $\KHiMLlogic$).
%     \end{small}

%     \end{frame}

% %=============================================================

%     \begin{frame}
%         {Example}
%         \begin{small}

%         Let $\model$ be s.t. $\Unc(i):=\cset{\cset{a,b}}$, $\model,w\not\models\khi(p,q)$ and $\model,w\models\defstyle{[!b]\khi(p,q)}$.
        
%         \begin{center}
%         \begin{tikzpicture}[->]
%         \node[state] at (0,1) (p) [label=left:$w$]{$p$};
%         \node[left = of p] (m) {$\model$};
%         \node[state] at (1.75,0.6) (nq) {{$q$}};
%         \node[state] at (1.75,1.4) (q) {\phantom{$p$}};
%         \path (p) edge [above] node {$a$} (q);
%         \path (p) edge [below] node {$b$} (nq);
%         \end{tikzpicture} 
%         \end{center} 
        
%         \begin{tabular}{llr}
%         & $\model,w\models\orange{[!b]}\khi(p,q)$ & {\bf (5)}\\ \pause 
%         iff & $\model,w\models\khi(\orange{[!b]p},\orange{[!b]q}) \vee \blue{\A}(\orange{[!b]p} \ra (\blue{\tup{b}\top} \wedge [b]\orange{[!b]q}))$ & {\bf (1)} \\ \pause
%         iff & $\model,w\models\khi(p,q) \vee \A(p \ra (\tup{b}\top \wedge [b]q))$  
%         \end{tabular}
%     \end{small}
%     \end{frame}

% %=============================================================
% \begin{frame}
%     {Axiomatization}
%     \begin{scriptsize}
%     \begin{table}
%         $$
%         \begin{array}{l@{\quad \quad  }l@{\quad}l}
%         \hline
%         \mbox{Axioms}
%         & \axm{Taut}  & \vdash \varphi \mbox{ for $\varphi$ a propositional tautology} \\
%         & \axm{DistA} & \vdash \A(\varphi\ra\psi) \ra (\A\varphi \ra \A\psi) \\
%         & \axm{TA}    & \vdash \A\varphi \ra \varphi \\
%         & \axm{Dist$\square$} & \vdash [a](\varphi\ra\psi) \ra ([a]\varphi \ra [a]\psi) \\
%         & \axm{A$\square$} & \vdash \A\varphi \ra [a]\varphi \\
%         \hline
%         & \axm{4KhA}  & \vdash \khi(\psi,\varphi) \ra \A\khi(\psi,\varphi) \\
%         & \axm{5KhA}  & \vdash \neg\khi(\psi,\varphi) \ra \A\neg\khi(\psi,\varphi) \\
%         & \axm{KhE}   & \vdash \left(\E\psi \land \khi(\psi,\varphi)\right) \rightarrow \E\varphi \\
%         & \axm{KhA}   & \vdash \left(\A(\chi \rightarrow \psi) \land \khi(\psi,\varphi) \land \A(\varphi \rightarrow \theta)\right) \rightarrow \khi(\chi, \theta) \\
%         \hline
%         \mbox{Rules}
%         &  \axm{MP}   & \mbox{From $\vdash \varphi$ and $\vdash \varphi \rightarrow \psi$ infer $\vdash \psi$ }\\
%         &  \axm{NecA} & \mbox{From $\vdash \varphi$ infer $\vdash \A\varphi$} \\
%         \hline
%         \end{array}
%         $$
%        % \caption{Axiomatization $\axset_{\khi,\square}$ for $\KHiMLlogic$ w.r.t.\ $\ultss$.}\label{tab:khimlaxiom}
%         \end{table}

%         \begin{theorem}
%             \begin{enumerate}
%                \item The axiom system above is sound and complete for \KHiMLlogic.  \pause
%                \item The axiom system above + the reduction axioms is sound and complete for \KHiMLReflogic.
%             \end{enumerate}
%         \end{theorem}
    
%     \end{scriptsize}
% \end{frame}

%=============================================================

\begin{frame}
    {Bisimulation for \KHilogic}

    \begin{definition}\label{def:notation}
Let $\model=\tup{\W,\ACT,\R,\Unc,\V}$ be an \ults over \PROP and \AGT. Take $\pi \in \pow(\ACT^*)$, $X, T \subseteq \W$ and $i \in \AGT$. \pause
\begin{itemize} \itemsep 0pt
    \item Write $X \ultsExecStrat{\pi} T$ $\siffdefs$ $X \subseteq \stexec(\pi)$ and $\R_{\pi}(X) \subseteq T$. \pause

    \item Write $X \ultsExecAgi T$ $\siffdefs$ there is $\pi \in \Unc(i)$ such that $X \ultsExecStrat{\pi} T$. \pause
\end{itemize}
Additionally, $X \subseteq \W$ is propositionally definable in $\model$ iff there is a propositional formula $\varphi$ such that $X = \{ x \in \W \mid \model, x \models \varphi\}$.
\end{definition}

\pause
Now we introduce the notion of bisimulation for $\KHilogic$. 

\end{frame}

%=============================================================

\begin{frame}{Bisimulation for \KHilogic}
    \begin{definition}[$\KHilogic$-bisimulation]\label{def:bisim-khi}
    \begin{footnotesize}
    Let $\model = \tup{\W,\ACT,\R,\Unc,\V}$ and $\model' = \tup{\W',\ACT',\R',\Unc',\V'}$ be $\ultss$. 
    A non-empty $Z \subseteq \W \times \W'$ is called an \defstyle{$\KHilogic$-bisimulation} between $\model$ and $\model'$ iff $sZs'$ implies all of the following.
    \pause
    \begin{description} 
        \item[(Atom):] $\V(s)=\V'(s')$. \pause

        \item[($\khi$-zig):] for any \emph{propositionally} definable $P \subseteq \W$, if $P \ultsExecAgi T$ for some $T \subseteq \W$, then there is $T' \subseteq \W'$ such that: 
            1) $Z(P) \ultsExecAgi T'$, and
            2) $T' \subseteq Z(T)$. \pause

        \item[($\khi$-zag):] % analogous to \textbf{$\khi$-Zig}.
        for any \emph{propositionally} definable $P' \subseteq \W'$, if $P' \ultsExecAgi T'$ for some $T' \subseteq \W'$, then there is $T \subseteq \W$ such that: 
            1) $Z^{-1}(P') \ultsExecAgi T$, and
            2) $T \subseteq Z^{-1}(T')$. \pause

        \item[($\A$-zig):] for all $t\in\W$ there is a $t'\in\W'$ such that $tZt'$. \pause

        \item[($\A$-zag):] for all $t'\in\W'$ there is a $t\in\W$ such that $tZt'$. \pause
    \end{description}
    We write $\model \bisim \model'$ when there is an
    $\KHilogic$-bisimulation $Z$ between $\model$ and $\model'$. \\Also, we write $\model,s \bisim \model',s'$ if
    $sZs'$.
    \end{footnotesize}
    \end{definition}
\end{frame}

%=============================================================

\begin{frame}{Bisimulation for \KHilogic}
The following theorem establishes a classical adequacy result. \pause

\begin{theorem}[\gray{Areces et al. 2021}]
\label{th:adequacy}
Let $\model,s$ and $\model',s'$ be two \ultss. $\model,s\bisim\model',s'$ implies $\model,s\models\varphi$ iff $\model',s'\models\varphi$, for all $\KHilogic$-formulas $\varphi$. Moreover, if $\model$ and $\model'$ are finite domain, the converse also holds.
\end{theorem}
\end{frame}


%=============================================================


\begin{frame}{Bisimulation for \KHilogic}
    There are some aspects about this definition that can be target of criticism. \pause
    
    \begin{description}
        \item[($\khi$-zig):] for any \red{\emph{propositionally} definable} $P \subseteq \W$, if $P \ultsExecAgi T$ for some $T \subseteq \W$, then there is $T' \subseteq \W'$ such that: 
            1) $Z(P) \ultsExecAgi T'$, and
            2) $T' \subseteq Z(T)$.

        \item[($\khi$-zag):] % analogous to \textbf{$\khi$-Zig}.
        for any \red{\emph{propositionally} definable} $P' \subseteq \W'$, if $P' \ultsExecAgi T'$ for some $T' \subseteq \W'$, then there is $T \subseteq \W$ such that: 
            1) $Z^{-1}(P') \ultsExecAgi T$, and
            2) $T \subseteq Z^{-1}(T')$.
    \end{description}

    % \begin{itemize}
    %     \item It is asking for the existence of a purely syntactic object such as a propositional formula.
    %     \item Makes it difficult to operate computationally over bisimulations.
    % \end{itemize}

\end{frame}

%=============================================================

\begin{frame}{Redefining \KHilogic-bisimulation}
    In order to address these issues, we introduce a new definition for \KHilogic-bisimulation. \pause

    % First, we characterize the propositionally definable sets of an \ults.

    First, we will define a way of grouping the nodes of an \ults by only considering their valuation function $\V$.\pause

    \vspace{3mm}

    \begin{definition}
        \begin{small}
        \begin{itemize}
            \item Let $\model=\tup{\W,\ACT,\R,\Unc,\V}$ be an \ults, we define the relation \\\quad\quad\quad\quad$\quotient{\model}\eqdef\set{(s,t) \subseteq \W\times\W\mid \V(s)=\V(t)}$.\pause
            \item Let $[s]\eqdef \set{t\in\W\mid (s,t)\in\quotient{\model}}$, we define $\qpart{\model}\eqdef\set{[s]\mid s\in\W}$.
        \end{itemize}
        \end{small}
    \end{definition}
\end{frame}

%=============================================================

\begin{frame}{Redefining \KHilogic-bisimulation}
    Now, we will characterize the propositionally definable sets of an \ults. \pause

    \vspace{5mm}


    \begin{lemma}
        \begin{small}
        Let $\model=\tup{\W,\ACT,\R,\Unc,\V}$ be an \ults, and let $P\subseteq \W$ be a propositionally definable set. Then, for all $[s]\in\qpart{\model}$, either $[s]\cap P =\emptyset$ or $[s]\subseteq P$. 
        \\If $\model$ is finite domain, then the converse also holds.
        \end{small}
    \end{lemma} \pause
    This lemma motivates our new proposal for the ($\khi$-zig) and ($\khi$-zag) conditions.
\end{frame}

%=============================================================

\begin{frame}{Redefining \KHilogic-bisimulation}

    \begin{footnotesize}
    \begin{definition}[$\KHilogic^*$-bisimulation]
        Let $\model = \tup{\W,\ACT,\R,\Unc,\V}$ and $\model' = \tup{\W',\ACT',\R',\Unc',\V'}$ be $\ultss$. 
        A non-empty $Z \subseteq \W \times \W'$ is called an \defstyle{$\KHilogic^*$-bisimulation} between $\model$ and $\model'$ iff $sZs'$ implies (Atom), ($\A$-zig), ($\A$-zag) and:
        \pause
        \begin{description} 
            \item[($\khi$-zig*):] 
            for all $P\subseteq\W$ such that {\color{NavyBlue}{for any $[t]\in\qpart{\model}$ either 
            $[t]\cap P =\emptyset$ or $[t]\subseteq P$}}, then, 
            if $P \ultsExecAgi T$ for some $T \subseteq \W$, then there is $T' \subseteq \W'$ such that: 
                1) $Z(P) \ultsExecAgi T'$, and
                2) $T' \subseteq Z(T)$.\pause
                
            \item[($\khi$-zag*):] for all $P'\subseteq\W'$ such that {\color{NavyBlue}for any $[t']\in\qpart{\model'}$ either 
            $[t]'\cap P' =\emptyset$ or $[t']\subseteq P'$}, then, 
            if $P' \ultsExecAgi T'$ for some $T' \subseteq \W'$, then there is $T \subseteq \W$ such that: 
                1) $Z^{-1}(P') \ultsExecAgi T$, and
                2) $T \subseteq Z^{-1}(T')$.\pause

        \end{description}
        We write $\model \nbisim \model'$ when there is an
        $\KHilogic^*$-bisimulation $Z$ between $\model$ and $\model'$. \\Also, we write $\model,s \nbisim \model',s'$ if
        $sZs'$.
    \end{definition}
\end{footnotesize}
\end{frame}

%=============================================================

\begin{frame}{Redefining \KHilogic-bisimulation}
\begin{center}
    \begin{scriptsize}
    \begin{tikzpicture}[frame rectangle]
            \node[state] (p) {$p$};
            \node[left of=p, xshift=-0.75cm] (m) {$\model$};
            \node[state, right of=p, xshift=1cm] (p2) {$p$};
            \node[state, right of=p2, xshift=1cm] (q) {$q$};
            \node[state, right of=q, xshift=1cm] (q2) {$q$};
            
            \node[state, below of=p, yshift=-0.7cm] (p') {$p$};
            \node[left of=p', xshift=-0.75cm, yshift=-0.4cm] (m') {$\model'$};
            \node[state, below of=p', yshift=-0.8cm] (p2') {$p$};
            \node[state, right of=p', yshift=-0.6cm, xshift=1.3cm] (q') {$q$};
            \node[state, right of=q', xshift=1.3cm] (q2') {$q$};
            
            
            \node at ($(p)+(0,0.5)$) {$s_1$};
            \node at ($(p2)+(0,0.5)$) {$s_2$};
            \node at ($(q)+(0,0.5)$) {$s_3$};
            \node at ($(q2)+(0,0.5)$) {$s_4$};
            
            \node at ($(p')+(0.05,-0.6)$) {$s_2'$};
            \node at ($(p2')+(0.05,-0.6)$) {$s_1'$};
            \node at ($(q')+(0.05,-0.6)$) {$s_3'$};
            \node at ($(q2')+(0.05,-0.6)$) {$s_4'$};
            
            
            \path (p) edge[->] node [above] {$a$} (p2);
            \path (p2) edge[->] node [above] {$a$} (q);
            \path (q) edge[->] node [above] {$a$} (q2);
            \path (p') edge[->] node [above] {$d$} (q');
            \path (p2') edge[->] node [above] {$d$} (q');
            \path (q') edge[->] node [above] {$e$} (q2');

            \path[dashed,red] (p) edge [bend right=35] node [above] {} (p2');
            \path[dashed,red] (p2) edge node [above] {} (p');
            \path[dashed,red] (q) edge node [above] {} (q');
            \path[dashed,red] (q2) edge node [above] {} (q2');
            
        \end{tikzpicture}
    \end{scriptsize}
\end{center}
\begin{footnotesize}
Consider $\AGT = \{i\}$, and $\Unc(i) = \{\{aa\}\}$ and $\Unc'(i) = \{\{de\}, \{d\}\}$ the indistinguishability relations 
over plans for the agent $i$ in each model.
\end{footnotesize}
\end{frame}
        
%=============================================================

\begin{frame}{Redefining \KHilogic-bisimulation}

    Now, we would want our new definition to satisfy the previously introduced adequacy results. \pause

    \begin{lemma}
        If a relation $Z$ is an $\KHilogic^*$-bisimulation between two \ults 
        $\model$ and $\model'$, then it is an $\KHilogic$-bisimulation between them. 
        If $\model$ and $\model'$ are finite domain, then the converse also holds.
    \end{lemma}\pause

    An important observation is that in the infinite case the converse does \red{not} hold.

    This can be proven with a cardinality argument.
\end{frame}

%=============================================================

\begin{frame}{Redefining \KHilogic-bisimulation}
    As a consequence of that lemma, we can now prove the corresponding results for our new definition. \pause 
    
    \begin{theorem}
    Let $\model,s$ and $\model',s'$ be two \ultss. $\model,s\nbisim\model',s'$ implies $\model,s\models\varphi$ iff $\model',s'\models\varphi$, for all $\KHilogic$-formulas $\varphi$. If $\model$ and $\model'$ are finite domain, the converse also holds.
    \end{theorem}\pause

    This theorem supports our belief that this proposal is adequate. 

\end{frame}


%=============================================================

% Comentario hablado, now that we gave a more algoritmical nature to the notion of bisimulation, we will study some
% computational problems regarding this concept. 


\begin{frame}{Computational aspects of \KHilogic-bisimulation}
    We will study the problem of deciding whether there exists an \KHilogic-bisimulation between two \ultss. \pause
    
    \begin{definition}
        \begin{small}
        Define the problem $\khibisimpb$ as: 
            \[ 
            \khibisimpb \eqdef \set{\tup{\model,\model'} \mid \model,\model' \text{ are finite \ults  and }  \model \bisim\model'}.
            \]
        \end{small}
    \end{definition}\pause

    This problem can be thought as a ``search and verification'' problem.
\end{frame}

%=============================================================

\begin{frame}{Computational aspects of \KHilogic-bisimulation}

    We will first characterize the complexity of deciding if a binary relation is an \KHilogic-bisimulation between two \ultss. \pause

    \begin{definition}
    \label{def:checkkhibisimpb}
        \begin{small}        
            Define the problem $\checkkhibisimpb$ as: 
            \[ 
            \begin{array}{ll}
            \checkkhibisimpb \eqdef &  \{\tup{\model,\model',Z} \mid \text{ Z is an $\KHilogic$-bisimulation} \\ 
            & \ \text{ between }  \model \text{ and } \model'\}.
            \end{array}
            \]
        \end{small}
    \end{definition} \pause

    The conditions \blue{($\khi$-zig)} and \blue{($\khi$-zag)} suggest that this problem may be a good fit in the class $\coNP$.

\end{frame}

%=============================================================

\begin{frame}{Computational aspects of \KHilogic-bisimulation}
    To prove the upper bound it is enough to provide a polynomial non-deterministic algorithm that decides its complement. \pause
    
    For every ``no-instance'' there exist a counterexample $x$ of polynomial length that can be efficiently verified. \pause

    The counterexamples that we will consider are subsets of nodes of the two models, and the algorithm 
    will check if the ($\khi$-zig) or ($\khi$-zag) conditions are satisfied in that particular subset.\pause

    \begin{lemma}
        $\checkkhibisimpb \in \coNP$. 
    \end{lemma}
    
\end{frame}


%=============================================================

\begin{frame}{Computational aspects of \KHilogic-bisimulation}
    Now, we want to provide a lower-bound for the computational complexity of $\checkkhibisimpb$.\pause

    To do that we will reduce the following $\coNPComplete$ problem:\pause

    \begin{definition}
        \[ 
            \dnftautpb \eqdef \set{\varphi \mid \varphi \text{ is a propositional tautology in DNF}}.
        \]
    \end{definition}
\end{frame}

%=============================================================

\begin{frame}{Computational aspects of \KHilogic-bisimulation}
    We define a polynomial reduction that maps a propositional formula $\varphi$ to a tuple $\tup{\model,\model',Z}$ 
    in a way that:\pause 
    \begin{center}
        $\varphi\in\dnftautpb$ iff $\tup{\model,\model',Z}\in\checkkhibisimpb$.
    \end{center}
\end{frame}

%=============================================================
\begin{frame}{Computational aspects of \KHilogic-bisimulation}
    \begin{small}
    \begin{center}
        $\varphi = (p_1 \wedge \neg p_2) \vee (p_2) \vee (\neg p_1)$  
    \end{center}
    \end{small}\pause
    \begin{figure}[t]
        \begin{small}
        \begin{center}
        \begin{tikzpicture}
            % Nodes at rotated positions
            \node[state] (phi1) at (0,0) {$q_3$};
            \node[state] (phi2) at (0,1.5) {$q_4$};
            \node[state] (phi3) at (0,3) {$q_5$};
            
            \node[state] (p1) at (2.6,0.5) {$q_1$};
            \node[state] (p2) at (2.6,1.5) {$q_2$};
            
            \node[state] (e) at (5.2,1.5) {$q_6$};

            % Labels
            \node at ($(phi1)+(-0.5,0)$) {$t_1$};
            \node at ($(phi2)+(-0.5,0)$) {$t_2$};
            \node at ($(phi3)+(-0.5,0)$) {$t_3$};
            \node at ($(e)+(0.4,0)$) {$e$};
            
            \node at ($(p1)+(0.3,-0.3)$)  {$s_1$};
            \node at ($(p2)+(0.3,-0.3)$) {$s_2$};

            % Edges
            \path (p1) edge[->] node [above, inner sep=2pt] {\footnotesize $in_{\varphi_1}$} (phi1);
            \path (p1) edge[->] node [above, inner sep=1pt] {\footnotesize $in_{\varphi_2}$} (phi2);
            \path (p2) edge[->] node [above, inner sep=1pt] {\footnotesize $in_{\varphi_2}$} (phi2);
            \path (p2) edge[->, bend right=20] node [above, inner sep=1pt] {\footnotesize $in_{\varphi_3}$} (phi3);

            \path (phi1) edge[->, bend right=45] node [below, inner sep=1pt] {\footnotesize $out_{\varphi_1}$} (p1);
            \path (p1) edge [loop below] node [below] {\footnotesize $test$} (p1);
            \path (phi2) edge[->, bend left=40] node [above, inner sep=1pt] {\footnotesize $out_{\varphi_2}$} (p2);
            \path (p2) edge [loop above] node [above] {\footnotesize $test$} (p2);
            \path (phi3) edge[->, bend left=50] node [above] {\footnotesize $out_{\varphi_3}$} (e);
            \path (p1) edge[->, bend right] node [below] {\footnotesize $test$} (e);
            \path (p2) edge[->, bend left] node [above] {\footnotesize $test$} (e);
        \end{tikzpicture} \\
            \smallskip 

                    $\begin{array}{ll}
                        \Unc(agt) = &
                         \left\{\{in_{\varphi_1}out_{\varphi_1}\},
                        \{in_{\varphi_2}out_{\varphi_2}\},
                        \{in_{\varphi_3}out_{\varphi_3}\},
                        \{test\} \right\}\\
                        \Unc'(agt) = & 
                        \left\{\{in_{\varphi_1}out_{\varphi_1}\},
                        \{in_{\varphi_2}out_{\varphi_2}\},
                        \{in_{\varphi_3}out_{\varphi_3}\} \right\}
                    \end{array}$
    \end{center}
    \end{small}
    \end{figure}
\end{frame}

%=============================================================

\begin{frame}{Computational aspects of \KHilogic-bisimulation}
    \begin{lemma}
        $\checkkhibisimpb$ is $\coNPHard$.
    \end{lemma}\pause

    \vspace{5mm}

    Now, we can present the desired complexity result for $\checkkhibisimpb$.\pause

    \vspace{5mm}

    \begin{theorem}
        $\checkkhibisimpb$ is $\coNPComplete$.
    \end{theorem}
\end{frame}

%=============================================================

\begin{frame}{Computational aspects of \KHilogic-bisimulation}
    Going back to $\khibisimpb$, we characterized the complexity of the ``verification'' subproblem.\pause

    The ``search'' subproblem seems computationally expensive. There are an exponential number of binary relations to consider. \pause

    Remarkably, the ``search'' part is not actually needed, as we can use a particular relation that works universally. 
\end{frame}


%=============================================================

\begin{frame}{Computational aspects of \KHilogic-bisimulation}
    \begin{lemma}
        Let $\model$ and $\model'$ be \ults such that $\tup{\W,\ACT,\R,\Unc,\V}$ and $\model' = \tup{\W',\ACT',\R',\Unc',\V'}$.
        % $s\in\S$ and $s'\in\S'$. 
        $\model\bisim\model'$ implies that $\universalbis\eqdef\set{(t,t')\in\W\times\W'\mid \V(t)=\V'(t')}$ is an $\KHilogic$-bisimulation between $\model$ and $\model'$. 
    \end{lemma}\pause

    This lemma states that deciding the existence of a \KHilogic-bisimulation reduces to verifying if $\universalbis$ is a \KHilogic-bisimulation.\pause

    \begin{theorem}
        $\khibisimpb$ is $\coNPComplete$.
    \end{theorem}
\end{frame}

%=============================================================

\begin{frame}{Computational aspects of \KHilogic-bisimulation}
    We can also extend this results for the pointed version of the problem.\pause

    \begin{corollary}
        Deciding $\pkhibisimpb$ is $\coNPComplete$, where 
        \[ 
            \pkhibisimpb \eqdef \set{\tup{\model,s,\model',s'} \mid \model,s \nbisim\model',s'}.
        \]
    \end{corollary}
\end{frame}


%=============================================================


\begin{frame}
    {Final remarks}

    \begin{small}
        \begin{itemize}
            \item We presented a new definition for \KHilogic-bisimulation that requires only purely structural conditions, and proved the desired adequacy results. \pause
            \item We established that verification/existence of \KHilogic-bisimulation between two \ultss is $\coNPComplete$. \pause
            \item Also, that is the case for the pointed version of the problem. \pause
            \item We proposed two computationally efficient approaches for bisimulation-based model contraction. \pause
            \item Future work: extend this results to other knowing how logics, and also to the version of this logic interpreted 
            over \ults extended with finite automata.
        \end{itemize}
    \end{small}
\end{frame}

%=============================================================
\end{document}
