\chapter{\KHilogic-Bisimulación entre dos modelos}

Como mencionamos en el anterior capítulo, a la hora de definir una noción de bisimulación para una lógica modal, 
es deseable que la misma tenga una naturaleza procedural, es decir, que naturalmente exista un procedimiento que dados dos
modelos y una relación binaria entre ellos pueda decidir si dicha relación es una bisimulación.

Esta naturaleza procedural para verificar si una relación entre dos modelos es una bisimulación,
también nos permite encontrar un procedimiento efectivo que decida si existe una bisimulación entre dos modelos.

De esta discusión surge la pregunta que se encara a lo largo de este capítulo. ¿Cuál es la dificultad intrínseca 
de los problemas mencionados?.

\section{$\KHiBisim$}

\begin{definicion}
    Definimos el problema de decidir si existe una \KHilogic-bisimulación entre dos modelos como:
    \begin{center}
        $\KHiBisim := \{\tup{\modults_1,\modults_2} \mid$ existe una \KHilogic-bisimulación
        entre $\modults_1$ y $\modults_2\}$
    \end{center}
\end{definicion}

Una vez definido matemáticamente el problema, un primer análisis intuitivo que surge acerca
del mismo es que tiene una naturaleza, a priori, de `búsqueda+verificación', pues
para decidir si existe una bisimulación entre dos modelos, no parece haber una mejor
opción que buscar una relación binaria candidata y verificar si efectivamente es una bisimulación. 

A su vez, esta `búsqueda+verificación' parece ser costosa, pues hay una cantidad exponencial de relaciones binarias
entre dos modelos, y verificar las condiciones ($\khi$-zig) y ($\khi$-zag) llevan consigo una cantidad exponencial
de conjuntos a considerar.

Detalles reunión 17/07, nos dimos cuenta que capaz una mejor forma de presentar la complejidad del problema es primero caracterizar
la complejidad de la `verificación', es decir el problema de dados dos modelos y una relación binaria entre ellos decidir si dicha relación es
una bisimulación.

Una vez caracterizada la complejidad de ese problema (que modificando un poco la demo que está ahora se puede probar que es $\coNP$-completo), 
ahi presentamos el lema que caracteriza la existencia de bisimulación entre dos modelos que trivializa la parte de `búsqueda' del problema, 
y de ahí obtenemos que el problema es $\coNP$-completo retocando un poco la prueba de la verificación.

Creo que contar primero el problema $\KHiBisim$ y después analizar qué tan complicada es la `verificación' y después qué tan complicada es la `búsqueda' 
para terminar caracterizando la complejidad del problema es un buen hilo para contar la historia. Y después en la siguiente sección probar la equivalencia 
con el caso punteado.


(Texto introduciendo caracterización de existencia de bisimulación)


\begin{lema}
    Sean $\modults_1$ y $\modults_2$ dos \ultss tales que 
    $\model_1=\tup{\W,\R,\cset{\S_i}_{i \in \AGT},\V,\ACT}$ y 
    $\model_2=\tup{\W',\R',\cset{\S_i'}_{i \in \AGT},\V',\ACT'}$.
    Entonces, existe $Z \subseteq \W \times \W'$ \KHilogic-bisimulación entre ellos si y sólo si

    \begin{center}
        $Z' = \{(w,w') \in \W \times \W' \mid \V(w) = \V'(w')\}$
    \end{center}
    es una \KHilogic-bisimulación entre $\modults_1$ y $\modults_2$.
\end{lema}

Notemos que este lema nos dice que para decidir si existe una \KHilogic-bisimulación
entre dos modelos $\modults_1$ y $\modults_2$ basta con verificar que $Z'$ es una \KHilogic-bisimulación.

\begin{demostracion}

    Notar que la dirección ($\leftarrow$) es directa. Demostraremos entonces ($\rightarrow$).

    Queremos ver que $Z'$ es una \KHilogic-bisimulación entre $\modults_1$ y $\modults_2$,
    dado que existe $Z \subseteq \W \times \W'$, \KHilogic-bisimulación entre ellos.

    Notemos que como $Z$ es una bisimulación, entonces cumple (Atom). Luego, notemos que
    por la definición de $Z'$, $Z \subseteq Z'$, pues $Z'$ contiene todos los pares de
    $\W \times \W'$ que satisfacen (Atom). Esto nos dice que $Z'$ satisface (A-zig) y (A-zag).
    A su vez, cómo mencionamos, $Z'$ contiene todos los pares que satisfacen (Atom).

    Demostremos entonces que $Z'$ cumple ($\khi$-zig) y ($\khi$-zag).

    \begin{itemize}
        \item ($\khi$-zig) Sean $U,T \subseteq \W$ tales que para cada $s \in \rho_\modults$ se cumple que $s \subseteq U$ 
        o $s \cap U = \emptyset$, y $U \ultsExecAgi T$ queremos ver que existe $T' \subseteq \W'$ tal que:
    
        \begin{multicols}{2}
            \begin{itemize}
                \item $Z'(U) \ultsExecAgi T'$, 
                \item $T' \subseteq Z'(T)$.
            \end{itemize}
        \end{multicols}
    
        Notemos que como $Z$ es una bisimulación, entonces existe $T'' \subseteq \W'$ tal que:
    
        \begin{multicols}{2}
            \begin{itemize}
                \item $Z(U) \ultsExecAgi T''$, 
                \item $T'' \subseteq Z(T)$.
            \end{itemize}
        \end{multicols}
    
        Demostremos que $T' := T''$ cumple con lo mencionado. Es claro que como $Z \subseteq Z'$ y 
        $T'' \subseteq Z(T)$ entonces $T'' \subseteq Z'(T)$. Entonces, nos queda demostrar que $Z'(U) \ultsExecAgi T''$.
    
        Esto lo demostraremos analizando que $Z(U) = Z'(U)$. Nuevamente, como $Z \subseteq Z'$
        entonces $Z(U) \subseteq Z'(U)$. Luego, solo queda demostrar que $Z'(U) \subseteq Z(U)$.
    
        Sea $w' \in Z'(U)$, entonces existe $w \in U$ tal que $(w,w') \in Z'$, por lo que $\V(w) = \V'(w')$. 
        Como $Z$ cumple (A-zag), existe $v \in \W$ tal que $(v,w') \in Z$, y cómo $Z$ cumple 
        (Atom) entonces $\V(v) = \V'(w')$. Luego $\V(w) = \V(v)$, lo que nos dice que $v \in [w]$.
    
        Como $w \in U$, entonces $[w] \subseteq U$, por lo que $v \in U$. Luego, notar que como $(v,w') \in Z$, entonces
        $w' \in Z(U)$.
    
        Lo cuál demuestra que $Z(U) = Z'(U)$. Por lo que $Z'(U) \ultsExecAgi T''$. Finalmente, concluimos 
        que $Z'$ cumple ($\khi$-zig).
    
    
        \item ($\khi$-zag) Análogo a ($\khi$-zig).
    \end{itemize}
    Queda demostrado que $Z'$ es una \KHilogic-bisimulación, lo que demuestra ($\rightarrow$).
\end{demostracion}

Notemos como esta caracterización de la existencia de bisimulaciones entre modelos
nos permite profundizar nuestro primer análisis. Tenemos ahora un problema de `verificación'
únicamente, pues para decidir existencia solo basta con verificar que la relación $Z'$ es una bisimulación.

Es así como este lema motiva el siguiente resultado. 


\begin{lema}
    $\KHiBisim \in \coNP$ 
\end{lema}

\begin{demostracion}
    Para demostrar que $\KHiBisim \in \coNP$ debemos dar un algoritmo $A$ y un polinomio $p$ tal que dos modelos 
    $\modults_1$ y $\modults_2$ son \KHilogic-bisimilares si y sólo si para todo $x \in \{0,1\}^*$ con 
    $|x| \le p(|\modults_1|+|\modults_2|)$,  $A(\modults_1,\modults_2,x) = 1$. Esencialmente, a los $x$ los 
    llamaremos ``contraejemplos'' y sólo nos interesarán los que sean de la forma $\tup{U,b}$ con $U$ un 
    subconjunto del dominio de alguno de los modelos y $b$ un bit que nos indica de qué dominio es subconjunto $U$. 

    Dados que los contraejemplos que nos interesan son subconjuntos de los dominios de ambos modelos, un polinomio 
    que nos sirve para acotar los largos de los contraejemplos es $p(x) = x$. 

    Proponemos entonces el siguiente algoritmo (se escribió en la página que sigue, es modularizable porque las dos 
    guardas del if hacen lo mismo pero en distintos modelos):

    \begin{algorithm}
        \caption{Verificador de contraejemplos}
        \begin{algorithmic}[1]
            \Function{CheckBisimilarity}{$\modults_1,\modults_2,\tup{U,b}$} 
                \State Compute $Z'$
                \State Check (A-zig), if not \Return 0
                \State Check (A-zag), if not \Return 0
                \If{$b$}
                \State Check if $U$ is propositionally definable in $\modults_1$, if not \Return 1
                    \ForAll{$S_i$}
                        \ForAll{$\pi \in \S_i$}
                            \If{$U \subseteq \sexec(\pi)$}
                            \State $foundStrat \gets false$
                                \ForAll{$\pi' \in \S_i'$}
                                    \If{$Z'(U) \subseteq \sexec(\pi') \wedge \R'_{\pi'}(Z'(U)) \subseteq Z'(\R_\pi(U))$}
                                        \State $foundStrat \gets true$
                                    \EndIf         
                                \EndFor
                                \If{$\neg foundStrat$}
                                    \Return 0
                                \EndIf
                            \EndIf
                        \EndFor
                    \EndFor
                \Else
                    \State Check if $U$ is propositionally definable in $\modults_2$, if not \Return 1
                    \ForAll{$S_i'$}
                        \ForAll{$\pi' \in \S_i'$}
                            \If{$U \subseteq \sexec(\pi')$}
                            \State $foundStrat \gets false$
                                \ForAll{$\pi \in \S_i$}
                                    \If{$Z'^{-1}(U) \subseteq \sexec(\pi) \wedge \R_{\pi}(Z'^{-1}(U)) \subseteq Z'^{-1}(\R'_{\pi'}(U))$}
                                        \State $foundStrat \gets true$
                                    \EndIf         
                                \EndFor
                                \If{$\neg foundStrat$}
                                    \Return 0
                                \EndIf
                            \EndIf
                        \EndFor
                    \EndFor
                \EndIf
                \State \Return 1
            \EndFunction
        \end{algorithmic}
    \end{algorithm}

    Primero notemos que el algoritmo propuesto corre en tiempo polinomial en el tamaño de $\tup{\modults_1,\modults_2}$.
    (Esto va a haber que explicarlo un poco más en detalle después porque se ponen instrucciones atómicas que esconden 
    mucho cómputo, pero bueno creo que sirve tener un primer intento y demostrar el si y sólo si 
    que es lo más importante, la complejidad si ves cada instrucción es medio claro que se puede lograr en tiempo polinomial).

    Veamos ahora que $\tup{\modults_1,\modults_2} \in \KHiBisim$ si y sólo si
    CheckBisimilarity($\modults_1,\modults_2,\tup{U,b}$) = 1 para cada $\tup{U,b}$ donde $U$ es subconjunto 
    del dominio de alguno de ambos modelos y $b$ es un bit que nos indica de qué dominio es subconjunto $U$

    \begin{itemize}
        \item ($\rightarrow$)
        Sea $\tup{\modults_1,\modults_2} \in \KHiBisim$ y $\tup{U,b}$, 
        veamos que CheckBisimilarity($\modults_1,\modults_2,\tup{U,b}$) = 1

        Primero notemos que como $\tup{\modults_1,\modults_2} \in \KHiBisim$, entonces $Z'$ es una bisimulación 
        entre $\modults_1$ y $\modults_2$.

        Luego viendo la ejecución de CheckBisimilarity, los chequeos de (A-zig) y (A-zag) darán $true$ como resultado.
        
        Analicemos ahora los dos casos posibles

        \begin{itemize}
            \item $b = 1$
                Primero notemos que si $U$ no es proposicionalmente definible en $\modults_1$, entonces el algoritmo devuelve 1.

                Supongamos entonces $U$ proposicionalmente definible. Sea $\pi \in \S_i$ tal que $U \subseteq \sexec(\pi)$, 
                entonces el algoritmo ingresa al $if$ de la línea 8 e inicializa $foundStrat$ en $false$.
                
                Pero entonces tenemos que $U \ultsExecAgi \R_\pi(U)$, luego como $U$ es proposicionalmente definible y 
                $Z'$ cumple ($\khi$-zig), existe $T \subseteq \W'$ tal que $Z'(U) \ultsExecAgi T$ y,
                a su vez, $T \subseteq Z'(\R_\pi(U))$.
                
                Como $Z'(U) \ultsExecAgi T$, entonces existe $\pi' \in \S_i'$ tal que $Z'(U) \subseteq \sexec(\pi')$ y 
                $\R'_{\pi'}(Z'(U)) \subseteq T$.
                
                Pero como $T \subseteq Z'(\R_\pi(U))$, entonces $\R'_{\pi'}(Z'(U)) \subseteq Z'(\R_\pi(U))$, por lo que 
                el algoritmo al recorrer $\pi'$, actualiza el valor de $foundStrat$ a $true$.
                
                Finalmente, esto demuestra que el algoritmo finaliza su ejecución en la última línea donde devuelve 1.
            \item $b = 0$
                Este caso es análogo al caso $b = 1$, con la diferencia de que hay que considerar el hecho de que 
                $Z'$ satisface ($\khi$-zag).
        \end{itemize}

        \item ($\leftarrow$) Para demostrar esta dirección, lo que haremos será ver que siendo 
        $\tup{\modults_1,\modults_2} \not \in \KHiBisim$ entonces
        existe $\tup{U,b}$ tal que CheckBisimilarity($\modults_1,\modults_2,\tup{U,b}$) = 0.  
        
        Como $\tup{\modults_1,\modults_2} \not \in \KHiBisim$, entonces sabemos que $Z'$ no es una bisimulación.

        Notemos que por su definición, $Z'$ cumple (Atom). Analicemos los posibles casos donde podría fallar $Z$.

        \begin{itemize}
            \item (A-zig) Si $Z'$ no satisface (A-zig) entonces el algoritmo lo identifica en la segunda línea y devuelve 0, 
            por lo que cualquier $\tup{U,b}$ sirve como contraejemplo.
            \item (A-zag) Similarmente, si $Z'$ no satisface (A-zag) el algoritmo lo identifica en la tercer línea y 
            nuevamente, cualquier $\tup{U,b}$ nos sirve como contraejemplo.
            \item ($\khi$-zig) Supongamos entonces que $Z'$ no satisface ($\khi$-zig), luego existen $U, T \subseteq \W$ tal que
            $U \ultsExecAgi T$ con $U$ proposicionalmente definible, y a su vez, no existe $T' \subseteq \W'$ tal que $Z'(U) \ultsExecAgi T'$ y 
            $T' \subseteq Z'(T)$.

            Analicemos entonces la ejecución de CheckBisimilarity($\modults_1,\modults_2,\tup{U,1}$).

            Notemos que como $U \ultsExecAgi T$, entonces existe $\pi \in \S_i$, tal que $U \subseteq \sexec(\pi)$ y, 
            a su vez, $\R_\pi(U) \subseteq T$. 

            Esto nos dice que al considerar el plan $\pi$, el algoritmo ingresa al $if$ de la línea 8 e inicializa 
            la variable $foundStrat$ en $false$.

            Veamos entonces que para cada $\pi' \in \S_i'$ ocurre que $Z'(U) \not \subseteq \sexec(\pi')$ o que 
            $\R'_{\pi'}(Z'(U)) \not \subseteq Z'(\R_\pi(U))$, pues notar que si esto ocurre para cada $\pi'$ entonces 
            el valor de $foundStrat$ permanece en $false$ y el algoritmo devuelve 0 en la línea 15.

            Sea $\pi' \in \S_i'$, supongamos entonces que $Z'(U) \subseteq \sexec(\pi')$ y veamos que 
            $\R'_{\pi'}(Z'(U)) \not \subseteq Z'(\R_\pi(U))$.

            Notemos que como $Z'(U) \subseteq \sexec(\pi')$, entonces $Z'(U) \ultsExecAgi \R'_{\pi'}(Z'(U))$. 
            Como mencionamos que ($\khi$-zig) falla para este caso, entonces ocurre que 
            $\R'_{\pi'}(Z'(U)) \not \subseteq Z'(T)$, pues en caso contrario $T' := \R'_{\pi'}(Z'(U))$ serviría como 
            testigo de ($\khi$-zig).

            Pero como $\R_\pi(U) \subseteq T$, entonces es claro que $Z'(\R_\pi(U)) \subseteq Z'(T)$. Luego como 
            $\R'_{\pi'}(Z'(U)) \not \subseteq Z'(T)$ entonces también ocurre que $\R'_{\pi'}(Z'(U)) \not \subseteq Z'(\R_\pi(U))$, 
            que era lo que queríamos ver.

            Por lo que el algoritmo abandona el $for$ con la variable $foundStrat$ valiendo $false$, luego devuelve 0 en la línea 15.

            Finalmente, CheckBisimilarity($\modults_1,\modults_2,\tup{U,1}$) = 0.

            \item ($\khi$-zag) Análogo al caso recién analizado.
        \end{itemize}
    \end{itemize}
    Finalmente $\KHiBisim \in \coNP$.
\end{demostracion}

\begin{lema}
    $\KHiBisim$ es $\coNP$-hard.
\end{lema}

\begin{demostracion}

    Demostremos que $\KHiBisim$ es $\coNP$-hard.

    Para demostrarlo, reduciremos el problema $\DNFTAUT$ a $\KHiBisim$. El problema $\DNFTAUT$ está compuesto 
    por las fórmulas $\varphi$ proposicionales en forma disyuntiva normal que son tautologías. 
    $\DNFTAUT$ es $\coNP$ completo, por lo que si encontramos una reducción computable en tiempo polinomial de 
    $\DNFTAUT$ a $\KHiBisim$ habremos demostrado que $\KHiBisim$ es $\coNP$-hard.

    Entonces queremos encontrar una reducción $f$ computable en tiempo polinomial que transforme una $\varphi$ proposicional 
    en forma disyuntiva normal a un par de modelos $\tup{\modults_1,\modults_2}$ tal que $\varphi \in \DNFTAUT$ si y sólo si 
    $\tup{\modults_1,\modults_2} \in \KHiBisim$.

    Describamos a $f$.

    Sea $\varphi$ una fórmula proposicional en forma disyuntiva normal, entonces es de la forma 
    $\varphi = \varphi_1 \vee ... \vee \varphi_m$. Sean a su vez $p_1,...,p_n$ las variables proposicionales que aparecen en $\varphi$, 
    definimos los siguientes conjuntos:

    \begin{itemize}
        \item $\W := \{p_i\mid i\in\{1,...,n\}\} \cup \{\varphi_i \mid i \in \{1,...,m\}\} \cup \{e\}$
        \item $\ACT := \{in_{\varphi_i} \mid i \in \{1,...,m\}\} \cup \{out_{\varphi_i} \mid i \in \{1,...,m\}\}$
        \item $\S := \{\{in_{\varphi_i}out_{\varphi_i}\} \mid i \in \{1,...,m\}\}$
        \item $\R := \{(p_i,\varphi_j,in_{\varphi_j}) \mid p_i$ no aparece en forma negativa en $\varphi_j\} \cup 
                     \{(\varphi_j,p_i,out_{\varphi_j})\mid p_i$ aparece en forma positiva en $\varphi_j\} \cup 
                     \{(\varphi_j,e,out_{\varphi_j}) \mid \varphi_j$ no tiene variables positivas$\}$
        \item $\V := \{(p_i,\{q_i\}) \mid i \in \{1,...,n\}\} \cup 
                     \{(\varphi_i,\{q_{i+n}\}) \mid i \in \{1,...,m\}\} \cup 
                     \{(e,\{q_{n+m+1}\})\}$ donde $q_1,...,q_{n+m+1} \in \PROP$ son $n+m+1$ variables proposicionales distintas.
    \end{itemize}

    Notar que decimos que $p_i$ aparece en forma `positiva' en $\varphi_j$ si $\varphi_j = ... \wedge p_i \wedge...$ . 
    A su vez, decimos que $p_i$ aparece en forma `negativa' en $\varphi_j$ si $\varphi_j = ...\wedge \neg p_i \wedge...$ .

    Ahora, $f$ mapeará $\varphi$ a $\tup{\modults_1,\modults_2}$, con $\tup{\modults_1,\modults_2}$ definidos de la siguiente manera:

    \begin{itemize}
        \item En el caso de que $\varphi$ evalúe a $false$ en la asignación que asigna a $p_1,...,p_n$ el valor $false$:

        \begin{itemize}
            \item $\model_1=\tup{\{1\},\{\},\{\},\{(1,p)\},\{a\}}$
            \item $\model_2=\tup{\{1\},\{\},\{\},\{(1,q)\},\{a\}}$
        \end{itemize}

        Notar que estos dos modelos son claramente no \KHilogic-bisimilares, dado que no existe relación binaria no vacía 
        que satisfaga (Atom). Pero es correcto porque sólo utilizaremos esta definición en uno de los casos donde 
        $\varphi \not \in \DNFTAUT$. 
        
        \item En el caso de que $\varphi$ evalúe a $true$ en la asignación que asigna a $p_1,...,p_n$ el valor $false$:
        
        \begin{itemize}
            \item $\model_1=\tup{\W,\R \cup \{(p_i,p_i,test) \mid i \in \{1,...n\}\} \cup 
                            \{(p_i,e,test)\mid i \in \{1,...,n\}\},\{\S \cup \{\{test\}\} \},\V,\ACT \cup \{test\}}$
            \item $\model_2=\tup{\W,\R,\cset{\S},\V,\ACT}$
        \end{itemize}
    \end{itemize}

    
    Notar que $\V$ está definida de forma tal que cada $w\in\W$ satisface una única variable proposicional distinta, es decir, 
    es inyectiva.


\medskip\medskip
    Estos dos párrafos que siguen eran explicativos, pensaba sacarlos pero quizás está bueno dar una explicación más humana 
    antes de la demo más técnica.

    La intuición de esta reducción es que queremos usar los conjuntos proposicionalmente definibles de este modelo para considerar 
    cada posible asignación de valores a las variables $p_1,...,p_n$. Dado un conjunto $U \subseteq \{p_1,...,p_n\}$, 
    estamos considerando la asignación que asigna $true$ a las variables dentro de $U$ y asigna $false$ a las de $\{p_1,...,p_n\}/U$. 
    Además, notemos que como la función $\V$ es inyectiva, entonces $U$ es proposicionalmente definible.
 
    El plan $test$ que sólo aparece en $\modults_1$ es la clave de la reducción. Lo que provoca es que dada una asignación 
    (un conjunto proposicionalmente definible $U$), a $\modults_2$ le hago buscar algún término $\varphi_i$ al cuál pueda moverse 
    (ninguna variable de $U$ aparezca negada en $\varphi_i$) y a su vez todo variable que aparecía positiva en 
    $\varphi_i$ estaba en $U$. (Porque el plan $test$ es como que simplemente va de $U$ a $U \cup \{e\}$), el $e$ es como que 
    considera los términos que no tienen ninguna variable positiva ($e$ de empty set). Si $\modults_2$ no encuentra ningún término 
    al cuál moverse es que dicho $U$ sirve como contraejemplo para afirmar que $\varphi$ no es una tautología.

\medskip\medskip

    Notemos que ambos modelos tienen tamaño polinomial con respecto al tamaño de $\varphi$.

    A su vez, la construcción de los modelos, solo tiene el costo extra computacional de analizar si $\varphi$ evalúa a $true$ 
    en la asignación que asigna $false$ a $p_1,...,p_n$, lo cuál es claramente computable en tiempo polinomial.

    Por lo que es fácil notar que $f$ es computable en tiempo polinomial en el tamaño de $\varphi$.

    Ahora queremos demostrar que $\varphi \in \DNFTAUT$ si y sólo si $\tup{\modults_1,\modults_2} \in \KHiBisim$.
    \begin{itemize}

    \item $(\rightarrow)$ Sea $\varphi \in \DNFTAUT$, donde $\varphi = \varphi_1 \vee ... \vee \varphi_m$ y $p_1,...,p_n$ 
    son las variables proposicionales que aparecen en $\varphi$, veamos que $\tup{\modults_1,\modults_2} \in \KHiBisim$.

    Notar que como $\varphi$ es una tautología, entonces $\varphi$ evalúa a $true$ en la asignación que asigna $false$ a 
    $p_1,...,p_n$, por lo que consideramos el segundo caso del mapeo $f$. 

    Veamos que $Z := \{(w,w) \mid w \in \W \}$ es una bisimulación entre $\modults_1$ y $\modults_2$. Es fácil ver que 
    $Z$ satisface (Atom), pues ambos modelos utilizan la misma función de valuación $\V$.

    A su vez, tanto (A-zig) como (A-zag) son claramente satisfechos por $Z$.

    Analicemos entonces ($\khi$-zig) y ($\khi$-zag).

    Primero, cabe resaltar que, por la definición de $Z$, para cada $X \subseteq \W$ se cumple que $X = Z(X)$.

    Notemos que cada plan de $\modults_2$ está también en $\modults_1$, y más aún, las aristas con las etiquetas 
    que aparecen en dichos planes son exactamente las mismas en ambos modelos. Juntando lo recién mencionado y el hecho de que 
    $U = Z(U)$, es fácil notar que Z satisface ($\khi$-zag).

    Luego solo queda ver que $Z$ satisface ($\khi$-zig).

    Sea $U, T \subseteq \W$ tal que $U$ es proposicionalmente definible y $U \ultsExecAgi T$, queremos ver que existe 
    $T' \subseteq \W$ tal que:

    \begin{multicols}{2}
        \begin{itemize}
            \item $Z(U) \ultsExecAgi T'$, 
            \item $T' \subseteq Z(T)$.
        \end{itemize}
    \end{multicols}

    Notemos primero que si el plan que atestigua $U \ultsExecAgi T$ es de la forma $in_{\varphi_j}out_{\varphi_j}$, 
    entonces $T$ sirve como $T'$ dado que ese mismo plan y las mismas aristas con dichas etiquetas están en $\modults_2$, 
    similar al análisis realizado para el caso ($\khi$-zag).

    Entonces consideremos el caso en el que el plan que atestigua $U \ultsExecAgi T$ es $test$. En dicho caso, 
    si analizamos la definición de $\R$, ese plan puede ser fuertemente ejecutable sólo si $U \subseteq \{p_1,...,p_n\}$ 
    dado que son los únicos nodos que tienen aristas con etiqueta $test$. Así que solo necesitamos considerar los casos 
    donde $U \subseteq \{p_1,...,p_n\}$.

    Notemos que $\R_{test}(U) = U \cup \{e\}$, entonces necesariamente $U \cup \{e\} \subseteq T$.

    Consideremos ahora la asignación $\overrightarrow{a}$ que asigna $true$ a las variables en $U$ y $false$ a las variables 
    en $\{p_1,...,p_n\} / U$. Como $\varphi$ es una tautología, entonces existe $\varphi_j$ que se vuelve verdadero a partir 
    de $\overrightarrow{a}$, es decir, en $\varphi_j$ las variables que aparecen en forma positiva son algún subconjunto de 
    $U$ y las variables que aparecen en forma negativa son algún subconjunto de $\{p_1,...,p_n\}/U$.

    Como las variables que aparecen en $\varphi_j$ en forma negativa no están en $U$, entonces, analizando la definición de $\R$, 
    podemos ver que el plan $in_{\varphi_j}out_{\varphi_j}$ es fuertemente ejecutable en $Z(U) = U$. 
    
    En particular, las aristas con etiqueta $in_{\varphi_j}$ llevarían de cada nodo de $U$ al nodo $\varphi_j$.
    
    Notemos que las aristas con etiqueta $out_{\varphi_j}$ llevan del nodo $\varphi_j$ a las variables 
    que aparecen en forma positiva en $\varphi_j$ o al nodo $e$ en caso de no tener variables positivas. Pero, como dijimos, 
    las variables positivas que aparecen en $\varphi_j$ son un subconjunto, posiblemente vacío, de $U$. 
    Entonces necesariamente de $\varphi_j$ las aristas con etiqueta $out_{\varphi_j}$ van a un conjunto $X \subseteq U \cup \{e\}$, 
    es decir, $\R_{in_{\varphi_j}out_{\varphi_j}}(U) = X \subseteq U \cup \{e\} \subseteq T$. 
    Finalmente, $T' := X$ nos sirve para demostrar que $Z(U) \ultsExecAgi T'$ y, a su vez, 
    $T' \subseteq X \subseteq U \cup \{e\} \subseteq T = Z(T)$.

    Lo cual demuestra que $Z$ satisface ($\khi$-zig). 

    Juntando los puntos mencionados, demostramos que $Z$ es una bisimulación entre $\modults_1$ y $\modults_2$, 
    por lo que $\tup{\modults_1,\modults_2} \in \KHiBisim$. 

    \item ($\leftarrow$) Para demostrar este caso, veremos que siendo $\varphi \notin \DNFTAUT$, 
    donde $\varphi = \varphi_1 \vee ... \vee \varphi_m$ y $p_1,...,p_n$ son las variables proposicionales que aparecen en $\varphi$, 
    entonces $\tup{\modults_1,\modults_2} \notin \KHiBisim$.

    Como $\varphi \notin \DNFTAUT$, existe una asignación $\overrightarrow{a}$ sobre las variables $p_1,...,p_n$ que hace fallar 
    cada $\varphi_j$. Denotemos a $U$ como el conjunto de variables a las que se le asigna $true$ en $\overrightarrow{a}$.

    Si $U = \emptyset$, es claro que $\tup{\modults_1,\modults_2} \notin \KHiBisim$, dado que en este caso se construyen 
    dos modelos concretos que no son \KHilogic-bisimilares.

    Así que consideremos el caso donde $U \neq \emptyset$.
    
    Primero notemos que la fórmula $\psi := \bigvee\limits_{p_i \in U} q_i$ define a $U$. 
    Pues, recordemos que por la definición de $\V$, $\V(p_i) = \{q_i\}$ para cada $i \in \{1,...,n\}$. 
    A su vez, como $\V$ es inyectiva, solo los elementos de $U$ satisfacerán $\psi$. Luego, como $\psi$ es proposicional, 
    $U$ es proposicionalmente definible.

    Como $U \subseteq \{p_1,...,p_n\}$ entonces el plan $test$ es fuertemente ejecutable en $U$ y 
    $\R_{test}(U) = U \cup \{e\}$, por lo que $U \ultsExecAgi U \cup \{e\}$.
    
    Veamos que no existe $T'$ tal que:

    \begin{multicols}{2}
        \begin{itemize}
            \item $Z(U) \ultsExecAgi T'$, 
            \item $T' \subseteq Z(U \cup \{e\})$.
        \end{itemize}
    \end{multicols}

    Notemos que los planes que tiene a su disposición $Z(U) = U$ son de la forma $in_{\varphi_j}out_{\varphi_j}$.

    Sea entonces $\varphi_j$, analicemos qué sucede en relación al plan $in_{\varphi_j}out_{\varphi_j}$. 
    Un primer detalle a tener en cuenta es que como $\overrightarrow{a}$ hace fallar a $\varphi_j$, entonces, 
    o bien ocurre que existe una variable en $U$ que aparece en forma negativa en $\varphi_j$, 
    o existe una variable en $\{p_1,...,p_n\}/U$ que aparece en forma positiva en $\varphi_j$.

    Si ocurre que una variable de $U$ aparece en forma negativa en $\varphi_j$ entonces el plan 
    $in_{\varphi_j}out_{\varphi_j}$ no será fuertemente ejecutable en dicha variable y, por lo tanto, 
    no será fuertemente ejecutable en $U$. Así que este plan no permitiría encontrar un $T'$ adecuado.

    Por otro lado, si ninguna variable de $U$ aparece en forma negativa en $\varphi_j$ entonces 
    $in_{\varphi_j}out_{\varphi_j}$ es fuertemente ejecutable en $U$. Sin embargo, notemos que 
    entonces existe una variable $p_k$ en $\{p_1,...,p_n\}/U$ que aparece en forma positiva en $\varphi_j$. 
    Como $p_k$ aparece en forma positiva en $\varphi_j$ entonces $p_k \in \R_{in_{\varphi_j}out_{\varphi_j}}(U)$. 
    Luego $\R_{in_{\varphi_j}out_{\varphi_j}}(U) \not \subseteq U \cup \{e\} = Z(U \cup \{e\})$, por lo que dicho plan no nos 
    permite encontrar un $T'$ adecuado.
    
    Hemos analizado entonces todos los planes y ninguno permite encontrar un $T'$ que cumpla con lo requerido, 
    luego $Z$ no satisface ($\khi$-zig).

    Hemos demostrado entonces que $Z$ no es una \KHilogic-bisimulación.

    Lo cual demuestra que $\tup{\modults_1,\modults_2} \not \in \KHiBisim$.
    \end{itemize}
    Finalmente, demostramos que $f$ es una reducción de $\DNFTAUT$ a $\KHiBisim$, la cuál puede ser computada en tiempo polinomial. 
    Lo cuál demuestra que $\KHiBisim$ es $\coNP$-hard.
\end{demostracion}


\begin{teorema}
    $\KHiBisim$ es $\coNP$-completo.
\end{teorema}


\begin{demostracion}

    A partir de los lemas 5 y 6, queda demostrado que $\KHiBisim$ es $\coNP$-completo.

\end{demostracion}


\section{Equivalencia vía Reducción de Karp con el Caso Punteado}