\chapter*{Agradecimientos}

A Raul, mi director. Por la inmensa ayuda en todos los aspectos, desde enseñanzas en cuestiones técnicas hasta 
consejos para la vida. También quiero mencionar a Carlos, que junto con Raul me invitaron al DL(R) 
Workshop en Chile y me permitieron participar en la escritura de un artículo de investigación.

A Susu y Adolfo, mis viejos. Me han acompañado a lo largo de toda mi vida y me han apoyado y aconsejado en absolutamente todas 
las decisiones que he tenido que tomar. Me escuchan en cada problema y me festejan en cada logro. Les agradezco infinitamente.

A Lorenzo y Rosario, mis hermanos. Por la excelente relación que tenemos. Siempre se han interesado por mi y me han aconsejado y ayudado 
cuando más lo necesitaba. Espero que sigamos igual o más unidos por siempre.

A Benja y Juli, mis amigos y grupo de trabajo por excelencia. Hemos compartido absolutamente todo en estos años y definitivamente este 
trayecto no hubiera sido tan lindo sin ustedes. Estamos juntos desde el cursillo de ingreso y es increíble pensar a dónde llegamos. Ha sido 
un placer.

A Santi, Jesus y el Viejo, mis amigos de toda la vida. He encontrado una pasión extraña en la computación que muchas veces me frustra pero 
me junto con ustedes y logro distraerme y disfrutar como cuando eramos unos pibes de secundaria. Son lo más grande que existe.

A Tincho y Leo, la MSN. Desde incontables charlas en temas de computación hasta salidas y noches de estudio, hemos compartido de todo estos últimos 
años. Por muchas más anécdotas. 

A FAMAF y a la Universidad Nacional de Córdoba. Me han brindado la oportunidad de formarme con docentes de excelente calidad, no solo académica 
sino que también humana. Me han otorgado financiamento para participar en competencias de programación y cursos de computación. Espero en un 
futuro poder retribuir aunque sea un poco de todo lo recibido. Es de crucial importancia defender la educación pública.