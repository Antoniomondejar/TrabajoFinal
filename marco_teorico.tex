\chapter{Lógica de ``Knowing How'' Multi-Agente Basada en Incertidumbre}
En este capítulo, introduciremos un enfoque para modelar el concepto de ``saber cómo'', presentado en \cite{ArecesFSV25,SaraviaPHD}, el cuál surge a partir de la 
lógica previamente introducida en \cite{Wang15KH, Wang2018GoalDirectedKH}. En esta primera propuesta, el conocimiento de un agente está 
representado en Sistemas de Transición Etiquetado (\lts por sus siglas en inglés), donde las transiciones etiquetadas representan los cambios de estados 
que produce cada acción que el agente puede realizar. A su vez, las habilidades de un agente están dadas no solo por las acciones básicas que tiene a 
su disposición en un determinado estado, sino que también por la composición de las mismas. A las composiciones de acciones las 
llamaremos planes. Se considera que un plan es ``adecuado'' para lograr un objetivo en cierto estado, cuando cada ejecución parcial del plan 
a partir de dicho estado se puede completar. En tal caso, diremos que el plan es ``fuertemente ejecutable'' en dicho estado.

La lógica de ``Knowing How'' multi-agente basada en incertidumbre toma este enfoque como punto de partida. La principal motivación por la que 
surge esta nueva propuesta es que la representación del conocimiento mencionada anteriormente supone que el agente cuenta con cierto nivel de omnisciencia, 
pues se asume que un agente conoce todo plan que pueda inferirse de su \lts asociado y que, además, es capaz de distinguir qué plan es adecuado de entre 
los que tiene a su disposición, en el sentido de que para todo par de planes es capaz de reconocer si provocan resultados diferentes.
Para esclarecer un poco más este argumento, veamos el siguiente ejemplo presentado en \cite{ArecesFSV25,SaraviaPHD}.

Consideremos un agente que quiere hacer una torta. El agente cuenta con la habilidad de realizar cualquiera de los cuatro métodos de mezcla diferentes 
(de batido, de la crema, de fusión y de frotamiento) y hasta podría ser capaz de identificarlos como acciones distintas entre sí. Sin embargo, 
el agente podría no tener noción de cuáles son los efectos de utilizar un método u otro, es decir, podría no distinguir que los mismos producen resultados 
diferentes. En este caso, podríamos considerar que el agente no sabe cómo hacer una torta, pues en los casos en los que selecciona el método correcto obtiene 
un buen resultado y en los demás casos no. 

Incluso, podríamos considerar más situaciones en las que la indistinguibilidad del agente no sea únicamente sobre acciones sino que sobre planes en su totalidad. Por ejemplo, 
el agente sabe la diferencia entre añadir leche y añadir harina a la mezcla, pero podría darse el caso en el que no sepa distinguir cuál es el orden adecuado en el que 
realizar ambas acciones. Incluso, podría ocurrir que el agente no sepa distinguir planes de distinta longitud, por ejemplo, en el proceso de horneado de la torta el agente 
debe eventualmente abrir el horno para verificar si el proceso de horneado ya está completo, pero hacerlo con demasiada frecuencia podría afectar al resultado de la torta.

Es así como estos ejemplos motivan una representación más general sobre el conocimiento de un agente, tomando en consideración no solo las habilidades con las que dispone sino 
también su capacidad de distinguir diferentes planes. A su vez, esto permite una representación para múltiples agentes, en la que todos los agentes disponen de las mismas acciones 
pero cada uno tiene su propia interpretación de los efectos de las mismas, así como su propia habilidad para distinguir distintos planes.

Así, la estrategia utilizada en \cite{ArecesFSV25,SaraviaPHD} consiste en fijar un conjunto de agentes $\AGT$ y dotar a los \lts con una 
relación de indistinguibilidad sobre planes para cada agente.

Una vez hechas estas consideraciones, introduciremos la lógica de ``Knowing-How'' multi-agente basada en incertidumbre (\KHilogic) definiendo 
su sintaxis y semántica, y su noción de bisimulación. Por último, mencionaremos algunos resultados relacionados a su complejidad 
computacional con respecto a los problemas de model-checking y satisfacibilidad, con el objetivo de brindar una perspectiva completa acerca 
del comportamiento computacional de la lógica.

\section{Sintaxis y Semántica}
En lo que sigue, consideraremos a $\PROP$ un conjunto no vacío de variables proposicionables con cardinalidad contable 
y a $\AGT$ un conjunto finito no vacío de nombres de agentes.

\begin{definicion}
    El lenguaje \KHilogic está compuesto por las fórmulas dadas por la gramática:
    \begin{center}
        $\varphi ::= p \mid \neg \varphi \mid \varphi \vee \varphi \mid \varphi \wedge \varphi \mid \kh_i(\varphi,\varphi)$,
    \end{center}
    donde $p \in \PROP$ e $i \in \AGT$. Las constantes booleanas son definidas de la forma usual. Las fórmulas de la forma 
    $\kh_i(\psi, \varphi)$ deben ser leídas como ``cuando vale $\psi$, el agente $i$ sabe cómo lograr $\varphi$''.
\end{definicion}

Formalizaremos matemáticamente algunos de los conceptos discutidos en la presentación de este capítulo. Como mencionamos, se 
utilizarán secuencias de acciones, o planes, para representar las habilidades de los agentes, que determinarán el conocimiento de los mismos. 
Las fórmulas de \KHilogic se interpretarán sobre sistemas de transición etiquetados extendidos con una familia de relaciones de 
indistinguibilidad entre planes.

\begin{definicion}[Acciones y planes]
    Sea $\ACT$ un conjunto enumerable de nombres de acciones, y sea $\ACT^*$ el conjunto de secuencias finitas sobre $\ACT$. 
    A los elementos de $\ACT^*$ los llamaremos planes, siendo $\epsilon$ el plan vacío. Sea $\sigma \in \ACT^*$, denotaremos $|\sigma|$ 
    al largo de $\sigma$ (notar $|\epsilon| = 0$). Para un plan $\sigma$ y $0 \leq k \leq |\sigma|$, el plan $\sigma[..k]$ es el prefijo 
    de $\sigma$ hasta la $k-$ésima posición inclusive. Para $0 < k \leq |\sigma|$, la acción $\sigma[k]$ es la que se encuentra en la 
    $k-$ésima posición de $\sigma$.  
\end{definicion}

Presentamos a continuación un ejemplo ilustrativo de las nociones mencionadas.

\begin{ejemplo}
    Consideremos un robot moviendose en una grilla, el cuál tiene a su disposición la opción de moverse a cualquiera de sus 4 casilleros vecinos; es decir,
    puede moverse a la izquierda ($`L'$), a la derecha ($`R'$), hacia arriba ($`U'$) o hacia abajo ($`D'$). En este contexto, el conjunto de acciones básicas 
    $\ACT$ estaría definido como $\ACT = \{L,R,U,D\}$, y $\ACT^*$ sería el conjunto de secuencias finitas sobre $\ACT$, el cuál representa los distintos 
    caminos que el robot podría tomar en la grilla. Por ejemplo, $`LLLDRR'$ y $`UURRU'$ están en $\ACT^*$. 

    Luego, a partir de la definición dada, tenemos:
    \begin{itemize}
        \item $(LLDDRRUU)[$..$0] = \epsilon$.
        \item $(LLDDRRUU)[$..$3] = LLD$.
        \item $(LLDDRRUU)[$..$7] = LLDDRRU$.
        \item $(LLDDRRUU)[5] = R$.
        \item $(LLDDRRUU)[8] = U$.
    \end{itemize}
\end{ejemplo}

Procederemos a introducir los modelos que representarán las acciones que los agentes tienen a su disposición: 
los sistemas de transición etiquetados.

\begin{definicion}[Sistema de Transición Etiquetado]
    Un \emph{Sistema de Transición Etiquetado (\lts, por sus siglas en inglés)} sobre $\PROP$ es una tupla $\modlts = \tup{\W, \R, \V, \ACT}$ 
    donde $\W$ es un conjunto no vacío de estados, $\R = \{\R_a \subseteq \W \times \W \mid a \in A$ para algún $A \subseteq \ACT\}$ es 
    una colección de relaciones binarias sobre $\W$, $\V : \W \to \mathcal{P}(\PROP)$ es una función de etiquetado y $\ACT$ es un 
    conjunto enumerable de nombres de acciones.
\end{definicion}

Gráficamente, un \lts se representa como un grafo dirigido etiquetado, donde los nodos serán los elementos de $\W$, cada uno marcado con las variables proposicionales 
dadas por la función de etiquetado $\V$ y las aristas entre dos nodos estarán dadas por los pares de $\R_a$ para cada $a \in \ACT$.

\begin{ejemplo}\label{ejemplo:lts}
    Sea $\modlts = \tup{\W, \R, \V, \ACT}$ un \lts con:
    
    \begin{itemize}
        \item $\W = \{w_1, w_2, w_3, w_4\}$.
        \item $\R = \{R_a, \R_b\}$ con $\R_a = \{(w_1,w_3), (w_1,w_4), (w_2,w_3)\}$ y $\R_b = \{(w_3,w_4)\}$.
        \item $\V = \{(w_1, \{p\}), (w_2, \{p\}), (w_3, \{q\}), (w_4, \{r\})\}$.
        \item $\ACT = \{a, b\}$.
    \end{itemize}

    Su representación gráfica es (\Cref{fig:lts}):
    \begin{figure}[h]
        \centering
        \begin{tikzpicture}
            \node[state] (p) {$p$};
            \node[state, below of=p, yshift=-1.5cm] (p2) {$p$};
            \node[state, right of=p, yshift=-1cm, xshift=1.5cm] (q) {$q$};
            \node[state, right of=q, xshift=1.5cm] (r) {$r$};
            
            \node at ($(p)+(0,0.5)$) {$w_1$};
            \node at ($(r)+(0,0.5)$) {$w_4$};
            \node at ($(q)+(0,0.5)$) {$w_3$};
            \node at ($(p2)+(0,-0.6)$) {$w_2$};

            \path (p) edge node [above] {$a$} (q)
                      edge [bend left] node [above] {$a$} (r);
            \path (p2) edge node [above] {$a$} (q);
            \path (q) edge node [above] {$b$} (r);
        \end{tikzpicture}
        \caption{Representación gráfica de $\modlts$}
        \label{fig:lts}
    \end{figure}

\end{ejemplo}

Como mencionamos, en \cite{ArecesFSV25,SaraviaPHD} se extienden los \lts con una noción de indistinguibilidad epistémica entre planes.

\begin{definicion}[\lts basado en incertidumbre]
    Un \emph{\lts-multi-agente basado en incertidumbre (\ults)} sobre $\PROP$ y $\AGT$ es una tupla $\model = \tup{\W,\R,\sim,\V,\ACT}$ 
    donde $\tup{\W,\R,\V,\ACT}$ es un \lts y $\sim$ asigna a cada $i \in \AGT$ una relación de equivalencia sobre un conjunto no vacío 
    de planes $P_i \subseteq \ACT^*$, también llamada relación de indistinguibilidad. Dado un \ults $\modults$ y $w \in \W$, se llamará 
    al par $(\modults,w)$ un \ults punteado y, usualmente, los paréntesis serán omitidos.
\end{definicion}

Notar que para cada agente $i \in \AGT$, $P_i$ representará los planes que el agente tiene a su disposición. 
Luego $\sim_i$ será una relación de equivalencia sobre $P_i$, donde dos planes estarán relacionados cuando un agente no sepa distinguirlos.

Sea $\model = \tup{\W,\R,\sim,\V,\ACT}$ un \ults y sea $i \in \AGT$, para un plan $\sigma \in P_i$, sea $[\sigma]_i$ su clase de 
equivalencia en $\sim_i$. Notar que hay una correspondencia uno-a-uno entre cada $\sim_i$ y la partición de $P_i$ en sus correspondientes 
clases de equivalencia $\S_i := \{[\sigma]_i \mid \sigma \in P_i\}$. Por lo tanto, de ahora en adelante nos referiremos a los \ults como
una tupla $\tup{\W,\R,\{\S_i\}_{i \in \AGT},\V,\ACT}$.

La representación gráfica de los \ultss será similar a la presentada para los \ltss, con la inclusión de la relación de indistinguibilidad 
entre planes de cada agente.

\begin{ejemplo}\label{ejemplo:ults}
    Siguiendo el \Cref{ejemplo:lts} y considerando $\AGT = \{i\}$, la representación gráfica 
    de $\model=\tup{\W,\R,\{\S_i\}_{i\in\AGT},\V,\ACT}$ es (\Cref{fig:ults}):
    \begin{figure}[h]
        \hspace{2.3cm}
        \begin{tikzpicture}
            \node[state] (p) {$p$};
            \node[state, below of=p, yshift=-1.5cm] (p2) {$p$};
            \node[state, right of=p, yshift=-1cm, xshift=1.5cm] (q) {$q$};
            \node[state, right of=q, xshift=1.5cm] (r) {$r$};
            
            \node at ($(p)+(0,0.5)$) {$w_1$};
            \node at ($(r)+(0,0.5)$) {$w_4$};
            \node at ($(q)+(0,0.5)$) {$w_3$};
            \node at ($(p2)+(0,-0.6)$) {$w_2$};

            \path (p) edge node [above] {$a$} (q)
            edge [bend left] node [above] {$a$} (r);
            \path (p2) edge node [above] {$a$} (q);
            \path (q) edge node [above] {$b$} (r);
        \end{tikzpicture}
        \hspace{1cm}
        \raisebox{1.8cm}{
            \begin{minipage}{0.45\textwidth}
                $\S_i = \left\{
                    \begin{array}{c}
                        \{ab,a\}, \{b\}
                    \end{array}
                \right\}$
            \end{minipage}
        }
        \caption{Representación gráfica de $\modults$}
        \label{fig:ults}
    \end{figure}

    En este caso, los planes que el agente $i$ tiene a su disposición son $\{ab,a,b\}$ y, lo que nos dice su relación de 
    indistinguibilidad $\S_i$ es que los planes $ab$ y $a$ son indistinguibles para $i$.
\end{ejemplo}

Dada entonces la incertidumbre de un agente sobre $\ACT^*$, sus habilidades dependerán no solo de lo que un plan puede lograr, sino de 
lo que toda una clase de equivalencia dentro de su relación de indistinguibilidad pueda garantizar.

\begin{definicion}
    Sea $\{\R_a \subseteq \W \times \W \mid a \in A,$ para algún $A \subseteq \ACT \}$ una colección de relaciones binarias sobre $\W$. 
    Se define $\R_\epsilon := \{(w,w) \mid w \in \W\}$ y, para $\sigma \in \ACT^*$ y $a \in \ACT$, 
    $\R_{\sigma a} := \{(w,v) \in \W \times \W \mid$ existe $u \in \W$ tal que $(w,u) \in \R_\sigma$ y $(u,v) \in \R_a \}$. 
    Luego sea $u \in \W$ y $\sigma \in \ACT^*$, se define $\R_\sigma(u) := \{v\in\W \mid (u,v) \in \R_\sigma\}$, y para $U\subseteq\W$ 
    definimos $\R_\sigma(U) := \bigcup\limits_{u \in U} \R_\sigma(u)$.

    Sea $\pi \subseteq \ACT^*$, $u \in \W$ y $U \subseteq \W$, se define
    \[
        \R_\strategy := \bigcup_{\sigma \in \strategy} \R_{\sigma},
    \qquad
        \R_{\strategy}(u) := \bigcup_{\sigma \in \strategy} \R_\sigma(u),
    \qquad
        \R_{\strategy}(U) := \bigcup_{u \in U} \R_{\strategy}(u).
    \]
\end{definicion}

\begin{ejemplo}
    Siguiendo con el \ults presentado en \Cref{ejemplo:ults}, tenemos:
    \begin{itemize}
        \item $\R_{ab} = \{(w_1,w_4), (w_2, w_4)\}$.
        \item $\R_{ab}(w_1) = \{w_4\}$, \quad $\R_{ab}(w_3) = \emptyset$. 
        \item $\R_{a}(\{w_1,w_2\}) = \R_a(w_1) \cup \R_a(w_2) = \{w_3,w_4\} \cup \{w_3\} = \{w_3, w_4\}$.
        \item $\R_{\{ab,b\}} = \R_{ab} \cup \R_b = \{(w_1,w_4),(w_2,w_4),(w_3,w_4)\}$.
        \item $\R_{\{ab,b\}}(w_1) = \R_{ab}(w_1) \cup \R_b(w_1) = \{w_3,w_4\} \cup \emptyset = \{w_3,w_4\}$.
        \item $\R_{\{ab,b\}}(\{w_1,w_3\}) = \R_{\{ab,b\}}(w_1) \cup \R_{\{ab,b\}}(w_3) = \{w_3,w_4\} \cup \{w_4\} = \{w_3,w_4\}$.
    \end{itemize}
\end{ejemplo}


La idea presentada en \cite{ArecesFSV25,SaraviaPHD} consiste en que un agente sabe cómo lograr que valga $\varphi$ dado que vale $\psi$ cuando exista un 
conjunto de planes `adecuado' que pueda ejecutar desde cualquier estado en el que valga $\psi$ que lleve sólo a estados donde valga $\varphi$. 
Entonces, una parte crucial es determinar qué se considerará un conjunto de planes `adecuado'. Como mencionamos en la presentación de este capítulo, y tomando 
inspiración del campo de ``planning'' en Inteligencia Artificial (IA), surge la noción de ejecutabilidad fuerte.

\begin{definicion}[Ejecutabilidad Fuerte]
    Sea $\{\R_a \subseteq \W \times \W \mid a \in A$, para algún $A \subseteq \ACT\}$ una colección de relaciones binarias. Un plan $\sigma \in \ACT^*$
    es fuertemente ejecutable ($\sexec$) en un estado $w \in \W$ si y sólo si $\R_\sigma$ está definido y, a su vez, $v \in \R_{\sigma[..k]}(w)$ implica que 
    $\R_{\sigma[k+1]}(v) \neq \emptyset$ para cada $k \in \{0,...,|\sigma|-1\}$. Se define el conjunto $\sexec$($\sigma$) $:= \{w \in \W \mid \sigma$ es $\sexec$ en $w\}$.
    
    Un conjunto de planes $\pi \subseteq \ACT^*$ es fuertemente ejecutable en un estado $u \in \W$ si y sólo si cada $\sigma \in \pi$ es fuertemente ejecutable en $u$.
    Por último, $\sexec$($\pi$) $:= \cap_{\sigma \in \pi}$ $\sexec$($\sigma$) es el conjunto de estados en $\W$ donde $\pi$ es fuertemente ejecutable. 
\end{definicion}

Intuitivamente, ejecutabilidad fuerte sobre planes pide que cada ejecución parcial de un plan (incluyendo $\epsilon$) pueda ser completada, y, ejecutabilidad fuerte sobre
clases de planes pide que cada plan de la clase sea fuertemente ejecutable.

\begin{ejemplo}
    Si consideramos el \ults presentado en \Cref{ejemplo:ults} entonces tenemos que:
    \begin{itemize}
        \item $\sexec(a) = \{w_1,w_2\}$.
        \item $\sexec(ab) = \{w_2\}$.

        Notar que, a pesar de que $\R_{ab}(w_1) \neq \emptyset$, el plan $ab$ no es fuertemente ejecutable en $w_1$ 
        dado que existe una ejecución parcial que no puede ser completada, pues notar que al tomar la arista 
        $(w_1,w_4)$ con etiqueta $a$ no existe arista con etiqueta $b$ desde $w_4$ para completar el camino.

        \item $\sexec(\{a,ab\}) = \sexec(a) \cap \sexec(ab) = \{w_1,w_2\} \cap \{w_2\} = \{w_2\}$.
    \end{itemize}
\end{ejemplo}

Ahora, estamos en condiciones de presentar la relación de satisfacibilidad que relaciona \ultss punteados con formulas de \KHilogic. 

\begin{definicion}[\KHilogic sobre \ultss]
    La relación $\models$ entre un \ults punteado $\modults,w$ (con $\model = \tup{\W,\R,\{S_i\}_{i\in\AGT},\V,\ACT})$ y las fórmulas de \KHilogic está definida inductivamente 
    de la siguiente forma:

    \begin{nscenter}
    \sloppy
    \begin{tabular}{@{}l@{\;\;\;}c@{\;\;\;}l@{}}
        $\modults,w \models p$ & \iffdef & $p\in\V(w)$, \\
        $\modults,w \models \neg\varphi$ & \iffdef & $\modults,w \not\models\varphi$, \\ 
        $\modults,w \models \varphi\vee\psi$ & \iffdef & $\modults,w \models \varphi \,\mbox{ o }\, \modults,w \models\psi$, \\
        $\modults,w \models \khi(\psi,\varphi)$ & \iffdef & \begin{minipage}[t]{0.68\textwidth}
                                                         existe $\strategy \in \S_i$ tal que \\
                                                         {\centering
                                                           \begin{inline-cond-kh}\item $\truthset{\modults}{\psi} \subseteq \sexec(\strategy)$ y \item $\R_\strategy(\truthset{\modults}{\psi}) \subseteq \truthset{\modults}{\varphi}$,\end{inline-cond-kh}
                                                          }
                                                       \end{minipage}
    \end{tabular}
    \end{nscenter}
    donde $\truthset{\modults}{\varphi} := \{w \in \W \mid \modults,w \models \varphi\}$. El conjunto de planes $\pi$ en la claúsula semántica de $\khi(\psi,\varphi)$ es llamado
    el testigo de $\khi(\psi,\varphi)$ en $\modults$.
\end{definicion}


Notar que $\khi(\psi,\varphi)$ vale en un estado $w$ cuando existe un conjunto de planes $\pi$ que el agente $i$ considera indistinguibles, tal que al ejecutar cualquier plan $\sigma \in \pi$ a partir de un estado donde vale $\psi$,
toda ejecución parcial puede ser completada terminando en un estado donde vale $\varphi$. También, cabe destacar que como $w$ no tiene ningún rol en la cláusula semántica de $\khi$,
dicho operador actúa \emph{globalmente}. Por lo tanto, $\truthset{\modults}{\khi(\psi,\varphi)}$ es $\W$ o $\emptyset$.

\begin{ejemplo}
    Nuevamente, consideremos el \ults presentado en \Cref{ejemplo:ults}, y veamos que:
    \begin{itemize}
        \item $\modults,w_1 \models p$ pero $\modults,w_3 \not\models p$.
        \item Notemos que $\sexec(\{ab,a\}) = \{w_2\}$ y $\sexec(\{b\}) = \{w_3\}$. 
        Luego $\truthset{\modults}{p} = \{w_1,w_2\} \not\subseteq \sexec(\pi)$ para todo $\pi \in \S_i$.
        Por lo que podemos afirmar que $\modults,w_1 \not\models \khi(p,q)$.
        Más aún, $\modults,w_j \not\models \khi(p,q)$ para cada $j \in \{1,..,4\}$.
        \item Como mencionamos en el item anterior, $\sexec(\{b\}) = \{w_3\}$. Luego, podemos ver que 
        $\modults, w_1 \models \khi(q,r)$ y $\{b\} \in \S_i$ es su testigo. 
        Pues, $\truthset{\modults}{q} = \{w_3\} \subseteq \sexec(\{b\})$ y $\R_b(w_3) = \{w_4\} \subseteq \truthset{\modults}{r}$.         

        A su vez, por la globalidad de $\khi$, ocurre que $\modults,w_j \models \khi(q,r)$ para cada $j \in \{1,...,4\}$.
    \end{itemize}
\end{ejemplo}


\section{Bisimulación}

La bisimulación es una herramienta crucial a la hora de analizar el poder expresivo de una lógica o un lenguaje formal. 
Informalmente, una bisimulación es una relación entre dos modelos que comparten ciertas características estructurales entre sí, las 
cuáles logran capturar el comportamiento de los mismos con respecto a la lógica en cuestión. 

Se presentará aquí la definición de bisimulación para la lógica \KHilogic, junto con algunos resultados deseables presentados originalmente 
en \cite{ArecesFSV25,SaraviaPHD}.

Primero, introduciremos una notación que nos será útil.

\begin{definicion}
    Sea $\model=\tup{\W,\R,\cset{\S_i}_{i \in \AGT},\V,\ACT}$ un \ults sobre $\PROP$ y $\AGT$.

    Tomemos un conjunto de planes $\pi \subseteq \ACT^*$, un conjunto de estados $U \subseteq \W$ y un agente $i \in \AGT$.
    \begin{itemize}
        \item Escribiremos $U \ultsExecStrat{\pi} T$ $\iffdef$ $U \subseteq \sexec(\pi)$ y $\R_\pi(U) \subseteq T$.
        \item Escribiremos $U \ultsExecAgi T$ $\iffdef$ existe $\pi \in \S_i$ tal que $U \ultsExecStrat{\pi} T$.
    \end{itemize}
    A su vez, decimos que $U \subseteq \W$ es \KHilogic-definible en $\modults$ si y sólo si existe una \KHilogic-fórmula $\varphi$ tal que
    $U = \truthset{\modults}{\varphi}$. Análogamente, decimos que $U \subseteq \W$ es proposicionalmente definible en $\modults$ 
    si y sólo si existe una fórmula proposicional $\varphi$ tal que $U = \truthset{\modults}{\varphi}$.
\end{definicion}

Una observación que surge de esta definición es que si un conjunto es proposicionalmente definible entonces es \KHilogic-definible, dado que 
toda fórmula proposicional sobre $\PROP$ es también una \KHilogic-fórmula.

La siguiente proposición presentada en \cite{ArecesFSV25,SaraviaPHD} nos dice que también vale la recíproca, 
es decir, que si un conjunto es \KHilogic-definible entonces es proposicionalmente definible.

\begin{proposicion}\label{prop:khi-implies-prop-definable}
    Sea $\model=\tup{\W,\R,\cset{\S_i}_{i \in \AGT},\V,\ACT}$ un \ults. Para todo $U \subseteq \W$, si $U$ es \KHilogic-definible, entonces $U$ es proposicionalmente definible.
\end{proposicion}

Ahora, estamos en condiciones de presentar la noción de bisimulación para \KHilogic.

\begin{definicion}[\KHilogic-bisimulación]\label{def:bisimulation}
    Sean $\modults$ y $\modults'$ dos \ultss, con dominios $\W$ y $\W'$ respectivamente. Sea $Z \subseteq \W \times \W'$.
    \begin{itemize}
        \item Sea $u \in W$ y $U \subseteq \W$, definimos
        \begin{nscenter}
            \begin{tabular}{@{}c@{}}
                $Z(u) := \csetsc{u' \in \W'}{(u,u') \in Z}$, \qquad $Z(U) := \bigcup_{u \in U} Z(u)$.
            \end{tabular}
        \end{nscenter}
        \item Sea $u' \in \W'$ y $U' \subseteq \W'$, definimos
        \begin{nscenter}
            \begin{tabular}{@{}c@{}}
                $Z^{-1}(u') := \csetsc{u \in \W}{(u,u') \in Z}$; \qquad $Z^{-1}(U') := \bigcup_{u' \in U'} Z^{-1}(u')$.
            \end{tabular}
        \end{nscenter}
    \end{itemize}

    Una relación binaria no vacía $Z \subseteq \W \times \W'$ es llamada una \KHilogic-bisimulación entre $\modults$ y 
    $\modults'$ si y sólo si $(w,w') \in Z$ implica lo siguiente:
    \begin{itemize}
        \item \textbf{Atom}: $\V(w)=\V'(w')$.

        \item \textbf{$\khi$-zig}: para cada conjunto \emph{proposicionalmente} definible $U \subseteq \W$, si $U \ultsExecAgi T$ para algún $T \subseteq \W$, entonces existe $T' \subseteq \W'$ tal que
        \begin{multicols}{2}
            \begin{cond-bisim}
                \item $Z(U) \ultsExecAgi T'$, 
                \item $T' \subseteq Z(T)$.
            \end{cond-bisim}
        \end{multicols}
        
        \item \textbf{$\khi$-zag}: para cada conjunto \emph{proposicionalmente} definible $U' \subseteq \W'$, si $U' \ultsExecAgi T'$ para algún $T' \subseteq \W'$, entonces existe $T \subseteq \W$ tal que
        \begin{multicols}{2}
            \begin{cond-bisim}
                \item $Z^{-1}(U') \ultsExecAgi T$,
                \item $T \subseteq Z^{-1}(T')$.
            \end{cond-bisim}
        \end{multicols}

        \item \textbf{A-zig}: para cada $u \in \W$ existe $u' \in \W'$ tal que $(u,u') \in Z$.

        \item \textbf{A-zag}: para cada $u' \in \W'$ existe $u \in \W$ tal que $(u,u') \in Z$.
    \end{itemize} 

    Escribiremos $\modults,w \bisim \modults',w'$ cuando exista una \KHilogic-bisimulación $Z$ entre
    $\modults$ y $\modults'$ tal que $(w,w') \in Z$.
\end{definicion}

Presentaremos ahora un ejemplo introducido en \cite{SaraviaPHD} con el objetivo de esclarecer cada una de las condiciones 
que se piden sobre una relación binaria para ser considerada una \KHilogic-bisimulación. 

\begin{ejemplo}\label{ejemplo:bisim}
    Sean $\modults$ y $\modults'$ dos \ults con $\AGT = \{i\}$, cuya representación gráfica está dada por (\Cref{fig:bisim}):
    \begin{figure}[h]
        \hspace*{1.65cm}
        \vspace{0.07cm}
        \begin{tikzpicture}
            \node[state] (p) {$p$};
            \node[state, right of=p, xshift=1cm] (p2) {$p$};
            \node[state, right of=p2, xshift=1cm] (q) {$q$};
            \node[state, right of=q, xshift=1cm] (q2) {$q$};
            
            \node[state, below of=p, yshift=-1cm] (p') {$p$};
            \node[state, below of=p', yshift=-0.8cm] (p2') {$p$};
            \node[state, right of=p', yshift=-0.6cm, xshift=1.3cm] (q') {$q$};
            \node[state, right of=q', xshift=1.3cm] (q2') {$q$};
            
            
            \node at ($(p)+(0,0.5)$) {$w_1$};
            \node at ($(p2)+(0,0.5)$) {$w_2$};
            \node at ($(q)+(0,0.5)$) {$w_3$};
            \node at ($(q2)+(0,0.5)$) {$w_4$};
            
            \node at ($(p')+(0,-0.6)$) {$w_2'$};
            \node at ($(p2')+(0,-0.6)$) {$w_1'$};
            \node at ($(q')+(0,-0.6)$) {$w_3'$};
            \node at ($(q2')+(0,-0.6)$) {$w_4'$};
            
            
            \path (p) edge node [above] {$a$} (p2);
            \path (p2) edge node [above] {$a$} (q);
            \path (q) edge node [above] {$a$} (q2);
            \path (p') edge node [above] {$d$} (q');
            \path (p2') edge node [above] {$d$} (q');
            \path (q') edge node [above] {$e$} (q2');

            \path[pointed] (p) edge [bend right=35] node [above] {} (p2');
            \path[pointed] (p2) edge node [above] {} (p');
            \path[pointed] (q) edge node [above] {} (q');
            \path[pointed] (q2) edge node [above] {} (q2');
            
        \end{tikzpicture}
        % \hspace{0.1cm}
        \hspace{1.4cm}
        \raisebox{2.8cm}{
            \begin{minipage}{0.45\textwidth}
                $\S_i = \left\{
                    \begin{array}{c}
                        \{aa\}
                    \end{array}
                \right\}$ \\ [1.6cm]
                $\S_i' = \left\{
                    \begin{array}{c}
                        \{de,d\}
                    \end{array}
                \right\}$
            \end{minipage}
        }
        \caption{Representación gráfica de $\modults$ y $\modults'$}
        \label{fig:bisim}
    \end{figure}

    $\modults$ está representado por el \ults con nodos $\{w_1,w_2,w_3,w_4\}$, mientras que $\modults'$ está representado 
    por el \ults con nodos $\{w_1',w_2',w_3',w_4'\}$. A su vez, $\S_i$ y $\S'_i$ son sus respectivas relaciones de indistinguibilidad 
    entre planes del agente $i$.

    Consideremos la relación binaria $Z = \{(w_j, w_j') \mid 1 \leq j \leq 4\}$, la cuál está representada gráficamente con las líneas 
    punteadas. Es posible ver que $Z$ satisface (Atom), (A-zig) y (A-zag). Analicemos como se comporta $Z$ con respecto a ($\khi$-zig), para ello 
    consideremos cada conjunto proposicionalmente definible del dominio de $\modults$:
    \begin{itemize}
       \item $U = \emptyset$. Definido por la fórmula $\varphi = \bot$.

       Notar que $U \ultsExecAgi \emptyset$, teniendo como testigo a $\pi = \{aa\}$.

       Luego, $Z(U) = \emptyset$ cumple $Z(U) \ultsExecAgi \emptyset$ a partir del testigo $\pi = \{de,d\}$ y, 
       como $Z(\emptyset) \subseteq \emptyset$, se cumple ($\khi$-zig) para dicho $U$.
       \item $U = \{w_1,w_2\}$. Definido por la fórmula $\varphi = p$.
       
       Aquí, tenemos que $U \ultsExecAgi \{w_3,w_4\}$ a partir del testigo $\pi = \{aa\}$.

       Luego, $Z(U) = \{w_1',w_2'\}$ cumple $Z(U) \ultsExecAgi \{w_3',w_4'\}$ a partir del testigo $\pi = \{de,d\}$ y, 
       como $\{w_3'w_4'\} \subseteq Z(\{w_3,w_4\})$, se cumple ($\khi$-zig) para dicho $U$.
       \item $U = \{w_3,w_4\}$. Definido por la fórmula $\varphi = q$.
       
       Notemos que no existe $\pi \in \S_i$ tal que $U \subseteq \sexec(\pi)$, luego ($\khi$-zig) se cumple trivialmente para dicho $U$.
       \item $U = \{w_1,w_2,w_3,w_4\}$. Definido por la fórmula $\varphi = p \vee q$. 
       
       Similarmente al caso analizado anteriormente, no existe $\pi \in \S_i$ tal que $U \subseteq \sexec(\pi)$, por lo que ($\khi$-zig) se 
       cumple trivialmente para dicho $U$.
    \end{itemize}
    Es posible ver que no existen más conjuntos proposicionalmente definibles en el dominio de $\modults$. Luego, $Z$ satisface ($\khi$-zig).
    
    Realizando un análisis similar sobre los conjuntos proposicionalmente definibles del dominio de $\modults'$, se puede ver que $Z$ también 
    satisface ($\khi$-zag).
    
    Por lo que podemos concluir que $Z$ es una \KHilogic-bisimulación. Más aún, analizando los elementos de $Z$, podemos decir que 
    $\modults,w_j \bisim \modults',w_j'$ para cada $1 \leq j \leq 4$. 
\end{ejemplo}


Para poder formalizar las propiedades cruciales de la bisimulación, definiremos primero la noción de 
equivalencia entre modelos con respecto a \KHilogic.

\begin{definicion}[\KHilogic-equivalencia]
    Dos \ultss punteados $\modults,w$ y $\modults',w'$ son \KHilogic-equivalentes ($\model, w \modequiv \model', w'$)
    si y sólo si, para cada $\varphi \in \KHilogic$,
    \begin{center}
        $\model, w \models \varphi$ \quad si y sólo \quad $\model', w' \models \varphi$.
    \end{center} 
\end{definicion}

Ahora, podemos presentar la correspondencia esperada entre $\bisim$ y $\modequiv$, demostrada en \cite{ArecesFSV25,SaraviaPHD}.
Usualmente nos referimos a este resultado como el Teorema de Invarianza por Bisimulación.

\begin{teorema}[\KHilogic-bisimilitud implica \KHilogic-equivalencia]\label{thm:bisim-implies-equivalence}
    Sean $\modults,w$ y $\modults',w'$ dos \ultss punteados, entonces
    \begin{center}
        $\modults,w \bisim \modults',w'$ implica $\modults,w \modequiv \modults',w'$.
    \end{center}
\end{teorema}

Este teorema caracteriza a la bisimulación como una noción que permite relacionar modelos \KHilogic-equivalentes, es decir, que la lógica no tiene una fórmula con la cuál distinguirlos,
a partir de propiedades puramente estructurales de los mismos.

En \cite[Sección 2]{FervariVQW21} se presenta un contraejemplo que atestigua que la recíproca del \Cref{thm:bisim-implies-equivalence} 
no es cierta para cualquier par de modelos. Un problema ampliamente estudiado en la literatura de las lógicas modales es el de analizar 
en qué clases de modelos vale la recíproca del teorema. Dichas clases son conocidas como clases de Hennessy-Milner.
En \cite{ArecesFSV25,SaraviaPHD}, se demuestra que la clase de \ults finitos (en el sentido de que tienen dominios finitos) es una clase 
de Hennessy-Milner, es decir, satisface la recíproca de \Cref{thm:bisim-implies-equivalence}.

\begin{definicion}\label{def:finite-ults}
    Se define la clase de \ultss finitos en su dominio (\textbf{FD}):
    \begin{center}
        \MFD $:= \{\modults \mid \modults = \tup{\W,\R,\{\S_i\}_{i\in\AGT},\V,\ACT}$ es un \ults tal que $\W$ es finito$\}$
    \end{center}

\end{definicion}


\begin{teorema}[\KHilogic-equivalencia implica \KHilogic-bisimilitud]\label{thm:finite-equivalence-implies-bisim}
    Sean $\modults,w$ y $\modults',w'$ dos \ultss punteados tal que $\modults, \modults' \in$ \MFD, entonces
    \begin{center}
        $\model,w \modequiv \model', w'$ implica $\model,w \bisim \model', w'$.
    \end{center}
\end{teorema}

Notemos que este teorema es un resultado de gran interés en términos computacionales. Como los algoritmos trabajan siempre con modelos 
finitos, este resultado nos dice que si se consigue un procedimiento efectivo que decida bisimilitud entre dos \ultss punteados, dicho 
procedimiento estará decidiendo a la vez equivalencia lógica entre los \ultss punteados en cuestión.


\section{Complejidad Computacional}

A lo largo de este trabajo, analizaremos la complejidad computacional de problemas relacionados con la noción de bisimulación. 
Por ello, vale la pena mencionar algunos resultados estudiados en \cite{ArecesFSV25,SaraviaPHD} referentes a la lógica presentada en este 
capítulo. 

A la hora de realizar un estudio computacional de una lógica, los dos problemas de decisión fundamentales a analizar son los de 
verificación de modelos (model-checking) y satisfacibilidad ($\SAT$).

El problema de model-checking se formula como "dado un modelo y una fórmula de la lógica, decidir si el modelo la satisface". Por otro lado, 
$\SAT$ se formula como "dada una fórmula de la lógica, decidir si existe un modelo que la satisfaga". 

Estos problemas han sido estudiados a lo largo de los años en numerosas lógicas y son centrales, no solo para la lógica computacional, sino que 
también para la teoría de la complejidad computacional en general. 

Para la lógica proposicional, el problema de model-checking está en la clase $\Poly$, mientras que $\SAT$ es $\NPComplete$ 
\cite[Capítulo 2, Sección 3]{Goldreich_2008}. Por otro lado, en la lógica modal básica tenemos que model-checking también 
está en $\Poly$ pero $\SAT$ es $\PSPACEComplete$ \cite[Capítulo 4]{HandbookModalLogic}. En la lógica de primer orden, el problema de 
model-checking es $\PSPACEComplete$ \cite[Capítulo 5, Sección 4]{Goldreich_2008} y $\SAT$ es indecidible. 

Una observación que surge a partir de estos resultados es que mientras mayor es el poder expresivo de la lógica, mayores son los recursos 
computacionales que se necesitan para resolver ambos problemas.

Metiéndonos de lleno en las lógicas de ``Knowing How'', en \cite{Demri_Fervari_2023} se demostró que para la lógica de ``Knowing How'' introducida 
en \cite{Wang15KH,Wang2018GoalDirectedKH} model-checking es $\PSPACEComplete$. Por otro lado, en \cite{SAT_Upper_Bound} se demostró que 
$\SAT \in \NP^{\NP}$, aunque no se ha encontrado todavía una cota inferior de su complejidad computacional.

Para el caso de la lógica \KHilogic, que estudiaremos en esta tesis, en \cite{ArecesFSV25,SaraviaPHD} se obtienen los siguientes resultados 
sobre los problemas de decisión mencionados:

\begin{teorema}\label{thm:model-checking-poly}
    El problema de model-checking para \KHilogic está en $\Poly$.
\end{teorema}

\begin{teorema}
    $\SAT$ para \KHilogic es $\NPComplete$.
\end{teorema}

Lo que nos dice que ambos problemas son potencialmente más fáciles, computacionalmente hablando, en su versión para \KHilogic con respecto a 
su versión para la lógica de ``Knowing How'' previamente mencionada. En esta tesis intentaremos completar la perspectiva algorítmica y 
computacional de \KHilogic, analizando diferentes problemas relativos a la comparación de modelos mediante bisimulación para \KHilogic. 
