\chapter{Lógica de `Knowing How' Multi-Agente Basada en Incertidumbre}
Se introducirá ahora la lógica de `Knowing-How' multi-agente basada en incertidumbre (\KHilogic), presentada en \cite{ArecesFSV25}.

\section{Sintaxis y Semántica}
Consideraremos a $\PROP$ como un conjunto no vacío de variables proposicionables con cardinalidad contable 
y a $\AGT$ como un conjunto finito no vacío de agentes.

\begin{definicion}
    El lenguaje \KHilogic está compuesto por las fórmulas dadas por la gramática:
    \begin{center}
        $\varphi ::= p \mid \neg \varphi \mid \varphi \vee \varphi \mid \varphi \wedge \varphi \mid \kh_i(\varphi,\varphi)$,
    \end{center}
    donde $p \in \PROP$ e $i \in \AGT$. Las constantes booleanas son definidas de la forma usual. Las fórmulas de la forma 
    $\kh_i(\psi, \varphi)$ deben ser leídas como ``cuando vale $\psi$, el agente $i$ sabe cómo hacer que $\varphi$ valga''.
\end{definicion}

Antes de definir la semántica de la lógica, se definirán algunos conceptos que serán útiles.

\begin{definicion}[Acciones y planes]
    Sea $\ACT$ un conjunto enumerable de nombres de acciones, y sea $\ACT^*$ el conjunto de secuencias finitas de elementos de $\ACT$. 
    A los elementos de $\ACT^*$ los llamaremos planes, siendo $\epsilon$ el plan vacío. Sea $\sigma \in \ACT^*$, denotaremos $|\sigma|$ 
    al largo de $\sigma$ (notar $|\epsilon| = 0$). Para un plan $\sigma$ y $0 \leq k \leq |\sigma|$, el plan $\sigma[..k]$ es el prefijo 
    de $\sigma$ hasta la $k-$ésima posición inclusive. Para $0 < k \leq |\sigma|$, la acción $\sigma[k]$ es la que se encuentra en la 
    $k-$ésima posición de $\sigma$.  
\end{definicion}

\begin{definicion}[Sistemas de Transiciones Etiquetados]
    Un \emph{Sistema de Transición Etiquetado (\lts, Labeled Transition System)} sobre $\PROP$ es una tupla $\modlts = \tup{\W, \R, \V, \ACT}$ donde $\W$ es un conjunto no 
    vacío de estados, $\R = \{\R_a \subseteq \W \times \W \mid a \in A$ para algún $A \subseteq \ACT\}$ es una colección de relaciones binarias
    sobre $\W$, $\V : \W \to \mathcal{P}(\PROP)$ es una función de etiquetado y $\ACT$ es un conjunto enumerable de nombres de acciones.
    Dado un \lts $\modlts$ y $w \in \W$, se llamará al par $(\modlts,w)$ un \lts punteado (los paréntesis usualmente se descartan).
\end{definicion}

\begin{definicion}[\lts basado en la incertidumbre]
    Un \emph{\lts-multi-agente basado en incertidumbre (\ults)} sobre $\PROP$ y $\AGT$ es una tupla $\model = \tup{\W,\R,\sim,\V,\ACT}$ 
    donde $\tup{\W,\R,\V,\ACT}$ es un \lts y $\sim$ asigna a cada $i \in \AGT$ una relación de equivalencia sobre un 
    $P_i \subseteq \ACT^*$, también llamada relación de indistinguibilidad. Dado un \ults $\modults$ y $w \in \W$, se llamará 
    al par $(\modults,w)$ un \ults punteado.
\end{definicion}

Se utilizarán secuencias de acciones, o planes, para representar el conocimiento de los agentes. Las fórmulas de \KHilogic 
se interpretarán sobre sistemas de transiciones etiquetados extendidos con una familia de relaciones de indistinguibilidad entre planes. 
Notar que para cada agente $i \in \AGT$, $P_i$ representará los planes que el agente tiene a su disposición. Luego $\sim_i$ 
será una relación de equivalencia sobre $P_i$, dos planes estarán relacionados cuando un agente no sepa distinguirlos.

Sea $\model = \tup{\W,\R,\sim,\V,\ACT}$ un \ults y sea $i \in \AGT$, para un plan $\sigma \in P_i$, sea $[\sigma]_i$ su clase de 
equivalencia en $\sim_i$. Notar que hay una correspondencia uno-a-uno entre cada $\sim_i$ y la partición de $P_i$ en sus correspondientes 
clases de equivalencia $\S_i := \{[\sigma]_i \mid \sigma \in P_i\}$. Por lo tanto, de ahora en adelante nos referiremos a los \ults como
una tupla $\tup{\W,\R,\{S_i\}_{i \in \AGT},\V,\ACT}$.

Dada entonces la incertidumbre de un agente sobre $\ACT^*$, sus habilidades dependerán no solo de lo que un solo plan puede lograr, sino que de 
lo que cada clase de equivalencia dentro de su relación de indistinguibilidad pueda garantizar.


\begin{definicion}
    Sea $\{\R_a \subseteq \W \times \W \mid a \in A,$ para algún $A \subseteq \ACT \}$ una colección de relaciones binarias sobre $\W$. 
    Se define $\R_\epsilon := \{(w,w) \mid w \in \W\}$ y, para $\sigma \in \ACT^*$ y $a \in \ACT$, 
    $\R_{\sigma a} := \{(w,v) \in \W \times \W \mid$ existe $u \in \W$ tal que $(w,u) \in \R_\sigma$ y $(u,v) \in \R_a \}$. 
    Luego sea $u \in \W$ y $\sigma \in \ACT^*$, se define $\R_\sigma(u) := \{v\in\W \mid (u,v) \in \R_\sigma\}$, y para $U\subseteq\W$ 
    definimos $\R_\sigma(U) := \bigcup\limits_{u \in U} \R_\sigma(u)$.

    Sea $\pi \subseteq \ACT^*$, $u \in \W$ y $U \subseteq \W$, se define
    \[
        \R_\strategy := \bigcup_{\sigma \in \strategy} \R_{\sigma},
    \qquad
        \R_{\strategy}(u) := \bigcup_{\sigma \in \strategy} \R_\sigma(u),
    \qquad
        \R_{\strategy}(U) := \bigcup_{u \in U} \R_{\strategy}(u).
    \]
\end{definicion}

La idea presentada en \cite{ArecesFSV25} consiste en que un agente sabe cómo lograr que valga $\varphi$ dado que vale $\psi$ cuando exista un 
conjunto de planes `adecuado' que pueda ejecutar desde cualquier estado en el que valga $\psi$ y que lleve sólo a estados donde valga $\varphi$. 
Una parte crucial entonces es determinar qué se considerará un conjunto de planes `adecuado'.

\begin{definicion}[Ejecutabilidad Fuerte]
    Sea $\{\R_a \subseteq \W \times \W \mid a \in A$, para algún $A \subseteq \ACT\}$ una colección de relaciones binarias. Un plan $\sigma \in \ACT^*$
    es fuertemente ejecutable ($\sexec$) en un estado $w \in \W$ si y sólo si $\R_\sigma$ está definido y, a su vez, $v \in \R_{\sigma[..k]}$ implica que 
    $\R_{\sigma[k+1]}(v) \neq \emptyset$ para cada $k \in \{0,...,|\sigma|-1\}$. Se define el conjunto $\sexec$($\sigma$) $:= \{w \in \W \mid \sigma$ es $\sexec$ en $w\}$.
    
    Un conjunto de planes $\pi \subseteq \ACT^*$ es fuertemente ejecutable en un estado $u \in \W$ si y sólo si cada $\sigma \in \pi$ es fuertemente ejecutable en $u$.
    Luego, $\sexec$($\pi$) $:= \cap_{\sigma \in \pi}$ $\sexec$($\sigma$) es el conjunto de estados en $\W$ donde $\pi$ es fuertemente ejecutable. 

\end{definicion}

Intuitivamente, ejecutabilidad fuerte sobre planes pide que cada ejecución parcial de un plan (incluyendo $\epsilon$) pueda ser completada, y, ejecutabilidad fuerte sobre
clases de planes pide que cada plan de la clase sea fuertemente ejecutable.

A partir de esta noción, las fórmulas de \KHilogic son interpretadas sobre un \ults de la siguiente manera.

\begin{definicion}[\KHilogic sobre \ultss]
    La relación $\models$ entre un \ults punteado $\modults,w$ (con $\model = \tup{\W,\R,\{S_i\}_{i\in\AGT},\V,\ACT})$ y las fórmulas de \KHilogic está definida inductivamente 
    de la siguiente forma:

    \begin{nscenter}
    \sloppy
    \begin{tabular}{@{}l@{\;\;\;}c@{\;\;\;}l@{}}
        $\modults,w \models p$ & \iffdef & $p\in\V(w)$, \\
        $\modults,w \models \neg\varphi$ & \iffdef & $\modults,w \not\models\varphi$, \\ 
        $\modults,w \models \varphi\vee\psi$ & \iffdef & $\modults,w \models \varphi \,\mbox{ o }\, \modults,w \models\psi$, \\
        $\modults,w \models \khi(\psi,\varphi)$ & \iffdef & \begin{minipage}[t]{0.68\textwidth}
                                                         existe $\strategy \in \S_i$ tal que \\
                                                         {\centering
                                                           \begin{inline-cond-kh}\item $\truthset{\modults}{\psi} \subseteq \sexec(\strategy)$ y \item $\R_\strategy(\truthset{\modults}{\psi}) \subseteq \truthset{\modults}{\varphi}$,\end{inline-cond-kh}
                                                          }
                                                       \end{minipage}
    \end{tabular}
    \end{nscenter}
    donde $\truthset{\modults}{\varphi} := \{w \in \W \mid \modults,w \models \varphi\}$. El conjunto de planes $\pi$ en la claúsula semántica de $\khi(\psi,\varphi)$ es llamado
    el testigo de $\khi(\psi,\varphi)$ en $\modults$.
\end{definicion}


Notar que $\khi(\psi,\varphi)$ vale en un estado $w$ cuando existe un conjunto de planes $\pi$ que el agente $i$ considera indistinguibles, tal que al ejecutar cualquier plan $\sigma \in \pi$ a partir de un estado donde vale $\psi$,
toda ejecución parcial puede ser completada terminando en un estado donde vale $\varphi$. También, cabe destacar que como $w$ no tiene ningún rol en la cláusula semántica de $\khi$,
dicho operador actúa \emph{globalmente}. Por lo tanto, $\truthset{\modults}{\khi(\psi,\varphi)}$ es $\W$ o $\emptyset$.


\section{Bisimulación}

La bisimulación es una herramienta crucial a la hora de analizar el poder expresivo de un lenguaje formal. Se presentará aquí su definición junto con algunos resultados 
deseables a la hora de estudiar la noción de bisimulación de una lógica modal, los cuáles fueron demostrados en \cite{ArecesFSV25}.

Primero, introduciremos una notación que nos será útil.

\begin{definicion}
    Sea $\model=\tup{\W,\R,\cset{\S_i}_{i \in \AGT},\V,\ACT}$ un \ults sobre $\PROP$ y $\AGT$.

    Tomemos un conjunto de planes $\pi \subseteq \ACT^*$, un conjunto de estados $U \subseteq \W$ y un agente $i \in \AGT$.
    \begin{itemize}
        \item Escribiremos $U \ultsExecStrat{\pi} T$ $\iffdef$ $U \subseteq \sexec(\pi)$ y $\R_\pi(U) \subseteq T$.
        \item Escribiremos $U \ultsExecAgi T$ $\iffdef$ existe $\pi \in \S_i$ tal que $U \ultsExecStrat{\pi} T$.
    \end{itemize}
    A su vez, decimos que $U \subseteq \W$ es \KHilogic-definible en $\modults$ si y sólo si existe una \KHilogic-fórmula $\varphi$ tal que
    $U = \truthset{\modults}{\varphi}$. Análogamente, decimos que $U \subseteq \W$ es proposicionalmente definible en $\modults$ 
    si y sólo si existe una fórmula proposicional $\varphi$ tal que $U = \truthset{\modults}{\varphi}$.
\end{definicion}

Una observación que surge de esta definición es que si un conjunto es proposicionalmente definible entonces es \KHilogic-definible, dado que 
toda fórmula proposicional sobre $\PROP$ es también una \KHilogic-fórmula.

La siguiente proposición presentada en \cite{ArecesFSV25} nos dice que también vale la recíproca, 
es decir, que si un conjunto es \KHilogic-definible entonces es proposicionalmente definible.

\begin{proposicion}\label{prop:khi-implies-prop-definable}
    Sea $\model=\tup{\W,\R,\cset{\S_i}_{i \in \AGT},\V,\ACT}$ un \ults. Para todo $U \subseteq \W$, si $U$ es \KHilogic-definible, entonces $U$ es proposicionalmente definible.
\end{proposicion}

Ahora si, presentamos la noción de bisimulación.

\begin{definicion}[\KHilogic-bisimulación]\label{def:bisimulation}
    Sean $\modults$ y $\modults'$ dos \ultss, con dominios $\W$ y $\W'$ respectivamente. Sea $Z \subseteq \W \times \W$.
    \begin{itemize}
        \item Sea $u \in W$ y $U \subseteq \W$, definimos
        \begin{nscenter}
            \begin{tabular}{@{}c@{}}
                $Z(u) := \csetsc{u' \in \W'}{(u,u') \in Z}$, \qquad $Z(U) := \bigcup_{u \in U} Z(u)$.
            \end{tabular}
        \end{nscenter}
        \item Sea $u' \in \W'$ y $U' \subseteq \W'$, definimos
        \begin{nscenter}
            \begin{tabular}{@{}c@{}}
                $Z^{-1}(u') := \csetsc{u \in \W}{(u,u') \in Z}$; \qquad $Z^{-1}(U') := \bigcup_{u' \in U'} Z^{-1}(u')$.
            \end{tabular}
        \end{nscenter}
    \end{itemize}

    Una relación binaria no vacía $Z \subseteq \W \times \W'$ es llamada una \KHilogic-bisimulación entre $\modults$ y 
    $\modults'$ si y sólo si $(w,w') \in Z$ implica lo siguiente:
    \begin{itemize}
        \item \textbf{Atom}: $\V(w)=\V'(w')$.

        \item \textbf{$\khi$-zig}: para cada \emph{proposicionalmente} definible $U \subseteq \W$, si $U \ultsExecAgi T$ para algún $T \subseteq \W$, entonces existe $T' \subseteq \W'$ tal que
        \begin{multicols}{2}
            \begin{cond-bisim}
                \item $Z(U) \ultsExecAgi T'$, 
                \item $T' \subseteq Z(T)$.
            \end{cond-bisim}
        \end{multicols}
        
        \item \textbf{$\khi$-zag}: para cada \emph{proposicionalmente} definible $U' \subseteq \W'$, si $U' \ultsExecAgi T'$ para algún $T' \subseteq \W'$, entonces existe $T \subseteq \W$ tal que
        \begin{multicols}{2}
            \begin{cond-bisim}
                \item $Z^{-1}(U') \ultsExecAgi T$,
                \item $T \subseteq Z^{-1}(T')$.
            \end{cond-bisim}
        \end{multicols}

        \item \textbf{A-zig}: para cada $u \in \W$ existe $u' \in \W'$ tal que $(u,u') \in Z$.

        \item \textbf{A-zag}: para cada $u' \in \W'$ existe $u \in \W$ tal que $(u,u') \in Z$.
    \end{itemize} 

    Escribiremos $\modults,w \bisim \modults',w'$ cuando exista una \KHilogic-bisimulación $Z$ entre
    $\modults$ y $\modults'$ tal que $(w,w') \in Z$.

\end{definicion}

Para poder formalizar las propiedades cruciales de una bisimulación, definiremos primero la noción de 
equivalencia entre modelos con respecto a \KHilogic.

\begin{definicion}[\KHilogic-equivalencia]
    Dos \ultss punteados $\modults,w$ y $\modults',w'$ son \KHilogic-equivalentes ($\model, w \modequiv \model', w'$)
    si y sólo si, para cada $\varphi \in \KHilogic$,
    \begin{center}
        $\model, w \models \varphi$ \quad si y sólo \quad $\model', w' \models \varphi$.
    \end{center} 
\end{definicion}

Ahora podemos presentar la correspondencia esperada entre $\bisim$ y $\modequiv$, demostrada en \cite{ArecesFSV25}.
Usualmente nos referimos a este resultado como el Teorema de Invarianza para Bisimulación.

\begin{teorema}[\KHilogic-bisimilitud implica \KHilogic-equivalencia]\label{thm:bisim-implies-equivalence}
    Sean $\modults,w$ y $\modults',w'$ dos \ultss punteados, entonces
    \begin{center}
        $\modults,w \bisim \modults',w'$ implica $\modults,w \modequiv \modults',w'$.
    \end{center}
\end{teorema}

Este teorema caracteriza a la bisimulación como una noción que permite relacionar modelos \KHilogic-equivalentes, es decir, que la lógica no tiene una fórmula con la cuál distinguirlos,
a partir de propiedades puramente estructurales de los mismos.

En \cite[Sección 2]{FervariVQW21} se presenta un contraejemplo que atestigua que la recíproca del teorema no es cierta para cualquier par de modelos.
Un problema ampliamente estudiado en la literatura de las lógicas modales es el de analizar en qué clases de modelos vale la recíproca del teorema.
En \cite{ArecesFSV25}, se demuestra que en el caso finito se satisface la recíproca.

Diremos que $\modults$, un \ults, es finito si y sólo si cada una de sus componentes tiene cardinalidad finita. Análogamente, diremos que 
($\modults,w$), un \ults punteado, es finito si y sólo $\modults$ es finito. 

\begin{teorema}[\KHilogic-equivalencia implica \KHilogic-bisimilitud]\label{thm:finite-equivalence-implies-bisim}
    Sean $\modults,w$ y $\modults',w'$ dos \ultss punteados finitos, entonces
    \begin{center}
        $\model,w \modequiv \model', w'$ implica $\model,w \bisim \model', w'$.
    \end{center}
\end{teorema}

Notemos que este teorema es un gran resultado en términos computacionales. Como los algoritmos trabajan siempre con modelos finitos, este resultado
nos dice que si se consigue un procedimiento efectivo que decida bisimilitud entre dos \ultss punteados, dicho procedimiento estará decidiendo a la vez
equivalencia lógica entre los \ultss punteados en cuestión.


\section{Complejidad Computacional}