% La equivalencia de modelos es un concepto crucial a la hora de entender el poder expresivo de un lenguaje 
% formal. En el contexto de las lógicas modales, la noción de bisimulación surge para caracterizar esta capacidad descriptiva a partir 
% de propiedades estructurales de los modelos considerados. 
% % Dos modelos serán bisimilares cuando sepan imitarse mutuamente en relación a cierto comportamiento relevante para la lógica en cuestión.
% En este trabajo investigamos aspectos computacionales de la noción de bisimulación para una lógica epistémica que busca modelar 
% el concepto de ``saber cómo'' (Lógica de \textit{Knowing How} Basada en la Incertidumbre). 

% En primer lugar, proponemos una nueva definición de bisimulación con el objetivo de otorgarle una naturaleza algorítmica y una mejor 
% alineación con las nociones clásicas. Además, demostramos que nuestra propuesta satisface las propiedades esperadas de 
% correctitud.

% Luego, estudiamos problemas computacionales asociados a esta noción. Determinamos que el problema de verificar 
% si una relación binaria es una bisimulación es \coNPComplete. A continuación, extendemos este resultado al problema de decidir la existencia de 
% una bisimulación entre dos modelos y a la versión punteada del mismo.

% Finalmente, abordamos el problema de la minimización de modelos a partir de la definición de contracciones por bisimulación. Presentamos 
% dos propuestas de contracción: la primera basada en el estudio de la máxima autobisimulación y la segunda formulada a partir de la propuesta 
% clásica de la Lógica Modal Básica. También, mostramos que ambas contracciones son computables en tiempo polinomial.
\textbf{Resumen:}\hspace{0.2em} 
La equivalencia de modelos es un concepto crucial a la hora de
entender el poder expresivo de un lenguaje formal. En el contexto de las lógicas
modales, la noción de bisimulación surge para caracterizar esta capacidad descriptiva
a partir del análisis de propiedades estructurales de los modelos considerados. Esto permite, por ejemplo, utilizar bisimulaciones para encontrar 
descripciones minimales, indistinguibles desde el punto de vista de la lógica, de ciertos escenarios.

En este trabajo investigamos aspectos computacionales de la noción de bisimulación para una lógica epistémica que busca modelar el 
“saber cómo”, a la que nos referimos como una Lógica de \textit{Knowing How} Basada en Incertidumbre. Esta terminología se 
debe a que el conocimiento de los agentes está sujeto a su incertidumbre sobre los efectos que producen los planes de los que disponen.
% Nos referiremos a la misma como una Lógica de Knowing How Basada en Incertidumbre, ya que el conocimiento 
% de los agentes está sujeto a su incertidumbre sobre los planes que tienen a disposición. 
En primer lugar, proponemos una nueva definición de bisimulación con el objetivo de otorgarle una naturaleza algorítmica y una mejor 
alineación con las nociones
clásicas. Además, demostramos que nuestra propuesta es adecuada, en el sentido de que captura la expresividad de la lógica en cuestión. 

Luego, estudiamos problemas computacionales asociados a esta noción. Primero, determinamos que el problema de verificar si una relación 
binaria es una bisimulación es
\coNPComplete. A continuación, extendemos este resultado al problema de decidir la
existencia de una bisimulación entre dos modelos y a la versión punteada del mismo.
Finalmente, abordamos el problema de la minimización de modelos a partir de
la definición de contracciones por bisimulación. Presentamos dos propuestas de contracción: la primera basada en el estudio de la 
máxima autobisimulación y la segunda formulada a partir de la propuesta clásica de la Lógica Modal Básica. También,
mostramos que ambas contracciones son computables en tiempo polinomial.

\medskip

\textbf{Palabras Clave:}\hspace{0.2em} Bisimulación, Saber cómo, Complejidad, Comparación de modelos.

\blankpagew

% Model equivalence is a crucial concept for understanding the expressive power of a formal language. In the context of 
% modal logics, the notion of bisimulation arises to characterize this descriptive capacity by analyzing structural properties of the 
% models considered. In this work, we investigate computational aspects of bisimulation for an epistemic logic that models the 
% concept of ``knowing how'' (Uncertainty-Based \textit{Knowing How} Logic).

% First, we propose a new definition of bisimulation with the aim of providing it with an algorithmic nature and a better alignment with 
% the classical notions. Moreover, we prove that our proposal satisfies the expected adequacy properties.

% Then, we study computational problems related to this notion. We determine that the problem of verifying whether a binary relation is 
% a bisimulation is \coNPCompleteEng. Next, we extend this result to the problem of deciding whether there exists a bisimulation between two 
% models and to the pointed version of this problem.

% Finally, we address the problem of model minimization by defining bisimulation contractions. We present two contraction proposals: 
% the first one based on the study of the maximal autobisimulation, and the second one built on the classical proposal of 
% Basic Modal Logic. In addition, we show that both contractions can be computed in polynomial time.
\textbf{Abstract:}\hspace{0.2em} 
Model equivalence is a crucial concept for understanding the expressive power of a formal language. 
In modal logic, bisimulation plays a central role in characterizing this expressive capacity by analyzing the structural properties 
of the models under consideration. For instance, it enables us to find minimal (but yet equivalent from the logic's point 
of view) representations of certain scenarios.

In this work, we investigate computational aspects of bisimulation for an epistemic logic that captures the notion of “knowing how”, 
to which we refer as an Uncertainty-Based \textit{Knowing How} Logic. The latter terminology comes from the fact that the knowledge of an agent 
is subject to her uncertainty about the effects of the plans she has at her disposal. We begin by proposing a new definition of 
bisimulation aimed at giving the concept an algorithmic nature and achieving a better alignment with classical notions. We also 
show that our definition satisfies the expected adequacy properties, in the sense that it captures the distinctive expressivity 
features of the logic we consider. 

We then study several computational problems related to this notion. In particular, we establish that verifying whether a 
given binary relation is a bisimulation is \coNPCompleteEng. We extend this result to the problem of deciding whether there exists a 
bisimulation between two models, including its pointed variant. Finally, we examine model minimization through the lens of bisimulation 
contractions. We present two contraction methods: one based on the maximal autobisimulation and another inspired by the classical 
contraction for Basic Modal Logic. Moreover, we show that both contractions can be computed in polynomial time.

\medskip

\textbf{Keywords:}\hspace{0.2em} Bisimulation, Knowing how, Complexity, Model Comparison.