\textbf{Resumen:}\hspace{0.2em} La equivalencia de modelos es un concepto crucial a la hora de entender el poder expresivo de un lenguaje 
formal. En el contexto de las lógicas modales, la noción de bisimulación surge para caracterizar esta capacidad descriptiva a partir 
de características estructurales de los modelos considerados. 
% Dos modelos serán bisimilares cuando sepan imitarse mutuamente en relación a cierto comportamiento relevante para la lógica en cuestión.
En este trabajo investigamos aspectos computacionales de la noción de bisimulación para una lógica epistémica que busca modelar 
el concepto de ``saber cómo'' (Lógica de \textit{Knowing How} Basada en Incertidumbre). 

En primer lugar, proponemos una nueva definición de bisimulación con el objetivo de otorgarle una naturaleza algorítmica y una mejor 
alineación con las nociones clásicas. Además, demostramos que nuestra propuesta satisface las propiedades esperadas de 
correctitud.

Luego, estudiamos problemas computacionales asociados a esta noción. Determinamos que el problema de verificar 
si una relación binaria es una bisimulación es \coNPComplete. A continuación, extendemos este resultado al problema de decidir la existencia de 
una bisimulación entre dos modelos y a la versión punteada del mismo.

Finalmente, abordamos el problema de la minimización de modelos a partir de la definición de contracciones por bisimulación. Presentamos 
dos propuestas de contracción: la primera basada en el estudio de la máxima autobisimulación y la segunda formulada a partir de la propuesta 
clásica de la Lógica Modal Básica. También mostramos que ambas contracciones son computables en tiempo polinomial.

\medskip

\textbf{Palabras Clave:}\hspace{0.2em} Bisimulación, Saber cómo, Complejidad, Comparación de modelos.