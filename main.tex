\documentclass[a4paper,10pt]{article}
\pagestyle{plain}

% this is for dealing with the Too many math alphabets used in version normal" compiling error
\newcommand\hmmax{0} 
\newcommand\bmmax{0}
%%%
\usepackage[utf8]{inputenc}
\usepackage[T1]{fontenc}
\usepackage[x11names,dvipsnames]{xcolor}
\usepackage{amssymb}
\usepackage{amsmath}
\usepackage{verbatim}
\usepackage{booktabs}
\usepackage{multicol}
\usepackage{xspace}
\usepackage{underscore}
\usepackage{upgreek}
\usepackage[textsize=scriptsize]{todonotes}
\usepackage{lineno}
\usepackage{pxfonts}
\usepackage{algorithm}
\usepackage{algpseudocode}
\usepackage{bm}
\usepackage{mathtools}
\usepackage[normalem]{ulem}
\usepackage[draft]{fixme}
\usepackage{comment}
\usepackage{url}

\usepackage[font=small,skip=0pt]{caption}

%------------------------------------------------------------------------------------------------


\usepackage{hyperref}

\hypersetup{
  %pdftitle          = {\titulo},
  pdfstartview      = {FitH},
  pdfpagelayout     = {OneColumn},
  plainpages        = false,            % page anchors with formatted form of the page number (i.e., different anchors for pages 'ii' and '2')
  bookmarksnumbered = {true},
  naturalnames      = {true},
  colorlinks        = {true},           % ligas con color; si es falso pone un recuadro que no aparece en la impresion
  linkcolor         = blue,             % color de las ligas dentro del documento (tabla de contenido, secciones, etc.)
  anchorcolor       = red,              % ?
  urlcolor          = NavyBlue,         % color de las ligas externas
  citecolor         = BrickRed          % color de las referencias
}

%------------------------------------------------------------------------------------------------

\usepackage[spanish]{babel}

\usepackage{macros}

%Overfull
\overfullrule=5pt
\begin{document}

\title{Trabajo Final de Licenciatura en Ciencias de la Computación}

\author{Antonio Mondejar}

\maketitle

\section{Marco Teórico: Lógica `Knowing How' multi-agente}
    \chapter{Lógica de `Knowing How' Multi-Agente Basada en Incertidumbre}
Introduciremos un enfoque para modelar el concepto de saber cómo, presentado en \cite{ArecesFSV25,SaraviaPHD}, el cuál surge a partir de la 
lógica de `Knowing How' previamente introducida en \cite{Wang15KH, Wang2018GoalDirectedKH}. En este enfoque, el conocimiento de un agente está 
representado en Sistemas de Transición Etiquetados (\lts por sus siglas en inglés), donde las transiciones etiquetadas representan el cambio de estado 
que produce cada acción que el agente puede realizar. A su vez, las habilidades de un agente estarán dadas no solo por las acciones básicas que tiene a 
su disposición en un determinado estado, sino que también por la composición de las mismas, es decir, cada camino en el \lts. A las composiciones de acciones las 
llamaremos planes. Se considera que un agente `conoce' un plan en un determinado estado cuando cada ejecución parcial del plan a partir de dicho estado se puede 
completar.

La lógica de `Knowing How' multi-agente basada en incertidumbre toma este enfoque como punto de partida. La principal motivación por la que 
surge esta propuesta es que la representación del conocimiento mencionada anteriormente supone que el agente cuenta con cierto nivel de omnisciencia, 
pues se asume que un agente conoce todo plan que pueda inferirse de su \lts asociado y que, además, sabe distinguir qué plan es adecuado entre los que tiene en su 
disposición, en el sentido de que para todo par de planes es capaz de reconocer si provocan resultados diferentes.
Para clarificar un poco más este argumento, veamos el siguiente ejemplo presentado en \cite{ArecesFSV25,SaraviaPHD}.

Consideremos un agente que quiere hacer una torta. El agente cuenta con la habilidad de realizar cualquiera de los cuatro métodos de mezcla diferentes 
(de batido, de la crema, de fusión y de frotamiento) y hasta podría ser capaz de identificarlos como acciones distintas entre sí. Sin embargo, 
el agente podría no tener noción de cuáles son los efectos de utilizar un método u otro, es decir, podría no distinguir que los mismos producen resultados 
diferentes. En este caso, podríamos considerar que el agente no sabe cómo hacer una torta, pues en los casos en los que selecciona el método correcto obtiene 
un buen resultado y en los demás casos no. 

Incluso, podríamos considerar más situaciones en las que la indistinguibilidad del agente no sea únicamente en acciones sino que en planes en su totalidad. Por ejemplo, 
el agente sabe la diferencia entre añadir leche y añadir harina a la mezcla, pero podría darse el caso en el que no sepa distinguir cuál es el orden adecuado en el que 
realizar ambas acciones. Incluso, podría ocurrir que el agente no sepa distinguir planes de distinta longitud, por ejemplo, en el proceso de horneado de la torta el agente 
debe eventualmente abrir el horno para verificar si el proceso de horneado ya está completo, pero hacerlo con demasiada frecuencia podría afectar al resultado de la torta.

Es así como estos ejemplos motivan una representación más general sobre el conocimiento de un agente, tomando en consideración no solo las habilidades con las que dispone sino 
también su capacidad de distinguir diferentes planes. A su vez, esto permite una representación para múltiples agentes, en la que todos los agentes disponen de las mismas acciones 
pero cada uno tiene su propia interpretación de los efectos de las mismas, así como su propia habilidad para distinguir distintos planes.

Así, la estrategia utilizada en \cite{ArecesFSV25,SaraviaPHD} consiste en fijar un conjunto de agentes $\AGT$ y dotar a los \lts con una 
relación de indistinguibilidad sobre planes para cada agente.

Ahora si, introduciremos la lógica de `Knowing-How' multi-agente basada en incertidumbre (\KHilogic) definiendo su sintaxis y semántica, su noción de bisimulación y por último mencionaremos 
algunos resultados relacionados a su complejidad computacional con respecto a los problemas de verificación de modelos y satisfacibilidad.

\section{Sintaxis y Semántica}
Consideraremos a $\PROP$ como un conjunto no vacío de variables proposicionables con cardinalidad contable 
y a $\AGT$ como un conjunto finito no vacío de nombres de agentes.

\begin{definicion}
    El lenguaje \KHilogic está compuesto por las fórmulas dadas por la gramática:
    \begin{center}
        $\varphi ::= p \mid \neg \varphi \mid \varphi \vee \varphi \mid \varphi \wedge \varphi \mid \kh_i(\varphi,\varphi)$,
    \end{center}
    donde $p \in \PROP$ e $i \in \AGT$. Las constantes booleanas son definidas de la forma usual. Las fórmulas de la forma 
    $\kh_i(\psi, \varphi)$ deben ser leídas como ``cuando vale $\psi$, el agente $i$ sabe cómo hacer que $\varphi$ valga''.
\end{definicion}

Formalizaremos matemáticamente algunos de los conceptos discutidos en la presentación de este capítulo. Como mencionamos, se 
utilizarán secuencias de acciones, o planes, para representar el conocimiento de los agentes. Las fórmulas de \KHilogic se interpretarán 
sobre sistemas de transiciones etiquetados extendidos con una familia de relaciones de indistinguibilidad entre planes.

\begin{definicion}[Acciones y planes]
    Sea $\ACT$ un conjunto enumerable de nombres de acciones, y sea $\ACT^*$ el conjunto de secuencias finitas sobre $\ACT$. 
    A los elementos de $\ACT^*$ los llamaremos planes, siendo $\epsilon$ el plan vacío. Sea $\sigma \in \ACT^*$, denotaremos $|\sigma|$ 
    al largo de $\sigma$ (notar $|\epsilon| = 0$). Para un plan $\sigma$ y $0 \leq k \leq |\sigma|$, el plan $\sigma[..k]$ es el prefijo 
    de $\sigma$ hasta la $k-$ésima posición inclusive. Para $0 < k \leq |\sigma|$, la acción $\sigma[k]$ es la que se encuentra en la 
    $k-$ésima posición de $\sigma$.  
\end{definicion}

\begin{ejemplo}
    Consideremos un robot moviendose en una grilla, el cuál tiene a su disposición la opción de moverse a cualquiera de sus 4 casilleros vecinos; es decir,
    puede moverse a la izquierda ($`L'$), a la derecha ($`R'$), hacia arriba ($`U'$) o hacia abajo ($`D'$). En este contexto, el conjunto de acciones básicas 
    $\ACT$ estaría definido como $\ACT = \{L,R,U,D\}$, y $\ACT^*$ sería el conjunto de secuencias finitas sobre $\ACT$, el cuál representa los distintos 
    caminos que el robot podría tomar en la grilla. Por ejemplo, $`LLLDRR'$ y $`UURRU'$ están en $\ACT^*$. 

    Luego, a partir de la definición dada, tenemos:
    \begin{itemize}
        \item $(LLDDRRUU)[$..$0] = \epsilon$.
        \item $(LLDDRRUU)[$..$3] = LLD$.
        \item $(LLDDRRUU)[$..$7] = LLDDRRU$.
        \item $(LLDDRRUU)[5] = R$.
        \item $(LLDDRRUU)[8] = U$.
    \end{itemize}
\end{ejemplo}


\begin{definicion}[Sistemas de Transiciones Etiquetados]
    Un \emph{Sistema de Transición Etiquetado (\lts, Labeled Transition System)} sobre $\PROP$ es una tupla $\modlts = \tup{\W, \R, \V, \ACT}$ 
    donde $\W$ es un conjunto no vacío de estados, $\R = \{\R_a \subseteq \W \times \W \mid a \in A$ para algún $A \subseteq \ACT\}$ es 
    una colección de relaciones binarias sobre $\W$, $\V : \W \to \mathcal{P}(\PROP)$ es una función de etiquetado y $\ACT$ es un 
    conjunto enumerable de nombres de acciones.
\end{definicion}

Gráficamente, un \lts se representa como un grafo dirigido etiquetado, donde los nodos serán los elementos de $\W$, cada uno marcado con las variables proposicionales 
dadas por la función de etiquetado $\V$ y las aristas entre dos nodos estarán dadas por los pares de $\R_a$ para cada $a \in \ACT$.

\begin{ejemplo}\label{ejemplo:lts}
    Sea $\modlts = \tup{\W, \R, \V, \ACT}$ un \lts con:
    
    \begin{itemize}
        \item $\W = \{w_1, w_2, w_3, w_4\}$.
        \item $\R = \{R_a, \R_b\}$ con $\R_a = \{(w_1,w_3), (w_1,w_4), (w_2,w_3)\}$ y $\R_b = \{(w_3,w_4)\}$.
        \item $\V = \{(w_1, \{p\}), (w_2, \{p\}), (w_3, \{q\}), (w_4, \{r\})\}$.
        \item $\ACT = \{a, b\}$.
    \end{itemize}

    Su representación gráfica es (\Cref{fig:lts}):
    \begin{figure}[h]
        \centering
        \begin{tikzpicture}
            \node[state] (p) {$p$};
            \node[state, below of=p, yshift=-1.5cm] (p2) {$p$};
            \node[state, right of=p, yshift=-1cm, xshift=1.5cm] (q) {$q$};
            \node[state, right of=q, xshift=1.5cm] (r) {$r$};
            
            \node at ($(p)+(0,0.5)$) {$w_1$};
            \node at ($(r)+(0,0.5)$) {$w_2$};
            \node at ($(q)+(0,0.5)$) {$w_3$};
            \node at ($(p2)+(0,-0.6)$) {$w_4$};

            \path (p) edge node [above] {$a$} (q)
                      edge [bend left] node [above] {$a$} (r);
            \path (p2) edge node [above] {$a$} (q);
            \path (q) edge node [above] {$b$} (r);
        \end{tikzpicture}
        \caption{Representación gráfica de $\modlts$}
        \label{fig:lts}
    \end{figure}

\end{ejemplo}


\begin{definicion}[\lts basado en incertidumbre]
    Un \emph{\lts-multi-agente basado en incertidumbre (\ults)} sobre $\PROP$ y $\AGT$ es una tupla $\model = \tup{\W,\R,\sim,\V,\ACT}$ 
    donde $\tup{\W,\R,\V,\ACT}$ es un \lts y $\sim$ asigna a cada $i \in \AGT$ una relación de equivalencia sobre un conjunto de planes
    $P_i \subseteq \ACT^*$, también llamada relación de indistinguibilidad. Dado un \ults $\modults$ y $w \in \W$, se llamará 
    al par $(\modults,w)$ un \ults punteado y, usualmente, los paréntesis serán omitidos.
\end{definicion}

Notar que para cada agente $i \in \AGT$, $P_i$ representará los planes que el agente tiene a su disposición. 
Luego $\sim_i$ será una relación de equivalencia sobre $P_i$, donde dos planes estarán relacionados cuando un agente no sepa distinguirlos.

Sea $\model = \tup{\W,\R,\sim,\V,\ACT}$ un \ults y sea $i \in \AGT$, para un plan $\sigma \in P_i$, sea $[\sigma]_i$ su clase de 
equivalencia en $\sim_i$. Notar que hay una correspondencia uno-a-uno entre cada $\sim_i$ y la partición de $P_i$ en sus correspondientes 
clases de equivalencia $\S_i := \{[\sigma]_i \mid \sigma \in P_i\}$. Por lo tanto, de ahora en adelante nos referiremos a los \ults como
una tupla $\tup{\W,\R,\{\S_i\}_{i \in \AGT},\V,\ACT}$.

La representación gráfica de los \ultss será similar a la presentada para los \ltss, con la inclusión de la relación de indistinguibilidad 
entre planes de cada agente.

\begin{ejemplo}\label{ejemplo:ults}
    Siguiendo el \Cref{ejemplo:lts} y considerando $\AGT = \{i\}$, la representación gráfica 
    de $\model=\tup{\W,\R,\{\S_i\}_{i\in\AGT},\V,\ACT}$ es (\Cref{fig:ults}):
    \begin{figure}[h]
        \centering
            \begin{tikzpicture}
                \node[state] (p) {$p$};
                \node[state, below of=p, yshift=-1.5cm] (p2) {$p$};
                \node[state, right of=p, yshift=-1cm, xshift=1.5cm] (q) {$q$};
                \node[state, right of=q, xshift=1.5cm] (r) {$r$};
                
                \node at ($(p)+(0,0.5)$) {$w_1$};
                \node at ($(r)+(0,0.5)$) {$w_2$};
                \node at ($(q)+(0,0.5)$) {$w_3$};
                \node at ($(p2)+(0,-0.6)$) {$w_4$};

                \path (p) edge node [above] {$a$} (q)
                edge [bend left] node [above] {$a$} (r);
                \path (p2) edge node [above] {$a$} (q);
                \path (q) edge node [above] {$b$} (r);
            \end{tikzpicture}
            \hspace{1cm}
            \raisebox{1.8cm}{
                \begin{minipage}{0.45\textwidth}
                    $\S_i = \left\{
                        \begin{array}{c}
                            \{ab,a\}, \{b\}
                        \end{array}
                    \right\}$
                \end{minipage}
            }
            \caption{Representación gráfica de $\modults$}
            \label{fig:ults}
    \end{figure}

    En este caso, los planes que el agente $i$ tiene a su disposición son $\{ab,a,b\}$ y, lo que nos dice su relación de 
    indistinguibilidad $\S_i$ es que a los planes $ab$ y $a$ no sabe distinguirlos.
\end{ejemplo}

Dada entonces la incertidumbre de un agente sobre $\ACT^*$, sus habilidades dependerán no solo de lo que un solo plan puede lograr, sino que de 
lo que cada clase de equivalencia dentro de su relación de indistinguibilidad pueda garantizar.

\begin{definicion}
    Sea $\{\R_a \subseteq \W \times \W \mid a \in A,$ para algún $A \subseteq \ACT \}$ una colección de relaciones binarias sobre $\W$. 
    Se define $\R_\epsilon := \{(w,w) \mid w \in \W\}$ y, para $\sigma \in \ACT^*$ y $a \in \ACT$, 
    $\R_{\sigma a} := \{(w,v) \in \W \times \W \mid$ existe $u \in \W$ tal que $(w,u) \in \R_\sigma$ y $(u,v) \in \R_a \}$. 
    Luego sea $u \in \W$ y $\sigma \in \ACT^*$, se define $\R_\sigma(u) := \{v\in\W \mid (u,v) \in \R_\sigma\}$, y para $U\subseteq\W$ 
    definimos $\R_\sigma(U) := \bigcup\limits_{u \in U} \R_\sigma(u)$.

    Sea $\pi \subseteq \ACT^*$, $u \in \W$ y $U \subseteq \W$, se define
    \[
        \R_\strategy := \bigcup_{\sigma \in \strategy} \R_{\sigma},
    \qquad
        \R_{\strategy}(u) := \bigcup_{\sigma \in \strategy} \R_\sigma(u),
    \qquad
        \R_{\strategy}(U) := \bigcup_{u \in U} \R_{\strategy}(u).
    \]
\end{definicion}

\begin{ejemplo}
    Siguiendo con el \ults presentado en \Cref{ejemplo:ults}, tenemos:
    \begin{itemize}
        \item $\R_{ab} = \{(w_1,w_4), (w_2, w_4)\}$.
        \item $\R_{ab}(w_1) = \{w_4\}$, \quad $\R_{ab}(w_3) = \emptyset$. 
        \item $\R_{a}(\{w_1,w_2\}) = \R_a(w_1) \cup \R_a(w_2) = \{w_3,w_4\} \cup \{w_3\} = \{w_3, w_4\}$.
        \item $\R_{\{ab,b\}} = \R_{ab} \cup \R_b = \{(w_1,w_4),(w_2,w_4),(w_3,w_4)\}$.
        \item $\R_{\{ab,b\}}(w_1) = \R_{ab}(w_1) \cup \R_b(w_1) = \{w_3,w_4\} \cup \emptyset = \{w_3,w_4\}$.
        \item $\R_{\{ab,b\}}(\{w_1,w_3\}) = \R_{\{ab,b\}}(w_1) \cup \R_{\{ab,b\}}(w_3) = \{w_3,w_4\} \cup \{w_4\} = \{w_3,w_4\}$.
    \end{itemize}
\end{ejemplo}


La idea presentada en \cite{ArecesFSV25,SaraviaPHD} consiste en que un agente sabe cómo lograr que valga $\varphi$ dado que vale $\psi$ cuando exista un 
conjunto de planes `adecuado' que pueda ejecutar desde cualquier estado en el que valga $\psi$ y que lleve sólo a estados donde valga $\varphi$. 
Una parte crucial entonces es determinar qué se considerará un conjunto de planes `adecuado'.

\begin{definicion}[Ejecutabilidad Fuerte]
    Sea $\{\R_a \subseteq \W \times \W \mid a \in A$, para algún $A \subseteq \ACT\}$ una colección de relaciones binarias. Un plan $\sigma \in \ACT^*$
    es fuertemente ejecutable ($\sexec$) en un estado $w \in \W$ si y sólo si $\R_\sigma$ está definido y, a su vez, $v \in \R_{\sigma[..k]}(w)$ implica que 
    $\R_{\sigma[k+1]}(v) \neq \emptyset$ para cada $k \in \{0,...,|\sigma|-1\}$. Se define el conjunto $\sexec$($\sigma$) $:= \{w \in \W \mid \sigma$ es $\sexec$ en $w\}$.
    
    Un conjunto de planes $\pi \subseteq \ACT^*$ es fuertemente ejecutable en un estado $u \in \W$ si y sólo si cada $\sigma \in \pi$ es fuertemente ejecutable en $u$.
    Por último, $\sexec$($\pi$) $:= \cap_{\sigma \in \pi}$ $\sexec$($\sigma$) es el conjunto de estados en $\W$ donde $\pi$ es fuertemente ejecutable. 
\end{definicion}

Intuitivamente, ejecutabilidad fuerte sobre planes pide que cada ejecución parcial de un plan (incluyendo $\epsilon$) pueda ser completada, y, ejecutabilidad fuerte sobre
clases de planes pide que cada plan de la clase sea fuertemente ejecutable.

\begin{ejemplo}
    Si consideramos el \ults presentado en \Cref{ejemplo:ults} entonces tenemos que:
    \begin{itemize}
        \item $\sexec(a) = \{w_1,w_2\}$.
        \item $\sexec(ab) = \{w_2\}$.

        Notar que, a pesar de que $\R_{ab}(w_1) \neq \emptyset$, el plan $ab$ no es fuertemente ejecutable en $w_1$ 
        dado que existe una ejecución parcial que no puede ser completada, pues notar que al tomar la arista 
        $(w_1,w_4)$ con etiqueta $a$ no existe arista con etiqueta $b$ desde $w_4$ para completar el camino.

        \item $\sexec(\{a,ab\}) = \sexec(a) \cap \sexec(ab) = \{w_1,w_2\} \cap \{w_2\} = \{w_2\}$.
    \end{itemize}
\end{ejemplo}


Ahora si, estamos en condiciones de presentar la relación de satisfacibilidad que relaciona \ultss punteados con formulas de \KHilogic. 

\begin{definicion}[\KHilogic sobre \ultss]
    La relación $\models$ entre un \ults punteado $\modults,w$ (con $\model = \tup{\W,\R,\{S_i\}_{i\in\AGT},\V,\ACT})$ y las fórmulas de \KHilogic está definida inductivamente 
    de la siguiente forma:

    \begin{nscenter}
    \sloppy
    \begin{tabular}{@{}l@{\;\;\;}c@{\;\;\;}l@{}}
        $\modults,w \models p$ & \iffdef & $p\in\V(w)$, \\
        $\modults,w \models \neg\varphi$ & \iffdef & $\modults,w \not\models\varphi$, \\ 
        $\modults,w \models \varphi\vee\psi$ & \iffdef & $\modults,w \models \varphi \,\mbox{ o }\, \modults,w \models\psi$, \\
        $\modults,w \models \khi(\psi,\varphi)$ & \iffdef & \begin{minipage}[t]{0.68\textwidth}
                                                         existe $\strategy \in \S_i$ tal que \\
                                                         {\centering
                                                           \begin{inline-cond-kh}\item $\truthset{\modults}{\psi} \subseteq \sexec(\strategy)$ y \item $\R_\strategy(\truthset{\modults}{\psi}) \subseteq \truthset{\modults}{\varphi}$,\end{inline-cond-kh}
                                                          }
                                                       \end{minipage}
    \end{tabular}
    \end{nscenter}
    donde $\truthset{\modults}{\varphi} := \{w \in \W \mid \modults,w \models \varphi\}$. El conjunto de planes $\pi$ en la claúsula semántica de $\khi(\psi,\varphi)$ es llamado
    el testigo de $\khi(\psi,\varphi)$ en $\modults$.
\end{definicion}


Notar que $\khi(\psi,\varphi)$ vale en un estado $w$ cuando existe un conjunto de planes $\pi$ que el agente $i$ considera indistinguibles, tal que al ejecutar cualquier plan $\sigma \in \pi$ a partir de un estado donde vale $\psi$,
toda ejecución parcial puede ser completada terminando en un estado donde vale $\varphi$. También, cabe destacar que como $w$ no tiene ningún rol en la cláusula semántica de $\khi$,
dicho operador actúa \emph{globalmente}. Por lo tanto, $\truthset{\modults}{\khi(\psi,\varphi)}$ es $\W$ o $\emptyset$.

\begin{ejemplo}
    Nuevamente, consideremos el \ults presentado en \Cref{ejemplo:ults}, y veamos que:
    \begin{itemize}
        \item $\modults,w_1 \models p$ pero $\modults,w_3 \not\models p$.
        \item Notemos que $\sexec(\{ab,a\}) = \{w_2\}$ y $\sexec(\{b\}) = \{w_3\}$. 
        Luego $\truthset{\modults}{p} = \{w_1,w_2\} \not\subseteq \sexec(\pi)$ para todo $\pi \in \S_i$.
        Por lo que podemos afirmar que $\modults,w_1 \not\models \khi(p,q)$.
        Más aún, $\modults,w_j \not\models \khi(p,q)$ para cada $j \in \{1,..,4\}$.
        \item Como mencionamos en el item anterior, $\sexec(\{b\}) = \{w_3\}$. Luego, podemos ver que 
        $\modults, w_1 \models \khi(q,r)$ y $\{b\} \in \S_i$ es su testigo. 
        Pues, $\truthset{\modults}{q} = \{w_3\} \subseteq \sexec(\{b\})$ y $\R_b(w_3) = \{w_4\} \subseteq \truthset{\modults}{r}$.         

        A su vez, por la globalidad de $\khi$, ocurre que $\modults,w_j \models \khi(q,r)$ para cada $j \in \{1,...,4\}$.
    \end{itemize}
\end{ejemplo}



\section{Bisimulación}

La bisimulación es una herramienta crucial a la hora de analizar el poder expresivo de una lógica o un lenguaje formal. 
Nos permite relacionar modelos a partir de características estructurales entre sí, las cuáles logran capturar el comportamiento de 
los mismos con respecto a la lógica en cuestión. 

Se presentará aquí su definición junto con algunos resultados deseables a la hora de estudiar la noción de bisimulación de una 
lógica modal, los cuáles fueron demostrados en \cite{ArecesFSV25,SaraviaPHD}.

Primero, introduciremos una notación que nos será útil.

\begin{definicion}
    Sea $\model=\tup{\W,\R,\cset{\S_i}_{i \in \AGT},\V,\ACT}$ un \ults sobre $\PROP$ y $\AGT$.

    Tomemos un conjunto de planes $\pi \subseteq \ACT^*$, un conjunto de estados $U \subseteq \W$ y un agente $i \in \AGT$.
    \begin{itemize}
        \item Escribiremos $U \ultsExecStrat{\pi} T$ $\iffdef$ $U \subseteq \sexec(\pi)$ y $\R_\pi(U) \subseteq T$.
        \item Escribiremos $U \ultsExecAgi T$ $\iffdef$ existe $\pi \in \S_i$ tal que $U \ultsExecStrat{\pi} T$.
    \end{itemize}
    A su vez, decimos que $U \subseteq \W$ es \KHilogic-definible en $\modults$ si y sólo si existe una \KHilogic-fórmula $\varphi$ tal que
    $U = \truthset{\modults}{\varphi}$. Análogamente, decimos que $U \subseteq \W$ es proposicionalmente definible en $\modults$ 
    si y sólo si existe una fórmula proposicional $\varphi$ tal que $U = \truthset{\modults}{\varphi}$.
\end{definicion}

Una observación que surge de esta definición es que si un conjunto es proposicionalmente definible entonces es \KHilogic-definible, dado que 
toda fórmula proposicional sobre $\PROP$ es también una \KHilogic-fórmula.

La siguiente proposición presentada en \cite{ArecesFSV25,SaraviaPHD} nos dice que también vale la recíproca, 
es decir, que si un conjunto es \KHilogic-definible entonces es proposicionalmente definible.

\begin{proposicion}\label{prop:khi-implies-prop-definable}
    Sea $\model=\tup{\W,\R,\cset{\S_i}_{i \in \AGT},\V,\ACT}$ un \ults. Para todo $U \subseteq \W$, si $U$ es \KHilogic-definible, entonces $U$ es proposicionalmente definible.
\end{proposicion}

Ahora si, presentamos la noción de bisimulación.

\begin{definicion}[\KHilogic-bisimulación]\label{def:bisimulation}
    Sean $\modults$ y $\modults'$ dos \ultss, con dominios $\W$ y $\W'$ respectivamente. Sea $Z \subseteq \W \times \W'$.
    \begin{itemize}
        \item Sea $u \in W$ y $U \subseteq \W$, definimos
        \begin{nscenter}
            \begin{tabular}{@{}c@{}}
                $Z(u) := \csetsc{u' \in \W'}{(u,u') \in Z}$, \qquad $Z(U) := \bigcup_{u \in U} Z(u)$.
            \end{tabular}
        \end{nscenter}
        \item Sea $u' \in \W'$ y $U' \subseteq \W'$, definimos
        \begin{nscenter}
            \begin{tabular}{@{}c@{}}
                $Z^{-1}(u') := \csetsc{u \in \W}{(u,u') \in Z}$; \qquad $Z^{-1}(U') := \bigcup_{u' \in U'} Z^{-1}(u')$.
            \end{tabular}
        \end{nscenter}
    \end{itemize}

    Una relación binaria no vacía $Z \subseteq \W \times \W'$ es llamada una \KHilogic-bisimulación entre $\modults$ y 
    $\modults'$ si y sólo si $(w,w') \in Z$ implica lo siguiente:
    \begin{itemize}
        \item \textbf{Atom}: $\V(w)=\V'(w')$.

        \item \textbf{$\khi$-zig}: para cada conjunto \emph{proposicionalmente} definible $U \subseteq \W$, si $U \ultsExecAgi T$ para algún $T \subseteq \W$, entonces existe $T' \subseteq \W'$ tal que
        \begin{multicols}{2}
            \begin{cond-bisim}
                \item $Z(U) \ultsExecAgi T'$, 
                \item $T' \subseteq Z(T)$.
            \end{cond-bisim}
        \end{multicols}
        
        \item \textbf{$\khi$-zag}: para cada conjunto \emph{proposicionalmente} definible $U' \subseteq \W'$, si $U' \ultsExecAgi T'$ para algún $T' \subseteq \W'$, entonces existe $T \subseteq \W$ tal que
        \begin{multicols}{2}
            \begin{cond-bisim}
                \item $Z^{-1}(U') \ultsExecAgi T$,
                \item $T \subseteq Z^{-1}(T')$.
            \end{cond-bisim}
        \end{multicols}

        \item \textbf{A-zig}: para cada $u \in \W$ existe $u' \in \W'$ tal que $(u,u') \in Z$.

        \item \textbf{A-zag}: para cada $u' \in \W'$ existe $u \in \W$ tal que $(u,u') \in Z$.
    \end{itemize} 

    Escribiremos $\modults,w \bisim \modults',w'$ cuando exista una \KHilogic-bisimulación $Z$ entre
    $\modults$ y $\modults'$ tal que $(w,w') \in Z$.
\end{definicion}

Presentaremos ahora un ejemplo introducido en \cite{SaraviaPHD} con el objetivo de esclarecer cada una de las condiciones 
que se piden sobre una relación binaria para ser considerada una \KHilogic-bisimulación. 

\begin{ejemplo}
    Sean $\modults$ y $\modults'$ dos \ults con $\AGT = \{i\}$, cuya representación gráfica está dada por (\Cref{fig:bisim}):
    \begin{figure}[h]
        \centering
            \begin{tikzpicture}
                \node[state] (p) {$p$};
                \node[state, right of=p, xshift=1cm] (p2) {$p$};
                \node[state, right of=p2, xshift=1cm] (q) {$q$};
                \node[state, right of=q, xshift=1cm] (q2) {$q$};

                \node[state, below of=p, yshift=-1cm] (p') {$p$};
                \node[state, below of=p', yshift=-0.8cm] (p2') {$p$};
                \node[state, right of=p', yshift=-0.6cm, xshift=1.3cm] (q') {$q$};
                \node[state, right of=q', xshift=1.3cm] (q2') {$q$};


                \node at ($(p)+(0,0.5)$) {$w_1$};
                \node at ($(p2)+(0,0.5)$) {$w_2$};
                \node at ($(q)+(0,0.5)$) {$w_3$};
                \node at ($(q2)+(0,0.5)$) {$w_4$};

                \node at ($(p')+(0,-0.6)$) {$w_2'$};
                \node at ($(p2')+(0,-0.6)$) {$w_1'$};
                \node at ($(q')+(0,-0.6)$) {$w_3'$};
                \node at ($(q2')+(0,-0.6)$) {$w_4'$};


                \path (p) edge node [above] {$a$} (p2);
                \path (p2) edge node [above] {$a$} (q);
                \path (q) edge node [above] {$a$} (q2);
                \path (p') edge node [above] {$d$} (q');
                \path (p2') edge node [above] {$d$} (q');
                \path (q') edge node [above] {$e$} (q2');

                \path[pointed] (p) edge [bend right=35] node [above] {} (p2');
                \path[pointed] (p2) edge node [above] {} (p');
                \path[pointed] (q) edge node [above] {} (q');
                \path[pointed] (q2) edge node [above] {} (q2');

            \end{tikzpicture}
            \hspace{1cm}
            \raisebox{2.8cm}{
                \begin{minipage}{0.45\textwidth}
                    $\S_i = \left\{
                        \begin{array}{c}
                            \{aa\}
                        \end{array}
                    \right\}$ \\ [1.6cm]
                    $\S_i' = \left\{
                        \begin{array}{c}
                            \{de,d\}
                        \end{array}
                    \right\}$
                \end{minipage}
            }
            \caption{Representación gráfica de $\modults$ y $\modults'$}
            \label{fig:bisim}
    \end{figure}

    $\modults$ está representado por el \ults con nodos $\{w_1,w_2,w_3,w_4\}$, mientras que $\modults'$ está representado 
    por el \ults con nodos $\{w_1',w_2',w_3',w_4'\}$. A su vez, $\S_i$ y $\S'_i$ son sus respectivas relaciones de indistinguibilidad 
    entre planes del agente $i$.

    Consideremos la relación binaria $Z = \{(w_j, w_j') \mid 1 \leq j \leq 4\}$, la cuál está representada gráficamente con las líneas 
    punteadas. Es posible ver que $Z$ satisface (Atom), (A-zig) y (A-zag). Analicemos como se comporta $Z$ en torno a ($\khi$-zig), para ello 
    consideremos cada conjunto proposicionalmente definible del dominio de $\modults$:
    \begin{itemize}
       \item $U = \emptyset$. Definido por la fórmula $\varphi = \bot$.

       Notar que $U \ultsExecAgi \emptyset$, teniendo como testigo a $\pi = \{aa\}$.

       Luego, $Z(U) = \emptyset$ cumple $Z(U) \ultsExecAgi \emptyset$ a partir del testigo $\pi = \{dd,de\}$ y, 
       como $Z(\emptyset) \subseteq \emptyset$, se cumple ($\khi$-zig) para dicho $U$.
       \item $U = \{w_1,w_2\}$. Definido por la fórmula $\varphi = p$.
       
       Aquí, tenemos que $U \ultsExecAgi \{w_3,w_4\}$ a partir del testigo $\pi = \{aa\}$.

       Luego, $Z(U) = \{w_1',w_2'\}$ cumple $Z(U) \ultsExecAgi \{w_3',w_4'\}$ a partir del testigo $\pi = \{de,d\}$ y, 
       como $\{w_3'w_4'\} \subseteq Z(\{w_3,w_4\})$, se cumple ($\khi$-zig) para dicho $U$.
       \item $U = \{w_3,w_4\}$. Definido por la fórmula $\varphi = q$.
       
       Notemos que no existe $\pi \in \S_i$ tal que $U \subseteq \sexec(\pi)$, luego ($\khi$-zig) se cumple trivialmente para dicho $U$.
       \item $U = \{w_1,w_2,w_3,w_4\}$. Definido por la fórmula $\varphi = p \vee q$. 
       
       Similarmente al caso analizado anteriormente, no existe $\pi \in \S_i$ tal que $U \subseteq \sexec(\pi)$, por lo que ($\khi$-zig) se 
       cumple trivialmente para dicho $U$.
    \end{itemize}
    Es posible ver que no existen más conjuntos proposicionalmente definibles en el dominio de $\modults$. Luego, $Z$ satisface ($\khi$-zig).
    
    Realizando un análisis similar sobre los conjuntos proposicionalmente definibles del dominio de $\modults'$, se puede ver que $Z$ también 
    satisface ($\khi$-zag).
    
    Por lo que podemos concluir que $Z$ es una \KHilogic-bisimulación. Más aún, analizando los elementos de $Z$, podemos decir que 
    $\modults,w_j \bisim \modults',w_j'$ para cada $1 \leq j \leq 4$. 
\end{ejemplo}


Para poder formalizar las propiedades cruciales de la bisimulación, definiremos primero la noción de 
equivalencia entre modelos con respecto a \KHilogic.

\begin{definicion}[\KHilogic-equivalencia]
    Dos \ultss punteados $\modults,w$ y $\modults',w'$ son \KHilogic-equivalentes ($\model, w \modequiv \model', w'$)
    si y sólo si, para cada $\varphi \in \KHilogic$,
    \begin{center}
        $\model, w \models \varphi$ \quad si y sólo \quad $\model', w' \models \varphi$.
    \end{center} 
\end{definicion}

Ahora, podemos presentar la correspondencia esperada entre $\bisim$ y $\modequiv$, demostrada en \cite{ArecesFSV25,SaraviaPHD}.
Usualmente nos referimos a este resultado como el Teorema de Invarianza para Bisimulación.

\begin{teorema}[\KHilogic-bisimilitud implica \KHilogic-equivalencia]\label{thm:bisim-implies-equivalence}
    Sean $\modults,w$ y $\modults',w'$ dos \ultss punteados, entonces
    \begin{center}
        $\modults,w \bisim \modults',w'$ implica $\modults,w \modequiv \modults',w'$.
    \end{center}
\end{teorema}

Este teorema caracteriza a la bisimulación como una noción que permite relacionar modelos \KHilogic-equivalentes, es decir, que la lógica no tiene una fórmula con la cuál distinguirlos,
a partir de propiedades puramente estructurales de los mismos.

En \cite[Sección 2]{FervariVQW21} se presenta un contraejemplo que atestigua que la recíproca del \Cref{thm:bisim-implies-equivalence} 
no es cierta para cualquier par de modelos. Un problema ampliamente estudiado en la literatura de las lógicas modales es el de analizar 
en qué clases de modelos vale la recíproca del teorema, dichas clases son conocidas como clases de Hennessy-Milner.
En \cite{ArecesFSV25,SaraviaPHD}, se demuestra que la clase de \ults finitos es una clase de Hennessy-Milner, es decir, satisface la 
recíproca de \Cref{thm:bisim-implies-equivalence}.

Diremos que $\modults$, un \ults, es finito si y sólo si cada una de sus componentes tiene cardinalidad finita. Análogamente, diremos que 
($\modults,w$), un \ults punteado, es finito si y sólo $\modults$ es finito. 

\begin{teorema}[\KHilogic-equivalencia implica \KHilogic-bisimilitud]\label{thm:finite-equivalence-implies-bisim}
    Sean $\modults,w$ y $\modults',w'$ dos \ultss punteados finitos, entonces
    \begin{center}
        $\model,w \modequiv \model', w'$ implica $\model,w \bisim \model', w'$.
    \end{center}
\end{teorema}

Notemos que este teorema es un gran resultado en términos computacionales. Como los algoritmos trabajan siempre con modelos finitos, este resultado
nos dice que si se consigue un procedimiento efectivo que decida bisimilitud entre dos \ultss punteados, dicho procedimiento estará decidiendo a la vez
equivalencia lógica entre los \ultss punteados en cuestión.


\section{Complejidad Computacional}

A lo largo de este trabajo, analizaremos la complejidad computacional de problemas relacionados con la noción de bisimulación. 
Por ello, vale la pena mencionar algunos resultados estudiados en \cite{ArecesFSV25,SaraviaPHD} referentes a la lógica presentada en este 
capítulo. 

A la hora de realizar un estudio computacional de una lógica, los dos problemas de decisión fundamentales a analizar son los de 
verificación de modelos (Model Checking) y satisfacibilidad ($\SAT$).

El problema de Model Checking se formula como "dado un modelo y una fórmula de la lógica, decidir si el modelo la satisface". Por otro lado, 
$\SAT$ se formula como "dada una fórmula de la lógica, decidir si existe un modelo que la satisfaga". 

Estos problemas han sido estudiados a lo largo de los años en numerosas lógicas y son centrales, no solo para la lógica computacional, sino que 
también para la teoría de la complejidad computacional en general. 

Para la lógica proposicional, el problema de Model Checking está en la clase $\Poly$, mientras que $\SAT$ es $\NPComplete$ 
\cite[Capítulo 2, Sección 3]{Goldreich_2008}. Por otro lado, en la lógica modal básica tenemos que Model Checking también 
está en $\Poly$ pero $\SAT$ es $\PSPACEComplete$ \cite[Capítulo 4]{HandbookModalLogic}. En la lógica de primer orden, el problema de 
Model Checking es $\PSPACEComplete$ \cite[Capítulo 5, Sección 4]{Goldreich_2008} y $\SAT$ es indecidible. 

Una observación que surge a partir de estos resultados es que mientras mayor es el poder expresivo de la lógica, mayores son los recursos 
computacionales que se necesitan para resolver ambos problemas.

Metiendonos de lleno en las lógicas de `Knowing How', en \cite{Demri_Fervari_2023} se demostró que para la lógica de `Knowing How' introducida 
en \cite{Wang15KH,Wang2018GoalDirectedKH} Model Checking es $\PSPACEComplete$. Por otro lado, en \cite{SAT_Upper_Bound} se demostró que 
$\SAT \in \NP^{\NP}$, aunque no se ha encontrado todavía una cota inferior de su complejidad computacional.

Para \KHilogic, en \cite{ArecesFSV25,SaraviaPHD} se obtienen los siguientes resultados sobre los problemas de decisión mencionados:
\begin{teorema}\label{thm:model-checking-poly}
    El problema de Model Checking para \KHilogic está en $\Poly$.
\end{teorema}

\begin{teorema}
    $\SAT$ para \KHilogic es $\NPComplete$.
\end{teorema}

Lo que nos dice que ambos problemas son más fáciles, computacionalmente hablando, en su versión para \KHilogic con respecto a su versión para 
la lógica de `Knowing How' previamente mencionada.

\section{Contracción por Bisimulación para Lógica `Knowing How'}
    

\begin{lema}
    Sea $\model=\tup{\W,\R,\cset{\S_i}_{i \in \AGT},\V,\ACT}$ un modelo, y sean $w$, $v \in \W$ tales que $\V(w)$ = $\V(v)$ entonces para toda $\varphi$ proposicional se cumple que, 
    \begin{center}
    $\modults, w \models \varphi$ \quad si y sólo si \quad $ 
    \modults, v \models \varphi$
    \end{center}
\end{lema}

\begin{demostracion}
    Sea $\model=\tup{\W,\R,\cset{\S_i}_{i \in \AGT},\V,\ACT}$ y $w$, $v \in \W$ tales que $\V(w)$ = $\V(v)$. La demostración es por inducción estructural sobre $\varphi$. Recordar que $\varphi$ es una fórmula \textbf{proposicional}.

    \begin{itemize}
        \item Caso base: $\varphi = p$ donde $p \in \PROP$.

        Notar que  $\modults, w \models \varphi$ si y sólo si $p \in \V(w)$.

        Ahora bien, como $\V(w)$ = $\V(v)$, $p \in \V(w)$ si y sólo si $p \in \V(v)$.

        Finalmente, por la definición de $\models$, $p \in \V(v)$ si y sólo si $\modults, v \models \varphi$.  
    
        \item Caso inductivo: La hipótesis inductiva establece que, para $\psi$ una subfórmula de $\varphi$, se cumple que $\modults, w \models \psi$ si y sólo si $\modults, v \models \psi$.

        \begin{itemize}
            \item Caso $\varphi = \psi_1 \lor \psi_2$. 
    
            Por definición de $\models$, $\modults, w \models \psi_1 \lor \psi_2$ si y sólo si $\modults, w \models \psi_1$ o $\modults, w \models \psi_2$.
            
            Por hipótesis inductiva, $\modults, w \models \psi_1$ o $\modults, w \models \psi_2 $ si y sólo si $\modults, v \models \psi_1$ o $\modults, v \models \psi_2$. 
            
            Pero notar que, nuevamente por definición de $\models$, $\modults, v \models \psi_1$ o $\modults, v \models \psi_2$ si y sólo si $\modults,v \models \psi_1 \lor \psi_2$.  

            \item Caso $\varphi = \neg \psi$.
        
            La demostración es similar a la del caso analizado anteriormente.
        \end{itemize}
    
    \end{itemize}
    
\end{demostracion}


\begin{definicion}
    Sea $\model=\tup{\W,\R,\cset{\S_i}_{i \in \AGT},\V,\ACT}$ un modelo entonces se define, 
    \begin{center}
        $A_\modults := \{(w,v) \in \W \times \W \mid \V(w) = \V(v)\}$
    \end{center}
    Notar que $A_\modults$ es una relación de equivalencia sobre $\W$ (hace falta probarlo? es medio directo). Luego se denotará,
    \begin{center}
        $\rho_\modults := \{ [w] \mid w \in \W $ y $[w]$ su clase de equivalencia respecto a $A_\modults\}$
    \end{center}
\end{definicion}


\begin{lema}
    Sea $\model=\tup{\W,\R,\cset{\S_i}_{i \in \AGT},\V,\ACT}$ un modelo finito y sea $U \subseteq \W$ entonces,
    \begin{center}

    existe $S \in \mathcal{P}(\rho_\modults)$ finito tal que  $(U = \bigcup\limits_{s_i \in S} s_{i})$ \quad si y sólo si \quad $U$ es proposicionalmente definible. 
    \end{center}
\end{lema}

Intuitivamente, lo que nos dice este lema es que todo conjunto proposicionalmente definible viene de tomar una cierta cantidad de clases de equivalencia de $\rho_\modults$ y unirlas. Y, a su vez, cualquier unión de una cierta cantidad de clases de equivalencia es un conjunto proposicionalmente definible. 


\begin{demostracion}
    Se demostrará por separado los casos $(\rightarrow)$ y $(\leftarrow)$.

    Sea $\model=\tup{\W,\R,\cset{\S_i}_{i \in \AGT},\V,\ACT}$ un modelo finito y sea $U \subseteq \W$. Un primer detalle a tener en cuenta es que, dado que $\modults$ es finito, existen un número finito de variables proposicionales $p_1,...,p_n$ tales que $\{p_1,...,p_n\}$ es la imagen de $\V$.

    \begin{itemize}
        \item $(\rightarrow)$ Notemos primero que si $U = \emptyset$, luego la fórmula $ \varphi := \bot$ define a $U$. Entonces supongamos que $U \neq \emptyset$. (esto se podría analizar por separado o capaz permitir que S sea vacío y que en ese caso $\varphi$ sea una disyunción de 0 cosas por lo que sería el elemento neutro de la disyunción o sea $\bot$).
        
        Sea $S = \{s_1,...,s_m\}$ finito tal que $S \in \mathcal{P}(\rho_\modults)$ y $U = \bigcup\limits_{s_i \in S} s_{i}$, demostremos que $U$ es proposicionalmente definible. 

        Recordar que cada $s_i$ es una clase de equivalencia de la relación $A_\modults$. A su vez, para cada $w$, $v \in s_i$ se cumple que $\V(w) = \V(v)$. Se utilizará $\V(s_i)$ para hacer referencia a $\V(w)$ para cada $w \in s_i$.

        Sea $\varphi_i := \bigwedge\limits_{j = 1}^{n} l_i(p_j)$ donde $l_i(p_j) = $
        $\left\{ \begin{array}{rcl}
                p_j & \mbox{si}
                & p_j \in \V(s_i) \\ \neg p_j & \mbox{si} & p_i \notin \V(s_i) \\
                \end{array}\right.
        $.

        Veamos que por la forma en la que construimos $\varphi_i$, se cumple que $\modults, w \models \varphi_i$ si y sólo si $w \in s_i$. ($*$)
        (esto podría explicarlo/demostrarlo un poco más en profundidad si no es algo claro).
        
        Demostremos entonces que $\varphi := \bigvee\limits_{i = 1}^{m}\varphi_i$ define a $U$. Primero, notar que sea $w \in \W$, $\modults, w \models \varphi$ si y sólo si $\modults, w \models \varphi_i$ para algún $i \in \{1,...m\}$.  

        Queremos ver entonces que, $w \in U$ si y sólo si $\modults, w \models \varphi$. 

        Sea $w \in U$, por hipótesis, existe $s_i$ tal que $w \in s_i$. Ahora bien, por $(*)$, $\modults, w \models \varphi_i$, por lo que $\modults, w \models \varphi$.

        Sea $w \in \W$ tal que $\modults, w \models \varphi$, entonces existe $i \in \{1,...,m\}$ tal que $\modults, w \models \varphi_i$. Luego por $(*)$, $w \in s_i$, lo que nos dice que $w \in U$.

        Luego $\varphi$ define a $U$.
        
        Como $\varphi$ es proposicional, $U$ es proposicionalmente definible. 
    
        \item $(\leftarrow)$ Sea $\varphi$ la fórmula proposicional que define a $U$, se quiere demostrar que existe $S = \{s_1,...,s_m\}$ finito, tal que $S \in \mathcal{P}(\rho_\modults)$ tal que $U = \bigcup\limits_{s_i \in S} s_{i}$. 

        Definamos $U' = \bigcup\limits_{w \in U} [w]$, siendo $[w]$ la clase de equivalencia de $w$ con respecto a $A_\modults$. Notemos que por la definición de $U'$, existe $S \in \mathcal{P}(\rho_\modults)$ finito tal que $U' = \bigcup\limits_{s_i \in S} s_{i}$ dado que $U'$ está definido como una unión finita de elementos de $\mathcal{P}(\rho_\modults)$.

        Demostremos ahora que $U' = U$. Es claro que $U \subseteq U'$, dado que $w \in [w]$ para cada $w \in \W$.

        Queda demostrar que $U' \subseteq U$. 
        
        Sea $w' \in U'$, existe $w \in U$ tal que $w' \in [w]$. Ahora bien, como $w \in U$, $\modults, w \models \varphi$. Luego, notemos que por Lema 1, $\modults, v \models \varphi$ para cada $v \in [w]$, pues $\V(w) = \V(v)$ para cada $v \in [w]$.
        Pero como $w' \in [w]$ esto nos dice que $\modults, w' \models \varphi$. Luego $w' \in U$.

        Finalmente $U' = U$, lo cual demuestra $(\leftarrow)$.
    \end{itemize}

\end{demostracion}

% IDEA PARA PROBAR LO DE QUE CADA CLASE DE EQUIVALENCIA O ESTÁ POR COMPLETO O NO ESTÁ, HABRÍA QUE PROBAR EL SII, ACÁ SOLO PRUEBO EL ->

\mycomment{
\begin{corolario}
    Sea $\model=\tup{\W,\R,\cset{\S_i}_{i \in \AGT},\V,\ACT}$ un modelo finito y sea $U \subseteq \W$ un conjunto proposicionalmente definible, entonces
    \begin{center}
        $\forall s_i \in \rho_\modults.$ $(s_i \subseteq U \vee s_i \cap U = \emptyset)$
    \end{center}
\end{corolario}

\begin{demostracion}
    Sea $\model=\tup{\W,\R,\cset{\S_i}_{i \in \AGT},\V,\ACT}$ un modelo finito y sea $U \subseteq \W$ proposicionalmente definible.
    
    Supongamos que la propiedad es falsa, luego existe $s' \in \rho_\modults$ tal que $s' \nsubseteq U \wedge s_i \cap U \neq \emptyset$. Notemos que esto dice que existen $w,v \in s'$ tal que $w \in U$ y $v \notin U$.

    A su vez, por Lema 2, como $U$ es proposicionalmente definible existe $S \subseteq \mathcal{P}(\rho_\modults)$ tal que $U = \bigcup\limits_{s_i \in S} s_{i}$. 
    
    Como $w \in U$ y $w \in s'$, entonces ocurre que $s' \in S$, pues recordemos que los elementos de $\rho_\modults$ son clases de equivalencia, es decir, $s'$ es la única clase que contiene a $w$. Ahora bien, pero como $v \in s'$ entonces $v\in U$, lo cuál es absurdo, pues dijimos que $v \notin U$. 

    El absurdo vino de suponer que existía $s' \in \rho_\modults$ tal que  $s' \nsubseteq U \wedge s_i \cap U \neq \emptyset$, luego la propiedad queda demostrada.
\end{demostracion}
}


\begin{corolario}
    Sea $\model=\tup{\W,\R,\cset{\S_i}_{i \in \AGT},\V,\ACT}$ un modelo finito y sea $U \subseteq \W$ entonces el problema de decidir si $U$ es proposicionalmente definible está en $\Poly$.

    Más aún, en caso de ser definible, encontrar una fórmula $\varphi$ que define a $U$ es realizable en tiempo polinomial en el tamaño de $\modults$.
\end{corolario}
(este corolario es una idea que se me ocurrió que no es central central al problema)

\begin{demostracion}
    Queremos encontrar un algoritmo que dado $\model=\tup{\W,\R,\cset{\S_i}_{i \in \AGT},\V,\ACT}$ un modelo finito y $U \subseteq \W$ decida en tiempo polinomial en el tamaño de $\modults$ si $U$ es proposicionalmente definible.

    Por Lema 2, sabemos que basta con encontrar $S \in \mathcal{P}(\rho_\modults)$ tal que $U = \bigcup\limits_{s_i \in S} s_{i}$ o determinar que no existe tal $S$.
    
    Como $\rho_\modults$ es la partición de $\W$ con respecto a la relación de equivalencia $A_\modults$, dicho $S \in \mathcal{P}(\rho_\modults)$ existirá si y sólo si ocurre que para cada $s_i \in \rho_\modults$ se cumple que $s_i \subseteq U$ o $s_i$ $\cap$ $U = \emptyset$. (Puedo desarrollar esto acá o incluso probar un lema más que enuncie esto, pero la idea intuitiva es que como queremos chequear que U es unión de clases de equivalencia, entonces si o si tiene que ocurrir que para cada clase de equivalencia están todos sus nodos en U o ninguno está en U)

    Entonces notemos que basta con dar un algoritmo que encuentre las clases de equivalencia de $\rho_\modults$ y para cada una de ellas chequee que, o todos sus nodos pertenecen a $U$ o ninguno de ellos lo hace.

    % acá va algoritmo que chequea si u es definible

    


        

    Luego, en caso de ser $U$ definible, tendremos computado ya el conjunto $S \in \mathcal{P}(\rho_\modults)$ tal que $U = \bigcup\limits_{s_i \in S} s_{i}$, que estaría formado por las clases de equivalencia que encontramos en el paso anterior que están completamente contenidas en $U$.

    Ahora bien, a partir de dicho conjunto $S = \{s_1,...,s_m\}$, podemos computar una fórmula que define a $U$ siguiendo la misma estrategia mencionada en la demostración del Lema 2. 
    
    Es decir, podemos computar $\varphi := \bigvee\limits_{i = 1}^{m}\varphi_i$, con $\varphi_i := \bigwedge\limits_{j = 1}^{n} l_i(p_j)$ donde $l_i(p_j) = $
        $\left\{ \begin{array}{rcl}
                p_j & \mbox{si}
                & p_j \in \V(s_i) \\ \neg p_j & \mbox{si} & p_j \notin \V(s_i) \\
                \end{array}\right.
        $.

    Notar que la cantidad de disyunciones ($m$) de $\varphi$ está acotada por la cantidad de elementos de $\W$ y por otro lado la cantidad de conjunciones de cada término ($n$) es la cantidad de variables proposicionales que aparecen en la imagen de $\V$. Juntando estos dos argumentos, podemos deducir que el tamaño de $\varphi$ es $\bigO(|\W| *|\V|)$, lo cuál es polinomial en el tamaño de $\modults$.

    Falta escribir el algoritmo y probar su complejidad.
\end{demostracion}


\begin{teorema}
    Sea $\model=\tup{\W,\R,\cset{\S_i}_{i \in \AGT},\V,\ACT}$ un modelo, entonces se cumple que,
    \begin{enumerate}
        \item $A_\modults$ es una autobisimulación de $\modults$.
        \item Sea $Z \subseteq \W \times \W$ una autobisimulación de $\modults$ entonces $Z \subseteq A_\modults$. Es decir, $A_\modults $ es la máxima autobisimulación de $\modults$. 
    \end{enumerate}
\end{teorema}

\begin{demostracion}
    Sea $\model=\tup{\W,\R,\cset{\S_i}_{i \in \AGT},\V,\ACT}$ un modelo, probaremos (1) y (2) por separado:

    \begin{enumerate}
        \item Queremos ver que $A_\modults$ es una autobisimulación de $\modults$, para ello debemos verificar que:
        \begin{itemize}
            \item (Atom) Dado $(w,v) \in A_\modults$, por definición de $A_\modults$, se cumple que $\V(w) = \V(v)$.
            \item (A-Zig) Notemos que $A_\modults$ es una relación de equivalencia, luego para cada $w\in \W$ se cumple que $(w,w) \in A_\modults$.
            \item (A-zag) Podemos utilizar el mismo argumento que en (A-zig) para demostrar este caso.
            \item (\KHilogic-Zig) Queremos ver que para cada $U \subseteq \W$ proposicionalmente definible, si $U \ultsExecAgi T$ para algún $T \subseteq \W$, entonces existe $T' \subseteq \W$ tal que
                \begin{multicols}{2}
                    \begin{itemize}
                        \item $A_\modults(U) \ultsExecAgi T'$, 
                        \item $T' \subseteq A_\modults(T)$.
                    \end{itemize}
                \end{multicols}

            Sean $U,T$ subconjuntos de $\W$ tales que $U \ultsExecAgi T$ y $U$ proposicionalmente definible, queremos encontrar $T' \subseteq \W$ que cumpla lo mencionado.

            Como $U$ es proposicionalmente definible, entonces existe $\varphi$ proposicional tal que $w \in U$ si y sólo si $\modults, w \models \varphi$.
            
            Demostremos por un lado que $U = A_\modults(U)$:
            
            ($\subseteq$) Siguiendo el argumento usado en (A-Zig), es claro que para $w \in U$ se cumple que $w \in A_\modults(U)$, pues $(w,w) \in A_\modults$ para cada $w \in \W$. 
            
            ($\supseteq$) Sea $v \in A_\modults(U)$, queremos ver que $v \in U$.
            Como $v \in A_\modults(U)$, entonces existe $w \in U$ tal que $(w,v) \in A_\modults$. Luego, como $(w,v) \in A_\modults$ entonces $\V(w) = \V(v)$. 
            
            Ahora bien, como $w \in U$ y $\varphi$ define a $U$, esto nos dice que $\modults, w \models \varphi$. Pero por Lema 1, como $\V(w) = \V(u)$ y $\varphi$ es proposicional entonces $\modults, v \models \varphi$, luego $v \in U$.

            Entonces demostramos que $U = A_\modults(U)$.

            Ahora bien, notemos que siendo $X \subseteq \W$ entonces $X \subseteq A_\modults(X)$, pues como analizamos anteriormente $(w,w) \in A_\modults$ para cada $w \in \W$.

             Juntando lo mencionado, podemos afirmar que $T' = T$ cumple que:

             \begin{itemize}
                 \item $A_\modults(U) \ultsExecAgi T'$, pues dijimos $A_\modults(U) = U$, y por hipótesis, $U \ultsExecAgi T = T'$.  
                \item $T' \subseteq A_\modults(T')$ pues esto se cumple para todo $X \subseteq \W$.
            \end{itemize}
            Luego queda demostrado que $A_\modults$ satisface (\KHilogic-Zig).

            \item (\KHilogic-Zag) Análogo a (\KHilogic-Zig), pues notemos que como $A_\modults$ es una relación de equivalencia, es simétrica.
        \end{itemize}

        Luego $A_\modults$ es una autobisimulación de $\modults$.
        
        \item Queremos ver que dada $Z \subseteq \W \times \W$ autobisimulación de $\modults$, entonces $Z \subseteq A_\modults$.
        
        Supongamos que no es cierto, es decir, existe $Z \subseteq \W \times \W$ autobisimulación de $\modults$ tal que hay $w,v \in \W$ que cumplen que $(w,v) \in Z$ y $(w,v) \notin A_\modults$. 

        Como $Z$ es una autobisimulación entonces satisface (atom), es decir, $\V(w) = \V(v)$. Pero notemos que por definición de $A_\modults$, $(w,v) \in A_\modults$, lo cuál es absurdo, pues dijimos que $(w,v) \notin A_\modults$.

        El absurdo vino de suponer que $Z \nsubseteq A_\modults$.

        Luego queda demostrada la propiedad.

        
    \end{enumerate}
\end{demostracion}

% Se me desacomodó el cuadrado del final de la demo, no se cómo arreglarlo


Algo interesante es que hay varias contracciones que no funcionan:

Por ejemplo, dejar las estrategias y las acciones iguales y usar como aristas $\R' = \{(a,[w],[v]) \mid (a,w,v)\in \R\}$. Esta sería la contracción de la lógica modal básica

(Poner gráfico de un contraejemplo).

Otro ejemplo de contracción que no funciona es dejar las estrategias y las acciones iguales y usar como aristas $\R = \{(a,[w],[v]) \mid$ para todo $w' \in [w]$ existe $v' \in [v]$ tal que $(a,w',v') \in \R\}$.

(Poner gráfico de un contraejemplo).

Otro ejemplo sería definir $\R = \{(a,[w],[v]) \mid$ para todo $w' \in [w]$ se cumple que $\R_a(w') \neq \emptyset$ y $(w',v) \in \R_a \}$ y esto tampoco funcionaría.  

(Poner gráfico de un contraejemplo).

Todo este resultado se puede hacer con caso infinito, reescribiendo el lema en vez de si y sólo si a implica, porque lo finito solo se necesita para encontrar la fórmula que en realidad no interesa



\begin{definicion}
    Sea $\model=\tup{\W,\R,\cset{\S_i}_{i \in \AGT},\V,\ACT}$ un modelo finito. Se define su contracción por bisimulación como el modelo $\model'=\tup{\W',\R',\cset{\S_i'}_{i \in \AGT'},\V',\ACT'}$ donde 
    \begin{center}
    % esto centrarlo mejor de alguna forma
        \begin{itemize}
            \item $\W' := \W/A_\modults$
            \item $\R' := \{\R'_{a_\sigma} \subseteq \W' \times \W' \mid a_\sigma \in \ACT'\}$ donde $([w],[v]) \in \R'_{a_\sigma}$ si y sólo si
            \begin{enumerate}
                \item existen $w' \in [w]$ y $v' \in [v]$ tal que $(w',v')\in \R_\sigma$
                \item $\sigma$ es fuertemente ejecutable para cada $w' \in [w]$.
            \end{enumerate}
            \item $\S_i' := \{ \pi' = \{a_{\sigma_1},...,a_{\sigma_k}\} \mid  
            $ existe $ \pi \in S_i $ tal que $ \pi = \{\sigma_1,...,\sigma_k
            \}\}$
            \item $\V'([w]) := \V(w)$
            \item $\ACT' := \{a_\sigma \mid $ existe $ \pi\in S_i$ tal que $ \sigma \in \pi$ para algún $i \in \AGT \}$ 
        \end{itemize}
    \end{center}
\end{definicion}
    
\begin{teorema}
    Sea $\model=\tup{\W,\R,\cset{\S_i}_{i \in \AGT},\V,\ACT}$ un modelo finito y sea $\modults'$ su contracción por bisimulación entonces $\tup{\modults,w}$ y $\tup{\modults',[w]}$ son $\KHilogic$-bisimilares para cada $w \in \W$.
\end{teorema}

\begin{demostracion}
    Sea $\model=\tup{\W,\R,\cset{\S_i}_{i \in \AGT},\V,\ACT}$ un modelo finito
    y sea $\modults'$ su contracción por bisimulación, basta ver que $Z = \{(w,[w]) \in \W \times\W'\}$ es una \KHilogic-bisimulación para demostrar la propiedad.

    Dado que $\W$ es no vacío entonces es claro que $Z$ es no vacío también.

    Por otro lado, notemos que por la definición de $\V'$, es claro que $Z$ satisface (Atom).

    A su vez, por cómo definimos $Z$, es fácil ver que también (A-zig) y (A-zag) son satisfechos.
    
    Demostremos entonces que se satisfacen (\KHilogic-zig) y (\KHilogic-zag).

    \begin{itemize}
        \item (\KHilogic-zig) Sean $U, T$ subconjuntos de $\W$ tales que $U \ultsExecAgi T$ y $U$ proposicionalmente definible, queremos encontrar $T' \subseteq \W'$ que

        \begin{multicols}{2}
            \begin{itemize}
                \item $Z(U) \ultsExecAgi T'$, 
                \item $T' \subseteq Z(T)$.
            \end{itemize}
        \end{multicols}

        Veamos que $T' = \{ [w] \mid w \in T\}$ cumple con lo mencionado. Por como definimos $Z$, es claro que $T'  = Z(T)$, por lo que $T' \subseteq Z(T)$. Entonces demostremos $Z(U) \ultsExecAgi T'$.


        Como $U \ultsExecAgi T$, existe $\pi = \{\sigma_1,...,\sigma_k\} \in S_i$ tal que cada plan de $\pi$ es fuertemente ejecutable en cada nodo de $U$ y $\R_\pi(U) \subseteq T$. 

        Ahora bien, sabemos por la definición de $\modults'$ que existe $\pi' =\{a_{\sigma_1},...,a_{\sigma_k}\} \in S_i'$. Luego, basta ver que cada $a_{\sigma_i}$ es fuertemente ejecutable en $Z(U)$ y que además $\R'_{\pi'}(Z(U)) \subseteq T'$, pues, juntando las dos afirmaciones concluiríamos que $Z(U) \ultsExecAgi T'$.


        Notemos que los planes de $\pi'$ son simplemente de un paso, es decir, deberíamos chequear que para cada elemento de $Z(U)$ existe una arista con cada una de las acciones de $\pi'$ y que cada arista de $Z(U)$ con alguna de las etiquetas de $\pi'$ se dirige hacia un nodo de $T'$. 

        Por Lema 2, existe $S \in \mathcal{P}(\rho_\modults)$ tal que $U  = \bigcup\limits_{s_i \in S} s_{i}$. Luego, por como está definido $Z$, notemos que $Z(U) = S$.

        Sean entonces $a_{\sigma_i} \in \pi'$ y $s_j \in S$ queremos ver que $a_{\sigma_i}$ es fuertemente ejecutable en $s_j$. Como cada nodo de $s_j$ está en $U$ y $\sigma_i$ es fuertemente ejecutable en todo nodo de $U$ entonces para cada nodo $w$ de $s_j$ existe $v$ tal que $(w,v) \in \R_{\sigma_i}$. A su vez, juntando lo mencionado, se puede ver que por la definición de $\R'$ entonces existe $v \in \W$ tal que $(s_j,[v]) \in \R'_{a_{\sigma_i}}$, luego $a_{\sigma_i}$ es fuertemente ejecutable en $s_j$. Como $a_{\sigma_i}$ y $s_j$ eran elementos fijos pero arbitrarios de $\pi'$ y $S$ respectivamente, vale que $\pi'$ es fuertemente ejecutable en todo $S$.

        Ya demostramos entonces que $\pi'$ es fuertemente ejecutable en $S$, queda demostrar que $\R'_{\pi'}(S) \subseteq T'$.

        Sean $s_j \in S$ y $a_{\sigma_i} \in \pi'$ tales que $(s_j,[v]) \in \R'_{a_{\sigma_i}}$ entonces existen $w \in s_j$ y  $v' \in [v]$ tales que $(w,v') \in \R'_{\sigma_i}$. Recordemos que $\R_\pi(U) \subseteq T$.  Luego, como $w \in U$ y $\sigma_i \in \pi$ entonces $v' \in T$, lo que nos dice que $[v] \in T'$. Entonces queda demostrado que $\R'_{\pi'}(S) \subseteq T'$, dado que $s_j$ y $a_{\sigma_i}$ son elementos fijos pero arbitrarios de $S$ y $\pi'$ respectivamente.

        Probamos entonces que $\pi'$ es fuertemente ejecutable en $S$ y que $\R'_{\pi'}(S) \subseteq T'$, luego esto nos dice que $S \ultsExecAgi T'$. Y como $S = Z(U)$, $Z(U) \ultsExecAgi T'$.

        Queda demostrado entonces \KHilogic-zig. (MEJORAR ESTA DEMO).

        \item (\KHilogic-zag) Sean $U, T \in \W'$ tales que $U \ultsExecAgi T$ con $U$ proposicionalmente definible, queremos encontrar $T' \subseteq \W$ tal que
        \begin{multicols}{2}
            \begin{itemize}
                \item $Z^{-1}(U) \ultsExecAgi T'$, 
                \item $T' \subseteq Z^{-1}(T)$.
            \end{itemize}
        \end{multicols}

        Notar que $U = \{[w_1],...,[w_n]\mid w_i \in \W \}$ y $T = \{[v_1],...,[v_m]\mid v_i \in \W \}$.
        
        Veamos que $T' = \bigcup\limits_{[v_i] \in T} [v_i]$ cumple con lo mencionado. Notar que por la definición de $Z$, es claro que $T' = Z^{-1}(T)$, lo que nos dice que $T' \subseteq Z^{-1}(T)$. Demostremos entonces que $Z^{-1}(U) \ultsExecAgi T'$. 
    
        Como $U \ultsExecAgi T$, entonces existe $\pi' = \{a_{\sigma_1},...,a_{\sigma_k}\}\in S_i'$ tal que $\pi'$ es fuertemente ejecutable en cada elemento de $U$ y $\R'_{\pi'}(U) \subseteq T$.

        Luego por como está definido $\modults$ entonces existe $\pi = \{\sigma_1,...,\sigma_k\} \in S_i$. Veamos entonces que $\pi$ es fuertemente ejecutable en cada nodo de $Z^{-1}(U)$ y a su vez que $\R_\pi(Z^{-1}(U)) \subseteq T'$.

        Sean $w \in Z^{-1}(U)$ y $\sigma_i \in \pi$, veamos que $\sigma_i$ es fuertemente ejecutable en $w$. Notemos que como $w \in Z^{-1}(U)$ entonces $[w] \in U$ y, a su vez, como $a_{\sigma_i}$ es fuertemente ejecutable en $U$, $a_{\sigma_i}$ también es fuertemente ejecutable en $[w]$. Luego, existe $[v]$ tal que $([w],[v]) \in \R'_{a_{\sigma_i}}$. Ahora bien, notemos que por la definición de $\R'$, como existe esa arista entonces se cumple que $\sigma_i$ es fuertemente ejecutable en todo nodo de $[w]$. En particular, $\sigma_i$ es fuertemente ejecutable en $w$. Finalmente, demostramos que $\pi$ es fuertemente ejecutable en $Z^{-1}(U)$.

        Demostremos ahora que $\R_{\pi}(Z^{-1}(U)) \subseteq T'$. 

        Sean $w, v \in \W$ tales que $w \in Z^{-1}(U)$ y $(w,v) \in \R_{\sigma_i}$, queremos ver que $v \in T'$. Notemos que como $w \in Z^{-1}(U)$, entonces $[w] \in U$. Ahora bien como $a_{\sigma_i}$ es fuertemente ejecutable en $U$, $a_{\sigma_i}$ también es fuertemente ejecutable en $[w]$.
        
        Como $a_{\sigma_i}$ es fuertemente ejecutable en $[w]$ entonces existe $[v']$ tal que $([w],[v']) \in \R_{a_{\sigma_i}}$. Ahora bien, como existe dicha arista, por la definición de $\R'$, ocurre que $\sigma_i$ es fuertemente ejecutable en todo nodo de $[w]$. Pero como $\sigma_i$ es fuertemente ejecutable en todo nodo de $[w]$ y $(w,v) \in \R_{\sigma_i}$ entonces $([w],[v]) \in \R'_{a_{\sigma_i}}$. Pero como $\R'_{\pi'}(U) \subseteq T$, entonces $[v] \in T$, lo que nos dice que $v \in T'$.
        
        Luego demostramos que $\R_{\sigma_i}(Z^{-1}(U)) \subseteq T'$. Como $\sigma_i$ era un elemento fijo pero arbitrario de $\pi$ entonces vale para todo $\pi$. Luego $\R_\pi(Z^{-1}(U)) \subseteq T'$, lo cual demuestra \KHilogic-zag.
    \end{itemize}

    Demostramos entonces que $Z$ es una \KHilogic-bisimulación. Luego $\tup{\modults,w}$ y $\tup{\modults',[w]}$ son $\KHilogic$-bisimilares para cada $w \in \W$.
    
\end{demostracion}



\begin{teorema}
    La contracción por bisimulación de un modelo finito $\modults$ tiene cardinalidad mínima entre los modelos $\KHilogic$-bisimilares a $\modults$.
\end{teorema}
(este no estoy seguro de cómo probarlo pero estaría bueno que sea cierto para demostrar que efectivamente estamos minimizando el modelo).

update de esto: cardinalidad es bastante directo, pero cantidad de aristas debe ser bastante más complicado, por lo que me contaste de cocientar las clases de estrategias.

\begin{teorema}
    Sea $\modults$ un modelo finito, encontrar su contracción por bisimulación $\modults'$ es realizable en tiempo polinomial en el tamaño de $\modults$.
\end{teorema}

(la mejor complejidad con la que se podría hacer no estoy seguro, porque tampoco definimos bien cómo va a ser la contracción, pero lo de encontrar los nodos strongly executable de cada camino se me ocurre que puede salir en $\mathcal{O}(\sum_{i} |S_i|*n*(n+m))$)


Esta definición va a ir en los premilinares pero la escribo acá para tenerla a mano al ver esta demo
\begin{definicion}
    Supongamos $\ACT$ un conjunto de acciones no vacío enumerable.
    Sea $\modults = \tup{\W,\{\R_a\}_{a\in\ACT},\V}$ un modelo de Kripke. Sea $Z_\modults$ su autobisimulación máxima en Lógica Modal Básica (LMB) entonces definimos su contracción por LMB-bisimulación como $\modults_{LMB} = \tup{\W',\{\R'_a\}_{a\in\ACT},\V}$ donde
    \begin{center}
        \begin{itemize}
            \item $\W' := \W /Z_\modults$
            \item $\R_a' := \{([w],[v]) \mid$ existen $w' \in [w]$ y $v' \in [v]$ tal que $(w',v') \in \R_a \}$
            \item $\V'([w]) := \V(w)$
        \end{itemize}
    \end{center}
    
    
\end{definicion}

Notar que en el contexto de esta definición, $[w]$ se refiere a la clase de equivalencia de $w$ en la relación de equivalencia dada por $Z_\modults$. 


\begin{lema}
    Sea $\model = \tup{\W,\{\R_a\}_{a\in\ACT},\V}$ un modelo de Kripke, y sea $\modults_{LMB} = \tup{\W',\{\R_a'\}_{a\in\ACT},\V'}$ su contracción por LMB-bisimulación,
    \begin{center}
        Si $([w],[v]) \in \R'_a$ entonces para cada $w' \in [w]$ existe $v' \in [v]$ tal que $(w',v') \in \R_a$
    \end{center}
\end{lema}



\begin{demostracion}
    Sea $\model = \tup{\W,\{\R_a\}_{a\in\ACT},\V}$ un modelo de Kripke, y sea $\modults_{LMB} = \tup{\W',\{\R_a'\}_{a\in\ACT},\V'}$ su contracción por LMB-bisimulación. Sea $([w],[v]) \in \R'_a$, veamos que para cada $w' \in [w]$ existe $v' \in [v]$ tal que $(w',v') \in \R_a$.

    Como $([w],[v]) \in \R'_a$ existen $w_0 \in [w]$ y $v_0 \in [v]$ tales que $(w_0,v_0)\in \R_a$. Ahora bien, sea $w' \in [w]$, por estar ambos $w_0$ y $w'$ en $[w]$, existe una LMB-bisimulación $Z$ tal que $(w_0,w') \in Z$. Pero notemos que, por (zig) de $Z$, como $(w_0,v_0) \in \R_a$  entonces existe $v'$ tal que $(w',v') \in \R_a$ y, a su vez, $(v_0,v') \in Z$, es decir, $v' \in [v]$. Como demostramos esto para un $w'$ arbitrario en $[w]$, vale para todo elemento de $[w]$.
\end{demostracion}


\begin{teorema}
    Sea $\model=\tup{\W,\R,\cset{\S_i}_{i \in \AGT},\V,\ACT}$ un modelo y sea $\modults_{LMB} = \tup{\W',\{\R_a'\}_{a\in\ACT},\V'}$ la contracción por LMB-bisimulación del modelo $\tup{\W,\{\R_a\}_{a\in\ACT},\V}$, entonces $\tup{\modults,w}$ y $\tup{\modults',[w]}$ son $\KHilogic$-bisimilares para cada $w \in \W$ siendo $\modults' = \tup{\W',\R',\cset{\S_i}_{i \in \AGT},\V',\ACT}$.
\end{teorema}

Notar que en este teorema, al fijar $\model=\tup{\W,\R,\cset{\S_i}_{i \in \AGT},\V,\ACT}$ inmediatamente fijamos a $\ACT$ como el conjunto no vacío enumerable de acciones para la Lógica Modal Básica.

\begin{demostracion}
    Queremos ver que $Z = \{(w,[w]) \mid w \in \W\}$ es una \KHilogic-bisimulación.
    
    Por definición de la contracción por bisimulación en (LMB) se puede ver que se satisface (atom). Luego, por definición de Z, es fácil ver que (A-zig) y (A-zag) son satisfechas. Demostremos entonces (\KHilogic-zig) y (\KHilogic-zag).

    \begin{itemize}
        \item (\KHilogic-zig). Sean $U, T$ ambos subconjuntos de $\W$ tales que $U$ es proposicionalmente definible y $U \ultsExecAgi T$. Queremos ver que existe $T' \subseteq \W'$ tal que

        \begin{multicols}{2}
            \begin{itemize}
                \item $Z(U) \ultsExecAgi T'$, 
                \item $T' \subseteq Z(T)$.
            \end{itemize}
        \end{multicols}
        Veamos que $T' = \{[w] \mid w \in T\}$ cumple con lo mencionado. Notemos que $T' = Z(T)$, por lo que solo debemos demostrar $Z(U) \ultsExecAgi T'$.

        Como $U \ultsExecAgi T$, existe $\pi \in S_i$ tal que $\pi$ es fuertemente ejecutable en $U$ y a su vez $\R_\pi(U) \subseteq T$.

        Demostremos que $\pi$ es fuertemente ejecutable en $Z(U)$ y que $\R'_\pi(Z(U)) \subseteq T'$.

        Supongamos que $\pi$ no es fuertemente ejecutable en $Z(U)$. Luego existe $\sigma \in \pi$ y $[w_1],...,[w_k]$ con $0 \le k \le |\sigma|$ tal que $[w_1] \in Z(U)$, $([w_i], [w_{i+1}]) \in \R'_{\sigma[i]}$ y $\R'_{\sigma[k]}([w_k]) = \emptyset$.

        Ahora bien, como $[w_1] \in Z(U)$, existe $w_1' \in U$ tal que $w_1' \in [w_1]$. Luego notemos que aplicando sucesivamente el Lema 3 en todo el camino, existen $w_1',...w_k'$ tales que $w_i' \in [w_i]$ y $(w_i',w_{i+1}') \in \R_{\sigma[i]}$.

        Pero veamos que esto nos dice que $\R_{\sigma[k]}(w_k') = \emptyset$. Pues, si existiera $v$ tal que $(w_k',v) \in \R_{\sigma[k]}$ entonces ocurriría que $([w_k],[v]) \in \R'_{\sigma[k]}$. Luego $\sigma \in \pi$ no es fuertemente ejecutable en $w_1' \in U$, lo cuál es absurdo, pues dijimos que $\pi$ es fuertemente ejecutable en todo $U$.

        Esto nos dice que $\pi$ es fuertemente ejecutable en $Z(U)$.

        Veamos ahora que $\R'_\pi(Z(U)) \subseteq T'$.

        Sea $[v] \in \R'_\pi(Z(U))$, entonces existen $\sigma \in \pi$ y $[w_1], ..., [w_{|\sigma|+1}]$ tales que $[w_1] \in Z(U)$, $([w_i],[w_{i+1}]) \in \R'_{\sigma[i]}$ y $[w_{|\sigma|+1}] = [v]$.

        Ahora bien, como $[w_1] \in Z(U)$, entonces existe $w_1' \in U$ tal que $w_1'\in [w_1]$. Luego notemos que aplicando sucesivamente el Lema 3 sobre el camino, existen $w_1',...,w_{|\sigma|+1}'$ tales que $w_i' \in [w_i]$ y $(w_i',w_{i+1}')\in \R_{\sigma[i]}$. Como $\R_\pi(U) \subseteq T$ esto nos dice que $w'_{|\sigma|+1} \in T$. Finalmente, por definición de $T'$, $[w_{|\sigma|+1}] \in T'$. Luego como $[v] = [w_{|\sigma|+1}]$, $[v] \in T'$.

        Entonces demostramos que $\pi$ es fuertemente ejecutable en $Z(U)$ y que $\R'_\pi(Z(U)) \subseteq T'$. Juntando ambos resultados, concluimos que $Z(U) \ultsExecAgi T'$, lo cuál demuestra (\KHilogic-zig).

       \item (\KHilogic-zag) Sean $U,T$ ambos subconjuntos de $\W'$ tales que $U$ es proposicionalmente definible y $U \ultsExecAgi T$. Queremos ver que existe $T' \subseteq \W$ tal que

       \begin{multicols}{2}
            \begin{itemize}
                \item $Z^{-1}(U) \ultsExecAgi T'$, 
                \item $T' \subseteq Z^{-1}(T)$.
            \end{itemize}
        \end{multicols}

        Veamos que $T' = \{w \mid [w] \in T\}$ cumple con lo mencionado. Notemos que $T' = Z^{-1}(T)$ por lo que solo debemos demostrar $Z^{-1}(U) \ultsExecAgi T'$.

        Como $U \ultsExecAgi T$, existe $\pi \in S_i$ tal que $\pi$ es fuertemente ejecutable en todo $U$ y a su vez $\R'_\pi(U) \subseteq T$.

        Veamos que $\pi$ es fuertemente ejecutable en $Z^{-1}(U)$ y que $\R_\pi(Z^{-1}(U)) \subseteq T'$.

        Supongamos que $\pi$ no es fuertemente ejecutable en $Z^{-1}(U)$. Luego existe $\sigma \in \pi$ y $w_1,...,w_k$ con $0 \le k \le |\sigma|$ tal que $w_1 \in Z^{-1}(U)$, $(w_i,w_{i+1}) \in \R_{\sigma[i]}$ y $\R_{\sigma[k]}(w_k) = \emptyset$. 

        Ahora bien, notemos entonces que por la definición de $\R'$, $[w_1],...,[w_k]$ cumple que $([w_i],[w_{i+1}]) \in \R'_{\sigma[i]}$ y, a su vez, que $[w_1] \in U$, pues $w_1 \in Z^{-1}(U)$. Pero notemos que como $\R_{\sigma[k]}(w_k) = \emptyset$, por la contrarrecíproca del Lema 3, podemos ver que $\R'_{\sigma[k]}([w_k]) = \emptyset$, lo que nos dice que $\sigma \in \pi$ no es fuertemente ejecutable en $[w_1] \in U$. Absurdo, pues dijimos que $\pi$ es fuertemente ejecutable en todo $U$.

        Esto nos dice que $\pi$ es fuertemente ejecutable en todo $Z^{-1}(U)$.

        Veamos ahora que $\R_\pi(Z^{-1}(U)) \subseteq T'$.

        Sea $v \in \R_\pi(Z^{-1}(U))$, entonces existen $\sigma \in \pi$ y $w_1,...,w_{|\sigma|+1}$ tales que $w_1 \in Z^{-1}(U)$, $(w_i,w_{i+1}) \in \R_{\sigma[i]}$ y $w_{|\sigma|+1} = v$. 

        Ahora bien, por la definición de $\R'$, esto nos dice que $[w_1],...,[w_{|\sigma|+1}]$ cumple que $([w_i],[w_{i+1}]) \in \R'_{\sigma[i]}$ y, a su vez, como $w_1 \in Z^{-1}(U)$ entonces $[w_1] \in U$. Luego notemos que como $\R'_\pi(U) \subseteq T$, esto nos dice que $[w_{|\sigma|+1}] \in T$, lo cuál implica que $w_{|\sigma|+1} \in T'$. Como $v = w_{|\sigma|+1}$ entonces $v \in T'$. 

        Entonces demostramos que $\pi$ es fuertemente ejecutable en todo $Z^{-1}(U)$ y que $\R'_\pi(Z^{-1}(U)) \subseteq T'$. Juntando ambos resultados, concluimos que $Z^{-1}(U) \ultsExecAgi T'$, lo cuál demuestra (\KHilogic-zag).
    \end{itemize}

    Demostramos entonces que $Z$ es una \KHilogic-bisimulación. Luego $\tup{\modults,w}$ y $\tup{\modults',[w]}$ son $\KHilogic$-bisimilares para cada $w \in \W$.
    
\end{demostracion}


\section{\KHilogic-Bisimulación entre dos modelos}
    
\begin{lema}
    Sean $\model_1=\tup{\W,\R,\cset{\S_i}_{i \in \AGT},\V,\ACT}$ y $\model_2=\tup{\W',\R',\cset{\S_i'}_{i \in \AGT},\V',\ACT'}$ dos modelos tales que existe $Z \subseteq \W \times \W'$ \KHilogic-bisimulación entre ellos, entonces

    \begin{center}
        $Z' = \{(w,w') \in \W \times \W' \mid \V(w) = \V'(w')\}$
    \end{center}
    es una \KHilogic-bisimulación entre $\modults_1$ y $\modults_2$.
\end{lema}

Notemos que este teorema nos dice que para decidir si existe una \KHilogic-bisimulación entre dos modelos $\modults_1$ y $\modults_2$ basta con verificar que $Z'$ es una \KHilogic-bisimulación.

\begin{demostracion}
    Queremos ver que $Z'$ es una \KHilogic-bisimulación entre $\modults_1$ y $\modults_2$, dado que existe $Z \subseteq \W \times \W'$, \KHilogic-bisimulación entre ellos.

    Notemos que como $Z$ es una bisimulación, entonces cumple (atom). Luego, notemos que por la definición de $Z'$, $Z \subseteq Z'$, pues $Z'$ contiene todos los pares de $\W \times \W'$ que satisfacen (atom). Esto nos dice que $Z'$ satisface (A-zig) y (A-zag). A su vez, cómo mencionamos, $Z'$ contiene todos los pares que satisfacen (atom).

    Demostremos entonces que $Z'$ cumple (\KHilogic-zig) y (\KHilogic-zag).

    \begin{itemize}
        \item (\KHilogic-zig) Sean $U,T \subseteq \W$ tales que $U$ es proposicionalmente definible y $U \ultsExecAgi T$ queremos ver que existe $T' \subseteq \W'$ tal que:
    
        \begin{multicols}{2}
            \begin{itemize}
                \item $Z'(U) \ultsExecAgi T'$, 
                \item $T' \subseteq Z'(T)$.
            \end{itemize}
        \end{multicols}
    
        Notemos que como $Z$ es una bisimulación, entonces existe $T'' \subseteq \W'$ tal que:
    
        \begin{multicols}{2}
            \begin{itemize}
                \item $Z(U) \ultsExecAgi T''$, 
                \item $T'' \subseteq Z(T)$.
            \end{itemize}
        \end{multicols}
    
        Demostremos que $T' := T''$ cumple con lo mencionado. Es claro que como $Z \subseteq Z'$ y $T'' \subseteq Z(T)$ entonces $T'' \subseteq Z'(T)$. Entonces, nos queda demostrar que $Z'(U) \ultsExecAgi T''$.
    
        Esto lo demostraremos analizando que $Z(U) = Z'(U)$. Nuevamente, como $Z \subseteq Z'$ entonces $Z(U) \subseteq Z'(U)$. Luego, solo queda demostrar que $Z'(U) \subseteq Z(U)$.
    
        Sea $w' \in Z'(U)$, entonces existe $w \in U$ tal que $(w,w') \in Z'$, por lo que $\V(w) = \V'(w')$. Ahora bien, como $Z$ cumple (A-zag), existe $v \in \W$ tal que $(v,w') \in Z$, y cómo $Z$ cumple (atom) entonces $\V(v) = \V'(w')$. Luego $\V(w) = \V(v)$.
    
        Notemos que como $U$ es proposicionalmente definible entonces existe una fórmula $\varphi$ proposicional que lo define. Como $w \in U$, esto nos dice que $\modults, w \models \varphi$. Luego por Lema 1, $\modults, v \models \varphi$, por lo que $v \in U$. Como $v \in U$ y $(v,w') \in Z$ entonces $w' \in Z(U)$.
    
        Lo cuál demuestra que $Z(U) = Z'(U)$. Por lo que $Z'(U) \ultsExecAgi T''$. Finalmente, concluimos que $Z'$ cumple (\KHilogic-zig).
    
    
        \item (\KHilogic-zag) Análogo a (\KHilogic-zig).
    \end{itemize}
    
    Queda demostrado que $Z'$ es una \KHilogic-bisimulación.
\end{demostracion}


\begin{lema}
    $\KHiBisim \in \coNP$ 
\end{lema}

\begin{demostracion}
    Para demostrar que $\KHiBisim \in \coNP$ debemos dar un algoritmo $A$ y un polinomio $p$ tal que dos modelos $\modults_1$ y $\modults_2$ son \KHilogic-bisimilares si y sólo si para todo $x \in \{0,1\}^*$ con $|x| \le p(|\modults_1|+|\modults_2|)$,  $A(\modults_1,\modults_2,x) = 1$. Esencialmente, a los $x$ los llamaremos ``contraejemplos'' y sólo nos interesarán los que representen un subconjunto del dominio de $\modults_1$ o del dominio de $\modults_2$.

\medskip\medskip

    Dados que los contraejemplos que nos interesan son subconjuntos de los dominios de ambos modelos, un polinomio que nos sirve para acotar los largos de los contraejemplos es $p(x) = x$. 

\medskip\medskip
    El algoritmo $A$ que propondremos hará lo siguiente:
    
    Dados $\modults_1, \modults_2$ y $x$, verificar que el conjunto $Z'$ mencionado en el lema 4 sea una \KHilogic-bisimulación entre los dos modelos recibidos. Notar que podemos verificar polinomialmente que $Z'$ cumple (A-zig) y (A-zag).

    Luego el algoritmo verificará que $Z'$ cumpla con (\KHilogic-zig) o (\KHilogic-zag) con respecto al subconjunto $x$,  dependiendo de si $x$ es un subconjunto del dominio de $\modults_1$ o del dominio de $\modults_2$. Notar que sólo nos interesarán los $x$ que sean proposicionalmente definibles, así que el algoritmo deberá encargarse de chequear que $x$ efectivamente sea un conjunto proposicionalmente definible. 

    En caso de serlo, simplemente analizará a qué conjunto se puede llegar desde $x$ utilizando cada plan del modelo y para cada uno de ellos, verificar que $Z'(x)$ tenga algún plan con el que cumpla la condición deseada. 
    
    (Queda escribir este algoritmo y demostrar más en profundidad el si y sólo si mencionado). 

\medskip\medskip

    Finalmente $\KHiBisim \in \coNP$.

\end{demostracion}

\begin{lema}
    $\KHiBisim$ es $\coNP$-hard.
\end{lema}

\begin{demostracion}

    Demostremos que $\KHiBisim$ es $\coNP$-hard.

    Para demostrarlo, reduciremos el problema $\DNFTAUT$ a $\KHiBisim$. El problema $\DNFTAUT$ está compuesto por las fórmulas $\varphi$ proposicionales en forma disyuntiva normal que son tautologías. $\DNFTAUT$ es $\coNP$ completo, por lo que si encontramos una reducción computable en tiempo polinomial de $\DNFTAUT$ a $\KHiBisim$ habremos demostrado que $\KHiBisim$ es $\coNP$-hard.

    Entonces queremos encontrar una reducción $f$ computable en tiempo polinomial que transforme una $\varphi$ proposicional en forma disyuntiva normal a un par de modelos $\tup{\modults_1,\modults_2}$ tal que $\varphi \in \DNFTAUT$ si y sólo si $\tup{\modults_1,\modults_2} \in \KHiBisim$.

    Describamos a $f$.

    Sea $\varphi$ una fórmula proposicional en forma disyuntiva normal, entonces es de la forma $\varphi = \varphi_1 \vee ... \vee \varphi_m$. Sean a su vez $p_1,...,p_n$ las variables proposicionales que aparecen en $\varphi$, definimos los siguientes conjuntos:

    \begin{itemize}
        \item $\W := \{p_i\mid i\in\{1,...,n\}\} \cup \{\varphi_i \mid i \in \{1,...,m\}\} \cup \{e\}$
        \item $\ACT := \{in_{\varphi_i} \mid i \in \{1,...,m\}\} \cup \{out_{\varphi_i} \mid i \in \{1,...,m\}\}$
        \item $\S := \{\{in_{\varphi_i}out_{\varphi_i}\} \mid i \in \{1,...,m\}\}$
        \item $\R := \{(p_i,\varphi_j,in_{\varphi_j}) \mid p_i$ no aparece en forma negativa en $\varphi_j\} \cup \{(\varphi_j,p_i,out_{\varphi_j})\mid p_i$ aparece en forma positiva en $\varphi_j\} \cup \{(\varphi_j,e,out_{\varphi_j}) \mid \varphi_j$ no tiene variables positivas$\}$
        \item $\V := \{(p_i,\{q_i\}) \mid i \in \{1,...,n\}\} \cup \{(\varphi_i,\{q_{i+n}\}) \mid i \in \{1,...,m\}\} \cup \{(e,\{q_{n+m+1}\})\}$ donde $q_1,...,q_{n+m+1} \in \PROP$ son $n+m+1$ variables proposicionales distintas.
    \end{itemize}

    Notar que decimos que $p_i$ aparece en forma `positiva' en $\varphi_j$ si $\varphi_j = ... \wedge p_i \wedge...$ . A su vez, decimos que $p_i$ aparece en forma `negativa' en $\varphi_j$ si $\varphi_j = ...\wedge \neg p_i \wedge...$ .

    Ahora, $f$ mapeará $\varphi$ a $\tup{\modults_1,\modults_2}$, con $\tup{\modults_1,\modults_2}$ definidos de la siguiente manera:

    \begin{itemize}
        \item En el caso de que $\varphi$ evalúe a $false$ en la asignación que asigna a $p_1,...,p_n$ el valor $false$:

        \begin{itemize}
            \item $\model_1=\tup{\{1\},\{\},\{\},\{(1,p)\},\{a\}}$
            \item $\model_2=\tup{\{1\},\{\},\{\},\{(1,q)\},\{a\}}$
        \end{itemize}

        Notar que estos dos modelos son claramente no \KHilogic-bisimilares, dado que no existe relación binaria no vacía que satisfaga (atom). Pero es correcto porque sólo utilizaremos esta definición en uno de los casos donde $\varphi \not \in \DNFTAUT$. 
        
        \item En el caso de que $\varphi$ evalúe a $true$ en la asignación que asigna a $p_1,...,p_n$ el valor $false$:
        
        \begin{itemize}
            \item $\model_1=\tup{\W,\R \cup \{(p_i,p_i,test) \mid i \in \{1,...n\}\} \cup \{(p_i,e,test)\mid i \in \{1,...,n\}\},\{\S \cup \{\{test\}\} \},\V,\ACT \cup \{test\}}$
            \item $\model_2=\tup{\W,\R,\cset{\S},\V,\ACT}$
        \end{itemize}
    \end{itemize}

    
    Notar que $\V$ está definida de forma tal que cada $w\in\W$ satisface una única variable proposicional distinta, es decir, es inyectiva.


\medskip\medskip
    Estos dos párrafos que siguen eran explicativos, pensaba sacarlos pero quizás está bueno dar una explicación más humana antes de la demo más técnica.

    La intuición de esta reducción es que queremos usar los conjuntos proposicionalmente definibles de este modelo para considerar cada posible asignación de valores a las variables $p_1,...,p_n$. Dado un conjunto $U \subseteq \{p_1,...,p_n\}$ , estamos considerando la asignación que asigna $true$ a las variables dentro de $U$ y asigna $false$ a las de $\{p_1,...,p_n\}/U$. Además, notemos que como la función $\V$ es inyectiva, entonces $U$ es proposicionalmente definible.
 
    El plan $test$ que sólo aparece en $\modults_1$ es la clave de la reducción. Lo que provoca es que dada una asignación (un conjunto proposicionalmente definible $U$), a $\modults_2$ le hago buscar algún término $\varphi_i$ al cuál pueda moverse (ninguna variable de $U$ aparezca negada en $\varphi_i$) y a su vez todo variable que aparecía positiva en $\varphi_i$ estaba en $U$. (Porque el plan $test$ es como que simplemente va de $U$ a $U \cup \{e\}$), el $e$ es como que considera los términos que no tienen ninguna variable positiva ($e$ de empty set). Si $\modults_2$ no encuentra ningún término al cuál moverse es que dicho $U$ sirve como contraejemplo para afirmar que $\varphi$ no es una tautología.

\medskip\medskip
        
    Notemos que ambos modelos son polinomiales en el tamaño de $\varphi$.

    A su vez, la construcción de los modelos, solo tiene el costo extra computacional de analizar si $\varphi$ evalúa a $true$ en la asignación que asigna $false$ a $p_1,...,p_n$, lo cuál es claramente computable en tiempo polinomial.

    Por lo que es fácil notar que $f$ es computable en tiempo polinomial en el tamaño de $\varphi$.

    Ahora queremos demostrar que $\varphi \in \DNFTAUT$ si y sólo si $\tup{\modults_1,\modults_2} \in \KHiBisim$.
    \begin{itemize}

    \item $(\rightarrow)$ Sea $\varphi \in \DNFTAUT$, donde $\varphi = \varphi_1 \vee ... \vee \varphi_m$ y $p_1,...,p_n$ son las variables proposicionales que aparecen en $\varphi$, veamos que $\tup{\modults_1,\modults_2} \in \KHiBisim$.

    Notar que como $\varphi$ es una tautología, entonces $\varphi$ evalúa a $true$ en la asignación que asigna $false$ a $p_1,...,p_n$, por lo que consideramos el segundo caso del mapeo $f$. 

    Veamos que $Z := \{(w,w) \mid w \in \W \}$ es una bisimulación entre $\modults_1$ y $\modults_2$. Es fácil ver que $Z$ satisface (atom), pues ambos modelos utilizan la misma función de valuación $\V$.

    A su vez, tanto (A-zig) como (A-zag) son claramente satisfechos por $Z$.

    Analicemos entonces (\KHilogic-zig) y (\KHilogic-zag).

    Primero, cabe resaltar que, por la definición de $Z$, para cada $X \subseteq \W$ se cumple que $X = Z(X)$.

    Ahora bien, notemos que cada plan de $\modults_2$ está también en $\modults_1$, y más aún, las aristas con las etiquetas que aparecen en dichos planes son exactamente las mismas en ambos modelos. Juntando lo recién mencionado y el hecho de que $U = Z(U)$, es fácil notar que Z satisface (\KHilogic-zag).

    Luego solo queda ver que $Z$ satisface (\KHilogic-zig).

    Sea $U, T \subseteq \W$ tal que $U$ es proposicionalmente definible y $U \ultsExecAgi T$, queremos ver que existe $T' \subseteq \W$ tal que:

    \begin{multicols}{2}
        \begin{itemize}
            \item $Z(U) \ultsExecAgi T'$, 
            \item $T' \subseteq Z(T)$.
        \end{itemize}
    \end{multicols}

    Notemos primero que si el plan que atestigua $U \ultsExecAgi T$ es de la forma $in_{\varphi_j}out_{\varphi_j}$, entonces $T$ sirve como $T'$ dado que ese mismo plan y las mismas aristas con dichas etiquetas están en $\modults_2$, similar al análisis realizado para el caso (\KHilogic-zag).

    Entonces consideremos el caso en el que el plan que atestigua $U \ultsExecAgi T$ es $test$. En dicho caso, si analizamos la definición de $\R$, ese plan puede ser fuertemente ejecutable sólo si $U \subseteq \{p_1,...,p_n\}$ dado que son los únicos nodos que tienen aristas con etiqueta $test$. Así que solo necesitamos considerar los casos donde $U \subseteq \{p_1,...,p_n\}$.

    Notemos que $\R_{test}(U) = U \cup \{e\}$, entonces necesariamente $U \cup \{e\} \subseteq T$.

    Consideremos ahora la asignación $\overrightarrow{a}$ que asigna $true$ a las variables en $U$ y $false$ a las variables en $\{p_1,...,p_n\} / U$. Como $\varphi$ es una tautología, entonces existe $\varphi_j$ que se vuelve verdadero a partir de $\overrightarrow{a}$, es decir, en $\varphi_j$ las variables que aparecen en forma positiva son algún subconjunto de $U$ y las variables que aparecen en forma negativa son algún subconjunto de $\{p_1,...,p_n\}/U$.

    Como las variables que aparecen en $\varphi_j$ en forma negativa no están en $U$, entonces, analizando la definición de $\R$, podemos ver que el plan $in_{\varphi_j}out_{\varphi_j}$ es fuertemente ejecutable en $Z(U) = U$. 
    
    En particular, las aristas con etiqueta $in_{\varphi_j}$ llevarían de cada nodo de $U$ al nodo $\varphi_j$.
    
    Ahora bien, notemos que las aristas con etiqueta $out_{\varphi_j}$ llevan del nodo $\varphi_j$ a las variables que aparecen en forma positiva en $\varphi_j$ o al nodo $e$ en caso de no tener variables positivas. Pero, como dijimos, las variables positivas que aparecen en $\varphi_j$ son un subconjunto, posiblemente vacío, de $U$. Entonces necesariamente de $\varphi_j$ las aristas con etiqueta $out_{\varphi_j}$ van a un conjunto $X \subseteq U \cup \{e\}$, es decir, $\R_{in_{\varphi_j}out_{\varphi_j}}(U) = X \subseteq U \cup \{e\} \subseteq T$. Finalmente, $T' := X$ nos sirve para demostrar que $Z(U) \ultsExecAgi T'$ y, a su vez, $T' \subseteq X \subseteq U \cup \{e\} \subseteq T = Z(T)$.

    Lo cual demuestra que $Z$ satisface (\KHilogic-zig). 

    Juntando los puntos mencionados, demostramos que $Z$ es una bisimulación entre $\modults_1$ y $\modults_2$, por lo que $\tup{\modults_1,\modults_2} \in \KHiBisim$. 

    \item ($\leftarrow$) Para demostrar este caso, veremos que siendo $\varphi \notin \DNFTAUT$, donde $\varphi = \varphi_1 \vee ... \vee \varphi_m$ y $p_1,...,p_n$ son las variables proposicionales que aparecen en $\varphi$, entonces $\tup{\modults_1,\modults_2} \notin \KHiBisim$.

    Como $\varphi \notin \DNFTAUT$, existe una asignación $\overrightarrow{a}$ sobre las variables $p_1,...,p_n$ que hace fallar cada $\varphi_j$. Denotemos a $U$ como el conjunto de variables a las que se le asigna $true$ en $\overrightarrow{a}$.

    Si $U = \emptyset$, es claro que $\tup{\modults_1,\modults_2} \notin \KHiBisim$, dado que en este caso se construyen dos modelos concretos que no son \KHilogic-bisimilares.

    Así que consideremos el caso donde $U \neq \emptyset$.
    
    Primero notemos que la fórmula $\psi := \bigvee\limits_{p_i \in U} q_i$ define a $U$. Pues, recordemos que por la definición de $\V$, $\V(p_i) = \{q_i\}$ para cada $i \in \{1,...,n\}$. A su vez, como $\V$ es inyectiva, solo los elementos de $U$ satisfacerán $\psi$. Luego, como $\psi$ es proposicional, $U$ es proposicionalmente definible.

    Ahora bien, como $U \subseteq \{p_1,...,p_n\}$ entonces el plan $test$ es fuertemente ejecutable en $U$ y $\R_{test}(U) = U \cup \{e\}$, por lo que $U \ultsExecAgi U \cup \{e\}$.
    
    Veamos que no existe $T'$ tal que:

    \begin{multicols}{2}
        \begin{itemize}
            \item $Z(U) \ultsExecAgi T'$, 
            \item $T' \subseteq Z(U \cup \{e\})$.
        \end{itemize}
    \end{multicols}

    Notemos que los planes que tiene a su disposición $Z(U) = U$ son de la forma $in_{\varphi_j}out_{\varphi_j}$.

    Sea entonces $\varphi_j$, analicemos qué sucede en relación al plan $in_{\varphi_j}out_{\varphi_j}$. Un primer detalle a tener en cuenta es que como $\overrightarrow{a}$ hace fallar a $\varphi_j$, entonces, o bien ocurre que existe una variable en $U$ que aparece en forma negativa en $\varphi_j$, o existe una variable en $\{p_1,...,p_n\}/U$ que aparece en forma positiva en $\varphi_j$.

    Si ocurre que una variable de $U$ aparece en forma negativa en $\varphi_j$ entonces el plan $in_{\varphi_j}out_{\varphi_j}$ no será fuertemente ejecutable en dicha variable y, por lo tanto, no será fuertemente ejecutable en $U$. Así que este plan no permitiría encontrar un $T'$ adecuado.

    Por otro lado, si ninguna variable de $U$ aparece en forma negativa en $\varphi_j$ entonces $in_{\varphi_j}out_{\varphi_j}$ es fuertemente ejecutable en $U$. Sin embargo, notemos que entonces existe una variable $p_k$ en $\{p_1,...,p_n\}/U$ que aparece en forma positiva en $\varphi_j$. Ahora bien, como $p_k$ aparece en forma positiva en $\varphi_j$ entonces $p_k \in \R_{in_{\varphi_j}out_{\varphi_j}}(U)$. Luego $\R_{in_{\varphi_j}out_{\varphi_j}}(U) \not \subseteq U \cup \{e\} = Z(U \cup \{e\})$, por lo que dicho plan no nos permite encontrar un $T'$ adecuado.
    
    Hemos analizado entonces todos los planes y ninguno permite encontrar un $T'$ que cumpla con lo requerido, luego $Z$ no satisface (\KHilogic-zig).

    Hemos demostrado entonces que $Z$ no es una \KHilogic-bisimulación.

    Lo cual demuestra que $\tup{\modults_1,\modults_2} \not \in \KHiBisim$.
    \end{itemize}

     Finalmente, demostramos que $f$ es una reducción de $\DNFTAUT$ a $\KHiBisim$, la cuál puede ser computada en tiempo polinomial. Lo cuál demuestra que $\KHiBisim$ es $\coNP$-hard.

\end{demostracion}


\begin{teorema}
    $\KHiBisim$ es $\coNP$-completo.
\end{teorema}


\begin{demostracion}

    A partir de los lemas 5 y 6, podemos afirmar que $\KHiBisim$ es $\coNP$-completo.
    
\end{demostracion}


\section{Ideas sueltas}
    \chapter{Ideas sueltas}

\begin{teorema}
    Sea $\model_=\tup{\W,\R,\cset{\S_i}_{i \in \AGT},\V,\ACT}$ un modelo, y sea $Z \subseteq \W \times \W$ una relación binaria que satisface (Atom) y $Id := \{(w,w)\in \W\} \subseteq Z$, entonces $Z$ es una autobisimulación.
\end{teorema}

No se si dice mucho este teorema, pero al menos caracteriza un poco las autobisimulaciones.

Resumen de lo trabajado:

\begin{itemize}

\item Primera sección: Contracción por bisimulación

Conclusiones positivas acerca de las contracciones, nos dicen cosas sobre la expresividad de la lógica, en el sentido de que demostramos que dado un modelo siempre existe otro en el que la función de valuación es inyectiva (no hay dos nodos que satisfagan exactamente el mismo conjunto de variables proposicionales), usa planes de solo un paso y las clases de equivalencia de planes que conoce cada agente son simplemente singletons y aún así la lógica no sabe cómo diferenciarlos (son lógicamente equivalentes).

También, logramos contraer un modelo a otro donde chequear strong executability se logra en un solo paso, por lo que potencialmente se ahorra cómputo al no necesitar cada paso intermedio de un plan.

Ambas contracciones son computables en tiempo polinomial en el tamaño del modelo lo cuál es muy bueno.


Conclusiones negativas, no logramos encontrar una contracción que de verdad minimice el \rm\textbf{tamaño} del modelo en cuestión. Si bien, seguro encontramos un modelo lógicamente equivalente que minimiza la cantidad de nodos, puede darse el caso que la contracción aumente considerablemente la cantidad de aristas (sin embargo, hay que considerar el trade-off del ahorro en el chequeo de strong executability).

Cosas a analizar: 
\begin{itemize}
    \item Hay una contracción mejor?
    \item Puede tener que ver la definición de bisimulación con el hecho de que no encontramos una contracción totalmente convincente (ver teorema de esta sección capaz)? 
    \item Como dijo Carlos, puede tener que ver con la naturaleza de los modelos donde se interpreta la lógica? En ese caso considerar hacer un análisis desde teoría de modelos.
\end{itemize}


\item Segunda sección: Bisimulación entre dos modelos

Queda escribir el algoritmo para demostrar membership y revisar la demo de hardness para confirmar que la reducción está efectivamente bien.

Creo que este problema queda bastante completo una vez hechas esas dos cosas.

\end{itemize}

Anotaciones de Raul, reunión 19/06.

1. Re-definir la noción de Kh-bisimulación, usando las ideas de autobisimulación considerando la valuación, para el caso de modelos diferentes.

2. Demostrar que con la nueva definición valen: bisimulación implica equivalencia de fórmulas, y en el caso finito, vale la recíproca.

3. Demostrar que la unión de dos bisimulaciones es una bisimulación.

4. Darle nueva estructura a la parte de autobisimulación: dividir las dos dimensiones, es decir,contracción por estados (clásico) y luego la contracción por ''lenguaje'' (más raro).

A esto agregar:

1. Corregir sintaxis de la lógica.

2. Introducir en preliminares la noción existente de bisimulación.

\end{document}

