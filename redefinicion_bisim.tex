La noción de bisimulación de una lógica modal nos permite relacionar dos modelos a partir de características puramente estructurales 
que comparten entre sí, con las que se logra garantizar la equivalencia lógica entre ellos. En ese sentido, consideramos que la 
bisimulación modela la idea de equivalencia en cuanto al comportamiento de ambos modelos respecto a una lógica en particular.

Ahora bien, si analizamos la definición presentada en \ref{def:bisimulation}, podemos notar que en (\KHilogic-zig) y (\KHilogic-zag) se piden condiciones 
sobre los conjuntos proposicionalmente definibles de cada modelo, lo cuál naturalmente no es una condición estructural del modelo, 
ya que dichos conjuntos dependen de la existencia de una fórmula, es decir, un objeto puramente sintáctico de la lógica. 

Es así como surge la motivación de esta sección. Propondremos una nueva noción de bisimulación para la lógica presentada, modificando las condiciones
(\KHilogic-zig) y (\KHilogic-zag) para que solo analicen características estructurales de ambos modelos.

Primero veremos algunos lemas para caracterizar los conjuntos proposicionalmente definibles de un modelo $\modults$.

\begin{lema}\label{ref:propositional-equivalence}
    Sea $\model=\tup{\W,\R,\cset{\S_i}_{i \in \AGT},\V,\ACT}$ un \ults, y sean $w$, $v \in \W$
    tales que $\V(w)$ = $\V(v)$ entonces para toda $\varphi$ proposicional se cumple que, 
    \begin{center}
    $\modults, w \models \varphi$ \quad si y sólo si \quad $ 
    \modults, v \models \varphi$
    \end{center}
\end{lema}

\begin{demostracion}
    Sea $\model=\tup{\W,\R,\cset{\S_i}_{i \in \AGT},\V,\ACT}$ y $w$, $v \in \W$
    tales que $\V(w)$ = $\V(v)$. La demostración es por inducción estructural sobre $\varphi$. 
    Recordar que $\varphi$ es una fórmula \textbf{proposicional}.

    \begin{itemize}
        \item Caso base: $\varphi = p$ donde $p \in \PROP$.

        Notar que  $\modults, w \models \varphi$ si y sólo si $p \in \V(w)$.

        Ahora bien, como $\V(w)$ = $\V(v)$, $p \in \V(w)$ si y sólo si $p \in \V(v)$.

        Finalmente, por la definición de $\models$, $p \in \V(v)$ si y sólo si $\modults, v \models \varphi$.  
    
        \item Caso inductivo: La hipótesis inductiva establece que, para $\psi$ una subfórmula de $\varphi$, 
        se cumple que $\modults, w \models \psi$ si y sólo si $\modults, v \models \psi$.

        \begin{itemize}
            \item Caso $\varphi = \psi_1 \lor \psi_2$. 
    
            Por definición de $\models$, $\modults, w \models \psi_1 \lor \psi_2$ si y sólo si $\modults,
            w \models \psi_1$ o $\modults, w \models \psi_2$.
            
            Por hipótesis inductiva, $\modults, w \models \psi_1$ o $\modults, w \models \psi_2 $ si y sólo
            si $\modults, v \models \psi_1$ o $\modults, v \models \psi_2$. 
            
            Pero notar que, nuevamente por definición de $\models$, $\modults, v \models \psi_1$ o $\modults,
            v \models \psi_2$ si y sólo si $\modults,v \models \psi_1 \lor \psi_2$.  

            \item Caso $\varphi = \neg \psi$.
        
            La demostración es similar a la del caso analizado anteriormente.
        \end{itemize}
    
    \end{itemize}
    
\end{demostracion}

Ahora definimos para cada modelo una relación de equivalencia sobre su dominio, la cuál particiona a los nodos de acuerdo
a su función de valuación.

\begin{definicion}
    Sea $\model=\tup{\W,\R,\cset{\S_i}_{i \in \AGT},\V,\ACT}$ un \ults entonces se define, 
    \begin{center}
        $A_\modults := \{(w,v) \in \W \times \W \mid \V(w) = \V(v)\}$
    \end{center}
    Notar que $A_\modults$ es una relación de equivalencia sobre $\W$. Luego se denotará,
    \begin{center}
        $\rho_\modults := \{ [w] \mid w \in \W $ y $[w]$ su clase de equivalencia respecto a $A_\modults\}$
    \end{center}
\end{definicion}

Ahora estamos en condiciones de presentar esta propiedad de los conjuntos proposicionalmente definibles.

\begin{lema}\label{ref:propositionaly-definable-lemma}
    Sea $\model=\tup{\W,\R,\cset{\S_i}_{i \in \AGT},\V,\ACT}$ un \ults y sea $U \subseteq \W$ tal que $U$ es proposicionalmente definible, entonces
    \begin{center}
        para cada $s_i \in \rho_\modults$ se cumple que $s_i \cap U = \emptyset$ o $s_i \subseteq U$.
    \end{center}
\end{lema}

\begin{demostracion}
    $\model=\tup{\W,\R,\cset{\S_i}_{i \in \AGT},\V,\ACT}$ un \ults y sea $U \subseteq \W$ tal que $U$ es proposicionalmente definible, 
    veamos que se cumple la propiedad mencionada.

    Como $U$ es proposicionalmente definible entonces existe $\varphi$ proposicional que lo define.

    Sea $s_i \in \rho_\modults$, notemos que si $s_i \cap U = \emptyset$ entonces la propiedad vale.
    
    Supongamos entonces que $s_i \cap U \neq \emptyset$ y veamos que $s_i \subseteq U$. Como $s_i \cap U \neq \emptyset$, entonces existe 
    $w \in U$ tal que $[w] = s_i$, y como $w \in U$ entonces ocurre que $\modults,w \models \varphi$. Ahora bien, sea $v \in s_i$ 
    entonces sabemos que $\V(v) = \V(w)$, luego por \ref{ref:propositional-equivalence} tenemos que $\modults,v \models \varphi$, por 
    lo que $v \in U$. Lo que nos dice que $s_i \subseteq U$ que era lo que queríamos demostrar.  
\end{demostracion}

A partir de esta propiedad, es como surge la nueva noción de bisimulación que proponemos para la lógica `Knowing-How' multi-agente con 
incertidumbre.

\begin{definicion}[\KHilogic-bisimulación redefinición]
    Sean $\modults$ y $\modults'$ dos \ultss, con dominios $\W$ y $\W'$ respectivamente.
    Una relación binaria no vacía $Z \subseteq \W \times \W'$ es llamada una \KHilogic-bisimulación entre $\modults$ y 
    $\modults'$ si y sólo si $(w,w') \in Z$ implica lo siguiente
    \begin{itemize}
        \item \textbf{Atom}: $\V(w)=\V'(w')$.

        \item \textbf{$\khi$-Zig}: Sea $U \subseteq \W$ tal que para cada $s_i \in \rho_\modults$ se cumple que $s_i \cap U = \emptyset$ o $s_i \subseteq U$, si $U \ultsExecAgi T$ para algún $T \subseteq \W$, entonces existe $T' \subseteq \W'$ tal que
        \begin{multicols}{2}
            \begin{cond-bisim}
                \item $Z(U) \ultsExecAgi T'$, 
                \item $T' \subseteq Z(T)$.
            \end{cond-bisim}
        \end{multicols}
        
        \item \textbf{$\khi$-Zag}: Sea $U' \subseteq \W'$ tal que para cada $s_i' \in \rho_{\modults'}$ se cumple que $s_i' \cap U' = \emptyset$ o $s_i' \subseteq U'$, si $U' \ultsExecAgi T'$ para algún $T' \subseteq \W'$, entonces existe $T \subseteq \W$ tal que
        \begin{multicols}{2}
            \begin{cond-bisim}
                \item $Z^{-1}(U') \ultsExecAgi T$,
                \item $T \subseteq Z^{-1}(T')$.
            \end{cond-bisim}
        \end{multicols}

        \item \textbf{A-Zig}: para cada $u \in \W$ existe $u' \in \W'$ tal que $(u,u') \in Z$.

        \item \textbf{A-Zag}: para cada $u' \in \W'$ existe $u \in \W$ tal que $(u,u') \in Z$.
    \end{itemize} 

    Escribiremos $\modults,w \bisim' \modults',w'$ (quizás usar nuevo símbolo) cuando exista una \KHilogic-bisimulación $Z$ entre
    $\modults$ y $\modults'$ tal que $(w,w') \in Z$.
\end{definicion}

Una vez definida la nueva noción, notemos que \ref{ref:propositionaly-definable-lemma} nos permite demostrar de forma casi directa que
toda relación binaria que satisface las nuevas condiciones presentadas de (\KHilogic-zig) y (\KHilogic-zag) también satisface las condiciones
presentadas en \ref{def:bisimulation}.

\begin{lema}
    Sea $Z$ una bisimulación con la nueva def, entonces $Z$ es una bisimulación con la vieja def. (reescribir).
\end{lema}