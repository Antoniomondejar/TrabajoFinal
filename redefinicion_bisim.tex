\chapter{Redefiniendo la Noción de Bisimulación}

La noción de bisimulación de una lógica modal nos permite relacionar dos modelos a partir de características puramente estructurales 
que comparten entre sí, con las que se logra garantizar la equivalencia lógica entre ellos. En ese sentido, consideramos que la 
bisimulación modela la idea de equivalencia en cuanto al comportamiento de ambos modelos respecto a una lógica en particular, lo cuál nos brinda
una caracterización de la expresividad de la misma.

Si analizamos la noción presentada en la \Cref{def:bisimulation}, podemos notar que ($\khi$-zig) y ($\khi$-zag) requieren condiciones 
sobre los conjuntos proposicionalmente definibles de los dominios de cada modelo. Dicha condición establece la existencia de una fórmula proposicional
(es decir, de un objeto sintáctico), que defina exactamente un conjunto. Esto contradice la naturaleza estructural y puramente semántica 
de las bisimulaciones. 

A su vez, a la hora de presentar una noción de bisimulación, es deseable que la misma tenga una naturaleza `procedural', es decir,
que dada una relación binaria entre los dominios de dos modelos sea posible diseñar un procedimiento efectivo que verifique si dicha relación 
es una bisimulación.

Nuevamente, si analizamos la definicion presentada en la sección anterior, podemos notar que para verificar que una relación binaria es una bisimulación
tenemos que explorar necesariamente los conjuntos proposicionalmente definibles de los dominios de cada modelo, lo cuál lleva consigo una cantidad 
infinita de fórmulas proposicionales que analizar. Por este motivo, no parece una tarea sencilla encontrar un procedimiento efectivo que realice lo deseado.

Es así como surge la motivación de este capítulo. Propondremos una nueva definición de bisimulación para la lógica presentada, modificando las condiciones
($\khi$-zig) y ($\khi$-zag) para que sólo se consideren características estructurales de ambos modelos, y, a su vez, se tenga una naturaleza procedural.

Comenzaremos anunciando algunas propiedades que serán de utilidad para obtener una definición alternativa de bisimulación para \KHilogic.

\begin{lema}\label{lema:propositional-equivalence}
    Sea $\model=\tup{\W,\R,\cset{\S_i}_{i \in \AGT},\V,\ACT}$ un \ults, y sean $w$, $v \in \W$
    tales que $\V(w)$ = $\V(v)$. Entonces, para toda $\varphi$ proposicional se cumple que, 
    \begin{center}
    $\modults, w \models \varphi$ \quad si y sólo si \quad $ 
    \modults, v \models \varphi$
    \end{center}
\end{lema}

\begin{demostracion}
    Sea $\model=\tup{\W,\R,\cset{\S_i}_{i \in \AGT},\V,\ACT}$ un \ults y $w$, $v \in \W$
    tales que $\V(w)$ = $\V(v)$. La demostración es por inducción estructural sobre $\varphi$. 
    Recordar que $\varphi$ es una fórmula \textbf{proposicional}.

    \begin{itemize}
        \item Caso base: $\varphi = p$ donde $p \in \PROP$.

        Notar que  $\modults, w \models \varphi$ si y sólo si $p \in \V(w)$.

        Como $\V(w)$ = $\V(v)$, $p \in \V(w)$ si y sólo si $p \in \V(v)$.

        Finalmente, por la definición de $\models$, $p \in \V(v)$ si y sólo si $\modults, v \models \varphi$.  
    
        \item Caso inductivo: La hipótesis inductiva establece que, para $\psi$ una subfórmula de $\varphi$, 
        se cumple que $\modults, w \models \psi$ si y sólo si $\modults, v \models \psi$.

        \begin{itemize}
            \item Caso $\varphi = \psi_1 \lor \psi_2$. 
    
            Por definición de $\models$, $\modults, w \models \psi_1 \lor \psi_2$ si y sólo si $\modults,
            w \models \psi_1$ o $\modults, w \models \psi_2$.
            
            Por hipótesis inductiva, $\modults, w \models \psi_1$ o $\modults, w \models \psi_2 $ si y sólo
            si $\modults, v \models \psi_1$ o $\modults, v \models \psi_2$. 
            
            Pero notar que, nuevamente por definición de $\models$, $\modults, v \models \psi_1$ o $\modults,
            v \models \psi_2$ si y sólo si $\modults,v \models \psi_1 \lor \psi_2$.  

            \item Caso $\varphi = \neg \psi$.
        
            La demostración es similar a la del caso analizado anteriormente.
        \end{itemize}
        Luego, queda demostrada la propiedad.
    \end{itemize}
    
\end{demostracion}

Definiremos ahora para cada modelo, una relación de equivalencia sobre su dominio, la cuál particiona a los nodos de acuerdo
a su función de valuación.

\begin{definicion}\label{def:Am}
    Sea $\model=\tup{\W,\R,\cset{\S_i}_{i \in \AGT},\V,\ACT}$ un \ults entonces se define, 
    \begin{center}
        $A_\modults := \{(w,v) \in \W \times \W \mid \V(w) = \V(v)\}$.
    \end{center}
    Notar que $A_\modults$ es una relación de equivalencia sobre $\W$. Luego se denotará,
    \begin{center}
        $\rho_\modults := \{ [w] \mid w \in \W $ y $[w]$ su clase de equivalencia respecto a $A_\modults\}$.
    \end{center}
\end{definicion}

Ahora estamos en condiciones de presentar una propiedad fundamental de los conjuntos proposicionalmente definibles del dominio de un modelo $\modults$.

\begin{lema}\label{lema:propositionally-definable-lemma}
    Sea $\model=\tup{\W,\R,\cset{\S_i}_{i \in \AGT},\V,\ACT}$ un \ults y sea $U \subseteq \W$ tal que $U$ es proposicionalmente definible, entonces
    \begin{center}
        para cada $s \in \rho_\modults$ se cumple que $s \cap U = \emptyset$ o $s \subseteq U$.
    \end{center}
\end{lema}

\begin{demostracion}
    Sea $\model=\tup{\W,\R,\cset{\S_i}_{i \in \AGT},\V,\ACT}$ un \ults y sea $U \subseteq \W$ tal que $U$ es proposicionalmente definible, 
    veamos que se cumple la propiedad mencionada.

    Como $U$ es proposicionalmente definible entonces existe $\varphi$ proposicional que lo define.

    Sea $s \in \rho_\modults$, queremos ver que $s \cap U = \emptyset$ o $s \subseteq U$. Supongamos entonces que 
    $s \cap U \neq \emptyset$ y veamos que $s \subseteq U$. 
    
    Como $s \cap U \neq \emptyset$, entonces existe $w \in U$ tal que $[w] = s$, y como $w \in U$ entonces ocurre que 
    $\modults,w \models \varphi$. Sea $v \in s$, entonces sabemos que $\V(v) = \V(w)$, luego por 
    \Cref{lema:propositional-equivalence} tenemos que $\modults,v \models \varphi$, por lo que $v \in U$. Lo que nos dice que 
    $s \subseteq U$, lo que queríamos demostrar.  
\end{demostracion}

A partir de esta propiedad, surge la nueva definición de bisimulación que proponemos para la lógica de `Knowing-How' multi-agente basada en 
incertidumbre.

\begin{definicion}[\KHilogic-bisimulación$^*$]\label{def:bisim_redefinition}
    Sean $\modults$ y $\modults'$ dos \ultss, con dominios $\W$ y $\W'$ respectivamente.
    Una relación binaria no vacía $Z \subseteq \W \times \W'$ es llamada una \KHilogic-bisimulación$^*$ entre $\modults$ y 
    $\modults'$ si y sólo si $(w,w') \in Z$ implica lo siguiente:
    \begin{itemize}
        \item \textbf{Atom}: $\V(w)=\V'(w')$.

        \item \textbf{$\khi$-zig$^*$}: Sea $U \subseteq \W$ tal que para cada $s \in \rho_\modults$ se cumple que $s \cap U = \emptyset$ o $s \subseteq U$, si $U \ultsExecAgi T$ para algún $T \subseteq \W$, entonces existe $T' \subseteq \W'$ tal que
        \begin{multicols}{2}
            \begin{cond-bisim}
                \item $Z(U) \ultsExecAgi T'$, 
                \item $T' \subseteq Z(T)$.
            \end{cond-bisim}
        \end{multicols}
        
        \item \textbf{$\khi$-zag$^*$}: Sea $U' \subseteq \W'$ tal que para cada $s' \in \rho_{\modults'}$ se cumple que $s' \cap U' = \emptyset$ o $s' \subseteq U'$, si $U' \ultsExecAgi T'$ para algún $T' \subseteq \W'$, entonces existe $T \subseteq \W$ tal que
        \begin{multicols}{2}
            \begin{cond-bisim}
                \item $Z^{-1}(U') \ultsExecAgi T$,
                \item $T \subseteq Z^{-1}(T')$.
            \end{cond-bisim}
        \end{multicols}

        \item \textbf{A-zig}: para cada $u \in \W$ existe $u' \in \W'$ tal que $(u,u') \in Z$.

        \item \textbf{A-zag}: para cada $u' \in \W'$ existe $u \in \W$ tal que $(u,u') \in Z$.
    \end{itemize} 

    Escribiremos $\modults,w \bisim^* \modults',w'$ cuando exista una \KHilogic-bisimulación$^*$ $Z$ entre
    $\modults$ y $\modults'$ tal que $(w,w') \in Z$.
\end{definicion}

\begin{ejemplo}\label{ejemplo:bisim^*}
    Consideremos $\modults$ y $\modults'$ presentados en \Cref{ejemplo:bisim}. Veamos que la relación $Z$, representada 
    gráficamente por las líneas punteadas, es una \KHilogic-bisimulación$^*$. 

    Como las condiciones (Atom), (A-zig) y (A-zag) se mantienen en la definición de \KHilogic-bisimulación$^*$, utilizando el mismo 
    análisis que en \Cref{ejemplo:bisim} podemos afirmar que $Z$ las satisface.

    Veamos como se comporta $Z$ con respecto a ($\khi$-zig$^*$). Primero notemos que $\rho_\modults = \{\{w_1,w_2\}, \{w_3,w_4\}\}$, ahora 
    consideremos los subconjuntos $U$ del dominio de $\modults$ que satisfacen que para cada $s \in \rho_\modults$ se cumple que 
    $s \subseteq U$ o $s \cap U = \emptyset$. 
    
    Se puede ver que los subconjuntos que cumplen esa condición son exactamente los subconjuntos 
    proposicionalmente definibles, para los cuáles ya realizamos un análisis en el ejemplo previamente mencionado. 
    Por lo que podemos afirmar que $Z$ satisface ($\khi$-zig$^*$).
    
    A su vez, podemos realizar un análisis similar para demostrar que $Z$ satisface ($\khi$-zag$^*$). 

    Luego, $Z$ es una \KHilogic-bisimulación$^*$. Más aún, podemos decir que $\modults,w_j \bisim^* \modults',w_j'$ con $1 \leq j \leq 4$.
\end{ejemplo}


Una vez presentada la nueva definición, notemos que \Cref{lema:propositionally-definable-lemma} nos permite demostrar de forma casi directa que
toda relación binaria que satisface las nuevas condiciones presentadas de ($\khi$-zig$^*$) y ($\khi$-zag$^*$) también satisface las condiciones
presentadas en la \Cref{def:bisimulation}.

\begin{lema}[\KHilogic-bisimilitud$^*$ implica \KHilogic-bisimilitud]\label{lema:new-implies-old}
    Sean $\modults$ y $\modults'$ dos \ultss, con dominios $\W$ y $\W'$ respectivamente. Si $Z \subseteq \W \times \W'$ es una \KHilogic-bisimulación$^*$ entonces $Z$ es una \KHilogic-bisimulación.
\end{lema}

\begin{demostracion}
    Queremos ver que $Z$ es una \KHilogic-bisimulación.

    Notemos que $Z$ satisface (Atom), (A-zig) y (A-zag) dado que $Z$ es una \KHilogic-bisimulación$^*$.

    Veamos entonces que satisface ($\khi$-zig) y ($\khi$-zag).

    \begin{itemize}
        \item ($\khi$-zig) Sea $U \subseteq \W$ proposicionalmente definible y sea $T \subseteq \W$ tal que $U \ultsExecAgi T$, queremos ver que existe
        $T' \subseteq \W'$ tal que

        \begin{multicols}{2}
            \begin{cond-bisim}
                \item $Z(U) \ultsExecAgi T'$, 
                \item $T' \subseteq Z(T)$.
            \end{cond-bisim}
        \end{multicols}
        Notemos que como $U$ es proposicionalmente definible entonces por \Cref{lema:propositionally-definable-lemma}, $U$ satisface que para cada $s \in \rho_\modults$ se cumple
        que $s \cap U = \emptyset$ o $s \subseteq U$.
    
        Luego como $Z$ satisface ($\khi$-zig$^*$) tenemos asegurada la existencia de dicho $T'$.
    
        \item ($\khi$-zag) Análogo a ($\khi$-zig).
    \end{itemize}

    Luego, $Z$ es una \KHilogic-bisimulación.
\end{demostracion}

Notemos como este lema nos permite demostrar el Teorema de Invarianza para nuestra nueva definición de bisimulación.

\begin{teorema}[\KHilogic-bisimilitud$^*$ implica \KHilogic-equivalencia]
    Sean $\modults,w$ y $\modults',w'$ dos \ultss punteados, entonces
    \begin{center}
        $\modults,w \bisim^* \modults',w'$ implica $\modults,w \modequiv \modults',w'$.
    \end{center}
\end{teorema}

\begin{demostracion}
    Sean $\modults,w$ y $\modults',w'$ dos \ultss punteados tales que $\modults,w \bisim^* \modults',w'$, entonces existe $Z$ \KHilogic-bisimulación$^*$
    entre $\modults$ y $\modults'$ tal que $(w,w') \in Z$.
    
    Luego por \Cref{lema:new-implies-old} sabemos que $Z$ es una \KHilogic-bisimulación, y como $(w,w') \in Z$ entonces $\modults,w \bisim \modults',w'$.

    Finalmente, por \Cref{thm:bisim-implies-equivalence}, tenemos que $\modults,w \modequiv \modults',w'$.
\end{demostracion}

Una vez demostrado el Teorema de Invarianza para la definición propuesta, nos gustaría demostrar que también se mantiene la otra dirección en
el caso finito. 

Para lograr esto, primero fortaleceremos la propiedad dada en \Cref{lema:propositionally-definable-lemma} para \ultss con dominio finito. 

\begin{lema}\label{lema:finite-propositionally-definable-lemma}
    Sea $\model=\tup{\W,\R,\cset{\S_i}_{i \in \AGT},\V,\ACT}$ un \ults en \MFD y sea $U \subseteq \W$, entonces
    para cada $s \in \rho_\modults$ se cumple que $s \cap U = \emptyset$ o $s \subseteq U$ si y sólo si $U$ es proposicionalmente definible.
\end{lema}

\begin{demostracion}
    Notar que la direccion ($\leftarrow$) está demostrada en \Cref{lema:propositionally-definable-lemma}. Entonces demostremos ($\rightarrow$).

    Supongamos que para cada $s \in \rho_\modults$ se cumple que $s \cap U = \emptyset$ o $s \subseteq U$.

    Por un lado si $U = \emptyset$, luego la fórmula $ \varphi := \bot$ define a $U$. Entonces supongamos que $U \neq \emptyset$. 
    
    Notemos que como $\modults \in$ \MFD, entonces $\rho_\modults$ es finito. Denotaremos a sus elementos como $\rho_\modults = \{s_1,...,s_n\}$. 
    A su vez, como para cada $v,w \in s_i$ se cumple que $\V(w) = \V(v)$ entonces denotaremos $\V(s_i)$ a $\V(w)$ con $w \in s_i$. 
    
    Luego, sean $i \neq j$ entonces existe una variable proposicional $p$ tal que $p \in \V(s_i)$ y $p \not\in \V(s_j)$ o $p \not\in \V(s_i)$ 
    y $p \in \V(s_j)$. A dicha variable la denotaremos $p_{i,j}$.



    Sea $\varphi_i := \bigwedge\limits_{\substack{j = 1 \\ j \neq i}}^{n} l(p_{i,j})$ donde $l(p_{i,j}) = $
    $\left\{ \begin{array}{rcl}
            p_{i,j} & \mbox{si}
            & p_{i,j} \in \V(s_i) \\ \neg p_{i,j} & \mbox{si} & p_{i,j} \notin \V(s_i) \\
            \end{array}\right. 
    $.

    Veamos que por la forma en la que construimos $\varphi_i$, se cumple que $\modults, w \models \varphi_i$ si y sólo si $w \in s_i$. ($*$)
    
    Demostremos entonces que $\varphi := \bigvee\limits_{s_i \subseteq U}\varphi_i$ define a $U$. Primero, notar que sea $w \in \W$, 
    $\modults, w \models \varphi$ si y sólo si $\modults, w \models \varphi_i$ para algún $i \in \{1,...n\}$ tal que $s_i \subseteq U$.  
    Queremos ver entonces que, $w \in U$ si y sólo si $\modults, w \models \varphi$. 

    Sea $w \in U$, entonces por hipótesis se cumple que $[w] \subseteq U$. Luego, existe $s_i \in \rho_\modults$ tal que 
    $s_i = [w]$, por lo que $s_i \subseteq U$. Por $(*)$, $\modults, w \models \varphi_i$, lo que nos dice que $\modults, w \models \varphi$.

    Sea $w \in \W$ tal que $\modults, w \models \varphi$, entonces existe $i \in \{1,...,n\}$ con $s_i \subseteq U$ tal que $\modults, w \models \varphi_i$. 
    Luego por $(*)$, $w \in s_i$, lo que nos dice que $w \in U$.

    Luego $\varphi$ define a $U$. Como $\varphi$ es proposicional, $U$ es proposicionalmente definible.
\end{demostracion}

Notemos como este lema justifica el argumento utilizado en \Cref{ejemplo:bisim^*}, en el que se menciona que los conjuntos proposicionalmente 
definibles de $\modults$ son exactamente los conjuntos $U$ que satisfacen que para cada $s \in \rho_\modults$ se cumple que 
$s \subseteq U$ o $s \cap U = \emptyset$.

De esta lema obtenemos un corolario sobre el problema de determinar si un subconjunto del dominio de un \ults es \KHilogic-definible.

\begin{corolario}[\KHilogic-definibilidad]
    Sea $\model=\tup{\W,\R,\cset{\S_i}_{i \in \AGT},\V,\ACT}$ un \ults en \MFD y sea $U \subseteq \W$, el problema de determinar si $U$ es 
    \KHilogic-definible está en $\Poly$.
\end{corolario}

\begin{demostracion}

    Primero, notemos que determinar si $U$ es \KHilogic-definible es equivalente a determinar si $U$ es proposicionalmente definible, por lo visto en 
    \Cref{prop:khi-implies-prop-definable}.

    Luego, por \Cref{lema:finite-propositionally-definable-lemma}, basta con dar un algoritmo que compute $\rho_\modults$ y para cada $s \in \rho_\modults$ 
    chequee que $s \subseteq U$ o $s \cap U = \emptyset$. Lo cuál es claro que puede realizarse en tiempo polinomial sobre el tamaño de $\modults$.
\end{demostracion}

Notemos como el fortalecimiento de la propiedad sobre los conjuntos proposicionalmente definibles nos permite fortalecer también \Cref{lema:new-implies-old}.

\begin{lema}[\KHilogic-bisimilitud implica \KHilogic-bisimilitud$^*$]\label{lema:old-implies-new-finite}
    Sean $\modults$ y $\modults'$ dos \ultss en \MFD, con dominios $\W$ y $\W'$ respectivamente. Sea $Z \subseteq \W \times \W'$, entonces 
    \begin{center}
        $Z$ es una \KHilogic-bisimulación$^*$ si y sólo si $Z$ es una \KHilogic-bisimulación.
    \end{center}
\end{lema}

\begin{demostracion}
    Notar que la dirección ($\rightarrow$) está demostrada en \Cref{lema:new-implies-old}, entonces demostremos ($\leftarrow$).
    
    Sea $Z$ una \KHilogic-bisimulación, veamos que es una \KHilogic-bisimulación$^*$.

    Notemos que $Z$ satisface (Atom), (A-zig) y (A-zag) dado que $Z$ es una \KHilogic-bisimulación.

    Así que demostraremos que $Z$ satisface ($\khi$-zig$^*$) y ($\khi$-zag$^*$).

    \begin{itemize}
        \item ($\khi$-zig$^*$) Sea $U \subseteq \W$ tal que para cada $s \in \rho_\modults$ se cumple que $s \cap U = \emptyset$ o $s \subseteq U$ 
        y sea $T \subseteq \W$ tal que $U \ultsExecAgi T$, queremos ver que existe $T' \subseteq \W'$ tal que

        \begin{multicols}{2}
            \begin{cond-bisim}
                \item $Z(U) \ultsExecAgi T'$, 
                \item $T' \subseteq Z(T)$.
            \end{cond-bisim}
        \end{multicols}

        Notemos que como para cada $s \in \rho_\modults$ se cumple que $s \cap U = \emptyset$ o $s \subseteq U$ y $\modults \in $ \MFD entonces 
        por \Cref{lema:finite-propositionally-definable-lemma} $U$ es proposicionalmente definible.
    
        Luego como $Z$ satisface ($\khi$-zig) tenemos asegurada la existencia de dicho $T'$.
    
        \item ($\khi$-zag$^*$) Análogo a ($\khi$-zig$^*$). 
    \end{itemize}

    Finalmente hemos demostrado que $Z$ es una \KHilogic-bisimulación$^*$.
\end{demostracion}

Y como demostramos que en el caso finito ambas definiciones de bisimulación son equivalentes, estamos en condiciones de demostrar el teorema deseado. 

\begin{teorema}[\KHilogic-equivalencia implica \KHilogic-bisimilitud$^*$]\label{thm:new-bisim-finite-equiv}
    Sean $\modults,w$ y $\modults',w'$ dos \ultss punteados tales que $\modults, \modults' \in$ \MFD, entonces
    \begin{center}
        $\model,w \modequiv \model', w'$ implica $\model,w \bisim^* \model', w'$.
    \end{center}
\end{teorema}

\begin{demostracion}
    Sean $\modults,w$ y $\modults',w'$ dos \ultss punteados con $\modults, \modults' \in $ \MFD tales que $\model,w \modequiv \model', w'$, 
    entonces por \Cref{thm:finite-equivalence-implies-bisim} tenemos que $\model,w \bisim \model', w'$. 
    Luego existe $Z$ \KHilogic-bisimulación tal que $(w,w') \in Z$.

    Por \Cref{lema:old-implies-new-finite} tenemos que $Z$ es una \KHilogic-bisimulación$^*$. Luego, como $(w,w') \in Z$ 
    entonces $\model,w \bisim^* \model', w'$.
\end{demostracion}

Hemos presentado una nueva definición de bisimulación para la lógica de `Knowing How' multi-agente basada en incertidumbre, la 
cuál pide condiciones puramente estructurales entre los modelos en cuestión. A su vez demostramos las principales propiedades que se 
esperan a la hora de proponer una definicion de bisimulación para una lógica modal, lo cuál sugiere que la nueva definición es adecuada.

También, podemos notar que la definicion propuesta evita la necesidad de listar cada fórmula proposicional para verificar que una relación binaria
es una bisimulación, lo cuál le aporta una naturaleza más procedural como mencionamos al principio de la sección.  

Teniendo en cuenta lo estudiado en esta sección, utilizaremos la nueva definición de bisimulación a lo largo de este trabajo.
Por una cuestión de simplicidad, descartaremos los $*$ cuando hablemos de una \KHilogic-bisimulación$^*$ y 
al relacionar dos \ultss punteados a partir de $\bisim^*$.