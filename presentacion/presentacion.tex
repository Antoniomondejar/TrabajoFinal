% \pdfminorversion=4
\documentclass{beamer}

\usepackage[spanish]{babel}
\usepackage[utf8x]{inputenc}
\usepackage{tikz}
\usepackage{courier}
\usepackage{array}
\usepackage{bold-extra}
\usepackage{minted}
\usepackage{algorithm}
\usepackage[noend]{algpseudocode}
\algrenewcommand{\algorithmiccomment}[1]{\hfill // #1}
\floatname{algorithm}{Algoritmo}
% \usepackage[thicklines]{cancel}
\usepackage{fancyvrb}
\usepackage{xspace}
\usepackage{xcolor}
\usepackage{amsthm}
\setlength{\parskip}{6pt}     % space between paragraphs
\xdefinecolor{dianablue}{rgb}{0.18,0.24,0.31}
\xdefinecolor{darkblue}{rgb}{0.1,0.1,0.7}
\xdefinecolor{darkgreen}{rgb}{0,0.5,0}
\xdefinecolor{darkgrey}{rgb}{0.35,0.35,0.35}
\xdefinecolor{darkorange}{rgb}{0.8,0.5,0}
\xdefinecolor{darkred}{rgb}{0.7,0,0}
\definecolor{darkgreen}{rgb}{0,0.6,0}
\definecolor{mauve}{rgb}{0.58,0,0.82}
%!TEX root = main.tex
%------------------------------------------------------------------------------------------------
% adjusting vertical space in the multicols environment
% default: \multicolsep = 12pt plus 4pt minus 3pt

\setlength{\multicolsep}{6pt plus 2pt minus 1.5pt}

%------------------------------------------------------------------------------------------------


% defining the format for theorems, propositions, corollaries and lemmas. The main point is the symbol at the end of them
\usepackage[hyperref,thmmarks]{ntheorem}
\usepackage{amsmath}
\usepackage{multicol}


% \theoremstyle{plain}
\theoremsymbol{\ensuremath{\triangleleft}}

\newtheorem{teorema}{Teorema}
\newtheorem{proposicion}{Proposición}
\newtheorem{corolario}{Corolario}
\newtheorem{lema}{Lema}
\newtheorem{hecho}{Hecho}

%------------------------------------------------

% defining the format for definitions, examples, fact and others. Same symbol at the end, but header and font different
\theoremheaderfont{\normalfont\bfseries}
\theorembodyfont{\normalfont}
\theoremsymbol{\ensuremath{\dashv}}

\newtheorem{definicion}{Definición}[section]
\newtheorem{ejemplo}{Ejemplo}
\newtheorem{nota}{Nota}

%------------------------------------------------

% defining the format for proofs
% \theoremstyle{nonumberplain}
\theoremheaderfont{\itshape}
\theorembodyfont{\normalfont}
\theoremsymbol{\ensuremath{\blacksquare}}

\newtheorem{demostracion}{Demostración.}

%-----------------------------------------------------------------------------

%------------------------------------------------
\usepackage{amssymb}


%------------------------------------------------




\newcommand{\modlts}{\mathcal{S}}
\newcommand{\modults}{\mathcal{M}}
\newcommand{\model}{\modults}


\newcommand{\tup}[1]{\langle #1 \rangle}

\newcommand{\W}{\mathrm{W}}
\newcommand{\R}{\mathrm{R}}
\renewcommand{\S}{\mathrm{S}}
\newcommand{\V}{\mathrm{V}}

\newcommand{\PROP}{{\rm \sf Prop}}
\newcommand{\ACT}{{\rm \sf Act}}
\newcommand{\AGT}{{\rm \sf Agt}}
\newcommand{\ACTP}{{\rm \sf A}}

\newcommand{\cset}[1]{\{ #1 \}}
% Sets with condition \csetsc{y}{xRy} = {y | xRy}
\newcommand{\csetsc}[2]{\{ #1  \mid #2 \}}
% Sets with domain and condition \csetc{y}{A}{xRy} = {y \in A | xRy}
\newcommand{\csetc}[3]{\{ #1 \in #2 \mid #3 \}}

\newcommand{\formatmodality}[1]{\ensuremath{\mathsf{#1}}}
\newcommand{\kh}{\formatmodality{Kh}}
\newcommand{\khi}{\formatmodality{Kh}_i}
% languages
\newcommand{\KHlogic}{\ensuremath{\mathsf{L}_{\kh}}}
\newcommand{\KHilogic}{\ensuremath{\mathsf{L}_{\khi}}}

\newcommand{\mycomment}[1]{}

\newcommand{\bigO}[0]{\mathcal{O}}

% complejidad 
\newcommand{\NP}{\rm\textsf{NP}}
\newcommand{\Poly}{\rm\textsf{P}}
\newcommand{\coNP}{\rm\textsf{coNP}}

\newcommand{\DNFTAUT}{\rm\textsf{DNF-TAUT}}

\newcommand{\KHBisim}{\rm\textsf{KhBisim}}

% estrategias
\newcommand{\mystackrel}[2]{%
  \mathrel{\vbox{\offinterlineskip\ialign{%
  \hfil##\hfil\cr
  \ensuremath{\scriptstyle#1}\cr
  \noalign{\kern.25ex}
  \ensuremath{#2}\cr
  }}}
}
\newcommand{\ultsExec}{\Rightarrow}

%\newcommand{\ultsExecStrat}[1]{\stackrel{#1}{\ultsExec}}
%\newcommand{\ultsExecStrat}[1]{\mystackrel{#1}{\ultsExec}}
\newcommand{\ultsExecStrat}[1]{\mathrel{\raisebox{-.2ex}{\ensuremath{\mystackrel{#1}{\ultsExec}}}}}

%\newcommand{\ultsExecAgi}{\stackrel{i}{\ultsExec}}
%\newcommand{\ultsExecAgi}{\mystackrel{i}{\ultsExec}}
\newcommand{\ultsExecAgi}{\mathrel{\raisebox{-.2ex}{\ensuremath{\mystackrel{i}{\ultsExec}}}}}

% no se que es esto
\usepackage{enumitem}
\newlist{cond-bisim}{enumerate}{1}
\setlist[cond-bisim]{label=\itmbisimformat{\arabic*}, leftmargin=3em, itemindent=1em}




\mode<presentation>
{
  \usetheme{default}
  \usecolortheme{default}
  \usefonttheme{default}
  \setbeamerfont{block title}{series=\bfseries}
  \setbeamertemplate{blocks}[rounded][shadow=true]
  \setbeamertemplate{navigation symbols}{}
  \setbeamertemplate{caption}[numbered]
  \setbeamertemplate{footline}[frame number]  % or "page number"
  \setbeamercolor{frametitle}{fg=dianablue}
  \setbeamercolor{footline}{fg=black}
  \setbeamercolor{title}{fg=dianablue}
  \setbeamercolor{block title}{fg=dianablue}
  % \setbeamercolor{proposition title}{fg=blue}
  % \setbeamercolor{definition title}{fg=teal}
  % \setbeamercolor{corollary title}{fg=purple}
  } 
  
% \useblocktheme{rounded}


% \usepackage[thmmarks]{ntheorem}
% \usepackage{amsmath}
% \usepackage{multicol}


% % \newtheorem{teorema}{Teorema}[chapter]
% % \newtheorem{proposicion}{Proposición}[chapter]
% % \newtheorem{corolario}{Corolario}[chapter]
% % \newtheorem{lema}{Lema}[chapter]
% % \newtheorem{hecho}{Hecho}[chapter]



% %------------------------------------------------

% % defining the format for definitions, examples, fact and others. Same symbol at the end, but header and font different
% \theoremheaderfont{\normalfont\bfseries}
% \theorembodyfont{\normalfont}
% \theoremsymbol{\ensuremath{\dashv}}

% % \newtheorem{definicion}{Definición}[chapter]
% % \newtheorem{ejemplo}{Ejemplo}[chapter]
% % \newtheorem{nota}{Nota}[chapter]

% %------------------------------------------------

% % defining the format for proofs
% \theoremstyle{nonumberplain}
% \theoremheaderfont{\itshape}
% \theorembodyfont{\normalfont}
% \theoremsymbol{\ensuremath{\blacksquare}}

% \newtheorem{demostracion}{Demostración.}



% \newcommand{\bigO}{\mathcal{O}}
% \newcommand{\zm}{Z_\mathcal{M}}
% \newtheorem{teorema}{Teorema}
% \newtheorem{proposicion}{Proposición}
% \newtheorem{definicion}{Definición}
% \newtheorem{corolario}{Corolario}




\title[yyyy-mm-dd-DESCRIPTION]{Aspectos Computacionales de Bisimulaciones para una Lógica de Knowing How}
\author{Antonio Mondejar\\[0.5em]
\small Director: Dr. Raul Fervari\\[0.5em]}
\institute{\small FAMAF - Universidad Nacional de Córdoba}
\date{\small 19 de Diciembre de 2025}

\usetikzlibrary{shapes.callouts}

\begin{document}

% \logo{\pgfputat{\pgfxy(0.11, 7.4)}{\pgfbox[right,base]{\tikz{\filldraw[fill=dianablue, draw=none] (0 cm, 0 cm) rectangle (50 cm, 1 cm);}\mbox{\hspace{-5 cm}\includegraphics[height=2 cm]{Logo_FAMAF_blanco.png}}}}}

\setbeamertemplate{headline}{%
  \begin{beamercolorbox}[wd=\paperwidth,ht=1.1cm,dp=0cm]{}
    \begin{tikzpicture}[remember picture,overlay]
      % Blue bar
      \fill[dianablue] (0,0) rectangle (\paperwidth,1.1cm);

      % Logo anchored and adjustable
      \node[anchor=east] at (\paperwidth-0.1cm,0.625cm) 
        {\includegraphics[height=1.85cm]{Logo_FAMAF_blanco.png}};
    \end{tikzpicture}
  \end{beamercolorbox}
}
% \setbeamertemplate{headline}{
%   \begin{tikzpicture}[remember picture,overlay]
%     % Blue bar
%     \fill[dianablue] (current page.north west) rectangle
%          ([yshift=-1.1cm]current page.north east);

%     % Logo
%     \node[anchor=north east] at ([yshift=-0.2cm]current page.north east)
%       {\includegraphics[height=1.8cm]{Logo_FAMAF_blanco.png}};
%   \end{tikzpicture}
% }

\begin{frame}
  \titlepage
\end{frame}

% \logo{\pgfputat{\pgfxy(0.11, 7.4)}{\pgfbox[right,base]{\tikz{\filldraw[fill=dianablue, draw=none] (0 cm, 0 cm) rectangle (50 cm, 1 cm);}\mbox{\hspace{-8 cm}\includegraphics[height=1 cm]{Logo_FAMAF_blanco.png}\includegraphics[height=1 cm]{Logo_FAMAF_blanco.png}}}}}

% Uncomment these lines for an automatically generated outline.
% \begin{frame}{Outline}
%  \tableofcontents
% \end{frame}

% START START START START START START START START START START START START START

% \begin{frame}
%   \frametitle{Motivación}
% Los grafos son utilizados para modelar numerosos escenarios:\pause
% \begin{itemize}
%   \item Bases de datos.\pause
%   \item Mapas de ciudades/países.\pause
%   \item Sistemas de computación.\pause
%   \item Usuarios de una red social.
% \end{itemize}

% \end{frame}

% =========================================================================================================== %

\begin{frame}
  \frametitle{Motivación}\pause

  \begin{figure}[h]
        \centering
        \begin{tikzpicture}
            \node[state] (u1) {\textsf{Arg}};
            \node[state, below of=u1, yshift=-1cm] (u2) {\textsf{Arg}};
            \node[state, below of=u2, yshift=-1cm] (u3) {\textsf{Arg}};
            \node[state, right of=u1, xshift=1.5cm] (u4) {\textsf{Chi}};
            \node[state, below of=u4, yshift=-1cm] (u5) {\textsf{Chi}};
            \node[state, below of=u5, yshift=-1cm] (u6) {\textsf{Chi}};

            

            \node at ($(u1)+(0,0.5)$) {$u_1$};
            \node at ($(u2)+(0,0.5)$) {$u_2$};
            \node at ($(u3)+(0,0.5)$) {$u_3$};
            \node at ($(u4)+(0,0.5)$) {$u_4$};
            \node at ($(u5)+(0,0.5)$) {$u_5$};
            \node at ($(u6)+(0,0.5)$) {$u_6$};

            \path (u1) edge [<->] node [above] {} (u4)
                       edge [<->] node [above] {} (u5);
                      %  edge [<->] node [above] {} (u6);

            \path (u2) edge [<->] node [above] {} (u4)
                       edge [<->] node [above] {} (u5)
                       edge [<->] node [above] {} (u6);
            
            \path (u3) edge [<->] node [above] {} (u5)
                       edge [<->] node [above] {} (u6);
                      %  edge [<->] node [above] {} (u6);


            % \path (p) edge [-] node [above] {$a$} (q)
            %           edge [-][bend left] node [above] {$a$} (r);
            % \path (p2) edge node [above] {$a$} (q);
            % \path (q) edge node [above] {$b$} (r);
        \end{tikzpicture}        % \label{fig:lts}
    \end{figure}\pause

    \begin{itemize}
      \item ¿Cada argentino es amigo de algún chileno?.
    \end{itemize} 

\end{frame}

% =========================================================================================================== %

\begin{frame}
  \frametitle{Motivación}\pause

   \begin{figure}
  \centering

  \begin{minipage}{0.45\textwidth}
  \centering
  % --- FIRST GRAPH ---
  \begin{tikzpicture}
    \node[state] (u1) {\textsf{Arg}};
            \node[state, below of=u1, yshift=-1cm] (u2) {\textsf{Arg}};
            \node[state, below of=u2, yshift=-1cm] (u3) {\textsf{Arg}};
            \node[state, right of=u1, xshift=1.5cm] (u4) {\textsf{Chi}};
            \node[state, below of=u4, yshift=-1cm] (u5) {\textsf{Chi}};
            \node[state, below of=u5, yshift=-1cm] (u6) {\textsf{Chi}};

            

            \node at ($(u1)+(0,0.5)$) {$u_1$};
            \node at ($(u2)+(0,0.5)$) {$u_2$};
            \node at ($(u3)+(0,0.5)$) {$u_3$};
            \node at ($(u4)+(0,0.5)$) {$u_4$};
            \node at ($(u5)+(0,0.5)$) {$u_5$};
            \node at ($(u6)+(0,0.5)$) {$u_6$};

            \path (u1) edge [<->] node [above] {} (u4)
                       edge [<->] node [above] {} (u5);
                      %  edge [<->] node [above] {} (u6);

            \path (u2) edge [<->] node [above] {} (u4)
                       edge [<->] node [above] {} (u5)
                       edge [<->] node [above] {} (u6);
            
            \path (u3) edge [<->] node [above] {} (u5)
                       edge [<->] node [above] {} (u6);
  \end{tikzpicture}\pause
  \end{minipage}
  \hfill
  \begin{minipage}{0.45\textwidth}
  \centering
  % --- SECOND GRAPH ---
  \begin{tikzpicture}
    \node[state] (u1) {\textsf{Arg}};
            % \node[state, right of=u1, yshift=-1cm] (u2) {\textsf{Arg}};
            % \node[state, below of=u2, yshift=-1cm] (u3) {\textsf{Arg}};
            \node[state, right of=u1, xshift=1.5cm] (u2) {\textsf{Chi}};
            % \node[state, below of=u4, yshift=-1cm] (u5) {\textsf{Chi}};
            % \node[state, below of=u5, yshift=-1cm] (u6) {\textsf{Chi}};

            

            \node at ($(u1)+(0,0.6)$) {$u_1'$};
            \node at ($(u2)+(0,0.6)$) {$u_2'$};
            % \node at ($(u3)+(0,0.5)$) {$u_3$};
            % \node at ($(u4)+(0,0.5)$) {$u_4$};
            % \node at ($(u5)+(0,0.5)$) {$u_5$};
            % \node at ($(u6)+(0,0.5)$) {$u_6$};

            \path (u1) edge [<->] node [above] {} (u2);
                      %  edge [<->] node [above] {} (u5);
                      %  edge [<->] node [above] {} (u6);

            % \path (u2) edge [<->] node [above] {} (u4)
            %            edge [<->] node [above] {} (u5)
            %            edge [<->] node [above] {} (u6);
            
            % \path (u3) edge [<->] node [above] {} (u5)
            %            edge [<->] node [above] {} (u6);
  \end{tikzpicture}
  \end{minipage}

  \end{figure}

\end{frame}

% =========================================================================================================== %


\begin{frame}
  \frametitle{Motivación}\pause

  \begin{figure}[h]
        \centering
        \begin{tikzpicture}
            \node[state] (u1) {\textsf{Nole}};
            \node[state, below of=u1, yshift=-1cm] (u2) {\textsf{Roger}};
            \node[state, below of=u2, yshift=-1cm] (u3) {\textsf{Jannik}};
            \node[state, right of=u1, xshift=1.5cm] (u4) {\textsf{Andy}};
            \node[state, below of=u4, yshift=-1cm] (u5) {\textsf{Charly}};
            \node[state, below of=u5, yshift=-1cm] (u6) {\textsf{Rafa}};

            

            \node at ($(u1)+(0,0.5)$) {$u_1$};
            \node at ($(u2)+(0,0.5)$) {$u_2$};
            \node at ($(u3)+(0,0.5)$) {$u_3$};
            \node at ($(u4)+(0,0.5)$) {$u_4$};
            \node at ($(u5)+(0,0.5)$) {$u_5$};
            \node at ($(u6)+(0,0.5)$) {$u_6$};

            \path (u1) edge [<->] node [above] {} (u4)
                       edge [<->] node [above] {} (u5)
                      %  edge [<->] node [above] {} (u6)
                       edge [<->][bend right = 50] node [above] {} (u3)
                       edge [<->][bend right = 32] node [above] {} (u2);

            \path (u2) edge [<->] node [above] {} (u4)
                       edge [<->][bend right = 30] node [above] {} (u3)
                       edge [<->] node [above] {} (u5)
                       edge [<->] node [above] {} (u6);
            
            \path (u3) edge [<->] node [above] {} (u5)
                       edge [<->] node [above] {} (u6);
            
            \path (u4) edge [<->][bend left=32] node [above] {} (u5)
                       edge [<->][bend left=50] node [above] {} (u6);
                      %  edge [<->] node [above] {} (u6);
            \path (u5) edge [<->][bend left=32] node [above] {} (u6);

            % \path (p) edge [-] node [above] {$a$} (q)
            %           edge [-][bend left] node [above] {$a$} (r);
            % \path (p2) edge node [above] {$a$} (q);
            % \path (q) edge node [above] {$b$} (r);
        \end{tikzpicture}        % \label{fig:lts}
    \end{figure}\pause
    \begin{itemize}
    \item ¿Existe alguna cadena de amistades que conecte a X con Y?. 
    \end{itemize}
\end{frame}

% =========================================================================================================== %

\begin{frame}
  \frametitle{Motivación}\pause

  \begin{figure}
  \centering

  \begin{minipage}{0.45\textwidth}
  \centering
  % --- FIRST GRAPH ---
  \begin{tikzpicture}
    \node[state] (u1) {\textsf{Nole}};
    \node[state, below of=u1, yshift=-1cm] (u2) {\textsf{Roger}};
    \node[state, below of=u2, yshift=-1cm] (u3) {\textsf{Jannik}};
    \node[state, right of=u1, xshift=1.5cm] (u4) {\textsf{Andy}};
    \node[state, below of=u4, yshift=-1cm] (u5) {\textsf{Charly}};
    \node[state, below of=u5, yshift=-1cm] (u6) {\textsf{Rafa}};

    

    \node at ($(u1)+(0,0.5)$) {$u_1$};
    \node at ($(u2)+(0,0.5)$) {$u_2$};
    \node at ($(u3)+(0,0.5)$) {$u_3$};
    \node at ($(u4)+(0,0.5)$) {$u_4$};
    \node at ($(u5)+(0,0.5)$) {$u_5$};
    \node at ($(u6)+(0,0.5)$) {$u_6$};

    \path (u1) edge [<->] node [above] {} (u4)
                edge [<->] node [above] {} (u5)
              %  edge [<->] node [above] {} (u6)
                edge [<->][bend right = 50] node [above] {} (u3)
                edge [<->][bend right = 32] node [above] {} (u2);

    \path (u2) edge [<->] node [above] {} (u4)
                edge [<->][bend right = 30] node [above] {} (u3)
                edge [<->] node [above] {} (u5)
                edge [<->] node [above] {} (u6);
    
    \path (u3) edge [<->] node [above] {} (u5)
                edge [<->] node [above] {} (u6);
    
    \path (u4) edge [<->][bend left=32] node [above] {} (u5)
                edge [<->][bend left=50] node [above] {} (u6);
              %  edge [<->] node [above] {} (u6);
    \path (u5) edge [<->][bend left=32] node [above] {} (u6);
  \end{tikzpicture}\pause
  \end{minipage}
  \hfill
  \begin{minipage}{0.45\textwidth}
  \centering
  % --- SECOND GRAPH ---
  \begin{tikzpicture}
    \node[state] (u1) {\textsf{Nole}};
    \node[state, below of=u1, yshift=-1cm] (u2) {\textsf{Roger}};
    \node[state, below of=u2, yshift=-1cm] (u3) {\textsf{Jannik}};
    \node[state, right of=u1, xshift=1.5cm] (u4) {\textsf{Andy}};
    \node[state, below of=u4, yshift=-1cm] (u5) {\textsf{Charly}};
    \node[state, below of=u5, yshift=-1cm] (u6) {\textsf{Rafa}};

    

    \node at ($(u1)+(0,0.6)$) {$u_1'$};
    \node at ($(u2)+(0,0.6)$) {$u_2'$};
    \node at ($(u3)+(0,0.6)$) {$u_3'$};
    \node at ($(u4)+(0,0.6)$) {$u_4'$};
    \node at ($(u5)+(0,0.6)$) {$u_5'$};
    \node at ($(u6)+(0,0.6)$) {$u_6'$};

    \path (u1) edge [<->] node [above] {} (u4);
              %   edge [<->] node [above] {} (u5)
              % %  edge [<->] node [above] {} (u6)
              %   edge [<->][bend right = 50] node [above] {} (u3)
              %   edge [<->][bend right = 32] node [above] {} (u2);

    \path (u2)  edge [<->][bend right = 30] node [above] {} (u3)
                edge [<->] node [above] {} (u5);
                % edge [<->] node [above] {} (u6);
    
    \path (u3) edge [<->] node [above] {} (u6);
                % edge [<->] node [above] {} (u6);
    
    \path (u4) edge [<->][bend left=32] node [above] {} (u5);
                % edge [<->][bend left=50] node [above] {} (u6);
              %  edge [<->] node [above] {} (u6);
    % \path (u5) edge [<->][bend left=32] node [above] {} (u6);
  \end{tikzpicture}
  \end{minipage}

  \end{figure}

    % \begin{itemize}
    % \item ¿Existe alguna cadena de amistades que me conecte con X?. 
    % \end{itemize}
\end{frame}

% =========================================================================================================== %

\begin{frame}
  \frametitle{Motivación}\pause

    
    \begin{itemize}
      \item Lógicas modales.\pause
      \begin{itemize}
        \item Familia de lenguajes formales que permiten matematizar este tipo de preguntas.\pause
        \item Distintos modos de verdad: posibilidad, necesidad/universalidad, obligaciones, etc.\pause
        \item Lógicas epistémicas.\pause
      \end{itemize}
      \item Equivalencia lógica.\pause
      \begin{itemize}
        \item Determina el poder expresivo de una lógica.\pause
        \item Utilidad en aplicaciones prácticas de las lógicas modales.
      \end{itemize}

    \end{itemize}
    
\end{frame}



% =========================================================================================================== %

% \begin{frame}
%   \frametitle{¿Knowing How?}\pause

%     \begin{itemize}
%       \item Lógicas epistémicas: lógicas modales que buscan representar y razonar sobre el conocimiento de agentes.\pause
%       \begin{itemize}
%         \item Usualmente modelan el ``saber qué''.
%         \item \textit{Juan sabe que está lloviendo en Río Cuarto}.\pause
%       \end{itemize}
%       \item Existen lógicas que estudian otras formas del conocimiento:\pause
%       \begin{itemize}
%         \item ``saber si''.\pause
%         \item ``saber por qué''.\pause
%         \item {\color{darkorange}{``saber cómo''}.}
%       \end{itemize}
%     \end{itemize}
    
% \end{frame}

% =========================================================================================================== %

\begin{frame}
  \frametitle{¿Knowing How?}\pause
  \begin{figure}[h]
        % \hspace{2.3cm}
        \vspace{-0.5cm}
        \begin{tikzpicture}
            \node[state] (phor) {$\textsf{Pq.Hor}$};
            \node[state, right of=phor, xshift=1.8cm] (gue) {$\textsf{Guemes}$};
            \node[state, below of=gue, yshift=-1.6cm, xshift=0.4cm] (cuniv) {$\textsf{C.Univ}$};
            \node[state, right of=gue, yshift=-1cm, xshift=1.8cm] (cen) {$\textsf{Centro}$};
            \node[state, right of=cen, xshift=1.8cm] (cof) {$\textsf{Cofico}$};
            \node[state, right of=cof, xshift=1.8cm] (acba) {$\textsf{Alta Cba}$};
            
            % \node at ($(p)+(0,0.5)$) {$w_1$};
            % \node at ($(r)+(0,0.5)$) {$w_4$};
            % \node at ($(q)+(0,0.5)$) {$w_3$};
            % \node at ($(p2)+(0.05,-0.6)$) {$w_2$};
            
            \path (phor) edge [bend right=12] node [below] {\tiny$30^v$} (gue);

            \path (gue) edge [bend right=12] node [below] {\tiny$30^v$} (cen)
                        edge [bend right=12] node [above] {\tiny$30^i$} (phor);
            
            \path (cen) edge [bend right=12] node [above] {\tiny$30^i$} (gue)
                        edge [bend right=12] node [above] {\tiny$18^i$} (cuniv)
                        edge [bend right=12] node [below] {\tiny{$18^v$, $30^v$}} (cof);
            
            \path (cof) edge [bend right=12] node [above] {\tiny{$18^i$, $30^i$}} (cen)
                        edge [bend right=12] node [below] {\tiny{$18^v$, $30^v$}} (acba);
            
            \path (acba) edge [bend right=12] node [above] {\tiny{$18^i$, $30^i$}} (cof);

            \path (cuniv) edge [bend right=12] node [below] {\tiny{$18^v$}} (cen);
                        % edge [bend left] node [above] {$a$} (r);
            % \path (p2) edge node [above] {$a$} (q);
            % \path (q) edge node [above] {$b$} (r);
        \end{tikzpicture}\pause
        \\
        \vspace{0.5cm}
        % \hspace{1cm}
        % \raisebox{1.8cm}{
            % \begin{minipage}{0.45\textwidth}
                \scriptsize$\S_{\textsf{Ana}} = \left\{
                    \begin{array}{c}
                        \{30^{i}30^{i}\}, \{18^{i}\}
                    \end{array}
                \right\}$ \\[0.1cm]\pause
                $\S_{\textsf{Pedro}} = \left\{
                    \begin{array}{c}
                        \{30^{i}30^{i},30^v30^v\}, \{18^{i}, 18^{v}\}
                    \end{array}
                \right\}$
            % \end{minipage}
        % }
        % \caption{Representación gráfica de $\modults$}
        % \label{fig:ults}
    \end{figure}\pause
% \vspace{0.1cm}
    ¿El agente X sabe cómo llegar a Parque Horizonte dado que está en el Centro?
\end{frame}

% =========================================================================================================== %

\begin{frame}
  \frametitle{Lógica de Knowing How Basada en Incertidumbre (\KHilogic)}\pause

  \begin{itemize}
  \item Conjunto de agentes: personas, robots, computadoras en sistemas distribuídos, etc. \pause
  \item Incertidumbre: considera la incapacidad de un agente para distinguir planes. \pause
  \item Propiedades referidas a las habilidades de los agentes. \pause
  \item Un agente sabe cómo lograr un objetivo a partir de una condición inicial. \pause Dispone de un 
  plan \textit{adecuado} que lo garantice. \pause
    \begin{itemize}
      \item ¿Cómo se representa el conocimiento de los agentes?
      \item ¿Qué consideramos \textit{adecuado}?
    \end{itemize}

  \end{itemize}

\end{frame}


% =========================================================================================================== %

\begin{frame}
  \frametitle{\lts basado en incertidumbre (\ults)}

  Sea $\PROP$ un conjunto contable de simbolos proposicionales y $\AGT$ un conjunto finito de nombres de agentes.\pause
  
  \begin{block}{\textbf{Definición}}
    Un \orange{\lts basado en incertidumbre (\ults)} es una tupla $\tup{\W,\ACT,\R,\{\S_i\}_{i\in\AGT},\V}$ donde \pause
    \begin{itemize}
      \item $\W$ es un conjunto contable no vacío de estados.\pause
      \item $\ACT$ es un conjunto no vacío de nombres de acciones.\pause
      \item $\R$ es una colección de relaciones binarias $\{\R_a \subseteq \W\times\W\}$ donde $a\in\ACT$.\pause
      \item $\S_i$ es una \orange{partición} sobre un conjunto no vacío de planes.\pause
      \item $\V : \W \to \mathcal{P}(\PROP)$.
    \end{itemize}
  \end{block}


\end{frame}


% =========================================================================================================== %

\begin{frame}
    \frametitle{Ejecutabilidad Fuerte}
    Los planes deben ser a \orange{prueba de fallas}: cada ejecución parcial debe poder completarse. \pause
    
    \vspace{-0.2cm}
    \renewcommand{\arraystretch}{1.6}
    \begin{ctabular}{cc}
      \begin{tabular}{@{\ \ \ \ \ \ \ }c}
        \begin{tikzpicture}
          \hspace{-0.5cm}
    \node [state, label = {[label-state]left:$w_1$}] (w1) {};
    \node [state, label = {[label-state]above:$w_2$}, above right = 0.5em and 3em of w1] (w2) {};
    \node [state, label = {[label-state]above:$w_3$}, right = 3em of w2] (w3) {};
    \node [state, label = {[label-state]below:$w_4$}, below right = 0.5em and 3em of w1] (w4) {};
    
    \path (w1) edge node [label-edge, pos = 0.35, above] {$a$} (w2)
                edge node [label-edge, pos = 0.35, below] {$a$} (w4)
            (w2) edge node [label-edge, above] {$b$} (w3);
        \end{tikzpicture}
    \end{tabular}
    &
    \begin{tabular}{l}
      \hspace{-0.9cm}
      \footnotesize  $ab$ no es fuertemente ejecutable en $w_1$
    \end{tabular}
    \end{ctabular}\pause
    \vspace{-0.5cm}
    \begin{block}{\textbf{Definición}}
      Un plan $\sigma$ es \orange{fuertemente ejecutable (FE)} en un estado $w$ si y sólo si cada ejecución parcial de $\sigma$ a partir de $w$ 
      puede ser completada.
    \end{block}
\end{frame}

%=============================================================

\begin{frame}
    \frametitle{Sintaxis y semántica}\pause
    \begin{small}
        \begin{block}{Definición (\KHilogic sobre \ults)}
            Las fórmulas de $\KHilogic$ están dadas por la gramática
            \[
                \varphi ::= p \mid \neg\varphi \mid \varphi\vee\varphi \mid \khi(\varphi,\varphi),
            \]
            donde $p \in \PROP$ e $i \in \AGT$. \pause $\khi(\psi,\varphi)$ debe ser leído como \orange{``el agente $i$ sabe cómo lograr $\varphi$ dado $\psi$''.} \pause
            \\ 
        \medskip
        $\modults,w \models \khi(\psi,\varphi)$ si y sólo si existe \orange{un conjunto de planes} $\strategy \in \S_i$ tal que:

        \begin{quote}
            \begin{enumerate}
            \item cada plan en $\strategy$ es FE en cada estado donde vale $\psi$ y
            \item desde los estados en los que vale $\psi$, cada plan en $\strategy$ termina su ejecución en un estado donde vale $\varphi$.
            \end{enumerate}
        \end{quote}
        \end{block}

        % \pause

        % \begin{block}{Property:}
        %     Define $\orange{\A\varphi} := \bigvee_{i\in\AGT}\khi(\neg\varphi,\bot)$, we have:
        %     \begin{nscenter}
        %         $\model,w\models\A\varphi \mbox{ iff for all $v$, } \model,v\models\varphi;$
        %     \end{nscenter}
        %     i.e., $\A$ is the standard \orange{universal} modality (and its dual: $\E\varphi:=\neg\A\neg\varphi$).
        % \end{block}

    \end{small}

\end{frame}

%=============================================================

\begin{frame}
  \frametitle{Sintaxis y semántica}
  \begin{figure}[h]
        % \hspace{2.3cm}
        \vspace{-0.5cm}
        \begin{tikzpicture}
            \node[state] (phor) {$\textsf{Pq.Hor}$};
            \node[state, right of=phor, xshift=1.8cm] (gue) {$\textsf{Guemes}$};
            \node[state, below of=gue, yshift=-1.6cm, xshift=0.4cm] (cuniv) {$\textsf{C.Univ}$};
            \node[state, right of=gue, yshift=-1cm, xshift=1.8cm] (cen) {$\textsf{Centro}$};
            \node[state, right of=cen, xshift=1.8cm] (cof) {$\textsf{Cofico}$};
            \node[state, right of=cof, xshift=1.8cm] (acba) {$\textsf{Alta Cba}$};
            
            % \node at ($(p)+(0,0.5)$) {$w_1$};
            % \node at ($(r)+(0,0.5)$) {$w_4$};
            % \node at ($(q)+(0,0.5)$) {$w_3$};
            % \node at ($(p2)+(0.05,-0.6)$) {$w_2$};
            
            \path (phor) edge [bend right=12] node [below] {\tiny$30^v$} (gue);

            \path (gue) edge [bend right=12] node [below] {\tiny$30^v$} (cen)
                        edge [bend right=12] node [above] {\tiny$30^i$} (phor);
            
            \path (cen) edge [bend right=12] node [above] {\tiny$30^i$} (gue)
                        edge [bend right=12] node [above] {\tiny$18^i$} (cuniv)
                        edge [bend right=12] node [below] {\tiny{$18^v$, $30^v$}} (cof);
            
            \path (cof) edge [bend right=12] node [above] {\tiny{$18^i$, $30^i$}} (cen)
                        edge [bend right=12] node [below] {\tiny{$18^v$, $30^v$}} (acba);
            
            \path (acba) edge [bend right=12] node [above] {\tiny{$18^i$, $30^i$}} (cof);

            \path (cuniv) edge [bend right=12] node [below] {\tiny{$18^v$}} (cen);
                        % edge [bend left] node [above] {$a$} (r);
            % \path (p2) edge node [above] {$a$} (q);
            % \path (q) edge node [above] {$b$} (r);
        \end{tikzpicture}
        \\
        \vspace{0.5cm}
        % \hspace{1cm}
        % \raisebox{1.8cm}{
            % \begin{minipage}{0.45\textwidth}
                \scriptsize$\S_{\textsf{Ana}} = \left\{
                    \begin{array}{c}
                        \{30^{i}30^{i}\}, \{18^{i}\}
                    \end{array}
                \right\}$ \\[0.1cm]
                $\S_{\textsf{Pedro}} = \left\{
                    \begin{array}{c}
                        \{30^{i}30^{i},30^v30^v\}, \{18^{i}, 18^{v}\}
                    \end{array}
                \right\}$
            % \end{minipage}
        % }
        % \caption{Representación gráfica de $\modults$}
        % \label{fig:ults}
    \end{figure}\pause
% \vspace{0.1cm}
  \small $\kh_{Ana}$(Centro,Pq.Hor) vale pero $\kh_{Pedro}$(Centro,Pq.Hor) no.
\end{frame}

% =========================================================================================================== %


\begin{frame}
  \frametitle{Comparando modelos}

  \begin{figure}
  \centering

  \begin{minipage}{0.45\textwidth}
  \centering
  % --- FIRST GRAPH ---
  \begin{tikzpicture}
    \node[state] (u1) {p};
            \node[state, right of=u1, xshift=1.5cm] (u2) {p};

            

            \node at ($(u1)+(0,0.5)$) {$u_1$};
            \node at ($(u2)+(0,0.5)$) {$u_2$};

            \path (u1) edge [bend left] node [above] {a} (u2);
                      %  edge [<->] node [above] {} (u6);

            \path (u2) edge [bend left] node [above] {a} (u1);
            
  \end{tikzpicture}
  \end{minipage}
  \hfill
  \begin{minipage}{0.45\textwidth}
  \centering
  \vspace{-1cm}
  % --- SECOND GRAPH ---
  \begin{tikzpicture}
    \node[state] (u1) {p};
      % \node[state, right of=u1, yshift=-1cm] (u2) {\textsf{Arg}};
      % \node[state, below of=u2, yshift=-1cm] (u3) {\textsf{Arg}};
      % \node[state, below of=u4, yshift=-1cm] (u5) {\textsf{Chi}};
      % \node[state, below of=u5, yshift=-1cm] (u6) {\textsf{Chi}};

      

      \node at ($(u1)+(0,0.6)$) {$u_1'$};
      % \node at ($(u3)+(0,0.5)$) {$u_3$};
      % \node at ($(u4)+(0,0.5)$) {$u_4$};
      % \node at ($(u5)+(0,0.5)$) {$u_5$};
      % \node at ($(u6)+(0,0.5)$) {$u_6$};

      \path (u1) edge [loop] node [above] {b} (u1);
                %  edge [<->] node [above] {} (u5);
                %  edge [<->] node [above] {} (u6);

      % \path (u2) edge [<->] node [above] {} (u4)
      %            edge [<->] node [above] {} (u5)
      %            edge [<->] node [above] {} (u6);
      
      % \path (u3) edge [<->] node [above] {} (u5)
      %            edge [<->] node [above] {} (u6);
  \end{tikzpicture}
  \end{minipage}
  \end{figure}
  \hspace{0.9cm}
  $\S_{i} = \left\{
                    \begin{array}{c}
                        \{aa,a\}
                    \end{array}
                \right\}$
                \hspace{3.3cm}
                $\S'_{i} = \left\{
                    \begin{array}{c}
                        \{b\}
                    \end{array}
                \right\}$

\end{frame}

% =========================================================================================================== %


\begin{frame}
  \frametitle{Bisimulación}\pause
  \begin{itemize}
    \item La noción de bisimulación surge para caracterizar la equivalencia lógica.\pause
    \item Una bisimulación es una relación binaria entre los dominios de dos modelos.\pause
    \begin{itemize}
      \item Cada par de estados de una bisimulación satisface propiedades relacionadas con las características de los modelos.
      \item Dos estados relacionados satisfacen exactamente las mismas fórmulas.
    \end{itemize}\pause
    \item Cada lógica modal tiene su propia definición.
  \end{itemize}
\end{frame}

%=============================================================

\begin{frame}
  \frametitle{Bisimulación}\pause
  \begin{footnotesize}
    
    \begin{block}{\textbf{Definición (\KHilogic-bisimulación)}}
      Sean $\model, \model'$ dos \ultss, una relación binaria $Z \subseteq \W\times\W'$ es una \orange{\KHilogic-bisimulación} entre $\model$ y $\model'$ si y sólo si se cumplen:\pause
  \begin{description}
    \item[(Atom)] $\V(w) = \V'(w')$ para todo $(w,w')\in Z$. \pause
    \item[($\khi$-zig):] para todo $U \subseteq \W$ \emph{proposicionalmente} definible, si $U \ultsExecAgi T$ para $T \subseteq \W$, existe $T' \subseteq \W'$ tal que: 
    1) $Z(U) \ultsExecAgi T'$ y
    2) $T' \subseteq Z(T)$. \pause
    
    \item[($\khi$-zag):] % analogous to \textbf{$\khi$-Zig}.
    para todo $U' \subseteq \W'$ \emph{proposicionalmente} definible, si $U' \ultsExecAgi T'$ para $T' \subseteq \W'$, existe $T \subseteq \W$ tal que: 
    1) $Z^{-1}(U') \ultsExecAgi T$ y
    2) $T \subseteq Z^{-1}(T')$. \pause
    
    \item[($\A$-zig):] para todo $w\in\W$ existe $w'\in\W'$ tal que $(w,w')\in Z$. \pause
    
    \item[($\A$-zag):] para todo $w'\in\W'$ existe $w\in\W$ tal que $(w,w')\in Z$.
  \end{description}
\end{block}
\end{footnotesize}
\end{frame}

%=============================================================

% \begin{theorem}[\gray{Areces et al. 2021}]
% \label{th:adequacy}
% Let $\model,s$ and $\model',s'$ be two \ultss. $\model,s\bisim\model',s'$ implies $\model,s\models\varphi$ iff $\model',s'\models\varphi$, for all $\KHilogic$-formulas $\varphi$. Moreover, if $\model$ and $\model'$ are finite domain, the converse also holds.
% \end{theorem}

\begin{frame}
  \frametitle{Bisimulación}\pause
  % \vspace{-0.2cm}
  \begin{small}
    
    \begin{teorema}[\gray{Areces et al. 2021}]
      Sean $\model$ y $\model'$ dos \ultss. Si existe una bisimulación $Z$ entre ellos tal que $(w,w')\in Z$, entonces 
      para cada \KHilogic-fórmula $\varphi$ se cumple que $\model,w \models \varphi$ si y sólo si $\model',w'\models\varphi$. Si ambos \ultss tienen 
      dominio finito entonces también vale la recíproca. 
    \end{teorema}\pause
  \end{small}
  
  % \vspace{0.15cm}

  \small Esta definición cuenta con algunos aspectos dignos de cuestionamiento. \pause
  % \vspace{0.1cm}
  
  \begin{footnotesize}
    \begin{description}
      \item[($\khi$-zig):] para cada $U \subseteq \W$ \red{\emph{proposicionalmente} definible}, si $U \ultsExecAgi T$ para $T \subseteq \W$, existe $T' \subseteq \W'$ tal que: 
      1) $Z(U) \ultsExecAgi T'$ y
      2) $T' \subseteq Z(T)$.
      
      \item[($\khi$-zag):] % analogous to \textbf{$\khi$-Zig}.
    para cada $U' \subseteq \W'$ \red{\emph{proposicionalmente} definible}, si $U' \ultsExecAgi T'$ para $T' \subseteq \W'$, existe $T \subseteq \W$ tal que: 
    1) $Z^{-1}(U') \ultsExecAgi T$ y
    2) $T \subseteq Z^{-1}(T')$. 
  \end{description}
\end{footnotesize}
\end{frame}

%=============================================================

\begin{frame}{Redefiniendo \KHilogic-bisimulación}
    Con el objetivo de abordar los problemas mencionados, proponemos una nueva definición de bisimulación para \KHilogic. \pause

    % First, we characterize the propositionally definable sets of an \ults.

    % First, we will define a way of grouping the nodes of an \ults by only considering their valuation function $\V$.\pause

    Primero, introducimos una forma de agrupar los nodos de un \ults de acuerdo a su función de valuación $\V$. \pause

    % \vspace{3mm}

    \begin{block}{\textbf{Definición}}
        \begin{small}
          Sea $\model$ un \ults, definimos:
        \begin{itemize}
            \item $A_{\model}\eqdef\set{(w,v) \subseteq \W\times\W\mid \V(w)=\V(v)}$.\pause
            \item $\rho_{\model}\eqdef\set{[w]\mid w\in\W}$, donde $[w] = \{v\in\W\mid(w,v)\in A_\model\}$.
        \end{itemize}
        \end{small}
    \end{block}
\end{frame}

%=============================================================

\begin{frame}{Redefiniendo \KHilogic-bisimulación}
    Luego, caracterizamos los conjuntos proposicionalmente definibles del dominio de un \ults. \pause

    % \vspace{5mm}


    \begin{lema}
        \begin{small}
        % Let $\model=\tup{\W,\ACT,\R,\Unc,\V}$ be an \ults, and let $P\subseteq \W$ be a propositionally definable set. Then, for all $[s]\in\qpart{\model}$, either $[s]\cap P =\emptyset$ or $[s]\subseteq P$. 
        % If $\model$ is finite domain, then the converse also holds.
        Sea $\model$ un \ults, y sea $U \subseteq \W$ un conjunto proposicionalmente definible. Entonces, para todo $s\in\rho_\model$ se cumple que $s \cap U = \emptyset$ o $s\subseteq U$.
        Si $\model$ tiene dominio finito entonces también vale la recíproca. 
      \end{small}
    \end{lema} \pause

    Este lema motiva nuestra reformulación de las cláusulas ($\khi$-zig) y ($\khi$-zag).
\end{frame}

%=============================================================

\begin{frame}{Redefiniendo \KHilogic-bisimulación}

    \begin{footnotesize}
    \begin{block}{\textbf{Definición (\KHilogic-bisimulación$^*$)}}\pause
        Sean $\model, \model'$ dos \ultss, una relación binaria $Z \subseteq \W\times\W'$ es una \orange{\KHilogic-bisimulación$^*$} entre $\model$ y $\model'$ 
        si y sólo si se cumplen \textbf{(Atom)}, \textbf{(A-zig)}, \textbf{(A-zag)} y:\pause
        \begin{description} 
            \item[($\khi$-zig$^*$):] 
            para todo $U\subseteq\W$ tal que {\color{darkgreen}{para cada $s\in\rho_{\model}$ se cumple que  
            $s\cap U =\emptyset$ o $s\subseteq U$}}, 
            si $U \ultsExecAgi T$ para $T \subseteq \W$, entonces existe $T' \subseteq \W'$ tal que: 
                1) $Z(U) \ultsExecAgi T'$ y
                2) $T' \subseteq Z(T)$.\pause
                
            \item[($\khi$-zag$^*$):] para todo $U'\subseteq\W'$ tal que {\color{darkgreen}para cada $s'\in\rho_{\model'}$ se cumple que 
            $s'\cap U' =\emptyset$ o $s'\subseteq U'$},
            si $U' \ultsExecAgi T'$ para $T' \subseteq \W'$, entonces existe $T \subseteq \W$ tal que: 
                1) $Z^{-1}(U') \ultsExecAgi T$ y
                2) $T \subseteq Z^{-1}(T')$.
        \end{description}
    \end{block}
\end{footnotesize}

\end{frame}

%=============================================================

\begin{frame}{Redefiniendo \KHilogic-bisimulación}

    % Now, we would want our new definition to satisfy the previously introduced adequacy results. \pause

    \begin{lema}
      \begin{small}
        % If a relation $Z$ is an \KHilogic-bisimulation$^*$ between two \ults 
        % $\model$ and $\model'$, then it is an $\KHilogic$-bisimulation between them. 
        % If $\model$ and $\model'$ are finite domain, then the converse also holds.
        Si una relación $Z$ es una bisimulación$^*$ entre $\model$ y $\model'$, entonces también es una bisimulación entre ellos.
        Si $\model$ y $\model'$ tienen dominio finito también vale la recíproca.
      \end{small}  
    \end{lema}\pause

%     An important observation is that in the infinite case the converse does \red{not} hold.

%     This can be proven with a cardinality argument.
% \end{frame}

% %=============================================================

% \begin{frame}{Redefiniendo \KHilogic-bisimulación}
    % As a consequence of that lemma, we can now prove the corresponding results for our new definition. \pause 
    Como consecuencia de este lema, podemos demostrar los resultados esperados de correctitud.\pause
    
    \begin{teorema}
    \begin{small}
      Sean $\model$ y $\model'$ dos \ultss. Si existe una bisimulación$^*$ $Z$ entre ellos tal que $(w,w')\in Z$, entonces 
      para cada \KHilogic-fórmula $\varphi$ se cumple que $\model,w \models \varphi$ si y sólo si $\model',w'\models\varphi$. Si ambos \ultss tienen 
      dominio finito entonces también vale la recíproca.     \end{small}
    \end{teorema}

    % This theorem supports our belief that this proposal is adequate. 

\end{frame}

%=============================================================

\begin{frame}
  \frametitle{Estudio computacional de \KHilogic-bisimulación}\pause

  Una vez presentada la nueva definición de bisimulación, caracterizamos la complejidad computacional de problemas relacionados con esta noción.\pause

  En primera instancia, estudiamos el problema de determinar la existencia de una bisimulación 
  entre dos \ultss.\pause

  \begin{small}
    \begin{block}{\textbf{Definición}}
      $\KHiBisim \eqdef \{\tup{\model,\model'} \mid $ existe una bisimulación entre $\model$ y $\model'\}$
    \end{block}\pause
  \end{small}

  $\KHiBisim$ puede considerarse como un problema de ``búsqueda y verificación''.

\end{frame}


%=============================================================

\begin{frame}
  \frametitle{Estudio computacional de \KHilogic-bisimulación}

  Primero, determinamos la complejidad computacional del subproblema de ``verificación''.\pause

  \begin{small}
    \begin{block}{\textbf{Definición}}
      $\CBisim$ $\eqdef \{\tup{\model,\model',Z} \mid Z$ es una bisimulación entre $\model$ y $\model'\}$
    \end{block}\pause
  \end{small}

  Las condiciones ($\khi$-zig) y ($\khi$-zag) sugieren que la clase $\coNP$ es adecuada para este problema.

\end{frame}

%=============================================================

% \begin{frame}
%   \frametitle{Estudio computacional de \KHilogic-bisimulación}\pause

%   \begin{tiny}
%    \begin{block}{}
%         % \caption{Verificador de contraejemplos}
%         \begin{algorithmic}
%             \Function{NotSimulates}{$\model,\model',Z,U$}
%                 \ForAll{$\S_i$}
%                     \ForAll{$\pi \in \S_i$}
%                         \If{$U \subseteq \sexec^{\model}(\pi)$}
%                         \State $T \gets \R_\pi(U)$
%                         \State $U' \gets Z(U)$
%                         \State $T' \gets Z(T)$ 
%                         \State \textit{foundWitness} $\gets 0$
%                             \ForAll{$\pi' \in \S_i'$}
%                                 \If{$U' \subseteq \sexec^{\model'}(\pi') \wedge \R'_{\pi'}(U') \subseteq T'$}
%                                     \State \textit{foundWitness} $\gets 1$
%                                 \EndIf         
%                             \EndFor
%                             \If{$\neg$ \textit{foundWitness}}
%                                 \Return 1
%                             \EndIf
%                         \EndIf
%                     \EndFor
%                 \EndFor
%                 \State \Return 0
%             \EndFunction

%             \vspace{1em}

%             \Function{CounterexampleChecker}{$\modults,\modults',Z,\tup{U,b}$} 
%                 \State Check (Atom), if not \Return 1
%                 \State Check (A-zig), if not \Return 1
%                 \State Check (A-zag), if not \Return 1
%                 \If{$b$}
%                     \State If there exists $s \in \rho_\model$ such that $s \not\subseteq U$ and $s \cap U \neq \emptyset$
%                     \Return 0
%                     \State \Return \Call{NotSimulates}{$\model,\model',Z,U$}
%                 \Else
%                     \State If there exists $s' \in \rho_{\model'}$ such that $s' \not\subseteq U$ and $s' \cap U \neq \emptyset$
%                     \Return 0
%                     \State \Return \Call{NotSimulates}{$\model',\model,Z^{-1},U$} \Comment{Notice the parameters.}
%                 \EndIf
%             \EndFunction
%         \end{algorithmic}
%       \end{block}
%     \end{tiny}

% \end{frame}

%=============================================================

\begin{frame}
  \frametitle{Estudio computacional de \KHilogic-bisimulación}
    % To prove the upper bound it is enough to provide a polynomial non-deterministic algorithm that decides its complement. \pause
    
    Para cada instancia negativa debe existir un certificado de tamaño polinomial que pueda ser verificado eficientemente.\pause 

    % For every ``no-instance'' there exist a counterexample $x$ of polynomial length that can be efficiently verified. \pause

    Los certificados a considerar serán subconjuntos de nodos de cada \ults, sobre los cuales se verificarán las condiciones ($\khi$-zig) o ($\khi$-zag).\pause

    % The counterexamples that we will consider are subsets of nodes of the two models, and the algorithm 
    % will check if the ($\khi$-zig) or ($\khi$-zag) conditions are satisfied in that particular subset.\pause
    \begin{small}
      \begin{lema}
        $\CBisim \in \coNP$. 
      \end{lema}
    \end{small}
    
\end{frame}

%=============================================================

\begin{frame}
  \frametitle{Estudio computacional de \KHilogic-bisimulación}
  Una vez demostrada la pertenencia a $\coNP$, queremos establecer que el problema es difícil para dicha clase.\pause

  Con este objetivo en mente, introducimos el siguiente problema $\coNP$-completo:\pause

  \begin{small}
    \begin{block}{\textbf{Definición}}
      $\DNFTAUT \eqdef \{\varphi\mid\varphi$ es una tautología proposicional en DNF$\}$
    \end{block}\pause
  \end{small}

  Reducimos polinomialmente $\DNFTAUT$ a $\CBisim$.
\end{frame}

%=============================================================

\begin{frame}
  \frametitle{Estudio computacional de \KHilogic-bisimulación}
    \begin{small}
    \begin{center}
        $\varphi = (p_1 \wedge \neg p_2) \vee (p_2) \vee (\neg p_1)$  
    \end{center}
    \end{small}\pause
    \vspace{-0.8cm}
    \begin{figure}[t]
        \begin{small}
        \begin{center}
        \begin{tikzpicture}
            % Nodes at rotated positions
            \node[state] (phi1) at (0,0) {$q_3$};
            \node[state] (phi2) at (0,1.5) {$q_4$};
            \node[state] (phi3) at (0,3) {$q_5$};
            
            \node[state] (p1) at (2.6,0.5) {$q_1$};
            \node[state] (p2) at (2.6,1.5) {$q_2$};
            
            \node[state] (e) at (5.2,1.5) {$q_6$};

            % Labels
            \node at ($(phi1)+(-0.5,0)$) {$\varphi_1$};
            \node at ($(phi2)+(-0.5,0)$) {$\varphi_2$};
            \node at ($(phi3)+(-0.5,0)$) {$\varphi_3$};
            \node at ($(e)+(0.4,0)$) {$e$};
            
            \node at ($(p1)+(0.33,-0.33)$)  {$p_1$};
            \node at ($(p2)+(0.33,-0.33)$) {$p_2$};

            % Edges
            \path (p1) edge[->] node [above, inner sep=2pt] {\footnotesize $in_{\varphi_1}$} (phi1);
            \path (p1) edge[->] node [above, inner sep=1pt] {\footnotesize $in_{\varphi_2}$} (phi2);
            \path (p2) edge[->] node [above, inner sep=1pt] {\footnotesize $in_{\varphi_2}$} (phi2);
            \path (p2) edge[->, bend right=20] node [above, inner sep=2.8pt] {\footnotesize $in_{\varphi_3}$} (phi3);

            \path (phi1) edge[->, bend right=45] node [below, inner sep=1pt] {\footnotesize $out_{\varphi_1}$} (p1);
            \path (p1) edge [loop below] node [below] {\footnotesize $test$} (p1);
            \path (phi2) edge[->, bend left=40] node [above, inner sep=1pt] {\footnotesize $out_{\varphi_2}$} (p2);
            \path (p2) edge [loop above] node [above] {\footnotesize $test$} (p2);
            \path (phi3) edge[->, bend left=50] node [above] {\footnotesize $out_{\varphi_3}$} (e);
            \path (p1) edge[->, bend right] node [below] {\footnotesize $test$} (e);
            \path (p2) edge[->, bend left] node [above] {\footnotesize $test$} (e);
        \end{tikzpicture} \\
            
        \vspace{0.3cm}
                    $\begin{array}{ll}
                        \S_{agt} = &
                         \left\{\{in_{\varphi_1}out_{\varphi_1}\},
                        \{in_{\varphi_2}out_{\varphi_2}\},
                        \{in_{\varphi_3}out_{\varphi_3}\},
                        \{test\} \right\}\\[0.2cm]
                        \S'_{agt} = & 
                        \left\{\{in_{\varphi_1}out_{\varphi_1}\},
                        \{in_{\varphi_2}out_{\varphi_2}\},
                        \{in_{\varphi_3}out_{\varphi_3}\} \right\}
                    \end{array}$
    \end{center}
    \end{small}
    \end{figure}
\end{frame}

%=============================================================

\begin{frame}
  \frametitle{Estudio computacional de \KHilogic-bisimulación}
  \begin{small}
    \begin{lema}
      $\CBisim$ es $\coNPHard$.
    \end{lema}\pause
  \end{small}  

    % \vspace{5mm}

    Juntando ambos lemas, obtenemos el resultado deseado de complejidad.\pause

    % \vspace{5mm}
    \begin{small}
      \begin{teorema}
        $\CBisim$ es $\coNPComplete$.
      \end{teorema}\pause
    \end{small}

    Volviendo a $\KHiBisim$...
\end{frame}

%=============================================================

\begin{frame}
  \frametitle{Estudio computacional de \KHilogic-bisimulación}
    Estudiamos la complejidad de ``verificación''. \pause La ``búsqueda'' parece costosa: cantidad exponencial 
    de relaciones binarias a considerar.\pause 

    % \vspace{5mm}

    % Determinamos que existe una relación binaria particular que es suficiente para decidir existencia.\pause
    \begin{footnotesize}
      
      \begin{lema}\label{lema:bisim-existence}
        Sean $\model$ y $\model'$ dos \ultss.
        Entonces, existe $Z \subseteq \W \times \W'$ una bisimulación entre ellos si y sólo si  
        \begin{center}
          $\universalbis = \{(w,w') \in \W \times \W' \mid \V(w) = \V'(w')\}$
        \end{center}
        es una bisimulación entre $\model$ y $\model'$.
      \end{lema}\pause
    \end{footnotesize}

    Decidir existencia de una bisimulación se reduce a verificar que $\universalbis$ es una bisimulación.\pause

    \begin{small}
      \begin{teorema}
        $\KHiBisim$ es $\coNPComplete$.
      \end{teorema}
    \end{small}
    % Juntando ambos lemas, obtenemos el resultado deseado de complejidad.\pause
  \end{frame}

% \begin{frame}
%   \frametitle{Estudio computacional de \KHilogic-bisimulación}
%   % \vspace{5mm}
  
%   También, a partir de los resultados obtenidos podemos extender este teorema a la versión punteada del problema.\pause

%   \begin{small}  
%     \begin{teorema}
%       $\PKHiBisim$ es $\coNPComplete$, donde 
%       \begin{center}
%         $\PKHiBisim$ $\eqdef \set{\tup{(\model,w),(\model',w')} \mid$ existe una bisimulación entre $\model$ y $\model'$ que contiene al par $(w,w')}$.
%       \end{center} 
%       \end{teorema}
%   \end{small}
% \end{frame}

%=============================================================

\begin{frame}
  \frametitle{Contracción por \KHilogic-bisimulación}\pause
  \begin{itemize}
    \item Minimización de modelos: dado un modelo, encontrar el modelo de tamaño mínimo que sea lógicamente equivalente.\pause
    \item En el contexto de las lógicas modales:\pause
        \begin{itemize}
          \item Para cada nodo del modelo original debe existir uno lógicamente equivalente.
          \item Se aborda a partir del estudio de las autobisimulaciones de un modelo.
        \end{itemize}
  \end{itemize}
\end{frame}

%=============================================================

\begin{frame}
  \frametitle{Contracción por \KHilogic-bisimulación}
  En primer lugar, caracterizamos la máxima autobisimulación de un \ults.\pause
  
  \begin{small}
    \begin{teorema}
      Sea $\model$ un \ults, $A_\model$ es una autobisimulación de $\model$. Más aún, sea $Z$ una autobisimulación de $\model$, entonces 
      $Z \subseteq A_\model$.
    \end{teorema}\pause
  \end{small}

  Luego, definimos la contracción por bisimulación de un modelo $\model$ cocientando su dominio a partir de $A_\model$.

\end{frame}

%=============================================================

% \begin{frame}
%   \frametitle{Contracción por \KHilogic-bisimulación}
%   \begin{footnotesize}  
%     \begin{block}{\textbf{Definición}}
%       Sea $\model$ un \ults. Se define su \orange{contracción por \KHilogic-bisimulación}, 
%       $\fcont{\model}=\tup{\W',\ACT',\R',\cset{\S_i'}_{i \in \AGT},\V'}$ donde \pause
%       \begin{center}
%         \begin{itemize}
%           \item $\W' \eqdef \rho_\model$\pause
%           \item $\ACT' \eqdef \{a_\pi \mid \pi\in \S_i$ para algún $i \in \AGT \}$\pause
%           \item $\R' \eqdef \{\R'_{a_\pi} \subseteq \W' \times \W' \mid a_\pi \in \ACT'\}$ donde $([w],[v]) \in \R'_{a_\pi}$ si y sólo si
%           \begin{enumerate}
%             \item Existen $w' \in [w]$ y $v' \in [v]$ tal que $(w',v')\in \R_\pi$
%             \item Cada plan en $\pi$ es FE en cada nodo de $[w]$
%           \end{enumerate}\pause
%           \item $\S_i' \eqdef \{ \{a_\pi\} \mid \pi \in \S_i \}$\pause
%           \item $\V'([w]) \eqdef \V(w)$
%         \end{itemize}
%       \end{center}    
%     \end{block}
%   \end{footnotesize}
% \end{frame}


%=============================================================


% \begin{frame}
%   \frametitle{Contracción por \KHilogic-bisimulación}

%   Presentamos una segunda propuesta de contracción basada en la propuesta clásica de la Lógica Modal Básica ($\bml$).\pause

%   \begin{footnotesize}
%     \begin{block}{\textbf{Definición}}
%       Sea $\model$ un \ults y sea $\model' = \tup{\W',\R',\V'}$
%       la contracción por $\bml$-bisimulación de $\tup{\W,\R,\V}$, definimos la \orange{contracción por 
%       \KHilogic-bisimulación} de $\model$ como 
%       $\model_{\bml}=\tup{\W', \R', \cset{\S_i}_{i\in\AGT},\V',\ACT}$.
%     \end{block}\pause
%   \end{footnotesize}

%   Análogamente, demostramos que esta propuesta también garantiza equivalencia lógica.\pause

%   \begin{small}
%     \begin{teorema}
%     Sea $\model$ un \ults y sea $\model_{\bml}$ su contracción por \KHilogic-bisimulación, entonces existe $Z$ una \KHilogic-bisimulación 
%     tal que $(w,[w])\in Z$ para cada $w\in\W$.
%     \end{teorema}
%   \end{small}
% \end{frame}

%=============================================================
\begin{frame}
  \frametitle{Contracción por \KHilogic-bisimulación}

  \begin{figure}
  \centering

  \begin{minipage}{0.45\textwidth}
  \centering
  % --- FIRST GRAPH ---
  \begin{tikzpicture}
    \node[state] (u1) {p};
                     \node[state] (p) {$p$};
        \node[state, below of=p, yshift=-1cm] (p2) {$p$};
        \node[state, right of=p, xshift=1.5cm] (q) {$q$};
        \node[state, below of=q, yshift=-1cm] (q2) {$q$};
        \node[state, right of=p, xshift=3.5cm] (r) {$r$};
        \node[state, right of=p2, xshift=3.5cm] (r2) {$r$};
        
        
        \node at ($(p)+(0,0.5)$) {$w_1$};
        \node at ($(p2)+(0.05,-0.5)$) {$w_2$};
        \node at ($(q)+(0,0.5)$) {$w_3$};
        \node at ($(q2)+(0.05,-0.5)$) {$w_4$};
        \node at ($(r)+(0,0.5)$) {$w_5$};
        \node at ($(r2)+(0.05,-0.5)$) {$w_6$};
        
        
        \path (p) edge node [above] {$a$} (q)
        (p) edge node [above] {$a$} (q2);
        \path (p2) edge node [below] {$a$} (q)
        (p2) edge node [below] {$a$} (q2);       
        \path (q) edge node [above] {$b$} (r);
        \path (q2) edge node [above] {$c$} (r)
        (q2) edge [bend left=50] node [below] {$c$} (p2)
        (q2) edge node [above] {$b$} (r2);
            
  \end{tikzpicture}
  \end{minipage}
  \hfill
  \begin{minipage}{0.45\textwidth}
  \centering
  \vspace{-1cm}
  % --- SECOND GRAPH ---
  \begin{tikzpicture}
      % \node[state, right of=u1, yshift=-1cm] (u2) {\textsf{Arg}};
      % \node[state, below of=u2, yshift=-1cm] (u3) {\textsf{Arg}};
      % \node[state, below of=u4, yshift=-1cm] (u5) {\textsf{Chi}};
      % \node[state, below of=u5, yshift=-1cm] (u6) {\textsf{Chi}};

      

      \node[state, yshift=-1cm] (p) {$p$};
          \node[state, right of=p, xshift=1.5cm] (q) {$q$};
          \node[state, right of=p, xshift=3.5cm] (r) {$r$};
          
          \node at ($(p)+(0,-0.5)$) {$[w_1]$};
          \node at ($(q)+(0,-0.5)$) {$[w_3]$};
            \node at ($(r)+(0,-0.5)$) {$[w_5]$};
            
            \path (p) edge [bend left=40] node [above] {$a_{ab}$} (r);
  \end{tikzpicture}
  \end{minipage}
  \end{figure}
  \hspace{0.5cm}
  $\S_{i} = \left\{
                    \begin{array}{c}
                        \{ab\},\{ac\}
                    \end{array}
                \right\}$
                \hspace{2.4cm}
                $\S'_{i} = \left\{
                    \begin{array}{c}
                        \{a_{ab}\}, \{a_{ac}\}
                    \end{array}
                \right\}$

\end{frame}

% =========================================================================================================== %

\begin{frame}
  \frametitle{Contracción por \KHilogic-bisimulación}
  Demostramos que esta contracción garantiza equivalencia lógica y es computable eficientemente.\pause

  \begin{small}
    \begin{teorema}
      Sea $\model$ un \ults y sea $\fcont{\model}$ su contracción por bisimulación, entonces existe $Z$ una bisimulación 
      tal que $(w,[w])\in Z$ para cada $w\in\W$. Más aún, si $\model$ es finito entonces $\fcont{\model}$ es computable 
      en tiempo polinomial.
    \end{teorema}
  \end{small}

\end{frame}

%=============================================================

\begin{frame}
  \frametitle{Conclusiones}\pause
  \begin{itemize}
    \item Introdujimos una nueva definición para \KHilogic-bisimulación, la cual solo considera características estructurales 
    de los \ultss involucrados.\pause
    \item Establecimos que tanto verificar como decidir existencia de una bisimulación entre dos \ultss es $\coNPComplete$.\pause
    \item Abordamos el problema de la minimización de modelos, para el cual propusimos una contracción por \KHilogic-bisimulación que 
    puede ser computada en tiempo polinomial. 
  \end{itemize}
\end{frame}

%=============================================================

\begin{frame}
  \frametitle{Detrás de escena}\pause
  \begin{itemize}
    \item Determinamos que el problema de definibilidad para \KHilogic está en $\Poly$.\pause
    \item Presentamos un contraejemplo para demostrar que las definiciones de bisimulación no son equivalentes en el caso infinito.\pause
    \item En el contexto de $\KHiBisim$, mencionamos cómo modificar el algoritmo presentado para que compute una fórmula de tamaño polinomial que 
    distingue a los \ultss.\pause
    \item Introducimos una segunda propuesta de contracción por \KHilogic-bisimulación, la cual está basada en la contracción clásica de 
    la Lógica Modal Básica.
  \end{itemize}
\end{frame}

%=============================================================

\begin{frame}
  \frametitle{Trabajo futuro}\pause
  \begin{itemize}
    \item Adaptar el estudio realizado en esta tesis a la lógica de Knowing How presentada por Wang en 2018:\pause
    \begin{itemize}
      \item Reformular su definición de bisimulación para que solo considere características estructurales de los modelos.
      \item Estudiar la complejidad computacional de su versión de $\KHiBisim$, el cual conjeturamos que es $\PSPACE$-completo.\pause
    \end{itemize}
    \item Implementar la contracción por \KHilogic-bisimulación propuesta y analizar su utilidad en un contexto de 
    consultas de model-checking.
    % \item Abordamos el problema de minimización de modelos, para el cual propusimos dos contracciones por \KHilogic-bisimulación las 
    % cuales son computables eficientemente.
  \end{itemize}
\end{frame}

%=============================================================

\begin{frame}
  \begin{huge}
    \begin{center}
      \color{dianablue}\textbf{¡MUCHAS GRACIAS!}
    \end{center}
  \end{huge}
\end{frame}


\end{document}
