% \pdfminorversion=4
\documentclass{beamer}

\usepackage[spanish]{babel}
\usepackage[utf8x]{inputenc}
\usepackage{tikz}
\usepackage{courier}
\usepackage{array}
\usepackage{bold-extra}
\usepackage{minted}
\usepackage[thicklines]{cancel}
\usepackage{fancyvrb}
\usepackage{xcolor}
\usepackage{amsthm}
\xdefinecolor{dianablue}{rgb}{0.18,0.24,0.31}
\xdefinecolor{darkblue}{rgb}{0.1,0.1,0.7}
\xdefinecolor{darkgreen}{rgb}{0,0.5,0}
\xdefinecolor{darkgrey}{rgb}{0.35,0.35,0.35}
\xdefinecolor{darkorange}{rgb}{0.8,0.5,0}
\xdefinecolor{darkred}{rgb}{0.7,0,0}
\definecolor{darkgreen}{rgb}{0,0.6,0}
\definecolor{mauve}{rgb}{0.58,0,0.82}
%!TEX root = main.tex
%------------------------------------------------------------------------------------------------
% adjusting vertical space in the multicols environment
% default: \multicolsep = 12pt plus 4pt minus 3pt

\setlength{\multicolsep}{6pt plus 2pt minus 1.5pt}

%------------------------------------------------------------------------------------------------


% defining the format for theorems, propositions, corollaries and lemmas. The main point is the symbol at the end of them
\usepackage[hyperref,thmmarks]{ntheorem}
\usepackage{amsmath}
\usepackage{multicol}


% \theoremstyle{plain}
\theoremsymbol{\ensuremath{\triangleleft}}

\newtheorem{teorema}{Teorema}
\newtheorem{proposicion}{Proposición}
\newtheorem{corolario}{Corolario}
\newtheorem{lema}{Lema}
\newtheorem{hecho}{Hecho}

%------------------------------------------------

% defining the format for definitions, examples, fact and others. Same symbol at the end, but header and font different
\theoremheaderfont{\normalfont\bfseries}
\theorembodyfont{\normalfont}
\theoremsymbol{\ensuremath{\dashv}}

\newtheorem{definicion}{Definición}[section]
\newtheorem{ejemplo}{Ejemplo}
\newtheorem{nota}{Nota}

%------------------------------------------------

% defining the format for proofs
% \theoremstyle{nonumberplain}
\theoremheaderfont{\itshape}
\theorembodyfont{\normalfont}
\theoremsymbol{\ensuremath{\blacksquare}}

\newtheorem{demostracion}{Demostración.}

%-----------------------------------------------------------------------------

%------------------------------------------------
\usepackage{amssymb}


%------------------------------------------------




\newcommand{\modlts}{\mathcal{S}}
\newcommand{\modults}{\mathcal{M}}
\newcommand{\model}{\modults}


\newcommand{\tup}[1]{\langle #1 \rangle}

\newcommand{\W}{\mathrm{W}}
\newcommand{\R}{\mathrm{R}}
\renewcommand{\S}{\mathrm{S}}
\newcommand{\V}{\mathrm{V}}

\newcommand{\PROP}{{\rm \sf Prop}}
\newcommand{\ACT}{{\rm \sf Act}}
\newcommand{\AGT}{{\rm \sf Agt}}
\newcommand{\ACTP}{{\rm \sf A}}

\newcommand{\cset}[1]{\{ #1 \}}
% Sets with condition \csetsc{y}{xRy} = {y | xRy}
\newcommand{\csetsc}[2]{\{ #1  \mid #2 \}}
% Sets with domain and condition \csetc{y}{A}{xRy} = {y \in A | xRy}
\newcommand{\csetc}[3]{\{ #1 \in #2 \mid #3 \}}

\newcommand{\formatmodality}[1]{\ensuremath{\mathsf{#1}}}
\newcommand{\kh}{\formatmodality{Kh}}
\newcommand{\khi}{\formatmodality{Kh}_i}
% languages
\newcommand{\KHlogic}{\ensuremath{\mathsf{L}_{\kh}}}
\newcommand{\KHilogic}{\ensuremath{\mathsf{L}_{\khi}}}

\newcommand{\mycomment}[1]{}

\newcommand{\bigO}[0]{\mathcal{O}}

% complejidad 
\newcommand{\NP}{\rm\textsf{NP}}
\newcommand{\Poly}{\rm\textsf{P}}
\newcommand{\coNP}{\rm\textsf{coNP}}

\newcommand{\DNFTAUT}{\rm\textsf{DNF-TAUT}}

\newcommand{\KHBisim}{\rm\textsf{KhBisim}}

% estrategias
\newcommand{\mystackrel}[2]{%
  \mathrel{\vbox{\offinterlineskip\ialign{%
  \hfil##\hfil\cr
  \ensuremath{\scriptstyle#1}\cr
  \noalign{\kern.25ex}
  \ensuremath{#2}\cr
  }}}
}
\newcommand{\ultsExec}{\Rightarrow}

%\newcommand{\ultsExecStrat}[1]{\stackrel{#1}{\ultsExec}}
%\newcommand{\ultsExecStrat}[1]{\mystackrel{#1}{\ultsExec}}
\newcommand{\ultsExecStrat}[1]{\mathrel{\raisebox{-.2ex}{\ensuremath{\mystackrel{#1}{\ultsExec}}}}}

%\newcommand{\ultsExecAgi}{\stackrel{i}{\ultsExec}}
%\newcommand{\ultsExecAgi}{\mystackrel{i}{\ultsExec}}
\newcommand{\ultsExecAgi}{\mathrel{\raisebox{-.2ex}{\ensuremath{\mystackrel{i}{\ultsExec}}}}}

% no se que es esto
\usepackage{enumitem}
\newlist{cond-bisim}{enumerate}{1}
\setlist[cond-bisim]{label=\itmbisimformat{\arabic*}, leftmargin=3em, itemindent=1em}




\mode<presentation>
{
  \usetheme{default}
  \usecolortheme{default}
  \usefonttheme{default}
  \setbeamertemplate{navigation symbols}{}
  \setbeamertemplate{caption}[numbered]
  \setbeamertemplate{footline}[frame number]  % or "page number"
  \setbeamercolor{frametitle}{fg=dianablue}
  \setbeamercolor{footline}{fg=black}
  \setbeamercolor{title}{fg=dianablue}
  \setbeamercolor{block title}{fg=dianablue}
  % \setbeamercolor{proposition title}{fg=blue}
  % \setbeamercolor{definition title}{fg=teal}
  % \setbeamercolor{corollary title}{fg=purple}
} 



% \usepackage[thmmarks]{ntheorem}
% \usepackage{amsmath}
% \usepackage{multicol}


% % \newtheorem{teorema}{Teorema}[chapter]
% % \newtheorem{proposicion}{Proposición}[chapter]
% % \newtheorem{corolario}{Corolario}[chapter]
% % \newtheorem{lema}{Lema}[chapter]
% % \newtheorem{hecho}{Hecho}[chapter]



% %------------------------------------------------

% % defining the format for definitions, examples, fact and others. Same symbol at the end, but header and font different
% \theoremheaderfont{\normalfont\bfseries}
% \theorembodyfont{\normalfont}
% \theoremsymbol{\ensuremath{\dashv}}

% % \newtheorem{definicion}{Definición}[chapter]
% % \newtheorem{ejemplo}{Ejemplo}[chapter]
% % \newtheorem{nota}{Nota}[chapter]

% %------------------------------------------------

% % defining the format for proofs
% \theoremstyle{nonumberplain}
% \theoremheaderfont{\itshape}
% \theorembodyfont{\normalfont}
% \theoremsymbol{\ensuremath{\blacksquare}}

% \newtheorem{demostracion}{Demostración.}



% \newcommand{\bigO}{\mathcal{O}}
% \newcommand{\zm}{Z_\mathcal{M}}
% \newtheorem{teorema}{Teorema}
% \newtheorem{proposicion}{Proposición}
% \newtheorem{definicion}{Definición}
% \newtheorem{corolario}{Corolario}




\title[yyyy-mm-dd-DESCRIPTION]{Aspectos Computacionales de Bisimulaciones para una Lógica de Knowing How}
\author{Antonio Mondejar}
\institute{Facultad de Matemática, Astronomía, Física y Computación}
\date{19 de Diciembre de 2025}

\usetikzlibrary{shapes.callouts}

\begin{document}

% \logo{\pgfputat{\pgfxy(0.11, 7.4)}{\pgfbox[right,base]{\tikz{\filldraw[fill=dianablue, draw=none] (0 cm, 0 cm) rectangle (50 cm, 1 cm);}\mbox{\hspace{-5 cm}\includegraphics[height=2 cm]{Logo_FAMAF_blanco.png}}}}}

\setbeamertemplate{headline}{%
  \begin{beamercolorbox}[wd=\paperwidth,ht=1.1cm,dp=0cm]{}
    \begin{tikzpicture}[remember picture,overlay]
      % Blue bar
      \fill[dianablue] (0,0) rectangle (\paperwidth,1.1cm);

      % Logo anchored and adjustable
      \node[anchor=east] at (\paperwidth-0.1cm,0.625cm) 
        {\includegraphics[height=1.85cm]{Logo_FAMAF_blanco.png}};
    \end{tikzpicture}
  \end{beamercolorbox}
}
% \setbeamertemplate{headline}{
%   \begin{tikzpicture}[remember picture,overlay]
%     % Blue bar
%     \fill[dianablue] (current page.north west) rectangle
%          ([yshift=-1.1cm]current page.north east);

%     % Logo
%     \node[anchor=north east] at ([yshift=-0.2cm]current page.north east)
%       {\includegraphics[height=1.8cm]{Logo_FAMAF_blanco.png}};
%   \end{tikzpicture}
% }

\begin{frame}
  \titlepage
\end{frame}

% \logo{\pgfputat{\pgfxy(0.11, 7.4)}{\pgfbox[right,base]{\tikz{\filldraw[fill=dianablue, draw=none] (0 cm, 0 cm) rectangle (50 cm, 1 cm);}\mbox{\hspace{-8 cm}\includegraphics[height=1 cm]{Logo_FAMAF_blanco.png}\includegraphics[height=1 cm]{Logo_FAMAF_blanco.png}}}}}

% Uncomment these lines for an automatically generated outline.
% \begin{frame}{Outline}
%  \tableofcontents
% \end{frame}

% START START START START START START START START START START START START START

\begin{frame}
  \frametitle{Motivación}
Los grafos son utilizados para modelar numerosos escenarios:\pause
\begin{itemize}
  \item Bases de datos.\pause
  \item Mapas de ciudades/países.\pause
  \item Sistemas de computación.\pause
  \item Usuarios de una red social.
\end{itemize}

\end{frame}

% =========================================================================================================== %

\begin{frame}
  \frametitle{Motivación}\pause

  \begin{figure}[h]
        \centering
        \begin{tikzpicture}
            \node[state] (u1) {\textsf{Arg}};
            \node[state, below of=u1, yshift=-1cm] (u2) {\textsf{Arg}};
            \node[state, below of=u2, yshift=-1cm] (u3) {\textsf{Arg}};
            \node[state, right of=u1, xshift=1.5cm] (u4) {\textsf{Chi}};
            \node[state, below of=u4, yshift=-1cm] (u5) {\textsf{Chi}};
            \node[state, below of=u5, yshift=-1cm] (u6) {\textsf{Chi}};

            

            \node at ($(u1)+(0,0.5)$) {$u_1$};
            \node at ($(u2)+(0,0.5)$) {$u_2$};
            \node at ($(u3)+(0,0.5)$) {$u_3$};
            \node at ($(u4)+(0,0.5)$) {$u_4$};
            \node at ($(u5)+(0,0.5)$) {$u_5$};
            \node at ($(u6)+(0,0.5)$) {$u_6$};

            \path (u1) edge [<->] node [above] {} (u4)
                       edge [<->] node [above] {} (u5);
                      %  edge [<->] node [above] {} (u6);

            \path (u2) edge [<->] node [above] {} (u4)
                       edge [<->] node [above] {} (u5)
                       edge [<->] node [above] {} (u6);
            
            \path (u3) edge [<->] node [above] {} (u5)
                       edge [<->] node [above] {} (u6);
                      %  edge [<->] node [above] {} (u6);


            % \path (p) edge [-] node [above] {$a$} (q)
            %           edge [-][bend left] node [above] {$a$} (r);
            % \path (p2) edge node [above] {$a$} (q);
            % \path (q) edge node [above] {$b$} (r);
        \end{tikzpicture}        % \label{fig:lts}
    \end{figure}\pause

    \begin{itemize}
      \item ¿Soy amigo de algún usuario con nacionalidad X?.
    \end{itemize} 

\end{frame}

% =========================================================================================================== %

\begin{frame}
  \frametitle{Motivación}\pause

   \begin{figure}
  \centering

  \begin{minipage}{0.45\textwidth}
  \centering
  % --- FIRST GRAPH ---
  \begin{tikzpicture}
    \node[state] (u1) {\textsf{Arg}};
            \node[state, below of=u1, yshift=-1cm] (u2) {\textsf{Arg}};
            \node[state, below of=u2, yshift=-1cm] (u3) {\textsf{Arg}};
            \node[state, right of=u1, xshift=1.5cm] (u4) {\textsf{Chi}};
            \node[state, below of=u4, yshift=-1cm] (u5) {\textsf{Chi}};
            \node[state, below of=u5, yshift=-1cm] (u6) {\textsf{Chi}};

            

            \node at ($(u1)+(0,0.5)$) {$u_1$};
            \node at ($(u2)+(0,0.5)$) {$u_2$};
            \node at ($(u3)+(0,0.5)$) {$u_3$};
            \node at ($(u4)+(0,0.5)$) {$u_4$};
            \node at ($(u5)+(0,0.5)$) {$u_5$};
            \node at ($(u6)+(0,0.5)$) {$u_6$};

            \path (u1) edge [<->] node [above] {} (u4)
                       edge [<->] node [above] {} (u5);
                      %  edge [<->] node [above] {} (u6);

            \path (u2) edge [<->] node [above] {} (u4)
                       edge [<->] node [above] {} (u5)
                       edge [<->] node [above] {} (u6);
            
            \path (u3) edge [<->] node [above] {} (u5)
                       edge [<->] node [above] {} (u6);
  \end{tikzpicture}\pause
  \end{minipage}
  \hfill
  \begin{minipage}{0.45\textwidth}
  \centering
  % --- SECOND GRAPH ---
  \begin{tikzpicture}
    \node[state] (u1) {\textsf{Arg}};
            % \node[state, right of=u1, yshift=-1cm] (u2) {\textsf{Arg}};
            % \node[state, below of=u2, yshift=-1cm] (u3) {\textsf{Arg}};
            \node[state, right of=u1, xshift=1.5cm] (u2) {\textsf{Chi}};
            % \node[state, below of=u4, yshift=-1cm] (u5) {\textsf{Chi}};
            % \node[state, below of=u5, yshift=-1cm] (u6) {\textsf{Chi}};

            

            \node at ($(u1)+(0,0.6)$) {$u_1'$};
            \node at ($(u2)+(0,0.6)$) {$u_2'$};
            % \node at ($(u3)+(0,0.5)$) {$u_3$};
            % \node at ($(u4)+(0,0.5)$) {$u_4$};
            % \node at ($(u5)+(0,0.5)$) {$u_5$};
            % \node at ($(u6)+(0,0.5)$) {$u_6$};

            \path (u1) edge [<->] node [above] {} (u2);
                      %  edge [<->] node [above] {} (u5);
                      %  edge [<->] node [above] {} (u6);

            % \path (u2) edge [<->] node [above] {} (u4)
            %            edge [<->] node [above] {} (u5)
            %            edge [<->] node [above] {} (u6);
            
            % \path (u3) edge [<->] node [above] {} (u5)
            %            edge [<->] node [above] {} (u6);
  \end{tikzpicture}
  \end{minipage}

  \end{figure}

\end{frame}

% =========================================================================================================== %


\begin{frame}
  \frametitle{Motivación}\pause

  \begin{figure}[h]
        \centering
        \begin{tikzpicture}
            \node[state] (u1) {\textsf{Nole}};
            \node[state, below of=u1, yshift=-1cm] (u2) {\textsf{Roger}};
            \node[state, below of=u2, yshift=-1cm] (u3) {\textsf{Jannik}};
            \node[state, right of=u1, xshift=1.5cm] (u4) {\textsf{Andy}};
            \node[state, below of=u4, yshift=-1cm] (u5) {\textsf{Charly}};
            \node[state, below of=u5, yshift=-1cm] (u6) {\textsf{Rafa}};

            

            \node at ($(u1)+(0,0.5)$) {$u_1$};
            \node at ($(u2)+(0,0.5)$) {$u_2$};
            \node at ($(u3)+(0,0.5)$) {$u_3$};
            \node at ($(u4)+(0,0.5)$) {$u_4$};
            \node at ($(u5)+(0,0.5)$) {$u_5$};
            \node at ($(u6)+(0,0.5)$) {$u_6$};

            \path (u1) edge [<->] node [above] {} (u4)
                       edge [<->] node [above] {} (u5)
                      %  edge [<->] node [above] {} (u6)
                       edge [<->][bend right = 50] node [above] {} (u3)
                       edge [<->][bend right = 32] node [above] {} (u2);

            \path (u2) edge [<->] node [above] {} (u4)
                       edge [<->][bend right = 30] node [above] {} (u3)
                       edge [<->] node [above] {} (u5)
                       edge [<->] node [above] {} (u6);
            
            \path (u3) edge [<->] node [above] {} (u5)
                       edge [<->] node [above] {} (u6);
            
            \path (u4) edge [<->][bend left=32] node [above] {} (u5)
                       edge [<->][bend left=50] node [above] {} (u6);
                      %  edge [<->] node [above] {} (u6);
            \path (u5) edge [<->][bend left=32] node [above] {} (u6);

            % \path (p) edge [-] node [above] {$a$} (q)
            %           edge [-][bend left] node [above] {$a$} (r);
            % \path (p2) edge node [above] {$a$} (q);
            % \path (q) edge node [above] {$b$} (r);
        \end{tikzpicture}        % \label{fig:lts}
    \end{figure}\pause
    \begin{itemize}
    \item ¿Existe alguna cadena de amistades que me conecte con X?. 
    \end{itemize}
\end{frame}

% =========================================================================================================== %

\begin{frame}
  \frametitle{Motivación}\pause

  \begin{figure}
  \centering

  \begin{minipage}{0.45\textwidth}
  \centering
  % --- FIRST GRAPH ---
  \begin{tikzpicture}
    \node[state] (u1) {\textsf{Nole}};
    \node[state, below of=u1, yshift=-1cm] (u2) {\textsf{Roger}};
    \node[state, below of=u2, yshift=-1cm] (u3) {\textsf{Jannik}};
    \node[state, right of=u1, xshift=1.5cm] (u4) {\textsf{Andy}};
    \node[state, below of=u4, yshift=-1cm] (u5) {\textsf{Charly}};
    \node[state, below of=u5, yshift=-1cm] (u6) {\textsf{Rafa}};

    

    \node at ($(u1)+(0,0.5)$) {$u_1$};
    \node at ($(u2)+(0,0.5)$) {$u_2$};
    \node at ($(u3)+(0,0.5)$) {$u_3$};
    \node at ($(u4)+(0,0.5)$) {$u_4$};
    \node at ($(u5)+(0,0.5)$) {$u_5$};
    \node at ($(u6)+(0,0.5)$) {$u_6$};

    \path (u1) edge [<->] node [above] {} (u4)
                edge [<->] node [above] {} (u5)
              %  edge [<->] node [above] {} (u6)
                edge [<->][bend right = 50] node [above] {} (u3)
                edge [<->][bend right = 32] node [above] {} (u2);

    \path (u2) edge [<->] node [above] {} (u4)
                edge [<->][bend right = 30] node [above] {} (u3)
                edge [<->] node [above] {} (u5)
                edge [<->] node [above] {} (u6);
    
    \path (u3) edge [<->] node [above] {} (u5)
                edge [<->] node [above] {} (u6);
    
    \path (u4) edge [<->][bend left=32] node [above] {} (u5)
                edge [<->][bend left=50] node [above] {} (u6);
              %  edge [<->] node [above] {} (u6);
    \path (u5) edge [<->][bend left=32] node [above] {} (u6);
  \end{tikzpicture}\pause
  \end{minipage}
  \hfill
  \begin{minipage}{0.45\textwidth}
  \centering
  % --- SECOND GRAPH ---
  \begin{tikzpicture}
    \node[state] (u1) {\textsf{Nole}};
    \node[state, below of=u1, yshift=-1cm] (u2) {\textsf{Roger}};
    \node[state, below of=u2, yshift=-1cm] (u3) {\textsf{Jannik}};
    \node[state, right of=u1, xshift=1.5cm] (u4) {\textsf{Andy}};
    \node[state, below of=u4, yshift=-1cm] (u5) {\textsf{Charly}};
    \node[state, below of=u5, yshift=-1cm] (u6) {\textsf{Rafa}};

    

    \node at ($(u1)+(0,0.6)$) {$u_1'$};
    \node at ($(u2)+(0,0.6)$) {$u_2'$};
    \node at ($(u3)+(0,0.6)$) {$u_3'$};
    \node at ($(u4)+(0,0.6)$) {$u_4'$};
    \node at ($(u5)+(0,0.6)$) {$u_5'$};
    \node at ($(u6)+(0,0.6)$) {$u_6'$};

    \path (u1) edge [<->] node [above] {} (u4);
              %   edge [<->] node [above] {} (u5)
              % %  edge [<->] node [above] {} (u6)
              %   edge [<->][bend right = 50] node [above] {} (u3)
              %   edge [<->][bend right = 32] node [above] {} (u2);

    \path (u2)  edge [<->][bend right = 30] node [above] {} (u3)
                edge [<->] node [above] {} (u5);
                % edge [<->] node [above] {} (u6);
    
    \path (u3) edge [<->] node [above] {} (u6);
                % edge [<->] node [above] {} (u6);
    
    \path (u4) edge [<->][bend left=32] node [above] {} (u5);
                % edge [<->][bend left=50] node [above] {} (u6);
              %  edge [<->] node [above] {} (u6);
    % \path (u5) edge [<->][bend left=32] node [above] {} (u6);
  \end{tikzpicture}
  \end{minipage}

  \end{figure}

    % \begin{itemize}
    % \item ¿Existe alguna cadena de amistades que me conecte con X?. 
    % \end{itemize}
\end{frame}

% =========================================================================================================== %

\begin{frame}
  \frametitle{Motivación}\pause

    
    \begin{itemize}
      \item Lógicas modales.\pause
      \begin{itemize}
        \item Familia de lenguajes formales que permiten matematizar este tipo de preguntas.\pause
        \item Distintos modos de verdad: posibilidad, necesidad/universalidad, obligaciones, etc.\pause
        \item Modelos de Kripke.\pause
      \end{itemize}
      \item Equivalencia lógica.\pause
      \begin{itemize}
        \item Determina el poder expresivo de una lógica.\pause
        \item Utilidad en aplicaciones prácticas de las lógicas modales.
      \end{itemize}

    \end{itemize}
    
\end{frame}



% =========================================================================================================== %

\begin{frame}
  \frametitle{¿Knowing How?}\pause

    \begin{itemize}
      \item Lógicas epistémicas: lógicas modales que buscan representar y razonar sobre el conocimiento de agentes.\pause
      \begin{itemize}
        \item Usualmente modelan el ``saber qué''.
        \item \textit{Juan sabe que está lloviendo en Río Cuarto}.\pause
      \end{itemize}
      \item Existen lógicas que estudian otras formas del conocimiento:\pause
      \begin{itemize}
        \item ``saber si''.\pause
        \item ``saber por qué''.\pause
        \item {\color{darkorange}{``saber cómo''}.}
      \end{itemize}
    \end{itemize}
    
\end{frame}

% =========================================================================================================== %

\begin{frame}
  \frametitle{Lógica de Knowing How Basada en Incertidumbre}\pause

\end{frame}


% =========================================================================================================== %


\end{document}
