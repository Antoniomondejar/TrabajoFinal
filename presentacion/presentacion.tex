% \pdfminorversion=4
\documentclass{beamer}

\usepackage[spanish]{babel}
\usepackage[utf8x]{inputenc}
\usepackage{tikz}
\usepackage{courier}
\usepackage{array}
\usepackage{bold-extra}
\usepackage{minted}
% \usepackage[thicklines]{cancel}
\usepackage{fancyvrb}
\usepackage{xspace}
\usepackage{xcolor}
\usepackage{amsthm}
\xdefinecolor{dianablue}{rgb}{0.18,0.24,0.31}
\xdefinecolor{darkblue}{rgb}{0.1,0.1,0.7}
\xdefinecolor{darkgreen}{rgb}{0,0.5,0}
\xdefinecolor{darkgrey}{rgb}{0.35,0.35,0.35}
\xdefinecolor{darkorange}{rgb}{0.8,0.5,0}
\xdefinecolor{darkred}{rgb}{0.7,0,0}
\definecolor{darkgreen}{rgb}{0,0.6,0}
\definecolor{mauve}{rgb}{0.58,0,0.82}
%!TEX root = main.tex
%------------------------------------------------------------------------------------------------
% adjusting vertical space in the multicols environment
% default: \multicolsep = 12pt plus 4pt minus 3pt

\setlength{\multicolsep}{6pt plus 2pt minus 1.5pt}

%------------------------------------------------------------------------------------------------


% defining the format for theorems, propositions, corollaries and lemmas. The main point is the symbol at the end of them
\usepackage[hyperref,thmmarks]{ntheorem}
\usepackage{amsmath}
\usepackage{multicol}


% \theoremstyle{plain}
\theoremsymbol{\ensuremath{\triangleleft}}

\newtheorem{teorema}{Teorema}
\newtheorem{proposicion}{Proposición}
\newtheorem{corolario}{Corolario}
\newtheorem{lema}{Lema}
\newtheorem{hecho}{Hecho}

%------------------------------------------------

% defining the format for definitions, examples, fact and others. Same symbol at the end, but header and font different
\theoremheaderfont{\normalfont\bfseries}
\theorembodyfont{\normalfont}
\theoremsymbol{\ensuremath{\dashv}}

\newtheorem{definicion}{Definición}[section]
\newtheorem{ejemplo}{Ejemplo}
\newtheorem{nota}{Nota}

%------------------------------------------------

% defining the format for proofs
% \theoremstyle{nonumberplain}
\theoremheaderfont{\itshape}
\theorembodyfont{\normalfont}
\theoremsymbol{\ensuremath{\blacksquare}}

\newtheorem{demostracion}{Demostración.}

%-----------------------------------------------------------------------------

%------------------------------------------------
\usepackage{amssymb}


%------------------------------------------------




\newcommand{\modlts}{\mathcal{S}}
\newcommand{\modults}{\mathcal{M}}
\newcommand{\model}{\modults}


\newcommand{\tup}[1]{\langle #1 \rangle}

\newcommand{\W}{\mathrm{W}}
\newcommand{\R}{\mathrm{R}}
\renewcommand{\S}{\mathrm{S}}
\newcommand{\V}{\mathrm{V}}

\newcommand{\PROP}{{\rm \sf Prop}}
\newcommand{\ACT}{{\rm \sf Act}}
\newcommand{\AGT}{{\rm \sf Agt}}
\newcommand{\ACTP}{{\rm \sf A}}

\newcommand{\cset}[1]{\{ #1 \}}
% Sets with condition \csetsc{y}{xRy} = {y | xRy}
\newcommand{\csetsc}[2]{\{ #1  \mid #2 \}}
% Sets with domain and condition \csetc{y}{A}{xRy} = {y \in A | xRy}
\newcommand{\csetc}[3]{\{ #1 \in #2 \mid #3 \}}

\newcommand{\formatmodality}[1]{\ensuremath{\mathsf{#1}}}
\newcommand{\kh}{\formatmodality{Kh}}
\newcommand{\khi}{\formatmodality{Kh}_i}
% languages
\newcommand{\KHlogic}{\ensuremath{\mathsf{L}_{\kh}}}
\newcommand{\KHilogic}{\ensuremath{\mathsf{L}_{\khi}}}

\newcommand{\mycomment}[1]{}

\newcommand{\bigO}[0]{\mathcal{O}}

% complejidad 
\newcommand{\NP}{\rm\textsf{NP}}
\newcommand{\Poly}{\rm\textsf{P}}
\newcommand{\coNP}{\rm\textsf{coNP}}

\newcommand{\DNFTAUT}{\rm\textsf{DNF-TAUT}}

\newcommand{\KHBisim}{\rm\textsf{KhBisim}}

% estrategias
\newcommand{\mystackrel}[2]{%
  \mathrel{\vbox{\offinterlineskip\ialign{%
  \hfil##\hfil\cr
  \ensuremath{\scriptstyle#1}\cr
  \noalign{\kern.25ex}
  \ensuremath{#2}\cr
  }}}
}
\newcommand{\ultsExec}{\Rightarrow}

%\newcommand{\ultsExecStrat}[1]{\stackrel{#1}{\ultsExec}}
%\newcommand{\ultsExecStrat}[1]{\mystackrel{#1}{\ultsExec}}
\newcommand{\ultsExecStrat}[1]{\mathrel{\raisebox{-.2ex}{\ensuremath{\mystackrel{#1}{\ultsExec}}}}}

%\newcommand{\ultsExecAgi}{\stackrel{i}{\ultsExec}}
%\newcommand{\ultsExecAgi}{\mystackrel{i}{\ultsExec}}
\newcommand{\ultsExecAgi}{\mathrel{\raisebox{-.2ex}{\ensuremath{\mystackrel{i}{\ultsExec}}}}}

% no se que es esto
\usepackage{enumitem}
\newlist{cond-bisim}{enumerate}{1}
\setlist[cond-bisim]{label=\itmbisimformat{\arabic*}, leftmargin=3em, itemindent=1em}




\mode<presentation>
{
  \usetheme{default}
  \usecolortheme{default}
  \usefonttheme{default}
  \setbeamerfont{block title}{series=\bfseries}
  \setbeamertemplate{blocks}[rounded][shadow=true]
  \setbeamertemplate{navigation symbols}{}
  \setbeamertemplate{caption}[numbered]
  \setbeamertemplate{footline}[frame number]  % or "page number"
  \setbeamercolor{frametitle}{fg=dianablue}
  \setbeamercolor{footline}{fg=black}
  \setbeamercolor{title}{fg=dianablue}
  \setbeamercolor{block title}{fg=dianablue}
  % \setbeamercolor{proposition title}{fg=blue}
  % \setbeamercolor{definition title}{fg=teal}
  % \setbeamercolor{corollary title}{fg=purple}
  } 
  
% \useblocktheme{rounded}


% \usepackage[thmmarks]{ntheorem}
% \usepackage{amsmath}
% \usepackage{multicol}


% % \newtheorem{teorema}{Teorema}[chapter]
% % \newtheorem{proposicion}{Proposición}[chapter]
% % \newtheorem{corolario}{Corolario}[chapter]
% % \newtheorem{lema}{Lema}[chapter]
% % \newtheorem{hecho}{Hecho}[chapter]



% %------------------------------------------------

% % defining the format for definitions, examples, fact and others. Same symbol at the end, but header and font different
% \theoremheaderfont{\normalfont\bfseries}
% \theorembodyfont{\normalfont}
% \theoremsymbol{\ensuremath{\dashv}}

% % \newtheorem{definicion}{Definición}[chapter]
% % \newtheorem{ejemplo}{Ejemplo}[chapter]
% % \newtheorem{nota}{Nota}[chapter]

% %------------------------------------------------

% % defining the format for proofs
% \theoremstyle{nonumberplain}
% \theoremheaderfont{\itshape}
% \theorembodyfont{\normalfont}
% \theoremsymbol{\ensuremath{\blacksquare}}

% \newtheorem{demostracion}{Demostración.}



% \newcommand{\bigO}{\mathcal{O}}
% \newcommand{\zm}{Z_\mathcal{M}}
% \newtheorem{teorema}{Teorema}
% \newtheorem{proposicion}{Proposición}
% \newtheorem{definicion}{Definición}
% \newtheorem{corolario}{Corolario}




\title[yyyy-mm-dd-DESCRIPTION]{Aspectos Computacionales de Bisimulaciones para una Lógica de Knowing How}
\author{Antonio Mondejar}
\institute{Facultad de Matemática, Astronomía, Física y Computación}
\date{19 de Diciembre de 2025}

\usetikzlibrary{shapes.callouts}

\begin{document}

% \logo{\pgfputat{\pgfxy(0.11, 7.4)}{\pgfbox[right,base]{\tikz{\filldraw[fill=dianablue, draw=none] (0 cm, 0 cm) rectangle (50 cm, 1 cm);}\mbox{\hspace{-5 cm}\includegraphics[height=2 cm]{Logo_FAMAF_blanco.png}}}}}

\setbeamertemplate{headline}{%
  \begin{beamercolorbox}[wd=\paperwidth,ht=1.1cm,dp=0cm]{}
    \begin{tikzpicture}[remember picture,overlay]
      % Blue bar
      \fill[dianablue] (0,0) rectangle (\paperwidth,1.1cm);

      % Logo anchored and adjustable
      \node[anchor=east] at (\paperwidth-0.1cm,0.625cm) 
        {\includegraphics[height=1.85cm]{Logo_FAMAF_blanco.png}};
    \end{tikzpicture}
  \end{beamercolorbox}
}
% \setbeamertemplate{headline}{
%   \begin{tikzpicture}[remember picture,overlay]
%     % Blue bar
%     \fill[dianablue] (current page.north west) rectangle
%          ([yshift=-1.1cm]current page.north east);

%     % Logo
%     \node[anchor=north east] at ([yshift=-0.2cm]current page.north east)
%       {\includegraphics[height=1.8cm]{Logo_FAMAF_blanco.png}};
%   \end{tikzpicture}
% }

\begin{frame}
  \titlepage
\end{frame}

% \logo{\pgfputat{\pgfxy(0.11, 7.4)}{\pgfbox[right,base]{\tikz{\filldraw[fill=dianablue, draw=none] (0 cm, 0 cm) rectangle (50 cm, 1 cm);}\mbox{\hspace{-8 cm}\includegraphics[height=1 cm]{Logo_FAMAF_blanco.png}\includegraphics[height=1 cm]{Logo_FAMAF_blanco.png}}}}}

% Uncomment these lines for an automatically generated outline.
% \begin{frame}{Outline}
%  \tableofcontents
% \end{frame}

% START START START START START START START START START START START START START

\begin{frame}
  \frametitle{Motivación}
Los grafos son utilizados para modelar numerosos escenarios:\pause
\begin{itemize}
  \item Bases de datos.\pause
  \item Mapas de ciudades/países.\pause
  \item Sistemas de computación.\pause
  \item Usuarios de una red social.
\end{itemize}

\end{frame}

% =========================================================================================================== %

\begin{frame}
  \frametitle{Motivación}\pause

  \begin{figure}[h]
        \centering
        \begin{tikzpicture}
            \node[state] (u1) {\textsf{Arg}};
            \node[state, below of=u1, yshift=-1cm] (u2) {\textsf{Arg}};
            \node[state, below of=u2, yshift=-1cm] (u3) {\textsf{Arg}};
            \node[state, right of=u1, xshift=1.5cm] (u4) {\textsf{Chi}};
            \node[state, below of=u4, yshift=-1cm] (u5) {\textsf{Chi}};
            \node[state, below of=u5, yshift=-1cm] (u6) {\textsf{Chi}};

            

            \node at ($(u1)+(0,0.5)$) {$u_1$};
            \node at ($(u2)+(0,0.5)$) {$u_2$};
            \node at ($(u3)+(0,0.5)$) {$u_3$};
            \node at ($(u4)+(0,0.5)$) {$u_4$};
            \node at ($(u5)+(0,0.5)$) {$u_5$};
            \node at ($(u6)+(0,0.5)$) {$u_6$};

            \path (u1) edge [<->] node [above] {} (u4)
                       edge [<->] node [above] {} (u5);
                      %  edge [<->] node [above] {} (u6);

            \path (u2) edge [<->] node [above] {} (u4)
                       edge [<->] node [above] {} (u5)
                       edge [<->] node [above] {} (u6);
            
            \path (u3) edge [<->] node [above] {} (u5)
                       edge [<->] node [above] {} (u6);
                      %  edge [<->] node [above] {} (u6);


            % \path (p) edge [-] node [above] {$a$} (q)
            %           edge [-][bend left] node [above] {$a$} (r);
            % \path (p2) edge node [above] {$a$} (q);
            % \path (q) edge node [above] {$b$} (r);
        \end{tikzpicture}        % \label{fig:lts}
    \end{figure}\pause

    \begin{itemize}
      \item ¿Soy amigo de algún usuario con nacionalidad X?.
    \end{itemize} 

\end{frame}

% =========================================================================================================== %

\begin{frame}
  \frametitle{Motivación}\pause

   \begin{figure}
  \centering

  \begin{minipage}{0.45\textwidth}
  \centering
  % --- FIRST GRAPH ---
  \begin{tikzpicture}
    \node[state] (u1) {\textsf{Arg}};
            \node[state, below of=u1, yshift=-1cm] (u2) {\textsf{Arg}};
            \node[state, below of=u2, yshift=-1cm] (u3) {\textsf{Arg}};
            \node[state, right of=u1, xshift=1.5cm] (u4) {\textsf{Chi}};
            \node[state, below of=u4, yshift=-1cm] (u5) {\textsf{Chi}};
            \node[state, below of=u5, yshift=-1cm] (u6) {\textsf{Chi}};

            

            \node at ($(u1)+(0,0.5)$) {$u_1$};
            \node at ($(u2)+(0,0.5)$) {$u_2$};
            \node at ($(u3)+(0,0.5)$) {$u_3$};
            \node at ($(u4)+(0,0.5)$) {$u_4$};
            \node at ($(u5)+(0,0.5)$) {$u_5$};
            \node at ($(u6)+(0,0.5)$) {$u_6$};

            \path (u1) edge [<->] node [above] {} (u4)
                       edge [<->] node [above] {} (u5);
                      %  edge [<->] node [above] {} (u6);

            \path (u2) edge [<->] node [above] {} (u4)
                       edge [<->] node [above] {} (u5)
                       edge [<->] node [above] {} (u6);
            
            \path (u3) edge [<->] node [above] {} (u5)
                       edge [<->] node [above] {} (u6);
  \end{tikzpicture}\pause
  \end{minipage}
  \hfill
  \begin{minipage}{0.45\textwidth}
  \centering
  % --- SECOND GRAPH ---
  \begin{tikzpicture}
    \node[state] (u1) {\textsf{Arg}};
            % \node[state, right of=u1, yshift=-1cm] (u2) {\textsf{Arg}};
            % \node[state, below of=u2, yshift=-1cm] (u3) {\textsf{Arg}};
            \node[state, right of=u1, xshift=1.5cm] (u2) {\textsf{Chi}};
            % \node[state, below of=u4, yshift=-1cm] (u5) {\textsf{Chi}};
            % \node[state, below of=u5, yshift=-1cm] (u6) {\textsf{Chi}};

            

            \node at ($(u1)+(0,0.6)$) {$u_1'$};
            \node at ($(u2)+(0,0.6)$) {$u_2'$};
            % \node at ($(u3)+(0,0.5)$) {$u_3$};
            % \node at ($(u4)+(0,0.5)$) {$u_4$};
            % \node at ($(u5)+(0,0.5)$) {$u_5$};
            % \node at ($(u6)+(0,0.5)$) {$u_6$};

            \path (u1) edge [<->] node [above] {} (u2);
                      %  edge [<->] node [above] {} (u5);
                      %  edge [<->] node [above] {} (u6);

            % \path (u2) edge [<->] node [above] {} (u4)
            %            edge [<->] node [above] {} (u5)
            %            edge [<->] node [above] {} (u6);
            
            % \path (u3) edge [<->] node [above] {} (u5)
            %            edge [<->] node [above] {} (u6);
  \end{tikzpicture}
  \end{minipage}

  \end{figure}

\end{frame}

% =========================================================================================================== %


\begin{frame}
  \frametitle{Motivación}\pause

  \begin{figure}[h]
        \centering
        \begin{tikzpicture}
            \node[state] (u1) {\textsf{Nole}};
            \node[state, below of=u1, yshift=-1cm] (u2) {\textsf{Roger}};
            \node[state, below of=u2, yshift=-1cm] (u3) {\textsf{Jannik}};
            \node[state, right of=u1, xshift=1.5cm] (u4) {\textsf{Andy}};
            \node[state, below of=u4, yshift=-1cm] (u5) {\textsf{Charly}};
            \node[state, below of=u5, yshift=-1cm] (u6) {\textsf{Rafa}};

            

            \node at ($(u1)+(0,0.5)$) {$u_1$};
            \node at ($(u2)+(0,0.5)$) {$u_2$};
            \node at ($(u3)+(0,0.5)$) {$u_3$};
            \node at ($(u4)+(0,0.5)$) {$u_4$};
            \node at ($(u5)+(0,0.5)$) {$u_5$};
            \node at ($(u6)+(0,0.5)$) {$u_6$};

            \path (u1) edge [<->] node [above] {} (u4)
                       edge [<->] node [above] {} (u5)
                      %  edge [<->] node [above] {} (u6)
                       edge [<->][bend right = 50] node [above] {} (u3)
                       edge [<->][bend right = 32] node [above] {} (u2);

            \path (u2) edge [<->] node [above] {} (u4)
                       edge [<->][bend right = 30] node [above] {} (u3)
                       edge [<->] node [above] {} (u5)
                       edge [<->] node [above] {} (u6);
            
            \path (u3) edge [<->] node [above] {} (u5)
                       edge [<->] node [above] {} (u6);
            
            \path (u4) edge [<->][bend left=32] node [above] {} (u5)
                       edge [<->][bend left=50] node [above] {} (u6);
                      %  edge [<->] node [above] {} (u6);
            \path (u5) edge [<->][bend left=32] node [above] {} (u6);

            % \path (p) edge [-] node [above] {$a$} (q)
            %           edge [-][bend left] node [above] {$a$} (r);
            % \path (p2) edge node [above] {$a$} (q);
            % \path (q) edge node [above] {$b$} (r);
        \end{tikzpicture}        % \label{fig:lts}
    \end{figure}\pause
    \begin{itemize}
    \item ¿Existe alguna cadena de amistades que me conecte con X?. 
    \end{itemize}
\end{frame}

% =========================================================================================================== %

\begin{frame}
  \frametitle{Motivación}\pause

  \begin{figure}
  \centering

  \begin{minipage}{0.45\textwidth}
  \centering
  % --- FIRST GRAPH ---
  \begin{tikzpicture}
    \node[state] (u1) {\textsf{Nole}};
    \node[state, below of=u1, yshift=-1cm] (u2) {\textsf{Roger}};
    \node[state, below of=u2, yshift=-1cm] (u3) {\textsf{Jannik}};
    \node[state, right of=u1, xshift=1.5cm] (u4) {\textsf{Andy}};
    \node[state, below of=u4, yshift=-1cm] (u5) {\textsf{Charly}};
    \node[state, below of=u5, yshift=-1cm] (u6) {\textsf{Rafa}};

    

    \node at ($(u1)+(0,0.5)$) {$u_1$};
    \node at ($(u2)+(0,0.5)$) {$u_2$};
    \node at ($(u3)+(0,0.5)$) {$u_3$};
    \node at ($(u4)+(0,0.5)$) {$u_4$};
    \node at ($(u5)+(0,0.5)$) {$u_5$};
    \node at ($(u6)+(0,0.5)$) {$u_6$};

    \path (u1) edge [<->] node [above] {} (u4)
                edge [<->] node [above] {} (u5)
              %  edge [<->] node [above] {} (u6)
                edge [<->][bend right = 50] node [above] {} (u3)
                edge [<->][bend right = 32] node [above] {} (u2);

    \path (u2) edge [<->] node [above] {} (u4)
                edge [<->][bend right = 30] node [above] {} (u3)
                edge [<->] node [above] {} (u5)
                edge [<->] node [above] {} (u6);
    
    \path (u3) edge [<->] node [above] {} (u5)
                edge [<->] node [above] {} (u6);
    
    \path (u4) edge [<->][bend left=32] node [above] {} (u5)
                edge [<->][bend left=50] node [above] {} (u6);
              %  edge [<->] node [above] {} (u6);
    \path (u5) edge [<->][bend left=32] node [above] {} (u6);
  \end{tikzpicture}\pause
  \end{minipage}
  \hfill
  \begin{minipage}{0.45\textwidth}
  \centering
  % --- SECOND GRAPH ---
  \begin{tikzpicture}
    \node[state] (u1) {\textsf{Nole}};
    \node[state, below of=u1, yshift=-1cm] (u2) {\textsf{Roger}};
    \node[state, below of=u2, yshift=-1cm] (u3) {\textsf{Jannik}};
    \node[state, right of=u1, xshift=1.5cm] (u4) {\textsf{Andy}};
    \node[state, below of=u4, yshift=-1cm] (u5) {\textsf{Charly}};
    \node[state, below of=u5, yshift=-1cm] (u6) {\textsf{Rafa}};

    

    \node at ($(u1)+(0,0.6)$) {$u_1'$};
    \node at ($(u2)+(0,0.6)$) {$u_2'$};
    \node at ($(u3)+(0,0.6)$) {$u_3'$};
    \node at ($(u4)+(0,0.6)$) {$u_4'$};
    \node at ($(u5)+(0,0.6)$) {$u_5'$};
    \node at ($(u6)+(0,0.6)$) {$u_6'$};

    \path (u1) edge [<->] node [above] {} (u4);
              %   edge [<->] node [above] {} (u5)
              % %  edge [<->] node [above] {} (u6)
              %   edge [<->][bend right = 50] node [above] {} (u3)
              %   edge [<->][bend right = 32] node [above] {} (u2);

    \path (u2)  edge [<->][bend right = 30] node [above] {} (u3)
                edge [<->] node [above] {} (u5);
                % edge [<->] node [above] {} (u6);
    
    \path (u3) edge [<->] node [above] {} (u6);
                % edge [<->] node [above] {} (u6);
    
    \path (u4) edge [<->][bend left=32] node [above] {} (u5);
                % edge [<->][bend left=50] node [above] {} (u6);
              %  edge [<->] node [above] {} (u6);
    % \path (u5) edge [<->][bend left=32] node [above] {} (u6);
  \end{tikzpicture}
  \end{minipage}

  \end{figure}

    % \begin{itemize}
    % \item ¿Existe alguna cadena de amistades que me conecte con X?. 
    % \end{itemize}
\end{frame}

% =========================================================================================================== %

\begin{frame}
  \frametitle{Motivación}\pause

    
    \begin{itemize}
      \item Lógicas modales.\pause
      \begin{itemize}
        \item Familia de lenguajes formales que permiten matematizar este tipo de preguntas.\pause
        \item Distintos modos de verdad: posibilidad, necesidad/universalidad, obligaciones, etc.\pause
        \item Modelos de Kripke.\pause
      \end{itemize}
      \item Equivalencia lógica.\pause
      \begin{itemize}
        \item Determina el poder expresivo de una lógica.\pause
        \item Utilidad en aplicaciones prácticas de las lógicas modales.
      \end{itemize}

    \end{itemize}
    
\end{frame}



% =========================================================================================================== %

\begin{frame}
  \frametitle{¿Knowing How?}\pause

    \begin{itemize}
      \item Lógicas epistémicas: lógicas modales que buscan representar y razonar sobre el conocimiento de agentes.\pause
      \begin{itemize}
        \item Usualmente modelan el ``saber qué''.
        \item \textit{Juan sabe que está lloviendo en Río Cuarto}.\pause
      \end{itemize}
      \item Existen lógicas que estudian otras formas del conocimiento:\pause
      \begin{itemize}
        \item ``saber si''.\pause
        \item ``saber por qué''.\pause
        \item {\color{darkorange}{``saber cómo''}.}
      \end{itemize}
    \end{itemize}
    
\end{frame}

% =========================================================================================================== %

\begin{frame}
  \frametitle{Lógica de Knowing How Basada en Incertidumbre (\KHilogic)}\pause

  \begin{itemize}
  \item Conjunto de agentes: personas, robots, computadoras en sistemas distribuídos, etc. \pause
  \item \textbf{Incertidumbre}: considera la incapacidad de un agente para distinguir planes. \pause
  \item Propiedades referidas a las habilidades de los agentes. \pause
  \item Un agente sabe cómo lograr un objetivo a partir de una condición inicial. \pause Conoce un 
  plan \textit{adecuado} que garantiza esta situación. \pause
    \begin{itemize}
      \item ¿Cómo se representa el conocimiento de los agentes?
      \item ¿Qué consideramos \textit{adecuado}?
    \end{itemize}

  \end{itemize}

\end{frame}


% =========================================================================================================== %

\begin{frame}
  \frametitle{\lts basado en incertidumbre (\ults)}\pause
  \begin{figure}[h]
        % \hspace{2.3cm}
        \vspace{-0.5cm}
        \begin{tikzpicture}
            \node[state] (phor) {$\textsf{Pq.Hor}$};
            \node[state, right of=phor, xshift=1.8cm] (gue) {$\textsf{Guemes}$};
            \node[state, below of=gue, yshift=-1.6cm, xshift=0.4cm] (cuniv) {$\textsf{C.Univ}$};
            \node[state, right of=gue, yshift=-1cm, xshift=1.8cm] (cen) {$\textsf{Centro}$};
            \node[state, right of=cen, xshift=1.8cm] (cof) {$\textsf{Cofico}$};
            \node[state, right of=cof, xshift=1.8cm] (acba) {$\textsf{Alta Cba}$};
            
            % \node at ($(p)+(0,0.5)$) {$w_1$};
            % \node at ($(r)+(0,0.5)$) {$w_4$};
            % \node at ($(q)+(0,0.5)$) {$w_3$};
            % \node at ($(p2)+(0.05,-0.6)$) {$w_2$};
            
            \path (phor) edge [bend right=12] node [below] {\tiny$30^v$} (gue);

            \path (gue) edge [bend right=12] node [below] {\tiny$30^v$} (cen)
                        edge [bend right=12] node [above] {\tiny$30^i$} (phor);
            
            \path (cen) edge [bend right=12] node [above] {\tiny$30^i$} (gue)
                        edge [bend right=12] node [above] {\tiny$18^i$} (cuniv)
                        edge [bend right=12] node [below] {\tiny{$18^v$, $30^v$}} (cof);
            
            \path (cof) edge [bend right=12] node [above] {\tiny{$18^i$, $30^i$}} (cen)
                        edge [bend right=12] node [below] {\tiny{$18^v$, $30^v$}} (acba);
            
            \path (acba) edge [bend right=12] node [above] {\tiny{$18^i$, $30^i$}} (cof);

            \path (cuniv) edge [bend right=12] node [below] {\tiny{$18^v$}} (cen);
                        % edge [bend left] node [above] {$a$} (r);
            % \path (p2) edge node [above] {$a$} (q);
            % \path (q) edge node [above] {$b$} (r);
        \end{tikzpicture}\pause
        \\
        \vspace{0.5cm}
        % \hspace{1cm}
        % \raisebox{1.8cm}{
            % \begin{minipage}{0.45\textwidth}
                \scriptsize$\S_{\textsf{Pedro}} = \left\{
                    \begin{array}{c}
                        \{30^{i}30^{i},30^{v}30^{v}\}, \{18^{i}\}
                    \end{array}
                \right\}$ \\[0.1cm]\pause
                $\S_{\textsf{Ana}} = \left\{
                    \begin{array}{c}
                        \{30^{i}30^{i}\}, \{30^v30^v\}, \{18^{v}\}
                    \end{array}
                \right\}$
            % \end{minipage}
        % }
        % \caption{Representación gráfica de $\modults$}
        % \label{fig:ults}
    \end{figure}\pause
% \vspace{0.1cm}
    ¿El agente X sabe cómo llegar a Parque Horizonte dado que está en el Centro?
\end{frame}

% =========================================================================================================== %

\begin{frame}
  \frametitle{\lts basado en incertidumbre (\ults)}

  Sea $\PROP$ un conjunto contable de simbolos proposicionales y $\AGT$ un conjunto finito de nombres de agentes.\pause
  
  \begin{block}{\textbf{Definición}}
    Un \lts basado en incertidumbre (\ults) es una tupla $\tup{\W,\R,\{\S_i\}_{i\in\AGT},\V,\ACT}$ donde \pause
    \begin{itemize}
      \item $\W$ es un conjunto contable no vacío de estados\pause
      \item $\R$ es una colección de relaciones binarias $\{\R_a \subseteq \W\times\W\}$ donde $a\in\ACT$\pause
      \item $\S_i$ es una \orange{partición} sobre un conjunto no vacío de planes\pause
      \item $\V : \W \to \mathcal{P}(\PROP)$\pause
      \item $\ACT$ es un conjunto no vacío de nombres de acciones
    \end{itemize}
  \end{block}


\end{frame}


% =========================================================================================================== %

\begin{frame}
    \frametitle{Ejecutabilidad Fuerte}
    Los planes deben ser a \orange{prueba de fallas}: cada ejecución parcial debe poder completarse. \pause
    
    \renewcommand{\arraystretch}{1.6}
    \begin{ctabular}{cc}
    \begin{tabular}{@{\ \ \ \ \ \ \ }c}
        \begin{tikzpicture}
    \node [state, label = {[label-state]left:$w_1$}] (w1) {};
    \node [state, label = {[label-state]above:$w_2$}, above right = 0.5em and 3em of w1] (w2) {};
    \node [state, label = {[label-state]above:$w_3$}, right = 3em of w2] (w3) {};
    \node [state, label = {[label-state]below:$w_4$}, below right = 0.5em and 3em of w1] (w4) {};
    
    \path (w1) edge node [label-edge, pos = 0.35, above] {$a$} (w2)
                edge node [label-edge, pos = 0.35, below] {$a$} (w4)
            (w2) edge node [label-edge, above] {$b$} (w3);
        \end{tikzpicture}
    \end{tabular}
    &
    \begin{tabular}{l}
      \footnotesize  $ab$ no es fuertemente ejecutable en $w_1$
    \end{tabular}
    \end{ctabular}\pause
    \vspace{-0.2cm}
    \begin{block}{\textbf{Definición}}
      Un plan $\sigma$ es fuertemente ejecutable (FE) en un estado $w$ si y sólo si cada ejecución parcial de $\sigma$ a partir de $w$ 
      puede ser completada.
    \end{block}
\end{frame}

%=============================================================

\begin{frame}
    \frametitle{Sintáxis y semántica}\pause
    \begin{small}
        \begin{block}{Definición (\KHilogic sobre \ults)}
            Las fórmulas de $\KHilogic$ están dadas por la gramática
            \[
                \varphi ::= p \mid \neg\varphi \mid \varphi\vee\varphi \mid \khi(\varphi,\varphi),
            \]
            donde $p \in \PROP$ e $i \in \AGT$. \pause $\khi(\psi,\varphi)$ debe ser leído como \orange{``el agente $i$ sabe cómo lograr $\varphi$ dado $\psi$''.} \pause
            \\ 
        \medskip
        $\modults,w \models \khi(\psi,\varphi)$ si y sólo si existe \orange{un conjunto de planes} $\strategy \in \S_i$ tal que:

        \begin{quote}
            \begin{enumerate}
            \item cada plan en $\strategy$ es FE en cada estado donde vale $\psi$ y
            \item desde los estados en los que vale $\psi$, cada plan en $\strategy$ termina su ejecución en un estado donde vale $\varphi$.
            \end{enumerate}
        \end{quote}
        \end{block}

        % \pause

        % \begin{block}{Property:}
        %     Define $\orange{\A\varphi} := \bigvee_{i\in\AGT}\khi(\neg\varphi,\bot)$, we have:
        %     \begin{nscenter}
        %         $\model,w\models\A\varphi \mbox{ iff for all $v$, } \model,v\models\varphi;$
        %     \end{nscenter}
        %     i.e., $\A$ is the standard \orange{universal} modality (and its dual: $\E\varphi:=\neg\A\neg\varphi$).
        % \end{block}

    \end{small}

\end{frame}

\end{document}
