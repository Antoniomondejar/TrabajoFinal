\chapter{Conclusiones y trabajo futuro}

\section{Conclusiones}


\section{Trabajo futuro}

Los resultados obtenidos en este trabajo dejan las puertas abiertas a numerosas posibilidades dentro de la familia de lógicas de `Knowing How'. 
En primera instancia, se podría realizar un desarrollo similar alrededor de la noción de bisimulación de la lógica de `Knowing How' presentada 
en \cite{Wang15KH,Wang2018GoalDirectedKH}. Consideramos que el estudiado realizado para \KHilogic es fácilmente adaptable a esta lógica; por lo que, siguiendo 
una estrategia similar, se podría redefinir la noción de bisimulación para que considere características puramente estructurales entre los modelos, 
logrando así darle una naturaleza más procedural. También, sería interesante analizar su versión de $\KHiBisim$ y proponer su contracción 
por bisimulación, donde creemos que algunos lemas fundamentales estudiados en este trabajo también son adaptables a dicha lógica.

Por otro lado, otro trabajo interesante sería el de implementar algunos de los algoritmos presentados en este trabajo. Particularmente, 
la implementación de la contracción por bisimulación sería muy beneficiosa en un contexto en el que se requiera realizar muchas consultas del 
estilo de Model Checking sobre un \ults.

Por último, a partir de los resultados estudiados en \cite{Demri_Fervari_2023} en relación a la lógica de `Knowing How' basada en incertidumbre, 
consideramos que es posible generalizar el resultado de complejidad del problema de $\KHiBisim$ para la lógica que se intepreta sobre $\regults$, los 
cuáles son \ults en donde la relación de indistinguibilidad entre planes de cada agente está dada por un conjunto de autómatas finitos en lugar 
de conjuntos explícitos de planes. 
Notar que a un nivel lógico o de teoría de modelos, la clase de modelos $\regults$ es una subclase de los modelos \ults (pues restringen las clases de equivalencia 
a lenguajes regulares). Sin embargo, a nivel computacional los $\regults$ son más poderosos, dado que permiten representar de forma finita un conjunto 
posiblemente infinito de planes. En \cite{Demri_Fervari_2023}, se demuestra que dado un $\regults$ verificar si el lenguaje de un autómata 
es fuertemente ejecutable en un nodo es realizable en tiempo polinomial. A su vez, se demuestra que dado un $\regults$ con $A$ un autómata dentro de la relación de indistinguibilidad de 
un agente, es posible computar en tiempo polinomial $\R_{L(A)}(w)$ para $w$ en el dominio del $\regults$, siendo $L(A)$ el lenguaje aceptado 
por el autómata $A$.
A partir de estos resultados, conjeturamos que es posible adaptar lo estudiado sobre $\KHiBisim$ para demostrar que su versión para 
esta lógica sobre $\regults$ también es $\coNPComplete$.