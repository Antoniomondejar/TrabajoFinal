\chapter{Conclusiones y trabajo futuro}\label{ch:conclusiones}

\section{Conclusiones}

En este trabajo presentamos novedosos aportes al estudio de la noción de bisimulación de \KHilogic. En primer lugar, propusimos una nueva 
definición para este concepto. Para ello, caracterizamos los conjuntos proposicionalmente definibles del dominio de un \ults, lo que nos 
aportó una forma intuitiva de redefinir las cláusulas ($\khi$-zig) y ($\khi$-zag) de la definición presentada en \cite{ArecesFSV25,SaraviaPHD}.

A su vez, esta caracterización nos permitió demostrar los resultados de correctitud esperados para esta noción y determinar que 
el problema de definibilidad para \KHilogic está en la clase $\Poly$. Consideramos que nuestra propuesta es adecuada y se alinea mejor 
con las nociones clásicas de bisimulación, dado que logra garantizar equivalencia lógica entre dos \ultss a partir del análisis de sus
características estructurales. Asimismo, la reformulación de las cláusulas mencionadas favorece al comportamiento computacional de la 
noción de bisimulación de \KHilogic, pues evita la necesidad de listar cada fórmula proposicional para descubrir los conjuntos 
proposicionalmente definibles del dominio de un \ults.

Luego, estudiamos problemas computacionales relacionados con esta noción. En primer instancia determinamos que el problema de verificar 
si una relación binaria es una \KHilogic-bisimulación (\CBisim) es \coNPComplete. Para establecer la pertenencia a la clase \coNP propusimos
un algoritmo que verifica contraejemplos en tiempo polinomial y, para demostrar \textit{hardness} para \coNP, redujimos polinomialmente 
\DNFTAUT a \CBisim. Por otra parte, extendimos este resultado al problema de determinar la existencia de una \KHilogic-bisimulación entre 
dos \ultss (\KHiBisim) y, lo mismo hicimos para la versión punteada de este problema (\PKHiBisim). A su vez, mostramos cómo modificar el 
algoritmo propuesto para \CBisim para que compute una fórmula de tamaño polinomial que distingue a los \ultss involucrados en caso 
de detectarse que no son bisimilares.

Finalmente, abordamos el problema de la minimización de modelos para \KHilogic. Siguiendo la estrategia utilizada para el caso de la 
Lógica Modal Básica, investigamos dos posibles soluciones a partir del estudio de contracciones por \KHilogic-bisimulación. La primera 
de ellas surge del análisis de las autobisimulaciones. Particularmente, demostramos que la relación que colapsa los estados de un \ults a partir 
de su función de valuación ($Z_\model$) es su autobisimulación máxima. Luego, nuestra primera propuesta de contracción ($\fcont{\model}$) consiste en 
cocientar el dominio del \ults con esta partición y abstraer los planes de mayor relevancia en el \ults a aristas en el modelo contraído. 
Esta contracción minimiza la cantidad de estados del \ults; sin embargo, un punto negativo es que la cantidad de aristas puede aumentar.

En nuestra segunda propuesta ($\model_\bml$), utilizamos la contracción por \bml-bisimulación. Cada \ults tiene naturalmente un modelo de Kripke asociado, 
por lo que la estrategia para este enfoque consistió en contraer las componentes asociadas a este modelo de Kripke según la noción de 
bisimulación de \bml. A pesar de no minimizar la cantidad de estados del \ults, la particular virtud de esta propuesta es que existen 
numerosos algoritmos con excelente complejidad que computan la contracción por \bml-bisimulación.

Demostramos que ambas contracciones son correctas (producen un \ults bisimilar) y que pueden ser computadas en tiempo 
polinomial con respecto al tamaño del \ults. A pesar de que ninguna de las dos propuestas resulta en una solución cerrada para el problema 
de la minimización de modelos, ambas proveen interesantes aproximaciones que pueden ser de utilidad en aplicaciones prácticas y que 
contribuyen al entendimiento del poder expresivo de \KHilogic y la clase de modelos \ults.

\section{Trabajo futuro}

Los resultados obtenidos en este trabajo dejan las puertas abiertas a numerosas posibilidades dentro de la familia de lógicas de \textit{Knowing How}. 
En primera instancia, se podría realizar un desarrollo similar alrededor de la noción de bisimulación de la lógica de \textit{Knowing How} presentada 
en \cite{Wang15KH,Wang2018GoalDirectedKH} (\KHlogic). Consideramos que el estudiado realizado para \KHilogic es fácilmente adaptable a \KHlogic. Siguiendo 
la estructura de trabajo de esta tesis, se podría redefinir la noción de bisimulación de \KHlogic para que considere características puramente estructurales de los modelos involucrados, 
logrando así darle una naturaleza más procedural. Además, conjeturamos que a partir de la adaptación de algunos lemas fundamentales estudiados 
en este trabajo y los resultados de complejidad obtenidos en \cite{Demri_Fervari_2023}, es posible demostrar que tanto $\CBisim$ como 
$\KHiBisim$ y $\PKHiBisim$ resultan \PSPACECompletes para sus versiones en \KHlogic.  

Por otro lado, una tarea que surge naturalmente de este trabajo es la de implementar los algoritmos presentados. Particularmente, 
la implementación de la contracción por bisimulación sería muy beneficiosa en un contexto en el que se requiera realizar muchas consultas del 
estilo de model-checking sobre un \ults.

Por último, a partir de los resultados estudiados en \cite{Demri_Fervari_2023} en relación a \KHilogic, 
consideramos que es posible generalizar el resultado de complejidad del problema de $\KHiBisim$ para la lógica que se intepreta sobre $\regults$, los 
cuales son \ults en donde la relación de indistinguibilidad sobre planes de cada agente está dada por un conjunto de autómatas finitos en lugar 
de conjuntos explícitos de planes. 
Notar que la clase de modelos $\regults$ es una subclase de los modelos \ults (pues restringen las clases de equivalencia 
a lenguajes regulares). Sin embargo, a nivel computacional los $\regults$ son más poderosos. Estos permiten representar de forma finita un conjunto 
posiblemente infinito de planes. En \cite{Demri_Fervari_2023}, se demuestra que dado un $\regults$ verificar si el lenguaje de un autómata 
es fuertemente ejecutable en un nodo es realizable en tiempo polinomial. A su vez, se demuestra que dado un $\regults$ con $A$ un autómata dentro de la relación de indistinguibilidad de 
un agente, es posible computar en tiempo polinomial $\R_{L(A)}(w)$ para $w$ en el dominio del $\regults$, siendo $L(A)$ el lenguaje aceptado 
por el autómata.
A partir de estos resultados, conjeturamos que es posible adaptar lo estudiado sobre $\KHiBisim$ para demostrar que su versión para 
esta lógica sobre $\regults$ también es \coNPComplete.