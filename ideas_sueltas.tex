\chapter{Ideas sueltas}

\begin{teorema}
    Sea $\model_=\tup{\W,\R,\cset{\S_i}_{i \in \AGT},\V,\ACT}$ un modelo, y sea $Z \subseteq \W \times \W$ una relación binaria que satisface (Atom) y $Id := \{(w,w)\in \W\} \subseteq Z$, entonces $Z$ es una autobisimulación.
\end{teorema}

No se si dice mucho este teorema, pero al menos caracteriza un poco las autobisimulaciones.

Resumen de lo trabajado:

\begin{itemize}

\item Primera sección: Contracción por bisimulación

Conclusiones positivas acerca de las contracciones, nos dicen cosas sobre la expresividad de la lógica, en el sentido de que demostramos que dado un modelo siempre existe otro en el que la función de valuación es inyectiva (no hay dos nodos que satisfagan exactamente el mismo conjunto de variables proposicionales), usa planes de solo un paso y las clases de equivalencia de planes que conoce cada agente son simplemente singletons y aún así la lógica no sabe cómo diferenciarlos (son lógicamente equivalentes).

También, logramos contraer un modelo a otro donde chequear strong executability se logra en un solo paso, por lo que potencialmente se ahorra cómputo al no necesitar cada paso intermedio de un plan.

Ambas contracciones son computables en tiempo polinomial en el tamaño del modelo lo cuál es muy bueno.


Conclusiones negativas, no logramos encontrar una contracción que de verdad minimice el \rm\textbf{tamaño} del modelo en cuestión. Si bien, seguro encontramos un modelo lógicamente equivalente que minimiza la cantidad de nodos, puede darse el caso que la contracción aumente considerablemente la cantidad de aristas (sin embargo, hay que considerar el trade-off del ahorro en el chequeo de strong executability).

Cosas a analizar: 
\begin{itemize}
    \item Hay una contracción mejor?
    \item Puede tener que ver la definición de bisimulación con el hecho de que no encontramos una contracción totalmente convincente (ver teorema de esta sección capaz)? 
    \item Como dijo Carlos, puede tener que ver con la naturaleza de los modelos donde se interpreta la lógica? En ese caso considerar hacer un análisis desde teoría de modelos.
\end{itemize}


\item Segunda sección: Bisimulación entre dos modelos

Queda escribir el algoritmo para demostrar membership y revisar la demo de hardness para confirmar que la reducción está efectivamente bien.

Creo que este problema queda bastante completo una vez hechas esas dos cosas.

\end{itemize}

Anotaciones de Raul, reunión 19/06.

1. Re-definir la noción de Kh-bisimulación, usando las ideas de autobisimulación considerando la valuación, para el caso de modelos diferentes.

2. Demostrar que con la nueva definición valen: bisimulación implica equivalencia de fórmulas, y en el caso finito, vale la recíproca.

3. Demostrar que la unión de dos bisimulaciones es una bisimulación.

4. Darle nueva estructura a la parte de autobisimulación: dividir las dos dimensiones, es decir,contracción por estados (clásico) y luego la contracción por ''lenguaje'' (más raro).

A esto agregar:

1. Corregir sintaxis de la lógica.

2. Introducir en preliminares la noción existente de bisimulación.